\textbf{John Ruskin} (1819--1900), foi um dos mais importantes intelectuais da era vitoriana. Principal teórico da preservação arquitetônica e ambiental da Inglaterra do século \textsc{xix} e crítico perspicaz das transformações sociais trazidas ao país pela industrialização, a qual veementemente combateu. Excêntrico, vinculado ao romantismo, grande esteta, valorizava a sensibilidade subjetiva em contraponto à razão; contraditório --- ao mesmo tempo aristocrático, reacionário e simpático ao socialismo.

\textbf{Pr\ae terita} é a autobiografia do autor, sua última obra e testamento literário. Escrita ao longo de 27 anos, a desigualdade de suas partes reflete o estado mental do autor durante o período de sua elaboração. O livro, heterodoxo como gênero e de caráter ``experimental'', não segue os modelos de autobiografias então vigentes --- geralmente apresentados em termos de confissão religiosa ---, e poderia também ser considerado como narrativa de viagem, elegia, memória filial ou coleção de excertos de diários. Esta tradução contempla apenas o primeiro dos três volumes de \textit{Pr\ae terita}.

\textbf{Marcos A.\,P. Ribeiro} \lipsum[3]





