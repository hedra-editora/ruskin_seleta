\textbf{John Ruskin} (Londres, 1819--1900) foi um dos mais importantes intelectuais da era vitoriana. Principal teórico da preservação arquitetônica e ambiental da Inglaterra do século \textsc{xix} e crítico perspicaz das transformações sociais trazidas ao país pela industrialização, a qual veementemente combateu. Excêntrico, vinculado ao romantismo, grande esteta, valorizava a sensibilidade subjetiva em contraponto à razão; contraditório --- ao mesmo tempo aristocrático, reacionário e simpático ao socialismo.

\textbf{Pr\ae t\textls[-470]{e˘}rita} é a autobiografia do autor, sua última obra e testamento literário, escrita ao longo de 27 anos. Heterodoxa como gênero e de caráter \textit{experimental}, não segue os modelos de autobiografias então vigentes --- geralmente apresentados em termos de confissão religiosa ---, e poderia também ser considerada uma narrativa de viagem, elegia, memória filial ou coleção de excertos de diários. Esta tradução contempla o primeiro dos três volumes de \textit{Pr\ae t\textls[-470]{e˘}rita}.

\textbf{Marcos A.\,P. Ribeiro} é escritor e tradutor literário. Publicou \textit{A faculdade de medicina da Bahia na visão de seus memorialistas (1854--1924)}; \textit{Contos do porto da barra}; \textit{Entre os ``bárbaros filosóficos''}; \textit{O discurso do tempo}; \textit{A Bahia e a segunda guerra mundial: o front da retaguarda}; \textit{Cerca trova}, entre outros. Traduziu Lawrence Ferlinghetti, Dashiell Hammett, Robert Lowell e Ernst Jünger.\looseness=-1


% a desigualdade de suas partes reflete o estado mental de Ruskin durante o período de sua elaboração.

%Realizou uma residência literária em Hamburgo, na Alemanha.

% \textbf{Hedra Edições} reúne obras máximas da área de humanidades. São textos que interpelam o presente, através de títulos interdisciplinares que sugerem uma base de formação crítica individual, e compõem o que chamamos clássicos de intervenção: clássicos que são clássicos, afinal, porque são atuais. 


