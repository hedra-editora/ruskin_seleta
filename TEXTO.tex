\chapter{Prefácio à edição original}

\begin{flushright}
\textit{Herne Hill, 10 de maio de 1885}
\end{flushright}

\noindent{}Escrevi esses esboços dos incidentes e esforços de meus primeiros anos
para meus amigos; e para o público que aprecia meus livros.

\textls[-20]{Escrevi"-os de modo franco e gárrulo, ao correr da pena; estendendo"-me o
quanto quisesse sobre o que me dava prazer recordar --- algumas vezes
detendo"-me cuidadosamente sobre aquilo que pensava ser útil tornar
conhecido; e guardando total silêncio sobre coisas que não me davam
qualquer prazer recordar, e cujo relato não interessaria ao leitor. A
descrição de minha vida revelou"-se mais divertida do que eu esperava, à
medida que evoquei cenas do passado longínquo para submetê"-las ao
escrutínio do presente: --- os principais métodos de estudo e princípios
de trabalho utilizados, perdoo"-me por recomendá"-los a outros escritores;
e provavelmente qualquer leitor contumaz de meus livros poderá
entendê"-los melhor, por conhecer, tão completamente quanto fui capaz de
mostrá"-lo, meu caráter pessoal, o qual, sem procurar esconder, nunca me
esforcei para exibir, e mesmo, eventualmente, senti certo caprichoso
prazer em expô"-lo ao risco da má"-interpretação.}

Escrevo essas poucas palavras introdutórias no aniversário de meu pai,
no quarto que foi meu na infância, nesta velha casa, --- para a qual ele,
há sessenta e dois anos, trouxe a mim e a minha mãe, eu então com dez
anos. Nas páginas seguintes, o que poderia ser apenas o passatempo de um
velho, que colhe flores visionárias do campo de sua juventude, adquiriu,
enquanto as escrevia, a nobreza de uma respeitosa oferenda, no túmulo de
pais que me educaram para buscar todo o bem possível, e cuja lembrança
alegra o ocaso de minha vida, na esperança de logo estar novamente com
eles. 

\part{\textsc{Pr\ae terita}}

\chapter{As fontes de Wandel\footnotemark{}} %Capítulo \textsc{i}. 

\footnotetext{O leitor deve ser avisado que os dois primeiros capítulos são
reimpressões, com pequenas revisões, provenientes de \textit{Fors Clavigera},\footnotemark{} tendo sido incluídos naquela obra primordialmente para ensinamento político, algo que agora me parece ter representado certa violência. {[}\textsc{n.\,a.}{]}}

\footnotetext{Obra de John Ruskin: \textit{Fors Clavigera.
  Letters to the Workmen and Labourers of Great Britain}. Série de
  cartas dirigidas aos trabalhadores e às mulheres inglesas, publicadas
  entre 1871 e 1884.}

%1.
 Sou, e meu pai o fora antes de mim, um ferrenho \textit{Tory}\footnote{Refere"-se
  ao \textit{Tory Party}: partido conservador britânico.} da
velha estirpe; --- da escola de Walter Scott, isso quer dizer, e da de
Homero. Nomeio esses dois entre inúmeros grandes escritores Tories
porque foram meus dois mestres. Lia constantemente os romances de Walter
Scott e a \textit{Ilíada} (na tradução de Pope) quando criança, em dias de
semana: aos domingos, o efeito desses livros era matizado por
\textit{Robson Crusoé} e por \textit{O peregrino};\footnote{\textit{The \label{peregrino}
  Pilgrim's Progress --- The Pilgrim's Progress From The World, to That
  Which is to Come: Delivered Under the Similitude of a Dream Wherein is
  Discovered, the Manner of His setting Out, His Dangerous Journey; and
  Safe Arrival at The Desired Countrey} --- é uma alegoria religiosa em
  duas partes, da autoria do ministro puritano e escritor inglês John
  Bunyan (1628--1688). A primeira parte foi publicada em 1678 e a segunda parte, em 1684. Considerada a mais característica expressão do puritanismo religioso,
  a obra, em sua época, estava presente em todos os lares ingleses e só
  perdia em popularidade para a Bíblia.} minha mãe
nutrindo grandes esperanças de fazer de mim um clérigo. Afortunadamente,
havia uma tia mais fervorosa que minha mãe; e ela me deu carne de
carneiro fria no jantar de domingo, e isso --- por mais que a preferisse
quente --- diminuiu a influência de \textit{O peregrino}; o resultado foi
que adquiri toda a nobre imaginação ensinada por Defoe e Bunyan, e,
contudo, não me tornei um clérigo.

%2.
Contei, entretanto, com ensinamentos ainda melhores que esses, aos
quais era compulsoriamente submetido, a cada dia da semana.

Walter Scott e o Homero de Pope eram leituras de minha própria eleição;
minha mãe me obrigava, através de árduo trabalho diário, a memorizar
longos capítulos da Bíblia; como também a ler, sílaba por sílaba, em voz
alta, nomes difíceis, do Gênesis ao Apocalipse, uma vez ao ano: e a essa
disciplina --- paciente, acurada e resoluta --- devo, não apenas o
conhecimento da Bíblia, o qual, ocasionalmente, achei útil, mas muito de
minha capacidade para o trabalho duro, além do melhor de meu gosto
literário. Os romances de Walter Scott certamente fizeram com que eu me
interessasse por romances de outros autores; e Pope, talvez tenha me
levado ao inglês de Johnson\footnote{Samuel Johnson (1709--1784) foi escritor e lexicógrafo britânico.} ou de Gibbon,\footnote{Edward Gibbon (1737--1794) foi historiador britânico, autor do clássico \textit{The History of the Decline and Fall of the Roman Empire}.} como modelos de linguagem; mas, uma vez conhecendo o capítulo 32 do Deuteronômio, o Salmo 119, o capítulo 15 da Primeira Epístola aos Coríntios, o Sermão da Montanha, e a maioria do
Apocalipse, cada sílaba de cor, e tendo sempre refletido sobre o
significado das palavras, não me era possível, mesmo nas épocas mais
tolas da juventude, escrever de modo inteiramente superficial ou
meramente formal, em inglês; e a afetação de tentar escrever como Hooker
e George Hebert foi a mais inocente a que eu poderia ter sucumbido.

%3.
Dos meus mestres eleitos, Scott e Homero, aprendi, então, o Torismo,
que minhas reflexões posteriores apenas serviram para confirmar.

Isso significa o mais sincero amor aos reis e a aversão a todos que
tentam desobedecê"-los. Com Scott e Homero aprendi estranhas ideias sobre
os reis, as quais creio serem agora muito obsoletas; pois, percebi que
tanto o autor da \textit{Ilíada} como o autor de \textit{Waverley}\footnote{Obra
de Walter Scott, publicada primeiramente em anonimato.}
fizeram seus reis, ou as pessoas que amam os reis, trabalharem mais
duramente que qualquer outra. Tydides ou Idommeneus\footnote{Personagens
  da mitologia grega, citadas na \textit{Ilíada}, de Homero.} sempre matavam vinte troianos para cada um morto pelos outros,
e Redgaunlet\footnote{Personagem do romance homônimo de Sir Walter Scott.} traspassava mais salmões que qualquer dos pescadores
de Solway; e --- o que era para mim objeto de particular admiração ---
observei que eles não apenas faziam mais, mas que em proporção ao que
faziam, \textit{obtinham} menos que as outras pessoas --- e não apenas
isso: que os melhores dentre eles estavam até prontos a governar a troco
de nada! e a deixar seus seguidores dividirem os saques e os lucros.
Ultimamente, quer me parecer que a ideia de um rei tornou"-se exatamente
o contrário disso, tendo sido presumida, geralmente, como o direito de
pessoas superiores governarem menos e obterem mais que qualquer outra. 
Isso talvez tenha sido benéfico para mim, que naqueles dias da
infância contemplava a realeza à grande distância.

%4.
A tia que me dava carne de carneiro fria aos domingos era irmã de meu
pai: morava em Bridgend, na cidade de Perth, e tinha um jardim repleto
de groselheiras espinhosas transbordando para o Tay; a porta da casa
abria"-se para a água castanho"-clara que corria sobre os seixos a três ou
quatro pés de profundidade; turbilhonando rápido, --- algo infinito, aos
olhos de uma criança.

%5.
Meu pai iniciou"-se no comércio de vinhos, sem capital, e com
considerável montante de débitos herdados de minha avó. Aceitou as
dívidas e pagou"-as todas antes de começar a retirar algo para si
próprio, --- motivo pelo qual seus amigos o chamaram de tolo, e eu, sem
ter manifestado qualquer opinião sobre sua conduta, a qual, nesses
assuntos, sei que corresponde exatamente à minha, escrevi na placa de
granito sobre seu túmulo que ele era um ``comerciante absolutamente
honesto''. Com o tempo, ele pôde alugar uma casa na rua Hunter,
Brunswick Square, número 54 (as janelas, para minha felicidade,
descortinavam vista para uma maravilhosa fonte vertical de ferro, na
qual os carros"-pipa eram abastecidos através de pequenos sifões, munidos
de tubos que pareciam jiboias; e eu nunca me cansava de contemplar
aquele mistério e o delicioso gotejamento da fonte); com o passar do
tempo, e ao atingir quatro ou cinco anos de idade, meu pai pôde dispor
de uma pequena carruagem e uma parelha de cavalos por dois meses, no
verão, com o auxílio da qual, em minha companhia e na de minha mãe,
visitava seus clientes no campo (que gostavam de ver o próprio
negociante como seu principal viajante); de modo que, a passo de trote,
e através das vistas panorâmicas das quatro janelas da carruagem, ainda
mais panorâmicas para mim porque meu assento era colocado um pouco mais
à frente (como costumávamos alugar regularmente a carruagem à empresa
Long Acre, durante dois meses, podíamos tê"-la equipada e adaptada como
quiséssemos), eu via todas as grandes estradas e a maioria dos
cruzamentos da Inglaterra e do País de Gales; e grande parte das
terras"-baixas da Escócia, chegando até Perth, onde, a cada ano,
passávamos todo o verão: e eu costumava ler o \textit{Abbot},\footnote{Título
  de um romance Walter Scott, publicado em 1820.}
em Kinross, e \textit{Monastério}, em Glenn Farg, que eu confundia com
``Glendearg'', e pensava que a White Lady tivesse vivido à beira do
riacho no vale profundo de Ochils, como a Rainha dos Escoceses na
distante ilha de Loch Leven.

%6.
Para grande regozijo de meu pai, à medida que eu crescia, ia
conhecendo todas as casas de nobres da Inglaterra; com o reverente e
saudável deleite de uma admiração desinteressada --- percebendo, logo que
pude compreender alguma verdade política plenamente, que era provavelmente
muito melhor viver numa pequena casa e maravilhar"-se com o Castelo de
Warwick do que viver no Castelo de Warwcik e não ter nada com que se
maravilhar; isso, entretanto, em qualquer caso, não faria a praça
Brunswick mais agradável para se viver, caso o Castelo de Warwick fosse
destruído. E, até o presente momento, embora tenha recebido convites
muito gentis para visitar a América, não pude, mesmo por poucos meses,
viver num país tão pobre a ponto de não possuir castelos.

%7.
Tendo formado, entretanto, minha noção da realeza principalmente da
\textit{Lady of the Lake},\footnote{Título de um longo poema de Sir Walter
  Scott.} de Fitzjames,\footnote{James Fitzjames
  (1670--1734) foi um nobre e militar britânico de origem francesa; primeiro e único Duque de Berwick; estimado por sua coragem, habilidades militares e
  integridade.} e a da nobreza do Douglas deste livro, e
do Douglas em \textit{Marmion},\footnote{Título de um poema épico de Sir
  Walter Scott.} uma dolorosa dúvida logo surgiu
em minha mente infantil, a de por que os castelos não deveriam estar
sempre vazios. Tantallon\footnote{Castelo na Escócia.}
estava lá; mas não Archibald of Angus:\footnote{Nobre escocês, quinto Conde
  de Angus, nascido no castelo de Tantallon.}
Stirling,\footnote{Castelo na Escócia.} mas não o
Cavaleiro de Snowdoun.\footnote{Personagem de Sir Walter Scott, do poema
  narrativo \textit{Lady of the Lake}.} As galerias e
jardins da Inglaterra eram belos de se ver --- mas seu Senhor e sua
Senhoria estavam sempre na cidade, diziam"-me as governantas e os
jardineiros. Profundo anseio por algum tipo de \textit{Restauração}
apoderou"-se de mim, o que me fez lentamente perceber que Carlos
Segundo\footnote{Charles \textsc{ii} of England (1630--1685) foi rei da Inglaterra, Escócia e Irlanda. Assumiu o trono da Inglaterra e da Irlanda em 1661
  (havia sido declarado rei da Escócia em 1649), no episódio conhecido
  como \textit{Restauração}, depois do chamado \textit{English Interregnum}, no
  qual a Inglaterra, Escócia e Irlanda, após a abolição da monarquia,
  experimentaram um período republicano, sob o comando do \textit{Lord
  Protector} Oliver Cromwell.} talvez apenas a tenha esboçado, embora
sempre tenha usado uma noz"-de"-galha dourada na botoeira da lapela no 29
de maio.\footnote{Data do aniversário de Charles \textsc{ii}. Neste dia, em 1660,
  aos 30 anos, o rei entrou em Londres, ao retornar de nove anos de
  exílio no continente europeu.} Parecia"-me que a
\textit{Restauração} de Carlos Segundo, quando comparada com a \textit{Restauração} que eu desejava, fosse como a noz"-de"-galha dourada comparada a uma galha real.
Ao tornar"-me mais maduro, o desejo por frutos doces em vez dos amargos,
e de Reis Vivos em vez dos mortos, pareceu"-me tanto racional quanto
romântico; e, gradualmente, converteu"-se no principal objetivo de minha
vida desenvolver os bons frutos, sendo minha maior esperança ver
Reis.\footnote{A Companhia de São Jorge foi fundada para promoção da vida
  rural em vez da urbana : e minha única esperança de prosperidade para
  a Inglaterra, ou para qualquer outro país, qualquer que seja o tipo de
  vida que eles levem, é a descoberta de homens capazes de exercer a
  Realeza, e a sua obediência {[}\textsc{n.\,a.}{]}}

%8.
Nunca fui capaz de atribuir essa predisposição a qualquer
descendência real: dos ancestrais de meu pai, nada sei; nem dos de minha
mãe, além de minha avó, que era a proprietária do Old King's Head na
Market Street, em Croydon; e gostaria que ela ainda estivesse viva e eu
lhe pudesse pintar a cabeça do rei à maneira de Simone Memmi,\footnote{Pintor
  italiano, nascido em Siena no ano de 1283 e morto em Siena ou Avignon, em 1344 ou 1349.} como um emblema.

Meu avô materno era, como já disse, um marinheiro, que costumava
embarcar, como Robson Crusoé, em Yarmouth, e voltava depois de longos
intervalos, fazendo"-se adorado em casa. Creio que tivesse algo a ver com
o comércio de arengues, mas não sei exatamente em que circunstâncias;
minha mãe nunca falou muito a respeito. Ele a mimava e a sua irmã (mais
jovem), com todo seu carinho, quando estava em casa; a menos que isso
advenha de alguma tendência para equivocar"-se, ou para afirmações
provenientes da imaginação, por parte das crianças, algo sempre
imperdoável. Tendo percebido, uma vez, que minha mãe decididamente lhe
contara uma mentira, enviou imediatamente o criado para comprar um novo
feixe de giesta"-das"-vassouras para açoitá"-la. ``O feixe não me machucou
mais do que apenas um galho teria feito'', disse minha mãe, ``mas
\textit{imaginei} uma boa quantidade deles''.

%9.
Meu avô materno foi morto aos trinta e dois anos, ao tentar entrar
cavalgando, em vez de andando, em Croydon; o cavalo esmagou sua perna
contra um muro; e morreu da ferida gangrenada. Minha mãe tinha então
sete ou oito anos de idade e, com sua irmã, foi enviada a um externato
um tanto elegante (para Croydon), o da senhora Rice: onde aprendeu os
princípios do evangelho, e tornou"-se um exemplo de garota e a melhor
costureira da escola; e onde minha tia absolutamente rejeitou"-os (os
princípios), tornando"-se o bem e o mal dos evangelhos.

%10. 
Minha mãe, sendo uma garota de grande capacidade mental, e não pouco
orgulho, desenvolveu"-se exemplarmente, com completa consciência de seus
objetivos, sendo muito motejada pela irmã, embora também muito amada por
ela, que tinha mais perspicácia, porém menos orgulho, e nenhuma
consciência. Afinal, minha mãe, consumada como dona de casa, foi enviada
à Escócia para cuidar da casa de meu avô paterno, em processo de ruína;
ele acabou, efetivamente, arruinado, e se matou. Meu pai veio para
Londres; foi vendedor numa casa comercial durante nove anos, sem férias;
então abriu seu próprio negócio; pagou os débitos de seu pai; e casou"-se
com sua exemplar prima de Croydon.

%11. 
Entrementes, minha tia ficara em Croydon e casara"-se com um padeiro.
Nessa época eu tinha quatro anos, e começavam minhas lembranças --- meu
pai rapidamente galgando posições no mundo comercial londrino --- isso
era visível --- embora para mim, como criança, completamente
incompreensível --- recordo apenas de lampejos de vida na Hunter Street,
Brunswick Square, depois em Marker Street, Croydon. Mas quando meu pai
adoecia --- e o trabalho duro e o sofrimento já tinham deixado sua marca
nele --- íamos todos para Croydon para sermos mimados por minha singela
tia; e para andar nas colinas Dupas, e no urzal de Addington.

%12. 
Minha tia vivia numa pequena casa que ainda existe --- pelo menos
existia há quatro meses --- a mais elegante da Market Street, com duas
janelas sobre a loja, no segundo andar; mas nunca me interessei pela
parte superior da casa, a menos que meu pai estivesse fazendo desenhos
em nanquim, quando, então, sentava"-me reverentemente para observá"-lo;
meus domínios preferidos, nas outras ocasiões, eram a loja, o forno, e
as pedras circundando a fonte de água cristalina na porta dos fundos (há
muito transformado em tubo de esgoto); minha principal companhia, o
cachorro de minha tia, Towzer, de quem ela se apiedara quando era um
rabugento e faminto animal vadio; e o transformou num valente e afável
cão: o mesmo que fizera com qualquer criatura viva que cruzasse seu
caminho, durante toda sua vida.

%13. 
Com a ajuda dessas ocasionais visões dos rios do Paraíso, vivi feliz
na rua Hunter, praça Brunswick, onde passava a maior parte do ano, até
ter mais do que quatro anos; no verão, durante algumas semanas,
respirava o ar do campo, hospedando"-me em pequenos chalés
(chalés\footnote{\textit{Cottages}, em inglês: autêntica cabana aldeã,
  camponesa, rústica.} verdadeiros, não as assim chamadas
vilas,\footnote{\textit{Villas}, em inglês: casa de campo elegante para
  vilegiatura, de padrão superior às habitações locais.})
tanto perto de Hampstead, como em Dulwich, com a sra.\,Ridley, esta
última numa fileira de casas numa rua que levara aos campos de Dulwich,
repleta de botões"-de"-ouro, na primavera, e de amoras"-pretas, no outono.
Mas minhas lembranças mais intensas desses dias estão ligadas à rua
Hunter. Os princípios gerais de minha mãe, aqui aplicáveis em primeira
instância, eram proteger"-me de qualquer dor evitável ou perigo, através
de constante vigilância; e, de resto, deixar que me divertisse como
quisesse, desde que não me tornasse nem irritável nem importuno. Mas a
regra era que eu deveria encontrar meu próprio divertimento. Nenhum tipo
de brinquedo me era, em princípio, permitido; --- e o compadecimento da
tia de Croydon por minha pobreza monástica era imenso. Num de meus
aniversários, numa tentativa de sobrepujar a resolução de minha mãe,
pelo esplendor da tentação, ela comprou o mais radiante polichinelo e
namorada que pôde encontrar nos bazares do Soho --- tão grandes quanto um
polichinelo e namorada verdadeiros, vestidos de escarlate e ouro, e que
podiam dançar, amarrados a uma perna de cadeira. Devo ter ficado muito
impressionado, porque me lembro claramente do aspecto dos dois bonecos,
enquanto minha própria tia exibia suas virtudes. Minha mãe foi obrigada
a aceitá"-los; mas depois, discretamente, disse"-me que não era correto eu
os ter; e nunca mais os vi.

%14. 
Nem eu ardorosamente desejei --- algo que nem por um instante me era
permitido querer --- a posse dessas coisas que se veem em lojas de
brinquedos. Dispunha de um molho de chaves para brincar, na medida em
que só era capaz de ter prazer com o que brilhasse e tinisse; quando
fiquei mais velho, tive um carrinho e uma bola; e aos cinco ou seis
anos, duas caixas de blocos de madeira feitos com esmero. Com essas
modestas, mas que ainda considero inteiramente suficientes propriedades,
e sendo sempre sumariamente castigado se chorasse, não fizesse o que me
era ordenado, ou ficasse caído pelas escadas, rapidamente logrei
alcançar métodos serenos e seguros de viver e agir; e podia passar meus
dias, contente, desenhando quadrados e comparando as cores de meu
tapete; --- examinando os nós da madeira do assoalho, ou contando os
tijolos das casas do lado oposto da rua; com intervalos de excitação
arrebatadora durante o abastecimento dos carros"-pipa através de seus
tubos de couro, a partir da gotejante fonte vertical de ferro na borda
do passeio; ou dos ainda mais admiráveis procedimentos do encarregado do
abastecimento de água, quando ele girava, girava a fonte, até que
jorrasse no meio da rua. Mas o tapete, e os padrões que encontrava nos
cobertores, vestidos e papéis de parede examinados, eram meus principais
recursos de diversão, e minha atenção a esses detalhes, tão acurada que,
quando, às três e meia era levado para ser retratado pelo Mr.\,Northcote,
e, estando a menos de dez minutos em sua companhia, perguntava"-lhe
por que havia buracos em seu tapete. O retrato reproduzia uma bela
criança com cabelos amarelos, usando um vestido branco como se fosse uma
garota, com uma larga cinta azul"-clara e sapatos azuis, para combinar;
os pés da criança saudavelmente desproporcionais em relação ao corpo; e
os sapatos ainda mais saudavelmente desproporcionais em relação aos pés.

%15. 
Esses itens de uso diário eram todos enviados ao velho pintor para
uma reprodução perfeita; mas eles pareciam, no quadro, mais
extraordinários do que em meu quarto, porque eu era representado
correndo por um campo, na orla de um bosque, cujos troncos das árvores
eram retratados em sucessão, à maneira de Sir Joshua Reynolds;\footnote{Pintor
  e crítico de arte inglês (1723--1792) que estudou a obra de Rubens;
  considerado um dos fundadores de escola inglesa de pintura.} enquanto duas colinas arredondadas, tão azuis quanto meus
sapatos, eram visíveis à distância, colocadas ali, pelo pintor, por
minha própria insistência; pois eu já fora, uma ou duas vezes, levado à
Escócia; e minha babá escocesa sempre cantava para mim, ao nos
aproximarmos do Tweed ou do Esk:
\bigskip

\begin{verse}
Pois a Escócia, meu querido, surge completamente diante de ti,\\
Com suas mocinhas descalças, e montanhas tão azuis\footnotemark
\end{verse}

\footnotetext{No original: \textit{For Scotland, my darling, lis full in thy view\,/\,With barefooted lassies, and mountains so blue}.}

A ideia de colinas distantes estava associada, em minha mente, à
aproximação a felicidades extremas no jardim de groselheiras de minha
tia escocesa, descendo até o Tay. Mas quando o velho Mr.\,Northcote
perguntou"-me (não custa muito, imaginei, obter"-se resposta tão
explícita) o que eu gostaria de ter no horizonte de minha pintura, que
eu tenha dito \textit{colinas azuis} em vez de \textit{groselheiras}, parece"-me ---
e o faço sem qualquer tendência mórbida a me superestimar --- um fato
muito interessante, e certamente não desprovido de certa promessa, numa
criança daquela idade.

%16. 
Acho que isso se deve também ao fato, já mencionado, de, tendo sido
sempre castigado quando me tornava inoportuno, a serenidade adquirida
agradar muito ao velho pintor; pois me sentava imóvel, contando os
buracos no tapete, ou observando"-o espremer seus tubos de tinta, ---
verdadeiramente uma bela operação, aos meus olhos; --- mas não recordo de
qualquer interesse pela aplicação das tintas na tela; minha ideia de
deleite artístico envolvia a indispensável presença de um grande pote,
repleto do mais brilhante verde, e um pincel embebido da tinta. Mas
minha quietude agradava tanto ao velho pintor que solicitou aos meus
pais permissão para que eu pousasse como modelo para o rosto de uma
criança que pintava, num contexto clássico; no qual fui representado
reclinado sobre uma pele de leopardo, e com um espinho sendo tirado de
meu pé por um selvagem dos bosques.

%17. 
Em todos esses aspectos, reputo o tratamento recebido, ou as
condições incidentais de minha infância, inteiramente corretos para uma
criança com o meu temperamento: mas a maneira pela qual fui apresentado
à literatura parece"-me questionável, e não estava preparado para
estudá"-la nas escolas de St.\,George sem que houvesse muita transformação.
Absolutamente declinei a aprender a ler pelas sílabas; mas poderia dizer
uma frase inteira de cor, com grande facilidade, e apontar com precisão
para cada palavra na página enquanto a repetia. Contudo, quando as
palavras eram mudadas de lugar, eu nada dizia, e minha mãe desistiu,
durante algum tempo, da tarefa de me ensinar a ler, esperando que eu,
com o tempo, acabasse por consentir em adotar o sistema popular de
aprendizado das sílabas. Mas, para me divertir, continuei com o meu
processo, aprendendo grupos de palavras de uma única vez, à medida que
estabelecia padrões; e aos cinco anos passei para os ``segundos
volumes'', na biblioteca circulante.

%18. 
Esse esforço para apreender as palavras em seu coletivo era
acompanhado pela minha verdadeira admiração pela aparência dos tipos
impressos, os quais comecei a copiar por prazer, como outras crianças
desenham cães e cavalos. As inscrições a seguir, fac"-similadas da guarda
de meu \textit{Seven Champions of Christendom}\footnote{Em português, \textit{Sete
  Campeões da Cristandade}. É um clássico infanto"-juvenil do escritor inglês
  William Henry Giles Kingston (1814--1880).} (teria uma
visão muito independente desses caracteres, a julgar pelo aspecto da
letra \textit{L}, e da relativa elevação do \textit{G}), acreditei que seriam um estudo
artístico muito precoce das formas; como em \textit{Fors}, em que as
primeiras linhas da nota, escritas após um intervalo de cinquenta anos,
na parte inferior de minha cópia, sob a direção de Mr.\,Burgess,
apresentava alguns notáveis pontos de correspondência com ele; pensei
que bem poderiam ter sido impressas juntas, tal como se apresentavam.

%19. 
Minha mãe havia, como posteriormente me disse, solenemente ``me
devotado a Deus'', antes de meu nascimento; imitando, assim, à
Anna.\footnote{Hannah, no original; refere"-se a Santa Anna, mãe da Virgem
  Maria.}

Boas mulheres são notavelmente habilidosas em se utilizar desse
expediente para livrar"-se precocemente de seus filhos: o verdadeiro
significado do ato piedoso é que, como os filhos de Zebedeu\footnote{Personagem
  bíblico; pescador da Galileia, pai de Tiago e João, discípulos de
  Jesus Cristo, e marido de Salomé.} não estão destinados
(ou pelo menos não se espera isso) a se sentar à direita e à esquerda de
Cristo, em Seu reino, seus próprios filhos talvez possam, pensam elas,
antecipadamente se candidatarem àquela respeitável posição na vida
eterna; particularmente se elas o pedirem a Cristo, muito humildemente,
todos os dias: mas elas sempre esquecem, da maneira mais ingênua, que
não compete a Ele conceder tal posição!

%20. 
``Devotar"-me a Deus'' significou, até onde minha mãe conhecia suas
próprias pretensões, que ela tentaria me mandar ao seminário e fazer de
mim um religioso: e que eu, portanto, nascera ``para a Igreja''. Meu
pai --- que sua alma descanse em paz ---, que tinha o terrível mau hábito
de ceder a minha mãe nas grandes coisas e agir segundo sua própria
vontade nas pequenas, permitiu, sem nada dizer, que eu fosse retirado do
comércio de \textit{sherry}, considerado como algo impuro; não sem a
desculpável influência daquele destino que minha mãe reservara para mim.
Pois, muitos e muitos anos depois, lembro, quando ele estava falando com
um de nossos amigos artistas, admirador de Rafael, e lamentando meu
empenho para me contrapor àquele gosto tão popular, --- quando meu pai e
ele estavam se consolando pelo fato de terem tido a pretensão de pensar
que eu poderia falar em público sobre Rafael e Turner, --- em vez de
contentar"-me, como deveria, em explicar"-lhes o caminho para a salvação
de suas almas --- e que afável religioso fora perdido em mim --- ``Sim'',
disse meu pai, com lágrimas nos olhos, --- (verdadeiras e ternas lágrimas,
como nunca papai derramara) ``ele teria sido um Bispo''.

%21. 
Para minha sorte, minha mãe, com essas distintas ideias sobre seu
próprio dever, e as esperanças latentes sobre a eminência de meu futuro,
conduziu"-me muito cedo à igreja; --- onde, a despeito de meus hábitos
discretos, e do frasco dourado de sais aromáticos de minha mãe, que
sempre me fascinava, com a tampa desatarraxada que me permitia ver o
padrão de flores entrelaçadas sob a esponja de banho, eu achava o banco
(da igreja) um lugar tão enfadonho para se permanecer quieto (meus
melhores livros de história também eram tirados de mim, pela manhã),
que, como disse antes em algum lugar, o horror do domingo costumava
projetar a presciência de sua melancolia até sexta"-feira --- e toda a
glória da segunda"-feira, distante sete dias da ida à igreja, não a
compensava.

%22. 
Não obstante, minha mente foi impregnada por alguns dos sermões do
reverendo Mr.\,Howell; e, eventualmente, imitava"-o, proferindo sermões em
casa, sobre as almofadas vermelhas do sofá; --- a performance sendo sempre
requerida pelos mais queridos amigos de minha mãe, que a consideravam a
grande realização de minha infância. O sermão, creio, continha onze
palavras; muito exemplar, parece"-me, a esse respeito --- e ainda penso
que deve ter sido o mais puro evangelho, pois lembro que começava com a
expressão: ``Sejam bons''.

%23. 
Raramente tínhamos companhia, mesmo nos dias da semana; e nunca me
era permitido descer para a sobremesa, até ficar bem mais velho --- e me
tornar capaz de quebrar nozes com habilidade. Era"-me permitido, então,
descer para quebrá"-las para os convidados --- (espero que tenham gostado
de meu desempenho) --- mas não podia comer nenhuma noz; nem qualquer
outra guloseima, nestas ocasiões ou em outras. Certa vez, em Hunter
Street, lembro"-me de minha mãe, depois do almoço, ter"-me dado três
passas, que ela apanhara da despensa; e recordo perfeitamente a primeira
vez que comi pudim, em nossos alojamentos na Norfolk Street --- onde
vivíamos enquanto nossa casa era pintada, ou limpa, ou algo do gênero.
Meu pai jantara na sala da frente e não terminara seu pudim; minha mãe
trouxe"-me o que sobrara, nos fundos da casa.

%24. .... 25 onde está? Conferir
Mas, para melhor compreensão, pelo leitor, dos progressos de minha
pobre existência, mesmo com o risco de abusar de sua paciência, faz"-se
necessário fornecer algumas informações acerca da posição de meu pai no
mundo mercantil de Londres.

Alguns dos comerciantes mais antigos talvez ainda se lembrem da firma da
qual meu pai era o sócio principal, que funcionava num pequeno
escritório num primeiro andar de escassas dimensões, numa viela de East
London, --- na Billiter Street, a principal transversal entre Leadenfall
Street e Fenchurch Street.

Os nomes completos dos três sócios constavam da placa de bronze colocada
sob o sino do escritório, --- Ruskin, Telford and Domecq.

O nome do Mr.\,Domecq, por direito, deveria ser o primeiro, pois meu pai e
o Mr.\,Telford eram apenas seus agentes. Ele era o único dono da
propriedade que constituía o principal capital da firma, --- o vinhedo de
Macharnudo, a mais valiosa encosta, para produção de vinho branco, da
península espanhola. A qualidade da vindima de Macharnudo fora essencial
para fixar os padrões do \textit{jerez sack}, ou \textit{dry} --- \textit{secco} --- \textit{sherris}, ou \textit{sherry}, dos dias de Henrique \textsc{v} aos nossos; o inalterável e incomparável solo de calcário argiloso nutrindo a vinha, a qual o tempo apenas enriquece e robustece, --- nunca debilita.

%26. 
O Mr.\,Domecq era, creio, espanhol de nascimento; educado na França e
na Inglaterra; homem da mais estrita honra e amável disposição; suas
origens, desconheço; como se tornou proprietário do vinhedo, desconheço;
que posição ocupava, quando jovem, na firma Gordon, Murphy \& Company,
desconheço; mas foi nesta empresa que pôde observar, durante nove anos,
o desempenho de meu pai, o principal empregado, e quando a casa faliu,
convidou"-o para ser seu agente na Inglaterra. Meu pai percebeu que
poderia confiar inteiramente na honra e na afeição do Mr.\,Domecq; mas não
tão completamente em seu senso comercial, ou em sua engenhosidade; e
insistiu, embora recebendo apenas sua comissão de agente, em ser, tanto
legalmente quanto operacionalmente, sócio principal da firma.

%27. 
Mr.\,Domecq passava a maior parte do tempo em Paris; raramente
visitava sua propriedade espanhola, mas tinha perfeito domínio dos
processos de cultivo e autoridade sobre seus empregados quase como um
chefe sobre seu clã. Mantinha as videiras dentro do padrão mais elevado;
e permitia a meu pai deliberar livremente sobre todos os aspectos
relativos à venda das uvas. O segundo sócio, Mr.\,Henry Telford, aportou
ao negócio o capital necessário ao estabelecimento da filial de Londres.
O escritório da Billiter Street pertencia"-lhe; e ele tinha uma agradável
casa de campo em Widmore, perto de Bromley; então um tranquilo e
distante vilarejo de Kent.

Ele personificava à perfeição o cavalheiro rural inglês de moderados
recursos; solteiro, vivia com três irmãs que não se casaram, --- as
quais, com seu refinamento, elevada educação, despretensão, benevolência
e felicidade, permanecem em minha memória mais como personagens de
contos de fada do que como pessoas verdadeiras. Nem em ficção nem em
realidade soube eu jamais de alguém como o Mr.\,Henry Telford; --- tão
amável, tão modesto, tão afetuoso, dotado de bom"-senso, grande amante
dos cavalos, --- e tão inteiramente incapaz de fazer, pensar ou dizer
algo que pudesse prejudicar o desempenho de um cavalo ou de um haras.

%28. 
Creio, entretanto, que nunca tenha perdido qualquer grande corrida;
que tenha passado a maior parte de sua vida sobre um cavalo; e que tenha
caçado durante toda a temporada de Leicestershire; mas nunca fez uma
aposta, nunca sofreu um acidente grave, nunca feriu um cavalo. Entre ele
e meu pai havia confiança absoluta e a maior e mais desinteressada
amizade possível. Meu pai orgulhava"-se da posição que o Mr.\,Telford
ocupava na nobreza rural; e o Mr.\,Telford tinha um respeito afetuoso pela
industriosidade e pelo infalível instinto comercial de meu pai. A
participação do Mr.\,Telford na condução do negócio limitava"-se ao
trabalho no escritório durante os dois meses de verão em que meu pai
tirava férias, e, algumas vezes, durante um mês, no início do ano,
quando meu pai viajava a serviço. Nessas ocasiões, o Mr.\,Telford
cavalgava diariamente de Widmore a Londres, assinava cartas e faturas,
lia os jornais, e cavalgava de volta a casa; qualquer questão que
requeresse decisão seria remetida a meu pai, ou aguardaria seu retorno.
Toda a família em Widmore teria nos recebido com grande amabilidade caso
os tivéssemos visitado; mas minha mãe sempre teve em mente, quando em
contato com a sociedade refinada, --- e seu orgulho não permitia
suportá"-lo pacientemente, --- das lacunas de sua própria educação; e,
consequentemente (o que era verdadeiro e decisivo sinal de tal falha),
nunca visitou a quem não percebia como sendo, de algum modo, socialmente
inferior.

Contudo, o Mr.\,Telford teve uma influência singularmente importante em
minha educação. Seguindo, creio, conselho de suas irmãs, ele me deu,
logo que publicada, a edição ilustrada da \textit{Itália}, de
Rogers.\footnote{\textit{Italy}, obra do poeta inglês Samuel Rogers
  (1763--1855), ilustrada por Joseph Mallord William Turner (1775--1851),
  através da qual Ruskin conheceu os trabalhos do pintor que seria uma
  de suas maiores influências.} Este livro me permitiu,
pela primeira vez, analisar detalhadamente o trabalho de Turner: e eu
poderia atribuir ao presente, aparentemente com razão, o direcionamento
das energias de minha vida. Mas é um grande descuido dos biógrafos
atribuírem a este acontecimento, que introduz alguns novos aspectos à
minha personalidade, toda a conformação de meu caráter, superestimando o
valor do episódio. O ponto essencial a ser observado, e considerado, foi
que pude entender o trabalho de Turner, quando o vi; --- não importa em
que circunstâncias, ou em que ano, o trabalho tenha sido visto pela
primeira vez. O pobre Mr.\,Telford, entretanto, foi sempre considerado,
por papai e mamãe, como o principal responsável por minha desvairada
paixão por Turner.

%29. 
Numa maneira mais direta, embora menos intencional, sua ajuda me foi
muito importante. Pois, antes que meu pai pensasse em alugar uma
carruagem para as já mencionadas férias de verão, o Mr.\,Telford sempre
nos emprestou sua própria carruagem de viagem.

Àquela época a velha carruagem inglesa era o mais luxuoso dos veículos
de viagem, para duas pessoas, ou mesmo para duas pessoas e um terceiro
passageiro, já que eu tinha três anos. A que utilizávamos era alta, e
podíamos facilmente ver de cima os diques de pedra e as sebes de
meia"-altura; subíamos à carruagem por um antiquado degrau dobrável, com
uma encantadora almofada que se ajustava perfeitamente ao recesso da
porta, --- um dos meus maiores deleites de viagem era ver o cavalariço
desdobrá"-lo e dobrá"-lo; embora meu prazer fosse maculado pela invejosa
ambição de obter permissão para fazê"-lo eu mesmo: --- mas nunca a obtive,
--- pelo receio de que pudesse apertar meus dedos.

%30. 
O \textit{dickey},\footnote{A palavra origina"-se de \textit{Dicky}, 
  apelido de \textit{Richard}.} --- (pensar que até este \label{dickey}
momento nunca deva ter me perguntado a origem desta palavra, e que agora
era incapaz de alcançá"-lo!) --- era, tipicamente, o assento em posição
dominante no correio de Sua Majestade, ocupado pela guarda; clássico,
mesmo na literatura moderna, como na cena do arranjo entre o Mr.\,Bob
Saywer e Sam,\footnote{Bob Sawyer e Sam Weller são personagens do romance
  \textit{The Pickwick Papers}, do escritor inglês Charles Dickens
  (1812--1870).} --- na carruagem do Mr.\,Telford
localizava"-se muito atrás, para permitir que as pernas ficassem
perfeitamente confortáveis (se alguém escolhesse viajar do lado de fora
nos dias de bom tempo), e para criar, embaixo do assento, um bom espaço
para o porta"-bagagem traseiro. Sobre o qual --- e também sobre toda a
bagagem à frente e colocada em cima da carruagem --- minha babá Anne
reinava, como guardiã e empacotadora; com sua correção e precisão
incomparáveis em bordar vestidos e virar panquecas. Esta delicada
habilidade, observe"-se, denota perspicácia e espírito criativo; pois, a
arte de arrumar bagagens e comandar uma campanha militar exige a
capacidade de conjeturar.

%31. 
Entre as pessoas que perdemos na vida, quando passamos dos
cinquenta, mortas, ou pior que isso, posso dizer que, de quem
verdadeiramente sinto mais falta, além de papai e mamãe 
(desconsiderando"-se as perdas imaginárias), é de Anne, babá de meu pai e
minha. Ela foi uma de nossas ``muitas''\footnote{\textit{Many}, no
  original; trocadilho com o arcaísmo \textit{Meinie}: dama de companhia;
  pessoa agregada à família.} (as \textit{muitas} sendo quase
sempre poucas), e da infância à velhice dedicou"-se inteiramente a nos
servir. Tinha natural talento e habilidade para fazer coisas
desagradáveis; sobretudo, os cuidados de um doente; assim seu desempenho
mais satisfatório era alcançado quando um de nós estava enfermo. E a
habilidade paralela de \textit{dizer} coisas desagradáveis; nesta questão
era confiável: apresentava o lado mais sombrio da situação antes de
atuar para resolvê"-la. Exibia também uma louvável e republicana aversão
a obedecer a ordens de fazer alguma coisa imediatamente; ao envelhecer
juntamente com minha mãe, esta tendo se tornado autoritária e caprichosa
sobre o lado da mesa redonda em que deveria ficar sua xícara de chá,
Anne sempre a punha, precisa e meticulosamente, do outro lado; o que
fazia mamãe me dizer gravemente, a cada manhã, após o desjejum, que se
alguma mulher já fora possuída pelo demônio, esta seria Anne. Mas, a
despeito dessas aspirações momentâneas e petulantes à liberalidade e à
independência de caráter, a pobre Anne permaneceu sempre servil em
espírito; e esteve sempre ocupada, dos quinze aos setenta e dois anos em
satisfazer o desejo de outras pessoas em vez do dela: nunca soube de que
tenha feito mal a um ser humano, exceto pelo fato de ter poupado pouco
mais de duzentas libras para seus parentes; como consequência, após seu
funeral, alguns deles não se falaram por muitos meses.

%32. 
O \textit{dickey} anteriormente mencionado, sendo indispensável que
acomodasse nossa vigia, Anne, era largo o suficiente para caber duas
pessoas, já que meu pai também poderia sentar"-se ao ar livre se a
paisagem e o dia fossem agradáveis. A carruagem não era das mais leves;
mas com a cuidadosa limitação da bagagem, seguia lépida atrás de bons
cavalos sobre as perfeitamente lisas estradas dos Correios; o serviço
dos Correios, nessa época, era universal, assim, nas principais
hospedarias de cada condado, à chegada de cada carro, o grito ``Soltem
os cavalos!'' era ouvido no pátio, sendo respondido, quase sempre
imediatamente, e no máximo dentro de cinco minutos, pelo alegre trote de
um cavaleiro com botas e jaqueta brilhante através do pórtico, com dois
cavalos ajaezados, --- não havia assento para o condutor, na frente: e as
quatro grandes janelas, muito bem encaixadas e com corrimento perfeito,
não permitiam a passagem de nenhuma gota de chuva quando fechadas e
nunca emperravam quando eram abertas, formando uma grande ogiva
envidraçada móvel, pela qual se podia ver a paisagem até a linha do
horizonte. Minha perspectiva ainda era mais ampla, porque meu assento
era a pequena caixa contendo minhas roupas, solidamente construída, com
uma almofada num dos lados; colocada verticalmente na frente (e bem para
adiante), entre meu pai e minha mãe.

%33. 
As férias de verão começavam geralmente por volta de quinze de maio;
para que a aproveitássemos melhor, o Mr.\,Telford nos disponibilizava tais
luxos; --- o aniversário de meu pai era no dia dez; neste dia eu sempre
tinha permissão para colher groselhas para a primeira torta de groselhas
do ano, da árvore entre os contrafortes da parede norte do jardim de
Herne Hill; não podíamos partir antes desta \textit{festa}.\footnote{Assim
  no original.} As férias propriamente ditas
consistiam num \textit{tour} a clientes por metade dos condados da
Inglaterra; e uma visita (se os condados estivessem na direção norte) a
minha tia na Escócia.

%34. 
Nosso ritmo de viagem era tão pré"-estabelecido quando nossa rotina
doméstica. Percorríamos de quarenta a cinquenta milhas por dia,
começando sempre cedo o suficiente para chegar confortavelmente a tempo
de jantar às quatro horas. Acordando, portanto, geralmente às seis
horas, uma parte ou duas da viagem eram feitas antes do desjejum, com o
orvalho sobre a grama, e o primeiro aroma dos espinheiros; se durante a
primeira metade da viagem a casa de algum gentil"-homem devesse ser
visitada, --- ou, melhor ainda, a casa de um Lord --- ou, no melhor dos
casos, a casa de um Duque, --- meu pai alimentava os cavalos e conduzia a
mim e a minha mãe, reverencialmente, pelos salões formais; dirigindo"-se
sempre à meia"-voz à governanta, ao mordomo, ou a outro responsável pela
casa; recolhendo com adoração qualquer migalha de informação sobre a
história e a vida doméstica daquela família, que escapasse de seus
lábios.

%35. 
Ao analisar acima, à página 5, o efeito de tudo isso sobre mim,
tenha talvez antecipado um pouco a impressão resultante de que era
provavelmente mais feliz viver numa pequena casa que numa grande. Mas,
seguramente, enquanto, até hoje, nunca passei diante de uma casa de
campo com janelas de treliça sem desejar ser o proprietário, nunca
conheci um castelo cujo senhor eu invejasse; e, embora no curso dessas
muitas peregrinações de adoração, eu tenha, por curiosidade, adquirido
extenso conhecimento tanto de arte como da paisagem natural, que me
seria infinitamente útil posteriormente, é evidente para mim,
retrospectivamente, que meu próprio caráter e inclinações tenham sido
pouco alterados pela experiência; e que o sentimento pessoal e o
instinto natural tenham se dirigido, irrevogavelmente, às coisas
simples, modestas, puras em sua paz, sob os baixos tetos vermelhos de
Croydon, à margem dos regatos floridos com agrião, ao fundo dos quais a
areia dançava e os peixinhos saltavam sobre as Fontes de Wandel.

\chapter*{As amendoeiras em flor\\de Herne Hill} %Capítulo \textsc{ii}. 
\markboth{As amendoeiras em flor}{}
\addcontentsline{toc}{chapter}{As amendoeiras em flor de Herne Hill}

%36. 
Quando eu tinha cerca de quatro anos, meu pai encontrou"-se em
condições de alugar uma casa em Herne Hill, uma colina sáfara a quatro
milhas ao sul do \textit{Standart de Cornhill}, cuja situação de isolamento
entre a vegetação cerrada permanece essencialmente inalterada até hoje:
certa pompa gótica, posteriormente introduzida por nossos ricos
vizinhos, foram as únicas modificações importantes; e estão tão
graciosamente escondidas pelas belas árvores das propriedades que não
incomodam o passante; e ainda posso subir e descer a estrada entre a
taverna Fox e a estação de Herne Hill, imaginando"-me aos quatro anos de
idade.

%37. 
Nossa casa era a mais ao norte do grupo que ficava exatamente no
topo ou cimo da colina, onde a superfície plana é estreita, como o
espaço que a neve ocupa (suponho), no cume do Monte Branco; logo
declinando, contudo, naquilo que, através da formação argilosa de
Londres, poderia ser considerado uma íngreme descida em direção a nosso
vale de Chamouni (ou de Dulwich), a leste; e com uma inclinação mais
suave em direção ao caminho de Cold Habour,\footnote{A História de
  Croydon informa que este nome há muito intriga os arqueólogos, sendo
  sempre encontrado próximo a ruínas militares romanas. {[}\textsc{n.\,a.}{]}}
a oeste; ao sul, declinando, não com menos beleza, em direção ao vale de
Effra.\footnote{Sem dúvida, forma abreviada de Effrena, que significa o rio
\textit{intransponível}; sobre o qual, recentemente, lamento dizê"-lo, foi
construída uma ponte, para a conveniência do sr.\,Biffin, o farmacêutico,
e de outros. {[}\textsc{n.\,a.}{]}} Enquanto, ao norte, a colina se prolongava numa ligeira
depressão de aproximadamente meia milha, recebendo, na paróquia de
Lambeth, o nome cavalheiresco de ``Champion Hill''; finalmente, mergulha
para extinguir"-se na planície de Peckham, e no barbarismo rural de Goose
Green.

%38. 
O grupo, do qual nossa casa era a quarta, consistia de dois pares de
casas em tudo iguais, inclusive os jardins; eram ainda os dois mais
altos blocos de construções que podiam ser vistos de Norwood, sobre a
crista da colina; como nossa própria casa, com três pavimentos e sótãos,
cujas janelas permitiam, naqueles incomparáveis dias sem fumaça, a
extraordinária visão das colinas de Norwood, de um lado, sob o nascente
sol de inverno; e do vale do Thames, do outro, Windsor telescopicamente
claro à distância, e Harrow, sempre nítido à visão contra o pôr do sol
do verão, em dias de tempo bom. A casa tinha jardins, na frente e nos
fundos, em proporções adequadas ao seu tamanho; o da frente, ricamente
plantado com sempre"-vivas antigas, lilases já crescidos e laburnos; o
dos fundos, setenta jardas\footnote{Uma \textit{yard}, medida inglesa de
  comprimento, corresponde a 91,44 centímetros.} de comprimento
por vinte de largura, reputado em toda a colina pela qualidade de suas
pereiras e macieiras, que foram escolhidas com extremo cuidado pelo
nosso antecessor (envergonho"-me por esquecer o nome de um homem a quem
devo tanto!) --- possuindo também uma velha e forte amoreira, uma alta
cerejeira ``coração"-banco'', outra cerejeira negra de Kent, e uma sebe
quase contínua em volta do jardim com groselheiras; cobertas, durante a
temporada (pois o solo era benéfico), com o mágico esplendor de
abundantes frutos: verde novo, depois âmbar suave, finalmente carmesim
áspero, eriçado, pendendo dos galhos espinhosos; cachos de peras e de
frutos rubis, que pareciam uvas, alegremente descobertos sob as folhas
largas.

%39. 
As diferenças primordiais que observei entre a natureza desse jardim
e aquele do Éden, como eu o havia imaginado, eram, que, nesse,
\textit{todas} as frutas eram proibidas; e não havia animais dóceis: em
outros aspectos, o pequeno jardim correspondia a todas minhas
expectativas sobre o Paraíso; e o clima, naquela época, atravessava um
período favorável, o que me permitia passar a maior parte do tempo no
jardim. Minha mãe nunca apresentou para meu aprendizado mais do que ela
sabia que eu poderia facilmente absorver se começasse a trabalhar
seriamente a partir das doze horas. Ela nunca permitiu que nada me
perturbasse enquanto eu estava trabalhando; se, às doze horas, não
apresentasse corretamente a lição, era obrigado a continuar a estudá"-la
e, geralmente, mesmo depois que a Gramática Latina fora adicionada aos
Salmos, dispunha de pelo menos uma hora livre antes do almoço, que
acontecia à uma e meia, voltando a ficar livre pelo resto da tarde.

%40. 
Minha mãe, cujo maior prazer eram suas flores, estava sempre
plantando ou podando ao meu lado; pelo menos nos momentos em que eu
escolhia permanecer ao \textit{seu} lado. Nunca pensei em fazer às suas
costas o que não teria feito diante dela; sua presença não era,
portanto, um estorvo; mas, também, não significava particularmente um
prazer, pois, tendo sido deixado tanto tempo sozinho, sempre tinha
minhas próprias pequenas coisas para cuidar; e, desse modo, quando
contava sete anos, já era mentalmente muito independente, até mesmo de
meus pais; e, não tendo ninguém mais de quem depender, comecei a levar
uma pequena, satisfeita, empertigada e presunçosa vida de jovem Robinson
Crusoé, no ponto central do universo que eu acreditava ocupar (como
devem naturalmente acreditar todos os animais de estrutura geométrica).

%41. 
Isso se devia, em parte, à modéstia de meu pai e, em parte, ao seu
orgulho. Em tais assuntos ele tinha muito mais confiança no julgamento
de minha mãe do que no seu próprio, e nunca ousou auxiliá"-la, muito
menos contradizê"-la, a respeito de minha educação; por outro lado,
contrariando seu permanente objetivo de fazer de mim um cavalheiro
eclesiástico, com as mais refinadas maneiras, e acesso aos mais altos
círculos das sociedades laica e espiritual, as visitas a Croydon, onde
vivia a tia que eu tanto amava, e meus jovens primos, filhos de padeiro,
foram se tornando cada vez mais raras: as relações sociais com nossos
vizinhos da colina não poderiam ser mantidas sem a modificação de nosso
regular e gentil egoísmo; em suma, de uma maneira tipicamente infantil,
eu não tinha interesse por nada, exceto por mim mesmo, alguns ninhos de
formigas, os quais o jardineiro jamais deixava para minhas brincadeiras,
e um ou dois pássaros sociáveis; embora eu nunca tivesse a perseverança
necessária para domesticá"-los. Mas isso, em parte, se devia ao fato de
que se eu conseguisse que um pássaro confiasse em mim, os gatos o
comeriam.

Nessas circunstâncias, todos os meus poderes de imaginação se dirigiam
às coisas inanimadas --- ao céu, às folhas, aos seixos, observáveis
dentro das muralhas do Éden, --- ou divagavam, à mínima oportunidade,
pelas regiões do romance, compatíveis com as realidades objetivas da
existência no século dezenove, a uma milha e um quarto de Camberwell
Green.

%42. 
Nesse momento, meu pai, de bom grado, embora sem outro objetivo além
de me agradar, quando acreditava que pudesse fazê"-lo sem infringir
nenhuma das regras estabelecidas por minha mãe, tornou"-se meu guia. Eu
gostava particularmente de vê"-lo se barbear; e sempre me era permitido
entrar em seu quarto pela manhã (sob aquele em que agora escrevo), para
ser a testemunha imóvel daquela operação. Sobre sua mesa de toalete
pendia uma de suas próprias aquarelas, realizada sob os auspícios do
primogênito dos Nasmyth; creio que na escola secundária de Edimburgo.
Fora pintada com os tons da época em que Dr.\,Monro era professor de Turner;
mesmo período em que meu pai frequentava aquela escola secundária;
nomeadamente, em subtons cinza do azul da Prússia e tinta inglesa,
posteriormente diluídas com cores quentes para efeitos luminosos.
Representava o castelo Conway, à margem do Frith, e, ao primeiro plano,
uma cabana,\footnote{\textit{Cottage}, no original.} um
pescador, e um barco à beira d'água.\footnote{A aquarela ainda está sob a
  cornija da lareira de meu quarto, em Brantwood. {[}\textsc{n.\,a.}{]}}

%43. 
Quando meu pai terminava de se barbear, sempre me contava uma
história sobre a pintura. O hábito desenvolveu"-se incidentalmente, em
resposta à minha inoportuna curiosidade sobre se o pescador vivia na
cabana, e aonde ele ia com o barco. Para ficar em paz, meu pai afirmou
que o pescador \textit{vivia} na cabana e que estava indo pescar perto do
castelo; posteriormente a trama do drama foi ficando mais complexa; e
confundiu"-se, creio, com aquela da tragédia de \textit{Douglas}, e aquela
do \textit{Castle Spectre}, peças nas quais meu pai atuara
amadoristicamente, diante de minha mãe, e de um público seleto, quando
tinha dezesseis anos, e ela, em seus severos vinte anos, era uma
dona de casa exemplar, com muito desdém e suspeitas religiosas em
relação ao teatro. Mas ela nunca receou em me contar, anos mais tarde,
quão bonito estava meu pai em seu traje das Terras Altas e o penacho de
penas negras.

%44. 
À tarde, quando meu pai retornava (sempre pontualmente) do trabalho,
jantava às quatro e meia, na sala da frente, minha mãe sentava a seu
lado para saber como fora seu dia, aconselhando"-o e encorajando"-o em
relação aos problemas; --- principalmente o encorajando, pois ele tendia
a ficar preocupado se os pedidos de \textit{sherry} estivessem abaixo do
esperado, até mesmo por um ou dois dias. Eu nunca presenciava esse
momento, contudo, e só posso assegurar o que narro por ouvir dizer e
provavelmente por conjectura; pois seria muito mau comportamento de
minha parte se, entre quatro e seis horas, aproximasse"-me da porta da
sala. Depois das seis, no verão, ficávamos todos no jardim enquanto
durasse o dia; chá sob a cerejeira ``coração"-branco''; ou no inverno ou
com tempo ruim, às seis horas na sala de visitas, --- bebia minha xícara
de leite e comia a fatia de pão com manteiga num pequeno recesso, com a
mesa diante dele, completamente sagrado para mim; e no qual eu
permanecia à noite como um Ídolo num nicho, enquanto minha mãe tricotava
e meu pai lia para ela, --- e para mim, na medida em que eu decidia
ouvir.

%45. 
As séries dos romances de Waverley, cumprindo seu objetivo, eram
ainda a principal fonte de deleite em todos os lares em que havia
interesse pela literatura; e não sou capaz de recordar o tempo em que
não os conhecia, como não posso recordar o tempo em que não conhecia a
Bíblia; mas tenho ainda viva lembrança da intensa expressão de
consternação mesclada ao desprezo de meu pai, quando lançou ao chão o
\textit{Count Robert of Paris},\footnote{\textit{Count Robert of Paris (Tales
  of my Landlord, Fourth Series):} romance de Sir Walter Scott,
  publicado em 1832.} após ler três ou quatro
páginas; e sabia que a vida de Scott estava terminada: o desprezo era
para ele um sentimento muito complexo e amargo, --- motivado, na verdade,
parcialmente pelo próprio livro, mas principalmente pelos patifes que
estavam vilipendiando e malbaratando o intelecto, e nem um pouco pela
sutil desonestidade que havia, essencialmente, causado sua ruína. Meu
pai nunca pôde perdoar a Scott por ter escamoteado sua associação com
Ballantyne.\footnote{Provavelmente James Ballantyne (1772--1833): editor
  inglês ao qual Walter Scott se associou secretamente.}

%46. 
Sendo esses os salutares prazeres de Herne Hill, tenho agora que
relatar, com profunda gratidão, o que devo à minha mãe por suas
decididas e consistentes lições sobre as Escrituras, de modo a tornar
cada palavra ali existente familiar aos meus ouvidos, acompanhada de sua
natural musicalidade, --- embora tal familiaridade fosse reverenciada, à
medida que transcendia todos meus pensamentos, e orientasse minha
conduta.

Ela o conseguiu, não através de palavras ou de autoridade pessoal, mas
simplesmente compelindo"-me a ler a obra por inteiro. Logo que me tornei
capaz de ler com fluência, começamos a trabalhar num curso sobre a
Bíblia, que nunca foi interrompido até que eu fosse para Oxford. Líamos
juntos versículos alternados, e ela observava, a princípio, cada
entonação de minha voz, corrigindo os erros, até que eu compreendesse o
versículo, se estivesse ao meu alcance, corretamente, vigorosamente. O
conjunto poderia estar além de meu entendimento, o que não a preocupava;
assegurava"-se, entretanto, que, quando eu compreendesse algo, o fizesse
corretamente; a apreensão do todo viria em tempo certo.

Assim, ela começou com o primeiro versículo do Gênesis, e avançou até o
último versículo do Apocalipse; nomes difíceis, números, leis levíticas,
e todo o resto; no dia seguinte, recomeçou com o Gênesis. Se um nome
fosse difícil, seria um bom exercício de pronúncia, --- se um capítulo
fosse cansativo, seria um bom exercício de paciência, --- se inspirasse
aversão, seria uma boa prova de fé, melhor do que se o capítulo fosse
atraente. Depois de alguns capítulos (de três a quatro por dia, a
depender da extensão de cada um, a primeira coisa após o desjejum,
proibida a interrupção pelos empregados, --- e pelos hóspedes, que, ou se
juntavam à leitura, ou tinham que ficar no andar superior, --- e pelos
visitantes ou excursionistas, exceto pelos verdadeiros viajantes), eu
tinha que aprender alguns versículos de cor, ou repeti"-los, para
assegurar que não perdera nada do que já fora estudado; e, à medida que
os capítulos chegavam ao fim, tinha que aprender belas paráfrases
escocesas, melodiosas e vigorosamente versificadas; às quais, junto com
a própria Bíblia, devo a educação de meus ouvidos aos sons.

É estranho que de todos os trechos da Bíblia que minha mãe me ensinou, o
que me custava mais a aprender, e que era, para minha mente infantil, o
mais repulsivo --- o Salmo 119 --- tenha agora se tornado o mais
precioso mim, em sua gloriosa e exuberante paixão de amor pela Lei do
Senhor, em oposição ao seu uso abusivo pelos modernos pregadores,
daquilo que eles imaginam ser o Evangelho do Senhor.

%47. 
Mas é apenas por esforço deliberado que relembro as horas matinais
de labor, regulares como o pôr do sol, --- labor igual para os dois ---
pelo qual, ano após ano, minha mãe obrigava"-me a aprender aquelas
paráfrases, e capítulos, (o oitavo do primeiro Livro dos Reis era um deles ---
experimente"-o, bom leitor, em seu tempo livre!) não admitindo nem uma
sílaba esquecida ou fora de lugar; cada sentença deveria ser repetida
até que ela estivesse satisfeita com a pronúncia. Lembro"-me de uma
discussão, que durou três semanas, a respeito da pronúncia de \textit{of} nas
seguintes linhas:

\begin{verse}
Deveria alguma próxima primavera\\
Reviver as cinzas da urna?\footnotemark
\end{verse}

\footnotetext{No original: \textit{Shall any following spring revive\,/\,The ashes of the urn?}}

Eu insistia, em parte por obstinação infantil, em parte devido a uma
verdadeira aptidão para o ritmo (não me interessavam as urnas nem seus
conteúdos), em recitar os versos com acento no \textit{of}. Confesso que
foi apenas após três semanas de trabalho que minha mãe conseguiu que o
acento fosse suavizado no \textit{of} e recaísse sobre as
cinzas,\footnote{\textit{Ashes}, no original.} segundo seu
desejo. Mas, mesmo que tivesse levado três anos, ela o teria conseguido,
desde que o tivesse encetado. E, seguramente, se ela não tivesse
conseguido, --- bem, não se sabe o que teria acontecido; mas estou muito
grato por ela o ter \textit{feito}.

%48. 
Acabo de abrir minha antiga Bíblia (ainda em uso), --- um volume
pequeno, compacto, e primorosamente impresso em Edimburgo por Sir D.
Hunter Blair e J. Bruce, Impressores de Sua Augusta Majestade Real, em
1816. Agora amarelo, com a idade; e acomodatício, mas não sujo, em muito
bom estado; exceto pelos ângulos inferiores das páginas do oitavo capítulo
do primeiro Livro dos Reis e o capítulo 32 do Deuteronômio, que estão gastas,
algo finas e escurecidas; o aprendizado desses dois capítulos custou"-me
muito sofrimento. A lista de minha mãe, com os capítulos, com os quais
ela pretendia estabelecer minha alma na vida,\footnote{Esta expressão, em
  \textit{Fors,} tem sido considerada por alguns leitores como
  significando que minha mãe, naquela época, fez"-me essencialmente e
  evangelicamente religioso. A verdade, entretanto, é bem outra. Quis
  dizer, apenas, que ela me deu \textit{base} segura para toda a vida
  futura, prática ou espiritual. Vejam os parágrafos seguintes. {[}\textsc{n.\,a.}{]}} caiu do livro. Apelo à indulgência do leitor indiferente, por
imprimir esta lista que me ocorreu acidentalmente: %---

\medskip

\begin{quote}
\begin{itemize}
\item \,\textsc{êxodo}\quad \textit{cap.\,15 e 20}
\item \,\textsc{2 samuel}\quad \textit{cap.\,1, do vers.\,17 até final}
\item \,\textsc{1 reis}\quad \textit{cap.\,8}
\item \,\textsc{salmos}\quad \textit{cap.\,23, 32, 90, 91, 103, 112, 119 e 139}
\item \,\textsc{provérbios}\quad \textit{cap.\,2, 3, 8 e 12}
\item \,\textsc{isaías}\quad \textit{cap.\,58}
\item \,\textsc{mateus}\quad \textit{cap.\,5, 6 e 7}
\item \,\textsc{atos}\quad \textit{cap.\,26}
\item \,\textsc{1 coríntios}\quad \textit{cap.\,13 e 15}
\item \,\textsc{james (\textit{sic})}\quad \textit{cap.\,4}
\item \,\textsc{revelação}\quad \textit{cap.\,5 e 6}
\end{itemize}
\end{quote}

\medskip

\noindent{}E, verdadeiramente, embora tenha adquirido um pouco mais de conhecimento
--- em matemática, meteorologia, e outras ciências afins, posteriormente
na vida, --- e não deva pouco ao ensinamento de muitas pessoas, esta
apropriação materna de minha mente com os capítulos mencionados,\footnote{Da
Bíblia.} reputo"-a, muito seguramente, como a mais valiosa e, no
cômputo geral, a parte \textit{essencial} de minha educação.

E talvez já seja tempo de assinalar as vantagens e desvantagens que as
vicissitudes da vida irrevogavelmente determinaram para mim, até os sete
anos.

Em primeiro lugar, relatarei minhas bênçãos (como um sábio amigo
recomendou"-me fazê"-lo, continuamente; considerando meu mau hábito de
sempre enumerar os espinhos em meus dedos e não os ossos que continham).

E para começar como a melhor e mais verdadeira de minhas bênçãos,
ensinaram"-me o perfeito significado da Paz, em pensamento, ato e
palavra.

Nunca ouvi meu pai ou minha mãe levantarem a voz um para o outro, em
qualquer questão; nem vi um olhar irado, ou mesmo ligeiramente ferido ou
ofendido, em seus olhos. Nunca ouvi um empregado ser repreendido; nem
mesmo irrefletidamente, impetuosamente, ou acusado de modo severo. Nunca
presenciei um momento de contrariedade ou de tumulto em qualquer questão
doméstica; nem nada feito às pressas ou não realizado no tempo
apropriado. Não tinha ideia do sentimento de ansiedade; o agastamento de
meu pai nas tardes em que recebera pedidos para apenas doze pipas de
vinho quando esperava receber para quinze, como já mencionei, nunca se
dirigia a \textit{mim}; referia"-se apenas à posição que seu nome ocuparia
na relação anual dos exportadores de \textit{sherry}; como nunca gastou mais do
que a metade de sua renda, era pouco afetado pelas variações ocasionais
de seu total. Nunca fiz nada errado de que tivesse consciência --- a não
ser a eventual demora em assimilar alguma frase edificante, porque
estava olhando uma vespa na vidraça da janela, ou um pássaro na
cerejeira; e nunca percebi nenhuma mágoa.

%49. 
Depois desta quase inestimável dádiva da Paz, recebi o perfeito
entendimento das naturezas da Obediência e da Fé. Obedecia a palavras,
ou a dedo em riste, do pai e da mãe, como um navio ao seu leme; não sem
tentar opor resistência, mas aceitando a orientação como parte de minha
própria vida e de minha virtude, como lei útil, tão necessária a mim em
qualquer ação moral como a lei da gravidade ao salto. E minha prática na
Fé logo estava completa: nunca nada me foi prometido sem ter sido
concedido; nenhuma ameaça deixou de ser infligida, e nunca nada me foi
dito que não fosse verdadeiro.

Paz, obediência, fé; essas três as virtudes essenciais; a seguir, o
hábito de fixar atenção com os dois olhos e a mente --- sobre o qual não
me estenderei no momento, e que constitui a principal faculdade prática
de minha vida, a cujo respeito Mazzini\footnote{Provavelmente Giuseppe
  Mazzini (1805--1872), político e revolucionário italiano,
  considerado o ``profeta do \textit{risorgimento''}, movimento pela
  unificação da Itália, em meados do século \textsc{xix}.} declarou,
um ano ou dois antes de sua morte, numa conversa que reproduzo
fidedignamente, ter eu ``a mente mais analítica da Europa''. Opinião
com a qual, pelo que conheço da Europa, estou inteiramente disposto a
concordar.

Por último, uma extrema perfeição de paladar e todos os outros sentidos
do corpo, obtida pela estrita proibição de bolo, vinho, doces e frutas;
as últimas permitidas apenas com grande parcimônia; e pela esmerada
preparação da comida que me era oferecida. Essas, estimo as principais
dádivas de minha infância; --- agora, deixem"-me contar os principais
infortúnios.

%50. 
Em primeiro lugar, eu nada tinha para amar.

Meus pais eram, para mim --- uma espécie de --- visíveis poderes da
natureza, não mais amados do que o sol e a lua: a diferença é que me
sentiria magoado e confuso se qualquer um deles desaparecesse; (quanto
sinto, agora que os dois se foram!) --- amava a Deus um pouco menos; não
que tivesse alguma querela com Ele, ou medo Dele; mas simplesmente achei
Seu ofício, a partir daquilo que me disseram dele, desagradável; e Seu
livro, a partir do que me disseram dele, desinteressante. Também não
tinha companhias com as quais brigar; ninguém a quem ajudar, e ninguém a
quem agradecer. Jamais um empregado foi designado para fazer qualquer
coisa para mim, além do que era seu dever; e por que deveria ser grato à
cozinheira por cozinhar, ou ao jardineiro por cuidar do jardim, ---
quando um não ousava dar"-me uma batata cozida sem pedir permissão, e o
outro não deixaria meus formigueiros em paz, porque fariam as aleias
parecerem negligenciadas? A consequência negativa disso tudo não foi,
contudo, aquela a que talvez se esperasse, que eu crescesse egoísta e
desafável; mas que, quando a afeição veio, veio com violência totalmente
pletórica e incontrolável, pelo menos para mim, que nunca tivera algo
para controlar.

%51. 
Pois nada tive que suportar (o segundo de meus maiores infortúnios).
Não conhecia perigo ou dor de qualquer tipo: minha força nunca era
exercitada, minha paciência nunca era testada, minha coragem nunca era
fortalecida. Não que tivesse medo de alguma coisa, --- mesmo fantasmas,
trovões, ou feras; e uma de minhas maiores aproximações à
insubordinação, à qual sempre estive tentado como criança, foi o
arrebatado esforço para obter permissão de brincar com os filhotes de
leão na \textit{menagerie}\footnote{Palavra inglesa, de origem francesa
  (\textit{ménagerie}), sem correspondência em português: lugar onde os
  animais são guardados e treinados especialmente para exibição; coleção
  de animais selvagens ou exóticos destinados à exibição.}
de Wombwell.

%52. 
Em terceiro lugar. Não me ensinaram delicadeza ou etiqueta de modos;
era suficiente, na pequena sociedade que frequentávamos, que não fosse
inoportuno e respondesse a uma pergunta sem timidez: mas a timidez veio
depois e aumentou à medida que me tornei consciente de minha rudeza, que
se revelou a partir da necessidade de disciplina social; considerei
impossível de adquirir, na maturidade, destreza nos exercícios
corporais, talento para atividades prazerosas, naturalidade e tato no
comportamento cotidiano.

%53. 
O último e o mais grave de meus infortúnios. Meu julgamento do que
era certo ou errado e a capacidade de ação\footnote{Observe que me
  refiro aqui à \textit{ação}, considerando"-se que eu fosse demasiadamente
  independente, como disse antes. {[}\textsc{n.\,a.}{]}} independente foram
deixados completamente subdesenvolvidos, porque as rédeas e os antolhos
nunca me foram retirados. As crianças deveriam ter um período de folga,
como os soldados; e uma vez que a obediência, se requerida, estivesse
assegurada, à pequena criatura deveriam, muito precocemente, ser
permitidos períodos de completa independência; durante os quais,
colocada sobre o dorso nu de sua própria vontade, teria que a controlar
com sua própria força. Mas a constante autoridade exercida sobre minha
infância fez com que eu, quando, afinal, lançado ao mundo, ficasse,
durante algum tempo, à deriva em seus vórtices.

%54. 
Meu veredicto atual sobre o teor geral de minha educação àquela
época, deve ser, portanto, que era, ao mesmo tempo, demasiadamente
formal e demasiadamente permissiva; mantendo meu caráter, no mais
importante momento de sua construção, sob restrição, mas sem
discipliná"-lo; protegendo minha inocência em vez de exercitar"-me na
prática da virtude. Minha própria mãe o percebeu com clareza,
posteriormente; e quando eu fazia algo errado, estúpido ou cruel, --- (e
fiz muitas coisas que eram os três) --- sempre dizia: ``Isso é porque
você foi muito mimado.''

%55. 
Até este ponto, com algumas omissões, fiz apenas reimprimir a
narrativa apresentada em \textit{Fors:} e receio que a continuação possa
ser mais trivial, porque muito do que é importante está concentrado nas
longas sentenças anteriores, que devem agora ser continuadas com passos
mais lentos; e também menos divertidos, porque, em \textit{Fors,} sempre
tentei dizer as coisas, na medida do possível, de maneira mordaz; o
restante deste livro será narrado o mais singelamente possível. Mas,
tendo ou não obtido sucesso na mordacidade, em \textit{Fors}, certamente
fui frequentemente obscuro; e a descrição anterior de Herne Hill
parece"-me necessitar, imediatamente, de redução a termos mais simples.

%56. 
A altura real da longa crista de Herne Hill, sobre o Tâmisa, --- pelo
menos acima de sua base, que fica quase ao nível do Tâmisa, em
Camberwell Green, é de, creio, não mais que cento e cinquenta
pés:\footnote{Um \textit{foot}, medida inglesa de comprimento, corresponde
  a 30,48 centímetros.} mas a crista aplana"-se rapidamente, em
ambos os lados, por cerca de um quarto de milha;\footnote{Uma
  \textit{mile}, medida inglesa de agrimensura, corresponde a 1.609,34 metros.} formando, a leste e a oeste, uma sucessão de belos
jardins e terrenos propícios ao lazer, imediatamente secos após a chuva,
e nos quais, para as crianças, rolar para baixo é uma brincadeira
prazerosa, e rolar para cima um exercício vigoroso. A vista da crista,
para ambos os lados, era, antes de chegarem as ferrovias, absolutamente
encantadora: para o oeste, à noite, quase sublime sobre extensões
suavemente entrelaçadas de florestas familiares; --- o Tâmisa,
propriamente, não era visível, ou qualquer outro campo, exceto os
imediatamente abaixo; apenas os cimos de vinte e quatro milhas quadradas
de bosques desabitados. Do outro lado, a leste e a sul, as colinas de
Norwood, parcialmente cobertas por rústicos tojos, parcialmente
florestadas com vidoeiros e carvalhos, parcialmente em matagal espinhoso
de verde puro e campos de altitude que traziam em si a promessa de todo
o encanto rural de Surrey e Kent, e, com tanto espaço e altura em sua
extensão, as colinas guardavam alguma semelhança com verdadeiras
montanhas. Semelhança agora inconcebível, pois o Palácio de
Cristal,\footnote{Construção em ferro e vidro erguida para a Exposição
  Universal de Londres, em 1851; transportado para a periferia da
  capital inglesa, foi destruído por um incêndio, em 1936.} mesmo sem atingir um tamanho considerável, e não tendo mais
sublimidade que um pepino entre duas chaminés, e com sua estupidez de
volume oco, diminui imediatamente as colinas; de modo que agora se as
considera apenas como três longos inchaços informes de argila, à espera
de construção. Mas então, o Nor"-wood, ou bosque do norte,\footnote{``North"-wood'',
  no original.} assim chamado, visto a partir de
Croydon, em oposição ao bosque do sul,\footnote{``South wood'', no
  original.} das colinas de Surrey, desenhava"-se em
forma de crescente elevada umas cinco milhas em torno de Dulwich, na
direção sul, interrompida por caminhos ascendentes que levavam a Gipsy
Hill e a outras colinas; e, do topo, vistas que dominavam Dartford e a
planície de Croydon, --- na contemplação das quais um dia levei minha mãe
ao desespero ao dizer que ``meus olhos estavam saindo de minha cabeça!''.
Ela pensou que eu fosse vítima de um ataque de insolação.

%57. 
No centro desse anfiteatro, a coroação da glória de Herne Hill \label{herne}
advinha do fato de que, após caminharmos, a partir de Londres, ao longo
da crista em direção sul, por uma milha de castanheiros, lilases e
macieiras, pendendo sobre paliçadas de madeira de cada lado ---
subitamente as árvores acabavam à esquerda e se atingia o topo de um
campo que descendia em direção sul até o vale de Dulwich --- campo
aberto, animado com vacas e botões"-de"-ouro; abaixo, os belos prados e as
altas avenidas de Dulwich; e mais além, toda a crescente das colinas de
Norwood; uma trilha, a qual se tinha acesso através de um torniquete,
que seguia para a esquerda, sempre tão aquecida que doentes poderiam ali
ser abrigados em março, quando caminhar por qualquer outro lugar os
teria matado; e tão tranquila, que quando tinha dificuldade para
escrever ou pensar, costumava percorrê"-la mais frequentemente do que
nosso jardim. O grande campo estava separado da trilha e da estrada
apenas por leve paliçada aberta de madeira, de quatro pés de altura,
necessária para manter as vacas em seu interior. Desde a última vez que
escrevi ou ali meditei, vários melhoramentos foram realizados; primeiro,
a vizinhança queria uma nova igreja e construiu um pobre templo gótico
com um inútil pináculo, colocado apenas por modismo, ao lado do campo;
então ergueram um presbitério atrás da igreja; as duas construções
esconderam metade da vista naquela direção. Depois veio o Palácio de
Cristal, estragando definitivamente a vista, por qualquer ângulo, e
trazendo de Londres, a cada dia de exposição, uma multidão de pedestres
através da trilha, que a deixavam suja com cinzas de charuto pelo resto
da semana: a seguir, vieram as ferrovias, e grosseirões falastrões em
cada trem de excursão, que derrubavam a paliçada, assustavam as vacas e
destruíam todos os ramos de flores que conseguiram alcançar por sobre a
cerca. Então os que residiam dentro da área fechada construíram um muro
de tijolos como defesa. E a trilha tornou"-se insuportavelmente quente e
suja e foi gradualmente abandonada aos grosseirões, sob a guarda de um
policial. Finalmente, este ano, uma paliçada cerrada de seis pés de
altura fechou o outro lado do campo e as procissões de excursionistas
ganharam a liberdade de obter a ideia do ar e do panorama campestre que
quisessem, perambulando entre o muro e a paliçada, com um charuto
malcheiroso adiante, outro às costas, e outro à boca.

%58. 
Não pretendo que este livro seja, na medida em que consiga evitá"-lo,
desagradável ou lamuriante; mas que expresse, de uma maneira geral, meu
caráter --- o qual, posso dizê"-lo, é extremamente amigável, quando não
estou aborrecido: resmungarei onde achar que devo, e registro o prejuízo
causado tanto aos residentes como aos excursionistas em Herne Hill 
apenas porque as questões relativas ao direito de passagem são agora
muito comuns; e, na maioria dos casos, a simples \textit{passagem}
representa a menor parte do Direito antigo, propriamente entendido. O
Direito refere"-se à bela vista e ao bom ar ao longo da passagem.

Devo também mencionar, rapidamente, que a despeito de todo o falatório
sobre a Magna Carta, muito poucos ingleses sabem que uma de suas
principais disposições é que a Lei não pode ser vendida;\footnote{``A
  ninguém Nós venderemos, a ninguém Nós negaremos ou diferiremos,
  Direito, ou Justiça''. {[}\textsc{n.\,a.}{]}} e parece"-me que a lei da
Inglaterra deva manter Banstead e outras colinas livres para o pobre da
Inglaterra, sem a necessidade de cobrar"-me, como foi feito recentemente,
uma centena de libras pelo usufruto temporário daquilo que deveria ser
gratuito.

%59. 
Devo retornar à época de meus primeiros anos, a fim de suprir
lacunas, após este pequeno avanço no livro; mas ainda arriscarei ser
enfadonho ao explicar que, ao escrever que ``no jardim Herne Hill todos
os frutos eram proibidos'', queria dizer, naturalmente, proibidos sob
restrições bem definidas; o que fazia a apanha de cada fruto, na estação
apropriada, uma espécie de festival da colheita; e que de sua aparente
severidade advinha um prazer futuro; que, quando finalmente permitido,
fazia com que o pêssego que minha mãe colhia para mim ao ter certeza que
estava maduro, e a torta de cereja para a qual eu escolhera redondas
cerejas vermelhas, tivessem para mim, creio, um sabor mais etéreo do que
teriam para uma criança a qual fosse permitida colher e comer frutas à
vontade; além disso, o genuíno e duradouro prazer que me era oferecido
pelas avenidas de árvores estava nas flores e não nos frutos. Para o
deleite epicurista da existência, batatas douradas, ervilhas verdes bem
cozidas, --- grandes favas de sabor intenso, --- e potes de ameixas roxas
e de groselhas, para cujo preenchimento, a cada ano, dependíamos mais do
verdureiro do que do jardim, eram centenas de vezes mais importantes
para mim do que as dúzias de nectarinas das quais poderia talvez obter
três metades, --- (as outras metades estavam mofadas) --- ou um
alqueire\footnote{Antiga medida de cereais, equivalente a 36,37 litros.} ou dois de peras que iam diretamente para as
prateleiras da despensa. Assim, muito cedo, mesmo em meus pensamentos
sobre árvores, descobri o princípio, repetido cinquenta anos depois em
\textit{Proserpina},\footnote{Título de uma obra de Ruskin sobre botânica.} que as sementes e os frutos existem por causa das
flores, e não as flores por causa dos frutos. A primeira alegria do ano
era a chegada da neve, a segunda, e fundamental, eram as amendoeiras em
flor, --- todos os outros jardins e bosques o seguiam alegremente numa
sequencia contínua de belas flores e folhas sombrosas; e, por muitos e
muitos anos, --- até que a vida se tornasse outono para mim, --- minha
maior prece à bondade celestial, nas estações floridas, era que a geada
poupasse as amendoeiras em flor.

\chapter{Às margens do Tay} %Capítulo \textsc{iii}. 

%60. 
O leitor deve ter observado, espero, que em tudo até agora dito,
foram enfatizadas apenas as condições favoráveis que envolviam a criança
cuja história escrevo, e a dócil e impressionável quietude de seu
temperamento.

Não se lhe reivindicou nenhuma aptidão ou habilidade especial; pois,
verdadeiramente, não existiam, exceto, talvez, a paciência para observar,
a precisão em sentir, as quais, junto com a necessária diligência,
formavam minha capacidade analítica.

Em todos os atributos essenciais do espírito, exceto nos citados, eu era
deficiente; apenas minha memória era de moderada capacidade.
Sinceramente, nunca conheci uma criança tão incapaz de desempenhar um
papel ou contar uma história. Por outro lado, nunca conheci outra cujo
interesse pelos fatos concretos fosse ao mesmo tempo tão intenso e tão
metódico.

%61. 
Percebi também, nos trechos anteriores, os quais pretendi que fossem
modestos, que a alta literatura foi citada com excessiva jactância como
minha primeira e exclusiva referência. Minha pequena \textit{Ilíada} de
Pope, e, qualquer que tenha sido meu entendimento deles, meu Gênesis e
meu Êxodo, foram certamente de pouca importância para mim antes dos dez
anos. O que me comprazia era a \textit{Dame Wiggins of Lee} e o
\textit{Peacock at Home}, e outras canções infantis, mais leves; e o
\textit{Frank} de Miss Edgeworth\footnote{Certamente Maria Edgeworth
  (1767--1849), novelista anglo"-irlandesa.} e o
\textit{Harry and Lucy}, junto com os \textit{Scientific} \textit{Dialogues}
de Joyce, mais sérios. Os primeiros esforços de que consigo lembrar,
indicando incipiente movimento de moléculas cerebrais, são seis
\textit{poemas} sobre temas extraídos dos livros mencionados; entre o quarto
e o quinto poema minha mãe escreveu: ``Janeiro, 1826. Este livro foi
começado em setembro ou outubro, 1826, acabado em janeiro, 1827.'' Foi
escrito numa imitação dos tipos de um livro impresso, no meu sétimo
aniversário. Era vermelho, com pautas azuis, seis polegadas de altura
por quatro de largura, contendo quarenta e cinco folhas escritas a
lápis, em ambos os lados, numa imitação de impressão, --- a folha de
rosto, escrita na forma aqui aproximadamente reproduzida:

\begin{center}
\textsc{harry e lucy}

\textsc{fim}

\textsc{a última parte das}

\bigskip

\textsc{primeiras lições}

\bigskip

em quatro volumes

Vol \textsc{i}

com pranchas

de cobre

\medskip

\textsc{impresso} e composto por um garotinho

e também ilustrado

%(Inserir aqui desenho da montanha)
\end{center}

%62. 
Dos quatro volumes previstos, parece que (segundo hábito que até
hoje mantenho) escrevi um e um quarto, o primeiro volume contendo apenas
quarenta folhas, o resto do livro ocupado pelos mencionados seis
\textit{poemas}, e das quarenta folhas, dez páginas foram reservadas para
\textit{pranchas de cobre}, uma das quais, propondo"-se a representar ``o novo
caminho de Harry'', é, segundo acredito, minha primeira tentativa de
desenhar uma montanha. O trecho que encerra o primeiro volume deste
trabalho, creio que seja, por vários motivos, digno de preservação. Eu o
imprimi, portanto, mantendo as divisões de linha, e as três variações de
tamanho das imitações dos tipos gráficos. A pontuação será deixada à
indulgente apreciação do leitor. Os hifens, deve ser observado, eram
longos ou curtos, para tornar a impressão uniforme, o que nem sempre foi
conseguido, mas a variação de espaço entre as linhas foi imitada à
perfeição.
\medskip
\begin{quote}
Harry sabia muito bem o que isso significava e continuou com seu desenho mas Lucy logo o chamou e mandou que observasse uma grande nuvem negra ao norte que parecia carregada de eletricidade. Harry correu para apanhar um aparelho elétrico que seu pai lhe dera e que a nuvem carregaria positivamente depois que outra o carregasse negativamente e então viria um longo trem de pequenas nuvens mas antes delas vinha um grande nuvem de poeira rosa elevada do solo que seguia a nuvem carregada positivamente e acabaria por entrar em contato com ela e quando a outra nuvem chegou um clarão de luz projetou"-se através da nuvem de poeira que a nuvem carregada negativamente envolvia e dissolveu"-se em chuva que imediatamente limpou o céu.
Depois que o fenômeno acabou e também a surpresa Harry começou a
imaginar como a eletricidade podia estar onde havia tanta água mas logo
observou um arco"-íris e uma névoa ascendendo abaixo dele que sua
fantasia transformou numa forma feminina. Lembrou então da bruxa das
águas dos Alpes que se elevava tomando água em suas mãos e a lançando ao
ar enquanto pronunciava palavras ininteligíveis. E embora fosse uma
lenda isso o afetou ao ver nas nuvens algo que se parecia com a bruxa.
\smallskip
\begin{flushright}
\textsc{fim de harry e lucy}
\end{flushright}
\end{quote}

%63. 
As várias razões apresentadas, que me induzem a reimprimir, de modo
demasiadamente literal, este trecho da \textit{composição}, são --- primeiro,
uma razoável amostra de minha ortografia aos sete anos; apenas razoável,
não \textit{exata}, porque era muito incomum cometer um erro, embora aqui
existam dois (\textit{takeing} e
\textit{unintelligable}),\footnote{O correto seria \textit{taking} e
  \textit{unintelligible}.} aos quais só posso atribuir
minha grande pressa em terminar o primeiro volume; --- segundo, que a
adaptação, à minha história, de materiais provenientes dos
\textit{Scientific Dialogues} de Joyce\footnote{A passagem original é a seguinte, volume \textsc{vi}, edição de 1812, p.\,138: 

`` Dr.\,Franklin menciona uma extraordinária aparição que ocorreu a Mr.\,Wilke, um respeitável eletricista. Em 20 de julho de 1758, às três da
tarde, ele observou uma grande quantidade de poeira erguendo"-se do
chão, e cobrindo o campo e parte da cidade na qual se encontrava. Não
havia vento, e a poeira movia"-se suavemente em direção ao leste, onde
surgiu uma grande nuvem negra que carregou positivamente seus
aparelhos. A nuvem moveu"-se para o oeste, a poeira a seguiu, e
continuou a se elevar cada vez mais alto, até que formasse uma pilar
espesso, na forma de um pão"-de"-açúcar, cuja extremidade parecia manter
contato com a nuvem. À alguma distância, surgiu outra grande nuvem,
acompanhada de uma longa cadeira de nuvens menores, as quais
carregaram negativamente seus aparelhos; e quando elas se aproximaram
da nuvem positiva, um clarão de luz foi visto projetando"-se através da
nuvem de poeira, sobre a qual as nuvens negativas se dispersaram e se
dissolveram em chuva, que imediatamente limpou a atmosfera." {[}\textsc{n.\,a.}{]}} e do \textit{Manfred},\footnote{Poema dramático de Lord Byron,
contendo elementos sobrenaturais, escrito entre 1816 e 1817.} é um exemplo perfeito do caráter amalgamado de minha mente,
desde os primeiros aos últimos dias de minha vida --- o que sempre
desacreditou as objeções dos leitores científicos em relação a meus
livros porque eles estavam repletos de amor à beleza, e as objeções dos
leitores estetas porque eles estavam repletos de amor à ciência; ---
terceiro, que o razoabilíssimo método de julgamento final, sobre o qual
fundamento minha reivindicação, ao leitor sensível, de respeito por
esses escritos bipartidos, não poderia ser mais bem ilustrado pelo fato
de que, mesmo aos sete anos, nenhuma fábula, não importa quão sedutora
fosse, poderia \textit{afetar} Harry, até que ele tivesse visto --- nas
nuvens, ou em qualquer outro lugar --- ``algo que se parecesse com ela''.

Dos seis poemas que se seguem, o primeiro é sobre a máquina a vapor,
começando assim:

\begin{verse}
Quando furiosamente das minas a água verte,\\
E limpa os minérios de toda rústica impureza;\footnotemark
\end{verse}
\footnotetext{No original: \textit{When
  furious up from mines the water pours,\,/\,And clears from rusty
  moisture all the ores}.}

E o último, sobre o arco"-íris, em \textit{versos brancos}, de caráter
didático, com observações sobre as disposições ignorantes e irrefletidas
de certas pessoas.

\begin{verse}
Mas aqueles que não conhecem esta luz,\\
Não refletem a seu respeito; e de toda ela,\\
Nenhuma das cores que a compõem conhecem.\footnotemark
\end{verse}
\footnotetext{No original: \textit{But those
  that do not know about the light,\,/\,Reflect not on it; and in all that
  light,\,/\,Not one of all the colours do they know.}}

%64. 
Foi apenas, creio, após meu sétimo ano ter sido consagrado a essas
meditações, que minha mãe acrescentou lições de latim à leitura da
Bíblia, e estabeleceu a rigorosa rotina diária esboçada no capítulo
anterior. Mas muito me surpreendeu que, ao tentar, se não para o deleite
do leitor, para meu próprio, descrever com exatidão essa rotina, não
fosse capaz de recordar o que costumava acontecer no início da manhã,
exceto pelo desjejum no quarto e, caso minha prima de Croydon Bridget
estivesse conosco, a discussão sobre quem deveria comer os pedaços de
torrada mais tostados. O que deve ter acontecido mais tarde, entretanto,
pois ainda não me era permitido comer torradas na época a qual me
refiro. Nada é muito claro para mim do curso do dia, até que, após meu
pai ter ido à City\footnote{Refere"-se à City of London, tradicional
  centro financeiro e de negócios do centro de Londres, normalmente
  designado apenas como \textit{City}.} de carruagem e as
ordens domésticas de minha mãe terem sido rapidamente emitidas, as
lições começassem às nove e meia, com as leituras da Bíblia já
descritas, e os dois ou três versículos a serem memorizados, além de um
versículo de paráfrase; --- então uma declinação latina ou um pouco de
verbo, e oito palavras do vocabulário da \textit{Gramática Latina}
de Adam (a melhor já publicada); o resto do dia era livre. Aritmética
foi saudavelmente postergada até muito mais tarde; geografia, aprendia"-a
sozinho e rapidamente; história, nunca foi ensinada além daquilo que
decidi ler dos \textit{Contos de um avô}, de Scott. Portanto, como
já foi dito, ao meio"-dia eu estava no jardim, em dias de tempo bom, ou
livre para meus próprios divertimentos, em dias de chuva; nos quais,
devo imediatamente acrescentar, logo que pude engatinhar, meus tijolos
de madeira foram companhia constante; e seria deselegância minha não
mencionar que foi graças a uma amiga excêntrica (suspeito fortemente de
minha tia de Croydon) que fui presenteado com uma ponte de dois arcos,
com a aduela e a chave da abóbada admiravelmente encaixadas, e os planos
de alvenaria com bordas bisotadas, articulados em rabo de andorinha, no
estilo da ponte de Waterloo. Arcos bem feitos e degraus embutidos até a
água tornavam o modelo largamente instrutivo: e nunca me cansava de
construir, \textit{des}construir, --- (era muito pesado para ser demolido,
tinha sempre que ser \textit{desmontado}) --- e reconstruir. A inconcebível
satisfação passiva --- ou melhor, impassível --- de fazer, ou ler, a mesma
coisa repetidamente, creio que tenha sido uma das grandes responsáveis
pelo meu futuro poder de ir até o fundo das coisas.

%65. 
Algumas pessoas poderiam dizer que esses brinquedos despertaram"-me
um amor precoce pela arquitetura; mas nunca vi ou ouvi falar de outra
criança tão aficionada por seus tijolos de brinquedo, exceto ``o Frank
de Miss Edgeworth''. Com certeza, na época atual, --- apesar da
ubiquidade das olarias, --- as pessoas não dão aos seus filhos tijolos de
brinquedo, mas pequenas marias"-fumaça; e os pequenos estão sempre
tirando tíquetes e chegando a estações, sem mesmo compreender --- nenhum
deles se dará ao trabalho de \textit{fazê"-lo}, --- o princípio da
maria"-fumaça! E que bom não seria para eles se pudessem também
compreender que o princípio da maria"-fumaça não pode superar o princípio
da Respiração?

Não apenas dominei, com Harry e Lucy, o inteiro princípio motor da
maria"-fumaça; mas também compreendi profundamente, com a ajuda de meus
bem"-cortados tijolos, as leis práticas da estabilidade em torres e
arcos, quando tinha sete ou oito anos; e esses estudos de estruturas
foram posteriormente impulsionados pelo meu constante hábito de
observar, com a mais cuidadosa atenção, os procedimentos de qualquer
pedreiro, talhador de pedra ou pavimentador, --- cujos trabalhos minha
babá me permitia parar para observar, em nossas caminhadas; ou, deleite
dos deleites, quando eles podiam ser vistos facilmente de alguma
favorável janela de hospedaria ou pousada, durante nossas viagens.
Nesses casos o dia não era suficiente longo para minha arrebatadora e
obsedante observação.

%66. 
Frequentemente, como já foi dito, se o clima o permitisse, passava
meu tempo no jardim, sobretudo observando atentamente as plantas. Não
tinha o menor interesse em plantá"-las, ou cultivá"-las, como não tinha em
cuidar de pássaros, ou de árvores, ou do céu, ou do mar. Passava todo
meu tempo a observá"-las de perto. Sem mórbida curiosidade, mas em
maravilhada admiração, reduzia as flores a pedaços até que conhecesse
delas tudo o que poderia ser visto com os olhos de uma criança;
costumava guardar pequenos tesouros de sementes, como se fossem pérolas
ou contas, --- jamais com a intenção de semeá"-las. O velho jardineiro
vinha apenas uma vez por semana, para varrer e capinar o que fosse
necessário; agradava"-me aprender a varrer as aleias; mas era
desencorajado e ficava envergonhado por ele sempre refazer as partes já
varridas por mim. Gostava muito de cavar buracos, mas esta forma de
jardinagem não era permitida. Forçosamente, retornava sempre ao meu
estado mental meramente contemplativo, e, aos nove anos, comecei um
poema, chamado \textit{Eudosia}, --- esqueci completamente onde obtive este nome,
ou o que significava para mim, --- \textit{no universo}, embora nada
possa entender dele, agora. Um ou dois dísticos, do verdadeiro início de
\textit{Deucalion}\footnote{Personagem da mitologia grega, filho de
  Prometeu e Pronoia.} e \textit{Proserpina},\footnote{Deus
  da agricultura, na mitologia romana, filha de Júpiter e Ceres.
  Corresponde a Perséfone, na mitologia grega, em que é filha de Zeus e
  Deméter.} talvez possam conseguir, juntos com os
versos precedentes, um lugar nesta grave autobiografia; tanto quanto eu
seja capaz de fornecer uma data precisa --- 28 de setembro de 1828 ---
para o início de seu \textit{primeiro livro}, como se segue: %---

\begin{verse}
Quando a cólera divina se abateu sobre o mundo,\\
E sobre as rochas, as colinas, as montanhas arrojou\\
As massas concentradas de água; e o mar sobre a praia, ---\\
Então montanhas desmoronaram, e vales, antes desconhecidos,\\
Permaneceram onde estavam. Muito diferente era a Terra,\\
Quando o primeiro dilúvio veio, da segunda Criação.\\
Agora veja o que produz! --- Rainha das flores, Ó rosa,\\
De cujas pétalas de belas cores tais odores exala,\\
Deves agora ante teus súditos ser citada,\\
Tanto por tua beleza quanto por tua delicadeza afamada.\\
Tua és a flor da Inglaterra, e a flor\\
Da Beleza --- da morada perfumada de Vênus.\\
E tu frequentemente espargirás doces perfumes a tua volta,\\
E frequentemente te inclinarás, esconder"-te"-ás no chão.\footnotemark\\
E então o lírio, erguendo"-se tão altaneiro,\\
Sobrelevará tua alegre corola por sobre as outras flores,\\
Lá onde as manchas escuras sobre o solo escarlate\\
E as folhas fusiformes podem ser encontradas.
\end{verse}
\footnotetext{Um modo desastrado --- principalmente pela rima --- de dizer que rosas frequentemente são pesadas demais para suas hastes. {[}\textsc{n.\,a.}{]}}

%67. 
Em 220 linhas, de tal qualidade, o primeiro livro alça"-se da rosa ao
carvalho. O segundo começa --- para minha surpresa, numa extremamente
excepcional violação do que me vangloriei acima --- com uma enlevada
apóstrofe ao que eu jamais vira!

\begin{verse}
Canto o pinheiro, que cobre os altos picos da Suíça,\footnotemark\\
E em seu alto trono, cresce num leito de rochas,\\
Em abismos tão profundos, em escarpas tão altas,\\
Que quem se aventurar a escalá"-los arriscar"-se"-á a morrer.
\end{verse}
\footnotetext{\textit{Switzer}, no original: evidente abreviação de Suíça,
  \textit{Switzerland}. {[}\textsc{n.\,a.}{]}}

\noindent{}Este entusiasmo, contudo, durou apenas --- quase se exaurindo na
descrição, emprestada de \textit{Harry e Lucy}, da descida a pique do
Alpnach, --- 76 linhas, quando os versos terminam, e o livro, virado de
cabeça para baixo, começa em sua outra extremidade, com a informação de
que ``o cristal de rocha é acompanhado por Actinolita, Axinita, e
Epídoto, em Bourg d'Oisans, no Delfinado.'' Mas essas meditações no
jardim nunca cessaram, e é impossível dizer quanta força retirei delas,
ou quanto tempo foi inutilmente gasto, exceto nos prazeres, nessas horas
de quietude e nesses versos pueris. A felicidade proveniente delas
tornava aborrecidas as obrigações da vida prática, e esses ociosos
devaneios, o leitor pode pensar, bem poderiam ter sido transformados no
início de um sólido conhecimento botânico, se minha mãe assim o
desejasse. Mas, enquanto existiam livros sobre geologia e sobre
mineralogia, os quais eu poderia entender, todos os sobre botânica eram
então, --- e são pouco melhores agora --- mais difíceis do que a gramática
latina. Mineralogia bastava"-me para um trabalho sério, e me inclino
finalmente a afirmar que o tempo gasto no jardim não poderia ser mais
bem empregado, exceto em capinação.

%68. 
Pontualmente, às seis horas, juntava"-me a meu pai e a minha mãe para
o chá, na sala de desenho, restrito ao nicho sagrado já mencionado, um
recesso ao lado da lareira, bem iluminado pela janela lateral, nas
tardes de verão, e pelo abajur sobre a cornija, no inverno, protegido do
calor inconveniente e das perigosas correntes de ar. Colocada à frente,
uma escrivaninha me encerrava, e recebia meu prato e minha xícara, ou os
livros que estava lendo. Após o chá, meu pai lia para minha mãe o que
lhes comprazia, eu tentava captar o que conseguisse, ou lia meus
próprios livros. Assim ouvi todas as comédias e peças históricas de
Shakespeare muitas vezes, --- toda a obra de Scott, todo o \textit{Don
Quixote}, um dos livros favoritos de meu pai, cuja leitura me levava a
um êxtase de gargalhadas: agora, é um dos livros mais tristes, e, sob
certos aspectos, um dos que me é mais nocivo.

Meu pai era um fiel leitor da \textit{melhor} poesia e da \textit{melhor}
prosa; --- de Shakespeare, Pope, Spenser,\footnote{Edmund Spenser (1552/\,53--1599) foi um poeta renascentista inglês, cuja obra mais conhecida é o
  longo poema alegórico, \textit{The Fairie Queene}, em que justifica o
  protestantismo e o puritanismo, e glorifica a Inglaterra e a rainha
  Elizabeth \textsc{i}.} Byron\footnote{Lord George Gordon Byron
  (1788--1824) foi um poeta romântico inglês, famoso em seu tempo pela
  poesia pessoal e satírica e pela conturbada vida amorosa. Sua lendária
  morte em Missolonghi, durante a campanha pela independência grega dos
  turcos, foi profundamente pranteada e o tornou herói nacional da
  Grécia.} e Scott; como também de Goldsmith, Addison e
Johnson. Quanto à poesia mais leve das baladas, faltava"-lhe refinamento
auditivo para apreciá"-las: a percepção do poder, da profundidade do
verdadeiro significado e da força das sílabas corretamente ordenadas,
fazia sua locução de \textit{Hamlet}, do \textit{Rei Lear}, de \textit{Júlio
César} ou de \textit{Marmion},\footnote{Título de um poema épico de Sir
  Walter Scott.} melodiosamente grandiosa e exata, mas
não tinha ideia da modulação do refrão de uma balada e pouca paciência
com a expressão de seus sentimentos. Procurava sempre, naquilo que lia,
pela determinação heroica e pelo triunfo da razão; não tolerava o amor
gratuito à miséria, e nunca lia, tanto para seu próprio prazer quanto
para minha instrução, baladas como as de \textit{Burd Helen}, \textit{The
Twa Corbies},\footnote{{[}\textit{Os dois corvos}, em celta{]}. Título de uma
  balada escocesa.} ou qualquer outro poema ou história
que despertasse interesse através de amor vão ou morte inútil.

Mas uma verdadeira, pura e enobrecedora tristeza logo começou a imiscuir
sua voz soturna na permanente felicidade daqueles dias; --- uma balada
musical, bela em sua sinceridade, que os consagrava como um cântico numa
catedral. Em relação a esses dias, --- devo agora retornar ao tempo em
que apenas os ouvia com minhas orelhas e, a despeito disso, são para mim
como se os tivesse visto com meus próprios olhos.

%69. 
Deve ter sido um pouco depois de 1780 que minha avó paterna,
Catherine Tweddale, fugiu com meu avô paterno, quando ela ainda não
tinha dezesseis anos; e minha tia Jessie, a única irmã de meu pai,
nasceu um ano depois; poucas semanas após o evento, minha avó, que ainda
não completara dezessete anos, foi surpreendida (por um amigo que
entrara inesperadamente em seu quarto) executando a dança escocesa para
três, tendo duas cadeiras como seus pares; naquele momento, não
encontrara outro modo de expressar adequadamente o prazer que sua vida
terrestre lhe proporcionava, com suas dádivas e promessas.

As promessas frustraram"-se algum tempo depois; e minha tia Jessie,
criatura elegante e perfeita, com seus belos olhos escuros, filha das
Terras Altas da Escócia, --- profundamente religiosa, em sua reservada
maneira puritana, --- e muito submissa aos destinos mais improváveis,
casou"-se com um curtidor de couros algo grosseiro, com um próspero
negócio na boa cidade de Perth: e quando eu tinha idade suficiente para
lhes fazer minha primeira visita, minha tia e meu tio, o curtidor de
couros, viviam numa casa quadrada de pedra cinzenta no subúrbio de Perth
conhecido como ``Bridgend'', a casa a cerca de cinquenta milhas ao
norte da ponte; seu jardim descendo abruptamente até o Tay, que
remoinhava águas de um cristalino melancólico, a três ou quatro pés de
profundidade, próximo aos degraus dos quais os criados mergulhavam seus
baldes.

%70. 
Um correspondente equivocado queixou"-se, em \textit{Fors}, de meu rude
hábito de escarnecer de pessoas sem ascendência nobre. Não tenho este
hábito; embora nem sempre estivesse inteiramente à vontade ao escrever
sobre meus tios padeiro e curtidor de couros. E meus leitores podem
confiar em mim quando lhes digo que, ao recordar minhas visões da casa
do absolutamente honesto principal padeiro de Market Street, Croydon, de
Peter, não Simon --- o curtidor de couros, cuja casa ficava na beira"-rio
de Perth, que não as alterarei, muito menos as aprazíveis realidades
daqueles primeiros dias, por nada que ouça agora, dos senhores e damas
daquele tempo da infância nos salões dos castelos e nos belos gramados e
lagos em florestas muradas como parques.

Gramado e lago, na verdade, eu os tinha em North Inch, em Perth, e
piscinas nos remansos do Tay, diante do Rose Terrace (onde vivi depois
da morte de meu tio, após breve apoplexia, em Bridgend), na paz
daqueles belos dias do verão escocês, com minha tia viúva, e minha
pequena prima Jessie, então atravessando um período brilhante entre seu
sexto e nono ano; olhos profundamente escuros,\footnote{Em contraste com
  o escuro da íris, fazendo os olhos parecem cerejas negras. {[}\textsc{n.\,a.}{]}} como sua mãe, e tão pia quanto ela; de modo que nós
costumávamos competir nos exames sobre as Escrituras de domingo à tarde;
e ficarmos orgulhosos como dois pavões porque os irmãos mais velhos de
Jessie, e a irmã Mary, eram geralmente ``batidos'', e eu ou Jessie éramos
sempre declarados \textit{Dux}.\footnote{\textit{Dux}, no original. Palavra
  latina que significa \textit{condutor}, \textit{guia}. Na hierarquia romana,
  título do comandante militar de regiões ou províncias de fronteira.
  Originou \textit{duque}, em português, e \textit{duce}, em italiano, por
  exemplo.} Concordamos que nos casaríamos quando
fôssemos um pouco mais velhos; sem considerar a necessidade de se
adquirir um pouco mais de sabedoria.

%71. 
Estranhamente, a empregada de cozinha, que fazia todo tipo de
trabalho, na casa em Rose Terrace, era a velha ``Mause'', --- que fora
empregada de minha avó em Edimburgo, --- que poderia muito bem ter sido o
protótipo da Mause de \textit{Old Mortality},\footnote{O vulgar
  moderno puritanismo mostrou sua degeneração em nada menos do que sua
  incapacidade de entender os retratos perfeitamente acabados de
  Covenanter, feitos por Scott. Só em \textit{Old Mortality} existem
  quatro que não podem ser superados; um típico, Elizabeth,
  perfeitamente sublime e pura; o segundo, Ephraim Mcbriar, fornecendo o
  aspecto demasiadamente banal do personagem, afetado por insanidade
  ascética; terceiro, Mause, colorida e algumas vezes burlesca, pela
  presunção escocesa, mas absolutamente forte e de coração puro; o
  último, Balfour, um estudo do maior interesse, mostrando o efeito da
  fé puritana, sinceramente professada, sobre um espírito natural e
  incuravelmente cruel. Acresça"-se a esses quatro estudos, de um único
  romance, aqueles de \textit{Heart of Midlothian}, e Nicol Jarvie e
  Andrew Fairservice, de \textit{Rob Roy}, e se terá uma série de análises
  teológicas muito superiores a de qualquer outra obra filosófica que eu
  conheça, de qualquer período. {[}\textsc{n.\,a.}{]}} mas com uma fé
mais solene, destemida e paciente, nela fortalecida pelo extremo
sofrimento; pois quase morrera de fome quando criança, e literalmente
apanhou os ossos jogados no lixo para roê"-los, e sempre, depois disso,
para ela, ver o desperdício de um átomo de comida na cozinha era tão
chocante quanto uma blasfêmia. ``Ó, Miss Margaret!'', dirigiu"-se certa
vez à minha mãe, que jogara algumas migalhas, de um prato sujo, pela
janela, ``Preferiria que a senhora tivesse me batido''. Seu jantar
consistia de tudo aquilo que os outros empregados se recusavam a comer;
--- frequentemente cascas de batatas, dando sua comida para qualquer pobre
que visse; e ficava em pé durante toda a missa (pelo menos até os
setenta anos, muito fraca), se pudesse persuadir, nas ruas, alguma
ovelha desgarrada a sentar em seu lugar. Sua face gasta e vincada,
imobilizada em resolução e paciência, incapaz de sorrir, algumas vezes
talvez excessivamente severa comigo e com Jessie, se quiséssemos mais
creme de leite no nosso mingau de aveia, ou se saltássemos de nosso
reservado favorito durante a missa dominical, --- ``Não se preocupe,
John'', disse"-me Jessie, vendo"-me num estado anticristão de provocação
em relação ao tema, ''quando estivermos casados, poderemos pular para
fora de reservados, o dia todo, se quisermos!'' --- pode ter sido
parcialmente responsável pelo meu ligeiro preconceito contra a religião
evangélica, do qual, confesso, traços podem ser encontrados em minhas
últimas obras; mas nunca poderia ser suficientemente agradecido por ter
visto, em nossa ``Old Mause'', o espírito puritano escocês em sua
perfeita fé e força; e ser capaz, em razão disso, de reconhecer seu
papel na reforma política da Escócia, com a reverência e a honra que lhe são
devidas.

%72. 
Minha tia, uma pura pomba"-sacerdotisa, se alguma já houve,
Dodona\footnote{Oráculo de localidade homônima na Grécia Antiga.} das Terras Altas da Escócia, era de um temperamento muito mais
brando; mas, ainda assim, permanece, para mim, a uma saudosa distância.
Ela foi muito abalada pela perda de três de seus filhos antes da morte
de seu marido. O pequeno Peter, particularmente, fora a pedra"-angular da
construção de seu amor; e lhe foi arrebatado repentinamente: --- inchaço
sem inflamação acometeu o joelho; ele sofreu muito, e tornou"-se
progressivamente mais fraco, sempre obediente, amoroso e
absolutamente paciente. Um dia ela quis que ele bebesse meio cálice de
Porto, e o colocou sobre os joelhos, levando o cálice a seus lábios.
``Agora não, mamãe; num minuto'', disse ele; e inclinou a cabeça sobre
seu ombro, deu um longo e baixo suspiro, e morreu. Então foi Catherine;
e --- esqueci o nome da outra pequena filha, eu não os via; minha mãe
contava"-me a respeito deles; sempre saudosa de Catherine, que fora sua
favorita. Um dia minha tia conversou seriamente com seu marido sobre as
duas crianças; planejando isso e aquilo para a educação delas e não sei
mais o quê; à noite, durante algum tempo, ela não conseguiu dormir; e,
enquanto pensava na cama, viu a porta do quarto se abrir, e duas pás de
coveiro entrarem, e pararem ao pé de sua cama. As duas crianças morreram
pouco tempo depois. Estive prestes a escrever ``dentro de quinze dias''
--- mas não posso lembrar com precisão as palavras de minha mãe.

%73. 
Mas quando vivi em Perth, ainda estavam lá --- Mary, sua filha mais
velha, que cuidava de nós, crianças, quando Mause estava muito ocupada;
James e John, William e Andrew; (não imagino em homenagem a quem o
incréu William foi batizado). Mas os rapazes estavam, então, todos na
escola ou na faculdade, --- os escolares, William e Andrew, vinham em
casa apenas para caçoar de Jessie e de mim, e comer as maiores peras
temporãs; os acadêmicos eram completamente indiferentes; e as duas
garotas e eu brincávamos, de maneira tranquila, em North Inch, e à
margem do Lead, um canal conduzido a partir do Tay, através de
Rose Terrace, até a cidade, com o objetivo de mover moinhos; e, creio,
há muito coberto com tijolos ou entulhado com detritos; mas, àquela
época, adorável, e um perpétuo tesouro de diamante fluido para nós,
crianças. Mary, a propósito, estava caminhando para os doze anos ---
loura, olhos azuis, e razoavelmente bonita; e tão pia quanto Jessie, sem
ser tão carola.

%74. 
Meu pai raramente ficava conosco em Perth, mas viajava a negócios
através da Escócia, e até minha mãe tornou"-se uma personagem
curiosamente secundária em Rose Terrace. Não entendo porque apenas
raramente passeava conosco, as crianças; ela e minha tia pareciam ter
sempre seus caminhos secretos. À Mary, Jessie e eu era permitido fazer o
que quiséssemos em North Inch: não me lembro de nenhum trabalho
escolar nesse tempo de Perth, exceto a já mencionada competição sobre as
Escrituras, aos domingos.

Se tivesse havido alguém para me ensinar algo sobre plantas ou seixos,
teria sido bom para mim; do modo como foi, passei meus dias como os
cardos e as atanásias, contemplando infinitamente os fluxos de água, ---
um medo singular desenvolveu"-se em mim, tanto dos remansos do Tay, onde
a água mudava de castanho para azul"-profundo, como dos precipícios de
Kinnoull; parcialmente originado de minha própria mente, parcialmente
devido à preocupação dos empregados quando subíamos o Kinnoull,
particularmente se eu desejasse parar para contemplar a pequena fonte
cristalina de Bower's Well.

%75. 
``Mas você não disse que não tinha medo de nada?'', escreve um
amigo, preocupado com a inatacabilidade da veracidade dessas memórias.
Bem, respondi, não tinha medo de fantasmas, trovão, ou feras, ---
referindo"-me especificamente aos terrores mais comuns da infância. Cada
dia, à medida que me tornava mais maduro, trazia"-me um medo racional;
acima, já me havido descrito como a pessoa mais racional que conhecia. E
pelos redemoinhos de calmo negror, sem máculas de espuma, onde o Tay
adquiria a forma de Medusa,\footnote{Sempre penso no Tay como uma
  deusa"-rio, como em Greta como uma ninfa. {[}\textsc{n.\,a.}{]}} nunca passei
sem medo, mesmo naqueles imprudentes dias; nem ousei acreditar que
pudesse caminhar entre lápides, à noite (nem durante o dia), como se
fossem apenas pedras de calçamento colocadas em pé. Muito pelo
contrário; mas é importante para a confiança do leitor em escritos que
parecem excessivamente impressionistas e emocionais que se saiba que
nunca estive submetido a --- talvez deva dizer, lamentavelmente, nunca
fui capaz de --- qualquer tipo de ilusão ou falsa imaginação, nem ao
menos sujeito a ter meus nervos abalados pela surpresa. Quando tinha
cerca de cinco anos, tendo estado em termos amigáveis com um Terra"-Nova
negro, então em experiência como cão de guarda em Herne Hill, após
nossas longas jornadas de verão, meu primeiro pensamento ao chegar à
casa era ver Leão. Minha mãe permitiu que eu fosse ao estábulo em
companhia de um de nossos empregados, Thomas, dando"-lhe ordens estritas
para que eu permanecesse a uma distância superior à corrente do cão.
Thomas, para maior segurança, carregou"-me nos braços. Leão estava
jantando, e não se surpreendeu com nenhum de nós; então implorei que o
deixasse acariciar. O aturdido Thomas colocou"-se diante do cão, que
saltou imediatamente sobre mim, me mordeu, arrancando um pequeno pedaço
da comissura labial esquerda. Fui levado para as escadas do fundo,
sangrando muito, mas nem um pouco assustado, exceto pela possibilidade
que Leão pudesse ser mandado embora. Leão realmente teve que ir; mas não
Thomas: minha mãe estava segura que ele não tinha culpa, e creio que se
considerou a maior responsável pelo acidente. O lado mordido da boca
(então realmente bonita), guarda para sempre a cicatriz, mas a ferida,
cujos bordos foram aproximados, fechou"-se rapidamente; a última vez que
movi meus lábios antes que Dr.\,Aveline ordenasse que ficassem em
silêncio por algum tempo, foi para dizer, ``Mamãe, se não posso falar,
posso tocar violino''. Mas a família tinha outra opinião, e nunca
alcancei, naquele instrumento, proficiência digna de meu gênio. O
incidente não provocou qualquer diminuição de meu amor pelos cães nem
qualquer temor de lidar com eles.

Não sabia se corria perigo real ou não, quando, noutro dia, estando
sozinho no mesmo estábulo, caí de cabeça no grande tanque da água
utilizada no jardim. Pensei que poderia ficar irremediavelmente preso se
tentasse colocar os pés para dentro do tanque: em vez disso, usei o
pequeno regador que tinha na mão para produzir um bom impulso a partir
do fundo e agarrar a outra borda do tanque com minha mão esquerda,
obtendo, posteriormente, um não pequeno elogio pela escolha do método.
Olhando retrospectivamente para as poucas chances em que tive, de alguma
maneira, de usar a cabeça, creio que nunca me faltou quando precisei
dela, e que sou mais propenso a me perturbar por súbita admiração que
por súbito perigo.

%76. 
Os remansos escuros do Tay, que levaram à minha jactância, ficavam
sob a margem esquerda, no início de North Inch, --- o caminho acima deles
era raramente percorrido por nós, crianças, exceto na época da colheita,
quando costumávamos ir respigar nos campos além dela; Jessie e eu depois
triturávamos nosso trigo no moedor de pimenta da cozinha, amassávamos e
assávamos nossos próprios bolos de pimenta, de impagável qualidade.

No curso geral desta minha cuidadosa narrativa, refuto, com tanta
indignação quanto é possível sem atingir a rudeza, a pecha de
parcialidade em relação a qualquer coisa apenas porque a vi quando era
jovem. Hesito, contudo, em registrar como permanente verdade para o
mundo, a impressão que ficou quando fui respigar com Jessie, de que as
gavelas escocesas são mais douradas que as de outras terras, e de que
não há outras plantações, aos olhos humanos, que sejam tão próximas ao
``paraíso de trigo''\footnote{Salmo \textsc{lxxviii}, 24. {[}\textsc{n.\,a.}{]}} como
aqueles de Strath"-Tay e Strath"-Earn.

\chapter{Sob novas tutorias} %Capítulo \textsc{iv}. 

%77. 
Quando tinha cerca de oito ou nove anos, fui acometido por uma febre
maligna em Dunkeld, durante a qual, creio, corri certo perigo, e, sem
dúvida, senti"-me muito mal. Surgiu após uma longa caminhada na qual
estivera colhendo dedaleiras e as despedaçando para examinar suas
sementes; houve alusões a que eu tivesse sido envenenado por elas, o que
era completamente absurdo, mas que estenderam o medo dos redemoinhos do
rio aos pequenos vales de dedaleiras. Pouco depois disso, quando
estávamos de volta a casa, minha prima Jessie adoeceu e morreu muito \label{jessie}
lentamente, com água no cérebro. Fiquei muito triste, não tanto pela
força do precoce afeto que desenvolvera por ela como pelo sentimento de
que os dias felizes em Perth tinham terminado para sempre, desde que já
não havia Jessie.

Antes que a doença adquirisse sua forma fatal, --- antes mesmo, creio,
que se tivesse declarado --- minha tia teve um de seus sonhos
premonitórios, simples e suficientemente vagos para permitir qualquer
interpretação; --- que estava se aproximando do vau de um rio escuro,
sozinha, quando a pequena Jessie surgiu correndo, a ultrapassou, e
cruzou o rio antes dela. Então atravessou o rio, olhou para trás da
outra margem, e, à distância, viu sua filha mais velha, Mause,
aproximando"-se do ponto perigoso do rio. E assim foi que Jessie, logo
após, adoeceu rapidamente e morreu; e poucos meses depois, ou talvez
quase um ano, minha tia morreu de definhamento; e dois ou três anos mais
tarde foi a vez de Mause, cuja única preocupação, após a partida de sua
mentora e de Jessie, era o momento de juntar"-se a elas.

%78. 
Estava em Plymouth com meu pai e minha mãe quando minha tia escocesa
morreu; encontrava"-me muito feliz com minha babá, numa colina a leste da
cidade, olhando a baía e o quebra"-mar; e retornei a casa para encontrar
meu pai, pela primeira vez, envolto na profunda dor de soluços e
lágrimas.

Senti muito a morte de minha tia, mas, àquela época (e durante um bom
tempo depois também), vivi sobretudo no presente, como um animal, e
minha principal sensação foi --- Que pena passar uma noite tão
desconfortável! E estávamos em Plymouth.

As mortes de Jessie e de sua mãe naturalmente findaram nossos dias
escoceses. A única filha sobrevivente, Mary, foi adotada por meu pai e
minha mãe, e cresceu junto comigo. Ela tinha quatorze anos quando veio
morar conosco, e eu era quatro anos mais jovem; --- assim, com os dias de
Perth, encerrou"-se a primeira década de minha vida. Mary era algo
bonita, olhos azuis, de compleição tosca, muito amável e afetiva de uma
maneira discreta, sem aptidões naturais, mas com bom senso e bons
princípios, honesta e inofensivamente devota, equilibrada, sem as
frivolidades e caprichos típicos das mocinhas. Tornou"-se um adicional
tom neutro e sereno na harmonia do lar; lia alternadamente comigo e minha
mãe versículos da Bíblia, no início da manhã, e frequentava a escola no
turno matutino. Quando viajávamos, ela assumia o papel de governanta em
relação a mim, nos sendo permitido explorar alguns lugares sem a
companhia da babá; --- mas geralmente levávamos também como companhia a
velha Anne.

%79. 
Começou então a ter alguma importância a igreja que eu frequentava
nas manhãs de domingo. Meu pai, ainda com a saúde frágil, não podia
assistir ao longo serviço da igreja anglicana, e, sendo minha mãe
evangélica, ele nos acompanhava, contente, ou pelo menos submisso, à
capela de Beresford, Walworth, onde o reverendo E. Andrews proferia,
regularmente, um sermão algo eloquente, enérgico e sincero, não
cansativo de ouvir: as preces eram abreviações do serviço da igreja
anglicana e a nós, sendo as pessoas mais importantes na congregação, era
permitido --- embora, como agora lembro, não sem olhares ofendidos e
reprovadores dos fiéis mais conscienciosos --- chegar quando as orações
estavam pela metade. Mary e eu costumávamos escrever um resumo do
sermão, à tarde, para nosso próprio prazer, --- Mary, por dever, e eu
para mostrar quão bem o podia fazer. Nunca íamos à igreja à tarde ou à
noite. Lembro ainda a surpreendente e assustadora sensação, como uma
visão preliminar do dia do Juízo Final, ao ir pela primeira vez, um ou
dois anos mais tarde, à igreja iluminada pela luz de velas.

%80. 
Não praticávamos o culto em casa, mas nossos empregados eram mais
bem tratados do que os de lares ostensivamente religiosos. Minha mãe
costumava buscá"-los, quando meninas, em famílias dela conhecidas, irmã
após irmã, e nunca tivemos problemas.

Nas noites de domingo, meu pai algumas vezes nos lia um sermão de Blair,
ou poderia ser que um empregado da loja ou um cliente viesse jantar,
quando então a conversa, meramente por cortesia, geralmente dirigia"-se
para o \textit{sherry}. Mary e eu passávamos a noite como podíamos, lendo
o \textit{Pilgrim's Progress}, a \textit{Holy War de Bunyan}, os
\textit{Emblems} de Quarles,\footnote{Francis Quarles (1592--1644) foi um poeta
  renascentista inglês, cujo livro mais conhecido é justamente
  \textit{Emblems}.} o \textit{Book of Martyrs} de
Foxe,\footnote{John Foxe (1516--1587) foi um escritor e religioso inglês.} a \textit{Lady of Manor} de Mrs.\,Sherwood,\footnote{Mary Martha
  Sherwood (1775--1851) foi uma escritora inglesa de livros infantis.} --- um livro terrível para mim, pelas histórias de garotas más
que tinham ido a bailes e morrido imediatamente de febre, --- e o
\textit{Henry Milner} de Mrs.\,Sherwood, --- este mais tarde, --- o
\textit{Youth's Magazine}, \textit{Alfred Campbell the Young
Pilgrim}, e, embora mais como uma indulgência profana, permitida devido
à dureza de nossos corações, a \textit{Natural History} de
Bingley.\footnote{Reverendo William Bingley (1774--1821) foi um naturalista e
  escritor inglês.} Nenhum de nós se interessava por
cantar hinos ou salmos, e éramos demasiadamente honestos para considerar
a música sacra como simples divertimento, além do fato de que não a
achávamos particularmente divertida.

%81. 
Meu pai e minha mãe, --- embora os cheques para caridade fossem
naturalmente enviados ao Dr.\,Andrews, e várias cortesias no Natal, como
perus e caixas de passas, tendo declarado sua satisfação com o estilo de
seu sermão e com a pureza de sua doutrina, --- com sua habitual timidez,
nunca tentaram travar relações com o reverendo, ou mesmo permitido que o
estado de suas almas fosse conhecido, nas visitas pastorais. Mary e eu,
contudo, nós sentíamos atraídos por ele, à distância, e costumávamos
caminhar com Anne, para cima e para baixo em Walworth, pela simples
esperança de vê"-lo passar pelo outro lado da rua. Finalmente, um dia,
quando, por extrema condescendência da Fortuna, encontrou"-nos, quando ia
a toda pressa, no nosso lado da rua e quase trombou comigo, Anne, quando
ele recuperou o equilíbrio, fez"-lhe uma rápida reverência; ao parar,
perguntou quem éramos, e foi extremamente amável conosco; e, ao
chegarmos a casa, numa febre de arrebatamento, dissemos, o que não
agradou muito a minha mãe, que o Doutor prometera nos visitar qualquer
dia! E assim, pouco a pouco, a bem"-aventurada relação se estabeleceu. Eu
devia ter onze anos ou estar me aproximando dos doze, nessa época. Miss
Andrews, a filha mais velha do ``anjo na casa'', era uma belíssima
garota de dezessete anos; cantava \textit{Tambourgi, Tambourgi}\footnote{Melodias
  hebraicas. {[}\textsc{n.\,a.}{]}} com muito espírito e uma voz rica, quando,
em época de amoras"-pretas, perambulava conosco por Norwood Spa, e me fez
perceber que havia algo nas garotas que não entendia, e isso era
curiosamente agradável. E finalmente, porque eu gostava tanto do Doutor,
e ele tinha a reputação (em Walworth) de bom erudito, meu pai pensou que
ele poderia, agradavelmente, ensinar"-me grego, iniciação já há muito
postergada. O Doutor, como se revelou mais tarde, sabia de grego pouco
mais que as letras e as declinações dos substantivos; mas as escrevia
com elegância, e tinha um ouvido acurado e sensível para o ritmo.
Começou a ensinar"-me as Odes de Anacreonte\footnote{Anacreonte ou Anakréon foi um poeta
  lírico grego da segunda metade do século \textsc{vi} a.C.} e me
fez estudá"-las e a Virgílio por inteiro, às vezes, como interlúdio,
recitava"-me passagens de Shakespeare com força e propriedade. A métrica
de Anacreonte agradava"-me por completo, não menos que o sentimento de
seus versos. Decorei metade das odes, apenas para meu próprio prazer, e
aprendi, o que em estudos posteriores da arte grega mostrou"-se
extremamente vantajoso para mim, que os gregos gostavam de pombas,
andorinhas e rosas, tanto quanto eu.

%82. 
No intervalo dessas agradáveis lições de grego, eu me divertia ---
parcialmente escrevendo versos de pé"-quebrado, parcialmente desenhando
mapas, ou copiando as ilustrações de Cruikshank\footnote{Certamente George Cruikshank (1792--1878), caricaturista e ilustrador inglês.} para os contos de
Grimm, o que fazia com grande exatidão, agora inacreditável para a
maioria das pessoas, mas, por feliz acaso, uma folha com desenhos feitos
quando tinha entre dez e onze anos, foi bem preservada. Mas nunca vi, em
toda a vida, obra infantil que mostrasse tão pouco talento, seja em
desenhos originais, seja em cópias. Não conseguia, literalmente,
desenhar nada, nem um gato, nem um rato, nem um barco, nem um arbusto,
\textit{saído de minha cabeça}, e não havia, naquele momento, felizmente,
quer da parte de meus pais, quer da parte de meu preceptor, a ideia de
ensinar"-me a desenhar a partir da mente de outras pessoas.

Mary, entretanto, na escola, estava tendo lições de desenho junto com as
outras crianças. Seu relato da amabilidade e do zelo do professor, e seu
modo franco e algo inusual de realizar as cópias interessou meu pai, que
ficou ainda mais impressionado com as cópias de Mary, como prova de sua
aplicação, quando estava distante, em viagem de inverno --- copiou a
lápis, de modo a produzir o efeito de uma vigorosa gravura, a pequena
aquarela de Prout\footnote{Samuel Prout (1783--1852) foi um aquarelista inglês,
  sobre o qual Ruskin escreveu as seguintes notas: \textit{Notes on the
  Fine Arts} e \textit{Society's Loan Collection of Drawings by Samuel Prout
  and William Hunt}).} de uma cabana à beira do caminho,
que foi o início de nossa futura coleção de aquarelas, então nosso único
quadro desse tipo --- de outro gênero, duas miniaturas de marfim
completavam a coleção.

%83. 
Percebo, ao refletir sobre o bom trabalho daquele paciente estudo em
preto e branco, que Mary poderia ter desenhado, se tivesse sido
orientada e apropriadamente encorajada. Mas sua capacidade de copiar
pacientemente não servia para desenhar a natureza e quando, naquele
mesmo verão, tendo eu entre dez e onze anos,\footnote{Em 1829. {[}\textsc{n.\,a.}{]}} passamos uma
temporada em Matlock, Derbyshire, tudo que ela se mostrou capaz de
conseguir foi um esboço do Caxton New Bath Hotel, com o qual seus
esforços em direção à arte findaram"-se naquele ano.

Mas, nos resplandecentes fragmentos brancos de espato, matizados com
galena, que faziam brilhar as alamedas do jardim do hotel, e nas lojas
da bela vila, e em muitos passeios alegres por suas falésias, continuei
meus estudos mineralógicos sobre o flúor, a calcita, e os minérios de
chumbo, com indescritível êxtase quando me era permitido entrar numa
caverna. Meu pai e minha mãe mostraram"-se mais benevolentes do que eu
esperava, ao condescenderem com minha paixão subterrânea; minha mãe não
podia suportar lugares sujos, e meu pai temia que as escadas pudessem
quebrar, ou o teto desabar, antes que conseguíssemos escapar. Eles
acompanharam"-me, entretanto, aonde eu quis ir, --- meu pai, até mesmo à
terrível mina Speedwell, em Castleton, onde, por um momento, fiquei um
pouco assustado.

De Matlock, devemos ter ido a Cumberland, pois encontrei nos escritos de
meu pai a anotação, ``Começado em 28 de novembro de 1830, terminado em
11 de janeiro de 1832'', na folha de rosto de \textit{Iteraid}, poema
em quatro livros, composto por mim, entre aquelas datas, cujo tema era
nossa viagem pelos Lagos, e do qual falaremos mais tarde.

%84. 
Deve ter sido na primavera de 1831 que importante passo foi dado
para que eu tivesse um professor de desenho. Mary não mostrava qualquer
talento para representar os cenários de nossas viagens, e comecei a
expressar o desejo de poder desenhá"-los. Assim, o agradável professor de
desenho de Mary, que, segundo meus pais, não era o responsável por sua
falta de talento, foi convidado para também \textit{me} dar uma hora de
aula por semana.

Creio que um professor de desenho só consiga se estabelecer, se assumir,
diante do público, a si mesmo como professor de um estilo; e ensinar a
partir dele. A persistente imagem negativa de Mr.\,Runciman em minha \label{84}
memória decorre, entretanto, dele não ter estimulado, nem mesmo pela
indulgência, o extraordinário talento que eu tinha para desenhar
delicadamente com o bico da pena. Qualquer trabalho desse tipo, desde
então, foi feito apenas para meu próprio prazer. Mr.\,Runciman não me deu
para copiar nada além de seus afetados e ineficientes desenhos, e minou,
enormemente, a força de minha mente e de minha mão.

Ele, contudo, ensinou"-me muito, e sugeriu ainda mais. Ensinou"-me
perspectiva, de um modo ao mesmo tempo preciso e simples --- uma
contribuição inestimável. Compeliu"-me à rapidez e à destreza manuais que
se revelaram, posteriormente, extremamente úteis, embora o que eu então
chamasse a ``força'', a extrema precisão de meu traço, tenha sido
perdida. Ele despertou em mim, --- na verdade, fez nascer, --- o hábito de
buscar os pontos essenciais daquilo que deveria desenhar, a fim de
abstraí"-los de modo decisivo, e explicou"-me o significado e a
importância da composição, embora ele próprio não fosse capaz de compor.

%85. 
Um período muito feliz seguiu"-se, por aproximadamente dois anos.

Eu estava, naturalmente, muito atrás de Mary, na habilidade com o lápis
de desenho, e foi bom para ela que esta superioridade tivesse sido
reconhecida, e a devida honra feita aos esforços constantes advindos de
sua prática sistemática e de sua incansável atenção. Pois, como ela não
escrevia poemas nem colecionava minerais como eu, nem mostrava
predisposição particular de espírito em qualquer área, foi, aos poucos,
decaindo para uma posição muito subordinada ao poder de minha grandeza.
Mas, durante algum tempo, não pude pretender rivalizar"-me com ela em
cópias à mão"-livre, e minhas primeiras tentativas de copiar da natureza
não foram vistas por meu pai como as menos lisonjeiras à sua vaidade.

Esses desenhos surgiram no curso de uma viagem a Dover, com cuja
promessa minha mãe pretendeu confortar"-me durante uma doença em 1829.
Encontro meu primeiro caderno de esboços, um pequeno e extremamente
inconveniente \textit{in"-octavo}, com capa estampada e flexível, o papel
branco puro, com nervuras granulosas, repleto de esboços, irregularmente
desfigurados por esforços impulsivos para concluí"-los, arbitrariamente
distribuídos pelas páginas e cantos, de Dover e Tunbridge Castles e da
torre principal da catedral de Canterbury. Estes, junto com um estudo
realmente bom, complementado por um detalhamento à parte, de Battle
Abbey, foram separados para preservação; o primeiro esboço que realmente
fiz a partir da natureza sendo o número um, de uma rua em Sevenoaks. Tenho
pouca satisfação e menor orgulho desses trabalhos; mas minha compreensão
inata da arquitetura revela"-se imediatamente neles, --- algo evidente
para os que se interessam pela percepção de tais aptidões. Os dois
esboços a lápis do pórtico sul e da torre central da catedral de
Canterbury, dei"-os para Miss Galé, de Burgate House, Canterbury: o
restante do caderno, a Mrs.\,Talbot, de Tyn"-y"-Ffynon, Barmouth, ambos
amigos muito queridos.

%86. 
Mas, antes de tudo, nessa época, meu prazer vinha de simplesmente
observar o mar. Não me era permitido remar, muito menos velejar, nem
andar sozinho ao longo do porto; então nada aprendi sobre navios nem
sobre qualquer outra coisa digna de interesse, mas gastava quatro ou
cinco horas por dia simplesmente contemplando o mar e divagando ---
atividade que se manteve até os quarenta anos. Sempre que ia a uma
praia, contentava"-me em olhar as ondas, ouvi"-las, acompanhá"-las e
devanear. Nunca me interessei pela história natural de conchas, ou
camarões, ou medusas. Seixos? --- sim, se estivessem à mão; do contrário,
meramente contemplava, durante todo o dia, a força confusa e espumante
do oceano. Tolo, parece"-me agora, desperdiçar toda essa inestimável
juventude apenas em sonho e transe de admiração; havia algo de paixão
byroniana nisso, o que é alguma coisa: mas foi uma terrível perda de
tempo.

%87. 
O verão de 1832, creio, deve ter sido passado em casa, porque meu
segundo livro de esboços contém apenas esforços para desenhar árvores em
Dulwich, e uma vista da ponte que agora havia sobre o Effra, pela
qual a estrada de Norwood cruzava o rio ao pé de Herne Hill: a própria
estrada, exatamente no lugar onde, de cima da ponte, abrangia"-se o
córrego com o olhar, reduzido a uma zona úmida, sombria e pútrida, perto
da estação de Herne Hill. Este esboço foi o primeiro no qual,
supostamente, mostrei algum talento para desenho. Mas no meu décimo terceiro aniversário, 8 de fevereiro de 1832, o sócio de meu pai, Mr.\,Henry Telford, deu"-me \textit{Italy} de Rogers,\footnote{Poema do poeta e
  banqueiro inglês Samuel Rogers (1763--1855), ilustrado por Turner.} que determinou a principal orientação de minha vida.

Nessa época nunca tinha ouvido falar de Turner, exceto pelas palavras
marcantes de Mr.\,Runcinam, de que ``o mundo tinha sido recentemente
fascinado e sacudido por algumas esplêndidas ideias lançadas por
Turner''. Mas assim que lancei os olhos sobre as vinhetas de Rogers, \label{87}
considerei"-as como meus únicos mestres, e encetei, por conta própria, a
imitá"-las da melhor maneira possível, com sombreado de pena fina.

%88. 
Contei esta história tantas vezes que começo a duvidar de quando
aconteceu. É curiosamente surpreendente que Mr.\,Telford não tenha
escrito, ele próprio, meu nome no livro, e meu pai, que nele escreveu,
``O presente de Henry Telford, Esq.'',\footnote{Abreviatura de
  \textit{Esquire}: membro da pequena nobreza em posição inferior a de um
  cavaleiro; escudeiro; ilustríssimo senhor; título de cortesia
  empregado após o nome caso não haja outro predicado, como \textit{Mr}., \textit{Dr}. etc.} ainda mais curiosamente, não tenha colocado data:
se foi um ano antes, não tem importância; não resta dúvida, contudo, que
no início da primavera de 1833, Prout tenha publicado seu \textit{Sketches
in Flanders and Germany}. Lembro bem de ir com meu pai à loja onde os
subscritores se inscreviam, tendo como referência a gravura de uma torre
com janela sobre o Mosela, em Koblenz. O livro chegou a Herne Hill antes
de nosso \textit{tour} anual; e, minha mãe, ao perceber nosso prazer em
olhar aqueles lugares maravilhosos, perguntou por que não íamos visitar
alguns deles? Meu pai hesitou um pouco, então com olhos brilhantes disse
--- por que não? E houve duas ou três semanas de preparativos
arrebatadores e maravilhados. Lembro"-me de, a cada noite, descer meu
grosso livro de geografia, agora ainda mais precioso para mim; (desço"-o
agora, e pela primeira vez escrevo minhas iniciais abaixo do nome de meu
pai) --- e de olhar, com Mary, para os contornos do Monte Branco, copiado
de Saussure,\footnote{Horace Bénédict Saussure (1740--1790), naturalista e \label{saussure}
  geólogo suíço, pioneiro da exploração dos Alpes; considerado o
  fundador do alpinismo.} à página 201, e de ler as
interessantes informações sobre os Alpes que o desenho ilustrava. De
modo que a Suíça deve ter sido imediatamente incluída nos planos --- logo
prosperamente, e com os resultados mais positivos, com a ajuda de Deus,
realizados.

%89. 
Fomos através de Calais e Bruxelas a Colônia; subimos o Reno até
Estrasburgo, cruzamos a Floresta Negra até Sachaffhausen, então fizemos
um giro através do norte da Suíça, por Basileia, Berna, Interlagos,
Lucerna, Zurique, até Constança, --- subindo o Reno até Coire, a seguir
por Splügen até Como, Milão e Gênova; com a intenção, como lembro agora,
de chegar a Roma. Mas já era junho, o calor de Gênova advertiu"-nos da
imprudência: mudamos de direção, e voltamos através de Simplon para
Genebra, vimos Chamonix, e então para casa através de Lyon e Dijon.

Fazer tudo isso da única maneira então possível, com cavalos de posta,
e, nos lagos, com barcos a remo, requeria planejamento minucioso a cada
dia. Meu pai gostava de chegar ao lugar em que dormiríamos o mais cedo
possível, e nunca parava o cavalo para que eu desenhasse (os
\textit{pence}\footnote{Plural de \textit{penny}, moeda
  divisionária inglesa.} extras que deveria dar ao
postilhão, pela espera, também pesavam na decisão); --- assim adquiri o
mau hábito, não sem certa disciplina, contudo, de fazer rascunhos
enquanto a carruagem se movia, e de trabalhá"-los, \textit{a partir de minha
mente}, à noite. Utilizando este método, produzi, durante toda a
viagem, cerca de trinta folhas com desenhos à pena fina e nanquim,
quatro ou cinco por folha; alguns não deselegantes, todos laboriosos,
mas em sua maioria um igual ao outro e, sem exceção, simplórios e
impessoais ao extremo.

%90. 
Desses rascunhos em voo pela estrada, fiz, durante a permanência nas \label{suica}
cidades, à pena e a lápis, alguns esboços aprimorados, dos quais, talvez
meia"-dúzia sejam dignos de registro e preservação. O orgulho de meu pai
em relação a um estudo da igreja renascentista com dupla torre de Dijon
foi grande. Um desenho ainda mais trabalhoso do Hôtel de Ville de
Bruxelas encontra"-se, junto com o de Dijon, em Brantwood. O desenho do
Hôtel de Ville que tenho comigo em Oxford é uma cópia de Prout que fiz
para ilustrar o volume no qual escrevi o começo de uma narrativa em
versos da viagem.

Pois a viagem despertara em mim todas as pobres pequenas faculdades que
jaziam sob extrema pressão, e, naqueles três meses, certamente tive mais
felicidade apaixonada, de um tipo absolutamente indescritível para
pessoas que nunca sentiram algo semelhante, ainda mais em tão sólida
quantidade, do que a maioria das pessoas ao longo de todas suas vidas.
Da impressão dos Alpes, vistos pela primeira vez de Schaffhausen, de
Milão e de Genebra, tentarei fornecer algum relato, posteriormente, --- 
agora, o mais importante é continuar.

%91. 
O inverno de 1833, e o tempo que pude roubar para meu próprio
divertimento, do ano de 1834, foram gastos na composição, revisão do
texto, e no desenho de vinhetas para a ilustração da citada narrativa
poética da viagem, em imitação à \textit{Italy} de Rogers. Os desenhos
foram feitos em pedaços separados de papel e colados nos livros; muitos
foram arrancados, outros permanecem sem que os versos correspondentes
jamais tenham sido escritos, pois todo meu entusiasmo foi gasto antes de
subirmos o Reno. Deixei a inacabada asneira aos cuidados de Joanie, para
que apenas amigos a pudessem ver.

Entrementes, tendo meu pai e minha mãe percebido que Dr.\,Andrews não
poderia me preparar nem para a universidade nem para as
responsabilidades de um bispado, fui enviado como aluno externo à escola
privada do reverendo Thomas Dale, em Grove Lane, a qual podia ir andando
de Herne Hill. Descia com meu pai após o desjejum, levando minha pasta
azul com os livros, e voltava para o almoço da uma e meia, preparando as
lições para o dia seguinte, à noite. Nessas circunstâncias, pouco via
meus colegas de escola, os dois filhos de Mr.\,Dale, Tom e James; e três
internos, os filhos do coronel Matson, de Woolwich; de Alderman Key, de
Denmark Hill; e um rapaz excelente e muito vivaz, Willoughby Jones,
depois Sir \textsc{w}., recentemente morto, para meu pesar.

%92. 
Comportando"-me, em todos os aspectos, de um modo que os rapazes só
poderiam considerar como inocente, trataram"-me como suponho que
tratariam uma garota; não me bateram nem zombaram de mim, ---
compreendendo, certamente, desde o início, que zombarias não teriam
efeito sobre mim. De modo geral, eu não as entendia, nem ao menos as
percebia, a fonte do puro amor"-próprio em meu coração protegendo"-me
serenamente contra toda depreciação, seja de professores ou de colegas.
Compreendia bem o conteúdo dos livros, tinha uma memória instantânea e
eficaz, aprendendo as lições o mais rapidamente e o melhor possível; e
desde que os outros garotos sempre aprendiam o mínimo que podiam, embora
estivesse muito atrás deles em conhecimento real, sempre sabia a lição
do dia melhor do que qualquer um. Já descrevi, no quarto capítulo de
\textit{Fiction, Fair and Foul},\footnote{\textit{Ficção, beleza e maldade}:
  título de uma coletânea de artigos de Ruskin.} a
recusa de Mr.\,Dale em aceitar meu conhecimento da gramática antiga, que
considerava como \textit{coisa escocesa}. A partir dessa atitude, eu o
rejeitei como meu mestre; e desde então aprendi tudo o que ele me
ensinou apenas por obrigação.

%93. 
Enquanto esses passos eram dados para meu desenvolvimento nos
estudos clássicos, encontrou"-se um professor, ainda na desafortunada
Walworth, para me ensinar matemática. Mr.\,Rowbotham era uma pessoa
extremamente diligente, merecedora, e muito bem informada em sua área de
trabalho, que, mesmo com a esposa, e dos vários impedimentos e
inconvenientes representados pelos filhos, mantinha uma \textit{academia para
jovens cavalheiros} perto de \textit{Elephant and Castle}, numa das
primeiras casas com canteiros negros de grama na frente, separados por
grades de ferro de Walworth Road.

Ele sabia latim, alemão e gramática francesa; era capaz de ensinar"-me o
\textit{uso dos globos}, no nível exigido numa escola preparatória, e era,
muito além do que se necessitava ali, um grande matemático. De resto,
absolutamente ignorante dos homens e de sua história, da natureza e de
seus significados; maçante e triste, incapaz de qualquer jovialidade ou
fantasia, pensava que a matemática fosse a única ocupação apropriada
para o intelecto humano, asmático a um grau que frequentemente causava
sofrimentos irremediáveis, e muito pobre, passando as noites, após o
penoso trabalho na escola, a escrever manuais de aritmética e álgebra, e
compilando gramáticas francesas e alemães, em cuja venda seria
trapaceado pelos livreiros, --- acrescentando, talvez, com um ano de
trabalho noturno, quinze ou vinte libras à sua renda; --- ser mais
desgraçado, inocente, paciente, insensível, não admirável,
desconfortável e intolerável jamais foi produzido nesta área da
Inglaterra pela cultura característica de sua metrópole.

%94. 
Sob a tutoria, duas vezes por semana, à noite, de Mr.\,Rowbotham %desalinhado
(sempre convidado para um substancial chá conosco antes das lições, como
auxílio realmente eficiente a sua ciência faminta, depois de ter subido
a pé Herne Hill, penosamente, devido à asma), prosperei razoavelmente em
1834, aprendendo algo da gramática francesa, da qual realmente sentia
necessidade, --- havia, antes, aprendido, de algum modo, palavras
suficientes para me desembaraçar, --- e, não sei como, lembro que, em
Paris, ao visitar o Louvre sob a supervisão de Salvador (queria fazer
um desenho da Ceia de Emaús, de Rembrandt), e quando Salvador solicitou
permissão ao custódio, parece que eu não tinha idade suficiente para
obter um tíquete de entrada, --- quinze anos era a idade mínima para a
admissão; mas vendo meu ar desolado, o amável custódio disse que achava
que se eu fosse ao ``Conselho'', ou algo semelhante, das autoridades, e
eu mesmo solicitasse a permissão, ela me seria dada. Após o que,
imediatamente roguei que fosse apresentado ao Conselho, e o custódio,
levando"-me sob as abas de seu casaco, eu, falando o francês desastrado
que me era possível, apresentei meu caso a vários cavalheiros com um
aspecto impressionantemente oficial, obtive minha permissão, e fiz um
esboço da Ceia de Emaús, conseguindo algum sucesso em termos de
expressão, e fiquei extremamente orgulhoso de mim. Mas meu parco
conhecimento da língua, ainda que suficiente para o dia a dia, deixou"-me
triste e envergonhado após o jantar fatal em casa de Mr.\,Domecq, quando a
pequena Elise, que acabara de fazer nove anos, percebendo que suas irmãs
mais velhas não fariam o esforço necessário para falar comigo, e sendo
de um temperamento inteiramente benevolente e compassivo, atravessou o
salão e veio até mim, em completa desolação, e apoiando um cotovelo
sobre meu joelho, dispôs"-se, deliberadamente, a falar de forma melíflua
comigo durante uma hora e meia, --- sem esperar resposta, a qual sabia
que eu seria incapaz de fornecer, mas satisfeita com minha grata e
respeitosa atenção, e admirado interesse, se não exatamente por aquilo
que ela dizia, pelo menos pela maneira como ela dizia. Contou"-se a
história completa de sua escola, e do caráter objetável de seus
professores, e do deleitável caráter de seus colegas, das brincadeiras
de mau"-gosto que lhe fizeram, e dos prazeres sub"-reptícios que eles
urdiam, e das alegrias da volta aos Champs Élysées, e da semelhança de
Paris com os Jardins do Éden. E a hora e meia pareceu"-me demasiadamente
curta, deixando"-me decidido, fosse como fosse, a fazer o melhor possível
para aprender francês.

%95. 
Assim, como disse, progredi nos estudos, para a alegria de Mr.\,Rowbotham, 
passando facilmente pelos três primeiros livros de Euclides,
e indo até as equações quadráticas, em álgebra. Mas, então, parei,
virtualmente, para sempre. Quando me deparei com somas de séries, ou
símbolos expressando as relações em vez das verdadeiras magnitudes das
coisas, --- parcialmente por falta de aptidão, parcialmente por uma já
bem"-desenvolvida e saudável aversão às coisas inutilmente aborrecidas e
inatingíveis, --- furtei"-me --- ou mantive"-me indiferente. Posteriormente,
em Oxford, obrigaram"-me a estudar algumas seções cônicas, cuja
representação, através do desenho, revelou"-se, depois, de extremo valor
para mim; e ensinaram"-me tanta trigonometria que meu desenho da montanha
mostrou"-se inatacável, tanto do plano como da elevação. Em geometria
elementar estava sempre bem, e, para um garoto, forte: e minha vaidade,
tornando"-se a cada hora mais venenosa, à medida que percebia a fraqueza
de meus mestres, fez"-me reservar todo o tempo possível para estudar, à
minha própria maneira, durante o ano de 1835, tentando obter a
trissecção do ângulo. Durante algum tempo depois, tive a tendência a
reprovar"-me por ter desperdiçado tantas horas ao longo daquele ano, não
sabendo nem imaginando quantos anos vindouros, a partir de então, seriam
desperdiçados de maneira ainda pior.

Enquanto o curso de minha educação estava, diariamente, transformando"-me
num pequeno e obstinado arbusto, vários golpes de geada estavam
arrancando de mim as pequenas flores --- ou plantas --- da floresta, que
haviam crescido, para minha felicidade, ao meu lado.

\chapter{Parnaso e Plynlimmon} %Capítulo \textsc{v}. 

%96. 
Permiti, no último capítulo, que o registro de minhas realizações e
experiências em arte se estendesse até data muito posterior aos meus
primeiros anos, período que compreende os acontecimentos mais
importantes para mim, tanto os benéficos como os maléficos. Retomo o fio
narrativo desses acontecimentos sem a menor hesitação, porque, tais como
são, ninguém mais os poderia contar; ao passo que, nos últimos anos,
meus amigos, em alguns aspectos, conhecem"-me melhor do que eu mesmo.

A segunda década de minha vida foi ainda mais abruptamente secionada do
tempo perfeitamente feliz da infância pela morte de minha tia de
Croydon; morte pelo \textit{frio},\footnote{\textit{Cold}, no original; jogo de
  palavras, já que o vocábulo tanto pode significar \textit{frio} como
  \textit{resfriado}.} literalmente, apanhada numa operação
doméstica de lavagem pelo vento do leste. Seu \textit{spaniel}\footnote{Raça
  de cão de caça inglês com pelos longos e encaracolados e orelhas
  caídas, cujo nome deriva de \textit{spanish}.} marrom e
branco, Dash, permaneceu deitado ao seu lado, e sobre seu caixão, até
que fosse afastado dele; então foi trazido para Herne Hill, e creio ter
sido minha companhia por algum tempo antes da chegada de Mary.

Com a morte de minha tia de Croydon findaram"-se, para mim, os dias
passados junto às fontes de Wandel, como aqueles em Perth, às margens do
Tay; e então, quando tinha dez anos, uma vida restrita exclusivamente ao
topo de Herne Hill iniciou"-se (quando não estávamos viajando), com
características não muito benéficas.

%97. 
Minha tia de Croydon deixou quatro filhos --- John, William, George e
Charles; e duas filhas --- Margaret e Bridget. Todos rapazes e moças
bonitos; mas Margaret, em tenra idade, sofreu um acidente que lhe
deformou para sempre a coluna vertebral. Ela era inteligente e
espirituosa, como sua mãe; mas nunca me despertou interesse, embora eu
sentisse por todos meus primos de Croydon afeição mais de irmão que de
primo. Mas nunca gostei de inválidos, algo que permanece até hoje; e
Margaret costumava usar o cabelo em cachos, o que não suportava ver.

Bridget era uma criatura muito diferente; olhos negros, ou, mais
precisamente, cor de avelã escura, magra, muito vivaz; de feições
excessivamente aguçadas para ser bonita, um pouco magra demais para ser
graciosa; caprichosa, e de temperamento mais ou menos egoísta,
entretanto, suficientemente agradável para ser convidada a vir conosco a
Perth, uma ou duas vezes, ou para ficar um mês ou dois em Herne Hill;
mas nunca se ligando muito a nós, nem nós a ela. Eu a via como
inconveniente em meu quarto, que se transformara em lugar de trabalho à
medida que me tornara estudioso; e ela não tinha vontade, ou talvez não
tivesse permissão, de brincar comigo no jardim.

%98. 
Os quatro rapazes eram todos de bom caráter, e muito empreendedores.
O mais velho, John, com maior tendência para os negócios que os outros,
logo foi buscar fortuna na Austrália, e a obteve; o segundo, William,
prosperou também em Londres.

O terceiro irmão, George, era o melhor dos homens, mas tinha pouco
espírito. Parecia"-se muito a um George \textsc{iv} rústico, com um expansivo,
saudável e benevolente anseio por simplicidade na face, o que muito o
beneficiava como representante do caráter britânico. Ele foi trabalhar
num negócio em Market Street, junto com seu pai, e ambos eram uma grande
alegria para todos nós, com sua amabilidade e sinceridade; nenhum deles,
em toda a vida, jamais fez nada desonesto, indelicado, ou cometeu
qualquer outra falta --- mas também nada fizeram de inteligente! Por
enquanto, eu os deixo pela manhã, felizes, carregar a carroça com pães
de quatro libras, e entregá"-los pelas ruas em torno de Market Street.

%99. 
O quarto, e mais jovem irmão, Charles, era como o último rebento de
um conto de fadas, rosado como o garoto David, generoso de coração, sem
lhe faltar senso"-comum, ou mesmo \textit{bom} senso; e afetuoso, como
todos os outros. Conduziu"-se bem nos estudos, e tornou"-se apresentável,
expressando"-se com correção gramatical e polidamente, em nosso ilustre
círculo de Herne Hill. Seu irmão mais velho, John, ocupou"-se de sua
educação nos mais importantes aspectos: muito precocemente ele o
colocou, sem sela, sobre um pônei, com a simples e elementar instrução
de que apanharia se caísse. E ele não caiu. Do mesmo modo, como primeira
lição de natação, ele atirou o garoto, como se simples seixo, ao meio do
Canal Croydon, lançando"-se à água, em seguida, naturalmente; mas creio
que o garoto conseguiu chegar à margem sem ajuda, e tornou"-se, quando
adulto, mestre destemido dos cavalos e das águas.

%100. 
Minha mãe costumava contar"-nos essas duas histórias com grande
satisfação, porque, na educação de seu próprio filho, sacrificara seu
orgulho pelo heroísmo ao seu desejo por segurança; e nunca me permitiu
ir à margem da lagoa ou ficar, no campo, junto a um pônei. A má"-sorte
quis que não houvesse fazenda ou pântano por perto, o que fez com que
essas restrições fossem alteradas; mas eu já percebera, com gratidão, o
prazer que tivera com o fosso habitado por girinos, em Croxted Lane; e
também, que entre onde vivíamos e a Walworth tutorial, existia, para
mim, um campo elísio, no gramado abandonado de Camberwell Green.
Havia uma lagoa num de seus ângulos, de tamanho considerável, e
profundidade desconhecida, --- provavelmente, mesmo no verão, cerca de
três pés no centro; a opacidade sombria de suas águas acrescentando"-se
ao mistério do perigo. Grande, como disse, para uma lagoa, talvez
sessenta ou setenta jardas de comprimento ao longo de Camberwell Green,
cinquenta jardas no menor comprimento; na margem ocidental crescia
majestosamente um olmo, de cujos ramos, contava"-se, e escrupulosamente
acreditava"-se, um garoto mau havia caído na lagoa no domingo, e,
imediatamente, sua alma, num poço mais escuro e profundo.

Um dos maiores privilégios de minha infância foi a permissão de minha
babá para que eu contemplasse, com temor, a impassível lagoa, a partir
do outro lado do caminho. O desaparecimento da lagoa, pela conversão
sanitária de Camberwell Green em buquê para a lapela de Camberwell, é
até hoje, para mim, objeto de eterna lamentação.

%101. 
As leis precautórias descritas acima, naturalmente não me
permitiam, durante minhas visitas a Croydon, sair com meus primos, a fim
de que eles não me induzissem à maldade; e nenhuma alegria intrépida foi
mais possível para mim em Croydon além de minhas caminhadas com Anne ou
com minha mãe por onde o canal da lagoa Scarborough cruzava a estrada;
ou pelo gramado viçoso de Duppas Hill; ou da observação de meu pai
desenhando com tinta nanquim, e minha própria incansável contemplação da
bomba de água e de seu gotejamento do outro lado da, assim chamada, rua,
mas, na verdade, viela, --- não mais que doze pés de parede a parede. De
modo que, quando finalmente se pensou que Charles, com todos seus dons e
talentos naturais, deveria ser trazido de Croydon para Londres, e
iniciado no elevado estilo de vida e no trabalho das ordens burguesas; e
quando, previsivelmente, após receber muitas recomendações e fazer
muitas perguntas, foi aceito como aprendiz na Messrs.\,Smith, Elder, \&
Co., no número 65 de Cornhill, com o grande privilégio de vir jantar em
Herne Hill todo domingo, a nova e brilhante presença de Charles
tornou"-se, para mim, vívida excitação, e admirável revelação das
atividades da juventude, e comecei a ficar realmente ligado a ele.

Eu não era o tipo de criatura pela qual um rapaz se interessaria muito,
--- ou, na verdade, qualquer ser humano, exceto papai e mamãe, e Mrs.\,Richard Gray (mais recentemente); sendo eu realmente nada mais que um
presunçoso, inoportuno e enfadonho pequeno macaco. Mas Charles era
sempre amável comigo, e naturalmente respondia com afeição de primo ou
mesmo de irmão minha admiração por ele, que a deleitava.

%102. 
Na Messrs.\,Smith \& Elder's, ele era um aprendiz reconhecidamente \label{102}
exemplar, tornando"-se rapidamente um proveitoso vendedor, executando as
ordens diligentemente, e conhecendo bem os livros e os clientes. Como
todos os aprendizes com boa disposição de espírito e os bons vendedores
fazem, orgulhava"-se dos produtos comercializados pela firma; e, aos
domingos, sempre trazia um ou dois volumes no bolso para nos mostrar as
características das publicações mais ambiciosas; escolhendo
particularmente, para me agradar, aquelas que poderiam conter boas
gravuras. Assim, familiarizei"-me com Stanfield\footnote{Certamente
  Clarkson Frederick Stanfield (1793--1867), pintor inglês de marinhas.} e Harding\footnote{Certamente James Duffield Harding
  (1798--1863), pintor inglês de paisagens.} muito antes
de possuir uma simples gravura de qualquer um deles; mas, realmente, o
mais precioso e contínuo efeito profundo sobre mim, de todos os
presentes recebidos na infância, foi o de minha tia de Croydon, o
\textit{Lembre"-se de mim}, de 1827, com uma bela gravura do \textit{Sepulchral
monument at Verona}, de Prout.

Estranho que o verdadeiro primeiro estímulo aos mais refinados instintos
de minha mente tenha sido fornecido pela totalmente sem educação formal,
mas completamente boa e bem intencionada irmã de minha mãe.

%103. 
Mas resultados mais magníficos vieram das conexões literárias de
Charles, através do interesse despertado em todos nós pelo pequeno
\textit{in"-octavo} com trabalhos em revelo dourado que Smith \& Elder
publicava anualmente, com o título de \textit{Friendship's Offering.} O
livro era editado por um pio missionário escocês, e poeta menor --- muito
menor --- Thomas Pringle; mencionado honrosamente uma ou duas vezes na
\textit{Life of Scott} de Lockhart.\footnote{Certamente John Gibson
  Lockhart (1794--1854), escritor e editor escocês, cuja obra mais
  conhecida é justamente a biografia de Sir Walter Scott, \textit{Life
  of Scott}.} Um homem estritamente honesto e
consciencioso, rigorosamente treinado, embora de cultura estreita,
dotado de todo o orgulho escocês, incansável viajante, e com a coragem
insolente dos Parks e dos Livingstones; com alguns belos toques de
romance e tintas de filosofia para suavizá"-lo, era membro reconhecido,
embora pouco considerado, de nossos melhores círculos literários, e
conhecera, no processo de distribuição de seu pequeno \textit{in"-octavo}
em relevo, todo o mundo dos círculos externos, e num nível mais abaixo, a
mim. Havia sido patrocinado por Scott; mantinha polida correspondência
com Wordsworth\footnote{William Wordsworth (1770--1850), um dos mais
  importantes poetas do romantismo inglês.} e Rogers;
relações familiares com o pastor Ettrick; e escrevera, ele próprio, um
livro de poemas sobre a África, no qual antílopes eram chamados de
gazelas sul"-africanas, e outros costumes e modos de vida africanos
cuidadosamente examinados.

%104. 
Em parte para agradar a Charles, o vendedor vivaz e de bom
temperamento, que dizia coisas maravilhosas de seu pequeno primo
estudante; --- em parte em busca de delicadas composições de maleável
ornamento, com as quais preencher interstícios na arquitetura do
\textit{Friendship's Offering}, Mr.\,Pringle visitava"-nos em Herne Hill,
ouvia sobre minha vida literária, expressava algum interesse por seu
progresso, --- e algumas vezes levava uma cópia de meus versos no bolso.
Ele foi a primeira pessoa que confidenciou a meu pai e a minha mãe, com
alguma segurança, que ainda não havia indicações seguras de que um dia
eu pudesse ocupar na literatura inglesa lugar mais elevado que os de
Milton ou Byron; adequadamente, nenhum de nós atribuiu muita
importância à sua opinião. Mas ele teve a sabedoria de reconhecer, a
despeito da vaidade de meus pais, as aptidões naturais de meu pai, e sua
refinada sensibilidade romântica; não menor que a sinceridade de minha
mãe em relação à fé evangélica, que ele próprio decidira professar: e
tornou"-se um honrado, embora jamais muito bem"-vindo convidado aos
jantares formais de domingo; e, a partir de então, conselheiro informal
de minha educação. Seu interesse por meu profundo amor à natureza e por
minha predisposição à rima, levou"-o a ler e criticar cuidadosamente
alguns de meus versos; e, afinal, como sagradas eleusínias e
peregrinações délficas, conduziu"-me pela mão a uma visita que tinha a
pagar ao poeta Rogers.

%105. 
O velho poeta, previamente avisado de minha aceitável pretensão, na
opinião de Mr.\,Pringle, ao caráter sagrado de tal apresentação, foi
razoavelmente amável comigo, embora o cultivo de gênios em germinação
jamais tenha sido considerado por Mr.\,Rogers uma atividade inteiramente
deleitável para o gênio em seu zênite. Além disso, fui infeliz nos
comentários pelos quais, em resposta às suas observações, pretendi
mostrar"-me à sua altura. Congratulei"-lhe entusiasticamente pela beleza
das gravuras que ilustravam seus poemas, --- mas revelei, creio, ao mesmo
tempo, ausência de interesse equivalente pela composição dos poemas. Em
todo caso, Mr.\,Pringle --- algo abruptamente, pensei no momento --- desviou
a conversa para assuntos relativos à África. Esses temas, sem dúvida,
eram calculados para interessar o refinado menestrel de St.\,James Place;
mas, outra vez, fui inepto, ao permitir que minha atenção, francamente
revelada por meus olhos divagadores, fosse atraída pelas pinturas
brilhantes nas paredes de seda carmesim; consequentemente, após termos
saído, Mr.\,Pringle aproveitou para me advertir que, no futuro, quando em
companhia de homens eminentes, eu deveria ouvi"-los mais atentamente.

%106. 
Essas, e algumas outras --- já descrevi a lisonjeira visita que o
pastor Ettrick nos fizera, também obtida por Mr.\,Pringle --- glórias e
progressos foram as recompensas pelos meus esforços literários;
entretanto, elas não me desviaram dos estudos científicos que eram
verdadeiro e natural deleite para mim. Registrei acima seu começo nas
caminhadas minerais em Matlock: mas os negócios de meu pai levavam"-me
frequentemente a Bristol, onde ele alojava minha mãe, com Mary e eu, em
Clifton. A história de Miss Edgeworth,\footnote{Maria Edgeworth
  (1767--1849), novelista anglo"-irlandesa.} \textit{Lazy
Lawrence},\footnote{Título de um conto de Miss Edgeworth.}
e a visita de Harry e Lucy a Matlock, deram um quase romântico e
visionário encanto à mineralogia naqueles pequenos vales; e o pedaço de
óxido de ferro com diamantes brilhantes de Bristol, --- número 51 da coleção
de Brantwood --- foi, creio, a primeira pedra na qual iniciei meus
estudos de sílica. O brilho dos diamantes evocava muitas lembranças,
porque de Clifton quase sempre cruzávamos para Chepstow, --- o
arrebatamento de estar flutuando, por uma hora e meia, naquele mar
turvo, concentrava em intensos minutos os prazeres que outros rapazes
tinham em um ano de passeios de barco, --- e, perto de Tintern e Malvern,
onde as colinas eram extremamente deleitáveis porque me permitiam correr
livremente por elas, não havendo precipícios ou fluxos d'água nos quais
cair, e também clássicas, para mim, através do \textit{Henry
Milner}\footnote{\textit{The History of Henry Milner}: título de um dos
  mais famosos livros de Mary Martha Sherwood.} de Mrs.\,Sherwood;\footnote{Mary Martha Sherwood (1775--1851), prolífica escritora
  inglesa de histórias infantis.} um livro que amava e
respeitava há muito tempo. Havia isso de curioso e valioso em minha
educação naqueles anos, que meu espírito aventureiro era sempre
ratificado pela chancela da localidade --- e que o encanto de cada
localidade era espiritualizado pelo ardor e pela paixão pela aventura.
%Parei a revisão aqui. (Suzana)

%107. 
Havia uma região, entretanto, aquela dos lagos de Cumberland, que
não necessitava de encantos por associação para intensificar os apelos
de sua realidade. Disse em algum lugar que minha primeira lembrança na
vida foi a de Friar's Crag em Derwentwater; --- significando, creio,
minha primeira lembrança de coisas que depois foram de grande
importância para mim; de qualquer modo, conheci Keswick antes de
conhecer Perth, e depois que os dias em Perth haviam terminado, minha
mãe e eu ficamos lá, em Royal Oak, ou na Lowwood Inn, ou em Coniston
Waterhead, enquanto meu pai viajava a negócios por Whitehaven,
Lancaster, Newcastle, e outras cidades do norte. A hospedaria em
Coniston situava"-se, então, na extremidade superior do lago, a estrada
de Ambleside ao povoado passando exatamente entre ela e a água; e a
visão da extensão maior do lago, com suas colinas laterais
agradavelmente florestadas, tinha para meu pai um encanto enternecedor
que despertava os mesmos sentimentos com os quais ele, posteriormente,
observou os lagos da Itália. Lowwood Inn era também, então, pouco mais
que uma casa de campo, --- e Ambleside, um povoado rural; e a paz e
felicidades absolutas que qualquer um, para quem colinas gramadas e
águas melodiosas fossem importantes, poderia encontrar a cada passo, e a
cada contorno de rochedo ou curva de baía, era totalmente diferente de
tudo que eu já havia visto, ou lido.

%108. 
A primeira visão impressionante de uma paisagem foi no País de
Gales; e já escrevi, --- mais do que seria sensato imprimir, --- sobre a
viagem de carro de Hereford a Rhaiadyr, e por Plynlimmon até
Pont"-y"-Monach: a alegria de uma caminhada com meu pai na tarde de
domingo em direção a Hafod, perturbado apenas por certo alarmado senso
de pecado por estar sendo tão feliz entre as colinas, em vez de estar em
casa escrevendo um sermão: --- a presença e a aprovação de meu pai não me
confortando inteiramente, pois ambos tínhamos reprimida consciência de
estarmos sendo profanos e rebeldes, comparados à minha mãe.

De Pont"-y"-Monach fomos para o norte, apanhando seixos na praia de
Aberystwith, e subindo o Cader Idris com a ajuda de pôneis: que para mim
permaneceu, por muitos anos, justificadamente, como a rainha das
montanhas. Seguiu"-se Harlech e suas areias, Festioniog, o passo de
Aberglaslyn, e a maravilha dos estreitos de Menai e a ponte, que então
vi, como Miss Edgeworth me havia ensinado, com reverência pela
habilidade mecânica do homem, --- visão singela, de um pobre inocente
incapaz de imaginar o uso que os homens fariam dessa habilidade no meio
século seguinte.

A \textit{ponte} do Menai era, lembre"-se, bom leitor, não um \textit{túnel}
\textit{subterrâneo}, --- mas uma passagem constituída de pranchas bem
aplanadas que oscilavam levemente de um pilar a outro por entre o
entrelaçado de ferro.

%109. 
E, a seguir, a Llanberis e ao Snowdon, de cuja ascensão, lembro
como acontecimento mais excitante o encontro, pela primeira vez na vida,
de um mineral ``real'' para mim, um pedaço de pirita de cobre! Mas a
impressão geral da forma de montanha galesa foi tão verdadeira e nítida
que viagens posteriores não a modificaram ou a aprofundaram.

E se então meu pai e minha mãe tivessem visto as reais forças e
fraquezas de seu pequeno John; --- se tivessem me dado apenas um
pedacinho de um pônei galês felpudo, e me deixado a cargo de um bom guia
galês, e de sua esposa, caso eu precisasse de cuidados, eles teriam
feito de mim, naquele momento, um homem, e posteriormente, para o
conforto de seus próprios corações, provavelmente o primeiro geólogo da
Europa, em meu tempo.

Se! Mas o que eles poderiam fazer melhor que isso seria lançar"-me, como
ao meu primo Charles, ao Canal de Croydon, esperando que eu conseguisse
encontrar, pelas leis da natureza, uma escapatória.

%110. 
Em vez disso, levaram"-me de volta a Londres, e meu pai utilizou
tempo que seria destinado ao trabalho, duas ou três vezes por semana,
para me conduzir à prisão de uma escola de equitação em Moorfields; um
edifício quadrado, iluminado apenas pela luz solar, com o chão
empoeirado, cujo cheiro, ao entrar pelo portão, era terror e horror e
abominação para mim: e ali fui colocado sobre grandes cavalos que
pulavam, e empinavam, e giravam, e moviam"-se lateralmente; eu caía
sempre que eles faziam qualquer um desses movimentos; e isso era uma
desonra para minha família, e uma terrível vergonha e sofrimento para
mim, até que, finalmente, a escola de equitação foi abandonada devido à
torção de meu dedo indicador direito (que nunca mais voltou ao normal),
--- e um bem"-adestrado pônei das Shetland foi comprado para mim, e
partimos pelas estradas de Norwood, conduzidos pela corda"-guia de um
mestre de equitação. Eu me saía muito bem enquanto cavalgávamos em linha
reta, começava a pensar em outra coisa, e caía quando fazíamos uma
curva. Poderia ter me saído maravilhosamente bem, não fossem os
mexericos a meu respeito e as constantes perguntas sobre se eu havia
caído ou não; mas como minha mãe, no momento em que chegava em casa,
fazia investigação minuciosa sobre os infortúnios do dia, tornei"-me mais
nervoso e desamparado a cada queda; e este aspecto de minha educação foi
enfim abandonado, meus pais consolando"-se, da melhor maneira possível,
com a conclusão de que o fato de eu não conseguir aprender a cavalgar
era um sinal de minha singular genialidade.

%111. 
O resto do ano foi gasto em ocupações domésticas tais como as acima
descritas; --- mas tanto neste como no ano precedente, meu interesse pela
mineralogia recebeu um novo e importante estímulo de um amigo que
posteriormente se tornara íntimo da família, mas de quem ainda não
falei.

Minha doença em Dunkeld, acima mencionada, foi tratada por dois médicos,
--- minha mãe, --- e Dr.\,Grant. O doutor deve ter sido um jovem
recém"-formado. Desconheço a origem de suas relações com meus pais; mas
sei que meu pai exercia uma influência quase paterna sobre ele; e que
lhe seria útil, até que ponto, não sei, mas certamente atuava de modo
contínuo e eficaz, no início de sua vida profissional. E, à medida que
cresci, ouvia, de meus pais, frequentes expressões de muita afeição e
respeito por Dr.\,Grant, acompanhadas de outras de lamento ou censura por
ele não ter se desenvolvido nem utilizado suas habilidades a contento.

Depois da doença de Dunkled, o nome do Dr.\,Grant sempre esteve
associado, em minha mente, a um pó castanho --- ruibarbo, ou algo
parecido --- de natureza granulosa e acre, o qual, por ordem dele, eu
tinha que tomar. O nome do remédio, desde então, soa para mim como
cr"-r"-a"-que\footnote{\textit{Gr"-r"-ish}, no original. Onomatopeia para o som
  de grãos sendo triturados; guarda semelhança fônica com a palavra
  inglesa \textit{grist}, que significa ``trigo a ser moído, trigo moído''.} e granular; e certo receio, que não chegava a aversão
--- mas, ao contrário, levava à afeição, (para \textit{mim}) --- fazia a
presença do doutor algo solene; mais ainda porque nunca gracejava, e
tinha um tipo de rosto pardacento, parcialmente austero, seco, vincado,
na verdade, ruibárbico. De resto, era um homem inteiramente amável e
consciencioso, muito afetuoso com meu pai, reconhecido como uma espécie
de tutor em relação a mim, e como tendo a responsabilidade de
conselheiro médico, funções influenciadas tanto por sua imaginação como
por sua personalidade.

%112. 
Conjeturo que se deve ao fato do Dr.\,Grant ser de boa família, e,
em todos os sentidos e acepções do termo, um gentil"-homem, que, logo
após sua chegada a Londres, tenha obtido o posto de cirurgião em uma das
fragatas de Sua Majestade, a ser enviada em cruzeiro pela costa
ocidental da América do Sul. Felizmente a saúde de seus companheiros
deu"-lhe pouco a fazer, profissionalmente; e o doutor pôde dedicar a
maior parte do tempo a estudar a história natural das costas do Chile e
do Peru. Um dos resultados dessas expedições a terra foi a descoberta de
um lucano como nunca tinha sido visto antes. O besouro tinha pinças
peculiares ou colossais, e --- esqueci o que ``chiasos'' significa em
grego, mas suas mandíbulas eram \textit{chiasoi}.\footnote{Palavra
  relacionada ao grego \textit{chíasma:} disposição em forma da letra
  grega \textit{qui}; cruzamento em forma de \textit{x}. Do termo grego deriva a palavra
  portuguesa ``quiasma'', utilizada em linguística para designar
  construção anômala que se origina de construções regulares, cruzamento
  ou contaminação sintática; em anatomia, para nomear cruzamento ou
  decussação de duas formações (quiasma óptico, por exemplo); em
  genética, para qualificar uma configuração específica que os
  cromossomos homólogos assumem durante a divisão celular; em
  diplomacia, para nominar a cruz em forma de \textsc{x} aposta à margem dos
  documentos para indicar a não aprovação de algum trecho.} Trouxeram"-no para casa cuidadosamente acomodado numa caixa de
algodão; e, quando a caixa foi aberta, despertou a admiração de todos
que o viram, e foi chamado de ``Chiasognathos Grantii''. Outro resultado
foi a coleção completa de beija"-flores de Valparaíso, da qual ele
separou, para presentear minha mãe, tantos pássaros, que podiam encher
com adejos púrpura e dourados vitrines tão grandes quanto as doadas por
Mr.\,Gould ao Museu Britânico, e se tornaram resplandecentes decorações na
sala de desenho de Herne Hill, --- constituindo"-se para mim, à medida que
crescia, padrões definitivos de textura de plumagem e de cor, --- e que se
encontram agora expostos no recesso mais bem iluminado da escola
paroquial de Coniston.

%113. 
O terceiro resultado foi ainda mais importante. Dr.\,Grant foi
presenteado pelos mestres de minas espanhóis com os mais belos e
característicos espécimes de pedras provenientes dos filões de Copiapo.
Foi um fato determinante para mim, no auge de meu interesse infantil por
minerais, ver a mesa de nosso salão repleta de prata folheada e ouro
arborescente. Não apenas o homem de ciência, mas também o avaro latente
em mim foi vivamente despertado em uma hora ou duas! Nos fragmentos que
Dr.\,Grant me deu, contei meu tesouro grão a grão; recordo hoje, com
grande simpatia, a indignação sentida quando não vi imediata mudança na
expressão do primo Charles, em reverência, quando lhe informei que a
película na superfície de um espécime despretensioso, que correspondia à
décima"-sexta parte de seis \textit{pences}, era de ``prata nativa''!

Logo após o retorno desta próspera viagem, Dr.\,Grant estabeleceu"-se numa
casa respeitável a meio"-caminho da descida de Richmond Hill, onde,
gradativamente, obteve clientela e posição social entre a pequena
nobreza\footnote{\textit{Gentry}, no original.} da cidade e
da vizinhança coberta por parques. E regularmente, nas manhãs de
domingo, ou naquelas alegres, geladas e brancas do inverno,
costumávamos, papai e mamãe, Mary e eu, irmos de carro até Clapham e
Wandsworth para desjejuar com o Dr.\,Grant na hospedaria \textit{Star and
Garter}. Os desjejuns deixaram forte impressão em mim, parcialmente
devido à bela vista das janelas; mas, mais ainda, porque enquanto meu
desjejum, mesmo quando em viagem, normalmente era de pão velho da
padaria, naqueles piqueniques estrelados, era"-me oferecido pão francês
fresco.

%114. 
Deixando, momentaneamente, Dr.\,Grant em tais agradáveis e cada vez
mais dignas circunstâncias, devo retornar aos amigos, os quais,
executando"-se os familiares, exerceram a mais poderosa influência sobre
minha vida infantil, --- Mr. e Mrs.\,Richard Gary.

Quando meu pai era apenas um vendedor, passou período considerável de
tempo na Espanha, aprendendo a conhecer o \textit{sherry}, e as maneiras
de prepará"-lo e estocá"-lo, em Jerez, Cádiz e Lisboa. Em Lisboa,
tornou"-se íntimo de um jovem escocês de sua idade, também vendedor em
alguma casa comercial espanhola, creio, mas sem a mentalidade estreita
desta profissão. Ao contrário, Richard Gray sobrepujava meu pai em
sentimento romântico, e em interesse erudito pela boa literatura, que
tão estranhamente entremeava os hábitos de trabalho de meu pai.
Igualmente enérgico, diligente, e de sólidos princípios, o entusiasmo de
Mr.\,Gray era, entretanto, irregular, e muito frequentemente errático,
coruscante; estando para o entusiasmo de meu pai, como disse Carlyle do
fogo francês contra o inglês em Dettingen: ``fogo de palha contra fogo
de carvão''. Contudo, não me atreveria absolutamente a afirmar que, sob
a influência errática e efervescente de Richard, uma excursão a Sintra,
ou a assistência a uma festa numa vila, ou mesmo a uma corrida de
touros, não poderiam, algumas vezes, de certo modo, invalidar minha
afirmação anterior de que, durante nove anos, meu pai não teve férias.
Em todo caso, os jovens tornaram"-se amigos íntimos e afetuosos; e a
amizade exerceu um efeito suavizante, animador e, de modo geral,
benéfico sobre o caráter de meu pai. Esta amizade fraterna não sofreu
qualquer mácula ou jaça, quando, pouco antes de deixar a Espanha, Mr.\,Gray casou"-se com uma moça escocesa muito bonita e de excelente caráter,
Mary Monro.

%115. 
De excelente caráter e de maneiras as mais amáveis; --- inteiramente
simples, meiga, afetuosa, e séria; não suficientemente inteligente a
ponto de se rebelar, mas salva da estupidez pela natureza vivaz, plena
de entusiasmo como seu marido. Ambos evangelicamente pios, de uma
maneira vívida, não rancorosa; e cada um sagradamente, mas não menos
passionalmente, apaixonados um pelo outro, formando o casal mais
perfeito que já vi neste mundo regido pela Providência em que cada um
busca seu par. Contudo, como quis o destino, eles tinham como único
desgosto o fato de não terem filhos, o que fez com que, nos anos
seguintes, a principal ocupação da vida de Mrs.\,Grant fosse
mimar"-me. Quando eu já estava suficientemente grande para ser
mimado, Mr.\,Gray, tendo obtido sucesso nos negócios, e vindo para
Londres, estabeleceu"-se, com sua esposa, sua mãe, e o poodle francês
branco desta, Petite, numa casa respeitável em Camberwell Grove. Uma
família inteiramente feliz; a velha Mrs.\,Monro tão amável quanto a filha,
talvez um pouco mais inteligente; Richard regozijando"-se de todo o
coração com a companhia de ambas; e Petite, tendo talvez tanto senso
quanto qualquer dois deles juntos, em deleite, e amada por todos os
três.

%116. 
A casa ficava perto do topo da colina Grove, --- que era um
verdadeiro bosque\footnote{\textit{Grove}, no original, que significa
  ``bosque, arvoredo'', em inglês; jogo de palavras com o nome da
  colina: Camberwell Grove.} naqueles dias, e grande,
cerca de três"-quartos de milha de comprimento, descendo abruptamente
a encosta, --- belo em perspectiva como uma nave de igreja com uma
extensão sem precedentes; árvores, olmos, olmos escoceses, sicômoros e
choupos, os ramos entrelaçando"-se no alto; de cada lado, caminhos de
pedras bem"-ordenadas através de pequenos gramados bem aparados levavam
às casas; com três ou quatro pavimentos, a maioria dispostas em
terraços, --- bem"-construídas, com tijolos de cor sóbria, com telhados de
ardósia altos e abruptos --- não pontiagudos, mas poligonais; todas
ricas, bem cuidadas, bem varridas, dignas e agradavelmente vulgares,
constituindo, dentro de seu próprio bosque, um mundo à parte. De Herne
Hill a Grove, era uma agradável caminhada de cerca de duas milhas; e
sempre que Mrs.\,Gray e minha mãe tinham algo a dizer uma a outra, elas
caminhavam --- colina abaixo ou acima --- para dizê"-lo; e a casa de Mr.\,Gray era como se fosse a nossa, a qualquer hora do dia ou da noite. Mas
nossa casa não era o mesmo para os Grays, mantendo suas formalidades
invioláveis; assim, durante toda a infância, tive o sentimento de
estarmos, de uma maneira ou outra, acima de nossos amigos e de nossas
relações, --- mais ou menos patrocinando a todos, favorecendo"-os com
nossos conselhos, instruindo"-os com nossos exemplos, e obrigados, tanto
pelo que éramos como pela constituição da sociedade, a mantê"-los a certa
distância.

%117. 
Com uma exceção; da qual lembro com muito prazer. No primeiro
capítulo de \textit{The Antiguary}, o proprietário da pousada Queen's
Ferry envia a seu estimado hóspede uma garrafa do melhor Porto produzido
por Robert Cockburn; com o qual Robert Cockburn supria o próprio Sir
Walter, sendo ele, àquela época, se não o maior, o mais importante
importador do mais fino vinho de Portugal, como meu pai dos vinhos da
Espanha. Mas Mr.\,Cockburn era essencialmente um velho gentil"-homem de
Edimburgo, e apenas por condescendência um comerciante de vinhos; um
homem de grande inteligência e agradável espírito sarcástico, que se
movia nos círculos mais elevados de Edimburgo; ligado a meu pai por
conexões relativas à ``velha cidade'',\footnote{``Auld toun'', no original:
  dialeto escocês.} e o respeitando sinceramente. Sem
dúvida, a personalidade mais imponente e original que frequentava as
festas dos comerciantes de vinhos.

Mrs.\,Cockburn era ainda um pouco mais altiva, --- uma senhora escocesa da
velha escola, --- condescendente, entretanto, com a nova geração. Fora o
primeiro amor dentre os primeiros amores de Lord Byron; ela era a Mary
Duff of Lachin"-y"-Gair.\footnote{Mary Duff: prima distante de Lord Byron,
  pela qual Byron se apaixonara na infância; posteriormente Mrs.\,Cockburn; citada como ``Mary'' no poema \textit{Lachin"-y"-Gair}, da
  autoria de Byron.} Ao lembrar"-me dela, ocorre"-me, em
primeiro lugar, sua grande beleza na meia"-idade, acompanhada de muito
bom senso, que, embora mesclado com algum traço de orgulhosa severidade,
resultava num caráter extremamente amável.

%118. 
Eles tinham dois filhos, Alexander e Archibald, ambos trabalhando
com o pai, ambos talentosos e diligentes, mas os dois categoricamente
decididos --- como era o desejo de seus pais --- a serem, em primeiro
lugar, gentis"-homens; depois, comerciantes: um tipo de personalidade a
ser honrada e preservada, entre nós; no caso deles, isso não era
ambicioso ou afetado: gentis"-homens eles eram, --- nascidos assim,
sentiam"-se mais à vontade no campo que numa casa comercial; sendo,
contudo, dedicados aos negócios. A casa comercial, entretanto, não se
desenvolveu como poderia se estivesse em mãos menos bem"-nascidas.

Os dois filhos, um ou outro, sempre jantavam conosco, e eram amigos mais
eminentes que a maioria de nossos convidados. Alexander tinha muito do
humor paterno; Archibald, um jovem bom, moreno, das Terras Altas da
Escócia, era"-me extremante agradável, e, em razão de meu amor por Scott,
deu"-se ao trabalho de me contar tudo que eu desejava ouvir sobre pesca
ou caça de veados. Pois, desde os mais ternos dias, eu me interessava
pelas aventuras de outras pessoas, embora nunca pretendesse vivê"-las
pessoalmente. Li todos os romances do Capitão Marryat,\footnote{Captain
  Frederick Marryat (1792--1848): novelista inglês, um dos pioneiros na
  escritura de histórias sobre a vida marítima.} sem
nunca querer ir para o mar; atravessei os campos de Waterloo sem a menor
inclinação para ser soldado; parti numa pescaria imaginária com Izaak
Walton sem nunca lançar um anzol; e conhecia \textit{Deerslayer} e
\textit{Pathfinder} de Cooper\footnote{James Fenimore Cooper (1789--1851):
  escritor norte"-americano prolífico e muito popular; sua obra mais
  conhecida é \textit{O Último dos Moicanos}.} quase de
cor, apenas com um revólver de brinquedo e sem ter caminhos para
encontrar para além das solidões de Gipsy"-Hill. Costumava contar"-me
histórias de campanhas nas quais eu era um brilhante general, ou sobre
cavernas nas quais descobria filões de ouro; mas eram apenas para
preencher vazios da imaginação, e não tinham relação com coisas reais ou
factíveis. Eu já não gostava da ideia de crescer, --- nunca esperei ser
mais sábio, e não fiz outros planos para o futuro além daqueles que o
pequeno bicho"-de"-seda escuro faz no centro de sua primeira folha de
amora.

\chapter{Schaffausen e Milão} %Capítulo \textsc{vi}.

%119. 
A visita ao campo de Waterloo, incidentalmente mencionada no
capítulo anterior, deve ter acontecido quando eu tinha cinco anos, ---
ocasião em que papai e mamãe fizeram a extravagância de ir a Paris ver
as festividades pela coroação de Charles \textsc{x}. Ficamos sete semanas em
Paris, numa tranquila hospedaria familiar, e depois alguns dias em
Bruxelas, --- mas não me lembro das etapas intermediárias. Parece"-me, ao
rever aqueles tempos matinais de minha vida, que recebia muito
lentamente as impressões, e que necessitava parar pelo menos dois ou
três dias num lugar, para que pudesse realmente percebê"-lo; mas a
percepção, uma vez ocorrida, permanecia, pelo tempo que durasse, sempre
correta; e desde que não tivesse, posteriormente, oportunidade de
modificá"-la, outras impressões se dissipavam em contato com a principal,
e desapareciam juntas. Então, o que as pessoas chamam de minha visão
preconceituosa das coisas, --- é, na verdade, exatamente o contrário,
nomeadamente, uma visão pós"-conceituosa.

%120. 
Outra característica de minhas percepções, curiosamente constante
--- era meu interesse apenas por coisas que estivessem à minha volta, ou
pelo menos claramente visíveis e presentes. Creio que geralmente é o que
acontece com as crianças; mas permaneceu --- e permanece --- como parte de
minha personalidade adulta. Nesta visita a Paris, minha atenção foi
atraída pelas macias almofadas vermelhas das poltronas, que demoravam
meia"-hora para ceder e acomodar o peso do corpo, --- pelo assoalho
perfeitamente polido do salão, e pelos bem"-dispostos lustradores
franceses que pela manhã deslizavam sobre ele até que se tornasse
reflexivo como uma mesa de mogno, --- pelo belo pátio repleto de flores e
plantas em canteiros e vasos entre as janelas de nosso rés"-do"-chão e o
portão externo, --- pelo amável criado negro que acompanhava outra
família, e que trazia para mim o gato da casa; e pela também amável
criada de quarto que devolvia o gato por medo que eu o irritasse, (sua
experiência com crianças inglesas fazia"-a duvidar de minhas intenções);
de todas essas coisas e pessoas lembro"-me, --- e dos jardins das
Tulherias, e dos jardins do ``Tivoli'', onde meu pai levou"-me à
``montanha russa'', e vi os mais belos fogos de artifício. Mas nada
recordo do Sena, nem da Notre Dame, nem de nada dentro ou fora da
cidade, exceto dos moinhos em Montmartre.

%121. 
Similarmente em Bruxelas. Não me lembro do Hôtel de Ville, de ruas
imponentes, de surpresas, ou interesses, apenas de nossa viagem de carro
a Waterloo e da lenta caminhada pelo campo. O monumento desfigurador da
paisagem ainda não fora construído --- a batalha ocorrera há apenas nove
anos; e cada encosta e depressão do terreno era ainda testemunha
verdadeira dos avanços e recuos. Fixada em minha mente por leituras
posteriores, esta visão do campo de batalha permanece completamente
clara, enquanto a resultante de uma visita após a construção do
monumento, praticamente desapareceu.

Devo também observar que o arrebatamento por estar a bordo de um vapor,
relatado em minha última carta, ocorreu em data posterior; como criança,
interessava"-me mais pela praia onde as ondas quebravam, ou pela areia
onde eu pudesse cavar, que pelo mar aberto. Não houve para mim ``a
primeira visão do mar''. Eu fora à Escócia no cúter \textit{Captain
Spinks}, então um barco regular de passageiros, quando tinha apenas três
anos; mas o tempo estava bom e, exceto pelo prazer de me tatuar com
piche por entre as cordas, foi como se estivesse em terra; mas minha
percepção do oceano, como agitador da Terra, ampliou"-se pelo chocalhar
das ondas na praia, e pelo ciciar da areia.

%122. 
Pretendi também, neste capítulo, dizer algumas palavras sobre outro
parente pobre, Nanny Clowsley, uma velha sempre alegre que vivia com um
relógio holandês e algumas velhas xícaras de chá numa peça única (com
uma pequena cama na alcova) no terceiro andar de uma casa de cumeeira,
parte de um grupo de velhas habitações à margem do Chelsea, próximas à
ponte Battersea, posteriormente demolidas. Mas é melhor que eu apresente
num único momento tudo que tenho a dizer sobre o Chelsea; apenas, sobre
seixos, devo mencionar a visão da janela de Nanny Clowsley, diretamente
para a maré do Tâmisa, dos barcos balançando na preamar e das barcaças
encalhadas na baixa"-mar.

E agora, devo continuar, e relatar como realmente vi pela primeira vez
várias coisas.

%123. 
Disse que, para nossas excursões pela Inglaterra, Mr.\,Telford
geralmente nos emprestava sua carruagem. Mas para a Suíça, agora levando
Mary, precisávamos de rodas mais fortes e mais espaço; e para tanto, e
para todas as outras excursões ao exterior, a primeira preparação e o
começo do deleite era a escolha de uma carruagem de nosso agrado, da
reserva de veículos alugáveis de Mr.\,Hopkinson, em Long Acre.

Os pobres escravos modernos e tolos que se deixam ser arrastados como
gado, ou madeira caída, pelos países que imaginam estar visitando, não
têm ideia das alegrias complexas, e das esperanças sinceras,
relacionadas à escolha e à preparação das carruagens de viagem, nos
velhos tempos. Em primeiro lugar, as questões mecânicas, da potência ---
da suavidade do movimento --- do equilíbrio e da segurança de pessoas e
bagagens; o efeito de pompa a ser obtido para embasbacar os observadores
plebeus; a engenhosidade do desenho e da distribuição das despensas sob
os assentos, gavetas secretas sob as janelas frontais, bolsos invisíveis
sob o forro do estofamento, proteção contra a poeira, acesso através de
insidiosas aberturas, ou dispositivos necromânticos como portas de
Aladim; o encaixe dos coxins de modo que não deslizassem, o
arredondamento das arestas para melhor descanso, a correta fixação e
abertura das cortinas; o perfeito encaixe das janelas, do qual metade do
conforto de uma viagem em carruagem realmente depende; e a adaptação de
todos esses luxos às necessidades dos que ali iriam sentar, neste
pequeno apartamento que seria, virtualmente, a casa dos passageiros por
cinco ou seis meses; --- tudo isso era uma viagem imaginária em si mesmo,
com todos os prazeres, e nenhum dos desconfortos, de uma viagem
verdadeira.

%124. 
Na grande ocasião de nossa primeira viagem continental --- planejada
para durar seis meses --- a carruagem escolhida fora equipada com um
assento frontal externo para meu pai e Mary, um \textit{dickey},\footnote{Conferir
  Capítulo \textsc{i}, nota à página \pageref{dickey}.} maior do que o normal, para
Anne e o \textit{courier}, e quatro assentos internos, embora os
dianteiros fossem muito pequenos, de modo que papai e Mary pudessem ser
abrigados em caso de tempo ruim. Recordo que, quando finalmente
decidimos qual carruagem alugaríamos, o amável Mr.\,Hopkinson, conhecendo
minha nascente reputação literária, perguntou"-me (para a alegria de meu
pai) se eu poderia traduzir a divisa do proprietário anterior, sob seu
brasão de armas, --- ``\textit{Vix ea nostra voco}'',\footnote{``Dificilmente
  chamo essas coisas de minhas.'' Ovidio, \textit{Metamorphoses}, 13.141.} --- no que fui bem"-sucedido, acrescentando
espirituosamente que, embora por direito a divisa pertencesse ao
proprietário anterior, poderia ser, com grande propriedade, aplicada a
\textit{nós}.

%125. 
Para uma carruagem familiar de tão sólida construção, com a
bagagem, e a carga de seis ou mais pessoas, eram necessários quatro
cavalos para fazê"-la avançar; e meia dúzia de equipagens deste tipo
estava disponível em cada casa de posta. O leitor moderno talvez tenha
tanta dificuldade em compreender estes períodos de transporte selvagem e
tosco, embora recentes, quanto qualquer aspecto da migração dos Saxões
ou Godos; e talvez não creia que minha descrição seja vã tagarelice.

Os cavalos franceses, e quase todos aqueles das grandes linhas de viagem
da Europa, eram, apropriadamente, robustos animais de trote para
carruagem, bem adaptados ao que se exigia deles e até para mais; pelos
não aparados, caudas longas, humor descontrolado, relinchando e
brincando um com o outro sempre que podiam; inteligentes no trabalho; na
maioria das vezes obedientes à voz, e às rédeas apenas em ordens mais
específicas; nunca tocados pelo chicote, que era utilizado apenas para
expressar o regozijo do cocheiro consigo mesmo e com os cavalos, ---
sinalizar para os veículos que obstruíam o caminho à nossa frente, e
avisar aos habitantes das cidades e vilas que atravessávamos que pessoas
distintas os estavam honrando com sua passagem. Se tudo corresse bem, os
quatro cavalos eram conduzidos pelo postilhão montado no cavalo guia;
mas se os animais fossem jovens, ou os cavaleiros inexperientes, havia
também um postilhão para os cavalos da frente. Como regra, havia quatro
animais de trote regular e um bom cocheiro, raramente bêbado, geralmente
muito jovem, os homens de constituição mais robustas mais aptos para
outros trabalhos, qualquer jovem hábil capaz de conduzir as
bem"-treinadas e dóceis bestas, além de pesarem menos sobre elas. Metade
do peso do cavaleiro, nestes casos, estava nas botas, frequentemente
penduradas na sela como dois baldes, o postilhão, depois que os cavalos
eram arreados, caminhava entre eles, ao longo da vara de atrelar.

%126. 
Ligeiramente menos oficial para uma viagem de carruagem de boa
classe que o postilhão, era o \textit{courier}, ou mais apropriadamente,
\textit{avant"-courier}, cuja principal obrigação era cavalgar à frente da
carruagem em galope constante, e, em cada casa de posta, mandar arriar
os cavalos, que esperavam prontos, de modo que não se perdesse tempo
entre as etapas. Outra função muito importante era barganhar e pagar
todas as contas, para poupar a família de preocupações inconvenientes e
fontes de ansiedade, além da dificuldade e da infelicidade de tentar
falar francês ou qualquer outra língua estrangeira. Além disso, conhecia
as boas pousadas em cada cidade, e todos os bons quartos em cada uma
delas, de modo que podia escrever antecipadamente e assegurar aqueles
adequados à família. Se fosse realmente inteligente e de alta categoria,
o \textit{courier} conhecia as atrações a serem visitadas em cada cidade,
e os meios secretos para se visitar aquelas que não estavam disponíveis
ao viajante vulgar. Murray,\footnote{Provavelmente George Gilbert Aimé
  Murray (1866--1957): helenista, tradutor, publicista e diplomata
  australiano radicado na Inglaterra.} o leitor se
lembrará, não existia naqueles dias; o \textit{courier} era um Murray
privado, que sabia, se tivesse alguma perspicácia, não apenas as coisas
que deveriam ser vistas, mas aquelas que você provavelmente mais
gostaria de ver, e informava seu \textit{valet"-de"-place}\footnote{Em cada
  cidade, valete ao serviço de viajantes.} a respeito,
intervindo como instância superior apenas em situações difíceis, a serem
superadas com dinheiro ou tato. O \textit{courier} acompanhava
invariavelmente as senhoras em suas compras, levando"-as às lojas da
moda, e encontrando preços que julgava apropriados. Por último,
conhecia, naturalmente, todos os outros \textit{couriers} de alta
categoria da estrada, e informava, em caso de interesse, quais as
pessoas importantes que incidentalmente estivessem na mesma pousada.

%127. 
Meu pai teria considerado uma insolente e revolucionária violação
dos privilégios da nobreza ter um \textit{courier} cavalgando a nossa
frente; além disso, sabiamente liberal com seu dinheiro em se tratando
de conforto e prazer, ele nunca teria pagado um cavalo extra por
exibição. Os cavalos, portanto, eram encomendados antecipadamente,
quando possível, pelos postilhões das carruagens que nos precediam (ou
não nos importávamos de esperar até que eles fossem arreados), e
transportávamos nosso \textit{courier} atrás, no \textit{dickey} com Anne, o
qual, em todas as outras funções e tarefas que desempenhava, era"-nos um
luxo indispensável. Indispensável, em primeiro lugar, porque nenhum de
nós sabia falar outra língua além do francês, e esta apenas o suficiente
para perguntar a direção; para todas as peculiaridades da barganha, ou
informações mais detalhadas, éramos inúteis, mesmo na França, --- e
teríamos sido também ovelhas migratórias, ou gansos, na Suíça e na
Itália. Indispensável, em segundo lugar, para a tranquilidade de meu
pai, que mesmo tendo grande generosidade de caráter, detestava ser
trapaceado. Ele sabia perfeitamente que o \textit{courier} receberia
comissão, a qual não questionava; mas sabia também que seu
\textit{courier} não seria logrado por estranhos, e estava satisfeito com
sua representação. Não por ostentação, mas por prazer e sensação de
mudança de sua vida suburbana, meu pai gostava de quartos grandes; e
minha mãe, para manter seus hábitos comuns e essenciais, de quartos
limpos; quartos limpos e grandes significavam uma boa pousada e um
primeiro andar. Meu pai também apreciava vista pela janela, e
sensatamente dizia, ``Por que devemos viajar para ver menos do que
podemos?'' --- isso significava um primeiro andar \textit{na frente}. Meu
pai também gostava de cozinha sofisticada, justamente porque era um
comensal parcimonioso e exigente; e minha mãe gostava de boa carne. Isso
significava jantar sem limite de preço. E também, embora meu pai não
frequentasse a sociedade, gostava de ver rapidamente, com reverência, as
pessoas da sociedade --- quer dizer, pessoas que ocupavam posições
importantes --- ele detestava moda, --- e era algo relevante perceber que
determinado Lord e Lady ocupavam o mesmo andar que ele, e que, a
qualquer momento, poderiam se encontrar nas escadas. Salvador,
devidamente avisado, e percebendo perspicazmente estas disposições de
espírito de meu pai, agradáveis e admiráveis à sua mente de
\textit{courier}, tinha carta branca em todas as funções administrativas e
barganhas. Encontrávamos nossos agradáveis quartos sempre prontos,
nossos bons cavalos sempre à espera, todos tiravam seus chapéus quando
chegávamos e partíamos. Salvador prestava contas semanalmente, que eram
aceitas sem uma palavra de objeção.

%128. 
Diante de todas essas condições de luxo e felicidade, pode o
moderno turista que solta baforadas de vapor entender outra e essencial
condição --- que nunca tínhamos pressa? Acrescida da possibilidade de
sempre começarmos no momento que escolhêssemos, e que se não
estivéssemos prontos, os cavalos esperariam? Com regra, desjejuávamos na
mesma hora que em casa --- oito; os cavalos estavam escarvando e
relinchando à porta (sob a arcada, deveria ter dito) às nove. Entre nove
e três, --- computando"-se sete milhas por hora, incluindo"-se as paradas,
como marcha mínima, --- fazíamos quarenta a cinquenta milhas de jornada,
sentávamos para jantar às quatro, --- e eu tinha duas horas para minhas
deliciosas explorações, ao entardecer; deveria voltar pontualmente às
sete para o chá, e concluir meus esboços até nove e meia, --- hora de ir
para a cama.

Nos dias de jornada mais longa, começávamos às seis, e fazíamos vinte
milhas antes do desjejum, chegando para jantar às quatro, como sempre.
Em dias excepcionalmente longos, fazíamos uma segunda parada, jantando
em alguma agradável hospedaria de vila, e tomando o chá mais tarde, após
percorrer oitenta ou noventa milhas. Mas esses esforços eram raramente
feitos, a menos que pretendêssemos alcançar alguma cidade com catedral,
num domingo, ou alguma agradável vila alpina. Nunca viajávamos aos
domingos; meu pai e eu quase sempre íamos --- como filósofos --- à missa,
pelas manhãs, e minha mãe, num gesto de boa"-vontade (raramente vi em seu
rosto traço de curiosidade feminina), juntava"-se a nós em alguma
atividade profana como passear de carruagem pelo Corso, à tarde. Mas
todos nós, inclusive meu pai, preferíamos uma caminhada pelos campos, em
torno de um chalé alpino.

%129. 40\footnote{\textit{Cf}. Capítulo \textsc{iv}, 90.}
À página \pageref{suica},
prometi descrição mais precisa de minhas primeiras impressões da Suíça e
da Itália, em 1833. Do conhecido Calais, falarei mais tarde, --- agora
relato apenas nossa subida do Reno até Estrasburgo, onde, mesmo com
todas suas prodigiosas construções, tive a sabedoria necessária para
perceber a imponência da catedral e seus trabalhos em ferro; mas fui
enormemente estimulado e impressionado pelos altos tetos e as ricas
fachadas das casas de madeira, em sua súbita indicação da proximidade da
Suíça; e particularmente por encontrar o cenário tão admiravelmente
expresso por Prout, na 36ª ilustração de seu \textit{Flanders and
Germany}, ainda intacto. E então, aconselhamo"-nos com Salvador, no salão
da pousada de Estrasburgo, se --- era sexta"-feira à tarde --- deveríamos,
no dia seguinte, tentar alcançar Basileia ou Schaffhausen, para nosso
descanso dominical.

%130. 
Muito dependia --- se alguma coisa ``depende'' de outra --- da
questão que estava sendo discutida! Salvador defendia seguir pela
estrada direta e plana que acompanhava o Reno, o que permitiria o luxo
de se alcançar a hospedaria \textit{Three Kings} ao entardecer. Mas em
Basileia, deveria se admitir, não havia vista dos Alpes, não se ouvia
nenhuma catarata, e Salvador honradamente sugeriu"-nos a esplêndida
possibilidade de atingirmos, através da estrada montanhosa da Floresta
Negra, os portões da própria Schaffhausen, antes que fechassem à noite.

A Floresta Negra! A catarata de Schaffhausen! A cadeia dos Alpes! Tudo
isso poderia ser alcançado domingo! Um domingo muito diferente dos
habitualmente passados nos campos de Walworth e Dulwich! Minha súplica
apaixonada teve êxito, e nas primeiras horas na manhã estávamos trotando
sobre a ponte de barcos para Kehl, e à luz que surgia a leste, lembro
bem de ter visto o contorno das montanhas da Floresta Negra crescer e
elevar"-se, enquanto cruzávamos a planície do Reno. ``Portões das
montanhas''; abrindo para mim uma nova vida --- vida que não terminaria
mais, exceto nos Portões das Montanhas, de onde não se retornava.

%131. 
E assim, atingimos os limites da Schwarzwald,\footnote{``Floresta
  Negra'', em alemão.} e a adentramos por um
desfiladeiro descendente; e apenas quinze minutos depois, creio, vimos
nossa primeira ``Swiss cottage''.\footnote{Suíça, em características e
  hábitos --- os limites políticos não importavam. {[}\textsc{n.\,a.}{]}}
Quanto isso significou para todos nós, --- quanta expectativa criou para
mim, nenhum viajante moderno poderá jamais conceber, mesmo que eu gaste
dias tentando explicá"-lo. Um guincho triunfante --- como se todos os
silvos das ferrovias se encontrassem no entroncamento de Clapham ---
jorrou da Tolice da Europa para destruir o mito de Guilherme Tell. Para
nós, tudo dito sobre ele era verdadeiro --- mais como mito luminoso que
como verdade mortal; e aqui, sob estas escuras florestas, resplandecia
seu visível, belo e tangível testemunho nos lariços púrpuras, esculpido
à perfeição pela alegria da vida camponesa, contínua, imóvel, sob a
sombra dos pinheiros na grama ancestral, --- perfeita e inexpugnável, na
benção da pobreza virtuosa, da paz religiosa.

O mito de Guilherme Tell foi realmente destruído? Você cruzou o túnel
sob o monte Saint"-Gothard e satisfez"-se, talvez, na Baía de Uri; e foi
apenas para você que as uvas verteram sangue da prensa de Saint Jacob e
a clava de pinho abateu cavalo e elmo em Morgarten Glen?

%132. 
É muito difícil para você imaginar o velho tempo dos viajantes,
quando a Suíça era ainda a terra dos suíços, a os Alpes não tinham sido
marcados por pegadas humanas. Só se falava do vapor para pequenas
travessias marítimas com bom tempo (havia paquetes a vapor através do
Atlântico? Não me lembro). De qualquer modo, as estradas terrestres eram
seguras; e uma vez dentro deste Paraíso montanhoso, avançávamos através
de seus vales perfumados, passando pelos chalés em seus gramados, ainda
brilhantes com o orvalho.

Pelo meio"-dia, a estrada atingiu alturas mais sáfaras, com montanhas
mais íngremes; por duas ou três vezes tivemos que esperar pelos cavalos,
e, ao pôr do sol, estávamos ainda a vinte milhas de Schaffhausen;
passava da meia"-noite quando atingimos seus portões fechados. O
sobressaltado porteiro fez a gentileza de abri"-los --- mas não o
suficiente; perdemos uma de nossas lanternas na colisão com a trave
oblíqua, ao passarmos pelo arco. Que felicidade superior o privilégio de
entrarmos, sonhadoramente, numa cidade medieval, produziu, mesmo com a
perda de uma lanterna, em comparação ao livre ingresso numa estação
ferroviária, após estarmos presos entre uma carroça pesada e um bonde!

%133. 
É estranho que apenas eu recorde vagamente da manhã seguinte;
imagino que devemos ter ido a algum tipo de igreja; e, sem dúvida,
passamos uma parte do dia a admirar as janelas de sacada em arco que se
projetavam sobre as ruas limpas. Nenhum de nós parece ter pensando que
os Alpes seriam visíveis sem o esforço profano da escalação de
montanhas. Jantamos às quatro, como sempre, e, estando a tarde muito
bela, fizemos uma caminhada, todos, --- meu pai e minha mãe e Mary e eu.

Devemos ter passado ainda algum tempo admirando a cidade, pois o sol
começava a se pôr quando chegamos a uma espécie de jardim público --- a
oeste da cidade, creio; alto, acima do Reno, a permitir visão do campo
que se estendia ao sul e a oeste do rio. Esse campo aberto com baixas
ondulações, que se perdia no horizonte azul, --- contemplamo"-nos como se
nossas próprias distâncias, de Malvern em Worcestershire, ou de Dorking
em Kent, --- até, subitamente, ver: além!

%134. 
Ninguém pensou, nem por um momento, que pudessem ser nuvens. Eram
tão claros como cristal, nítidos no horizonte puramente azul, e já
tingidos pelo róseo do pôr do sol. Infinitamente além de tudo que
pudéssemos ter imaginado ou sonhado, --- as muralhas do Éden perdido não
seriam mais belas para nós; nem mais terríveis, em torno do Paraíso, as
muralhas da Morte sagrada.

Não é possível imaginar, em qualquer época do mundo, introdução mais
abençoada à vida para uma criança de temperamento como o meu. É verdade
que o temperamento pertence à época: há poucos anos antes, --- dentro do
século, --- antes disso, nenhuma criança nascera com tal interesse pelas
montanhas ou pelos homens entre os quais vivia. Até Rousseau,\footnote{Jean"-Jacques
  Rousseau (1712--1778): filósofo e escritor suíço; um dos nomes mais
  importantes do Iluminismo francês; é dele a ideia de quem o homem
  nasce bom (``o bom selvagem'') e a sociedade o corrompe.} não havia amor ``sentimental'' pela natureza; e até Scott,
não havia algo como amor compreensivo pelos ``homens de todos os tipos e
condições'', não apenas em espírito, mas também em corpo. São Bernardo
de La Fontaine, contemplando o Mont Blanc com seus olhos de criança,
viu, acima do Mont Blanc, a Madona; São Bernardo de Talloires não viu o
lago de Annecy, mas os mortos entre Martigny e Aosta. Quanto a mim, os
Alpes e sua gente eram de igual beleza, tanto pela neve quanto pela
humanidade; e desejei, para mim e para eles, não a visão de tronos no
Paraíso, mas a de rochas, não a de qualquer espírito do Paraíso, mas a
de nuvens.

%135. 
Então, em perfeita saúde e com o coração em fogo, não querendo ser
nada além do garoto que eu era, não querendo ter nada mais do que tinha;
conhecendo a dor o suficiente para ver a vida com seriedade, mas não a
ponto de amortecer sua intensidade; e com uma amálgama de ciência e
sentimentos capaz de fazer da visão dos Alpes não apenas a revelação da
beleza terrestre, mas também a primeira página de seu livro, --- desci do
terraço do jardim de Schaffhausen, naquele entardecer, com meu destino
definido em relação a tudo que seria sagrado e útil. Àquele terraço, e
àquela praia do lago de Genebra, meu coração e minha fé retornam até
hoje, em busca de cada ímpeto que ainda está vivo neles, e de cada
pensamento em que exista altruísmo ou paz.

%136. 
A manhã seguinte àquela noite de domingo em Schaffhausen também foi
sem nuvens, e logo cedo fomos de carruagem às quedas"-d'água, vendo
novamente a cadeira dos Alpes na luz matinal, e conhecendo, em Lauffen,
um rio alpino. Ao sair da garganta de Balsthal, tive outra memorável
visão da cadeia dos Alpes, e essas imagens à distância, jamais vistas
pelo moderno viajante, ensinou"-me, e fez"-me perceber mais que as
maravilhas, vistas de perto, de Thun e Interlachen. Tivemos sorte
novamente ao adentrarmos a Itália pelo mais grandioso passo, --- de modo
que a primeira ravina dos Alpes principais que vi foi Via Mala, e o
primeiro lago da Itália, Como.

Tomamos um barco para navegar até o pequeno e esconso lago de Chiavenna,
e remamos por toda a extensão aquosa, passando outro domingo em
Cadenabbia, e então, de vila em vila, cruzamos um lago, e outro, até
Como, e depois a Milão, através de Monza.

Embora fosse o início do verão, havia muita gente; e a primeira
impressão da Itália sempre deve ser no verão. De outra parte, embora meu
coração estivesse com os camponeses suíços, o gosto artificial em mim
foi formado principalmente pela representação de Turner, destas mesmas
cenas, na \textit{Itália}, de Rogers. O ``Lago de Como'', as duas vilas ao
luar, o ``Adeus'', prepararam"-me para tudo que era belo e elegante nos
terraços dos jardins, nas arcadas bem proporcionadas, e nos espaços
brancos da muralha ensolarada, que, em geral, não atraem a mente
inglesa. Mas para mim, através de Turner, eles eram quase naturais, ---
imediatamente familiares, e reverenciadas. Então não fazia ideia do mal
do Renascimento existente nelas; estavam associadas apenas ao que me
contaram sobre a ``divina arte'' de Rafael e Leonardo e, pela minha
ignorância de datas, às histórias de Shakespeare. A vila Portia, --- o
palácio Juliet, --- pensei que fossem como as da Itália.

E também, como registrado na reimpressão do prefácio do volume \textsc{ii} dos
\textit{Pintores Modernos},\footnote{\textit{Modern Painters}, livro de
  autoria de Ruskin, publicado em 1843.} sempre tive uma
verdadeira percepção de dimensão, seja de montanhas ou de edifícios,
e esta percepção dava"-me grande alegria; de modo que a vastidão de
escala dos palácios milaneses, e a ``montanha de mármore, uma centena de
pináculos'' do Domo, impressionaram"-me enormemente, de imediato: e não
tendo ainda a capacidade de distinguir o bom gótico do ruim, a mera
riqueza e delicadeza do ornamento semelhante à renda contra o céu foi um
verdadeiro êxtase para mim --- que se intensificou ao escalá"-lo vendo o
Monte Rosa por entre seus pináculos, do outro lado da planície!

%137. 
Fui parcialmente preparado para esta visão por uma admirável
apresentação, em Londres, um ou dois anos antes, num local, cujo
desaparecimento foi posteriormente considerado por mim como grande
perda, --- o panorama de Burford, em Leicester Square, uma instituição
educacional dotada dos mais elevados e puros valores, e que deveria ter
sido mantida pelo governo, como uma das melhores escolas instrumentais
de Londres. Ali contemplei, magnificamente pintada, uma vista do teto da
catedral de Milão, num momento em que não havia qualquer esperança de
ver a realidade, mas com uma alegria e admiração das mais profundas; ---
e estar agora em Milão faz aquela profunda admiração insondável.

Outra vez, para nossa grande felicidade, o tempo esteve claro e sem
nuvens durante todo o dia, e à medida que o sol declinava em direção ao
oeste, fomos de carruagem ao Corso, onde, àquela época, a alta sociedade
milanesa passeava feliz e altiva como os ingleses por seus parques, e do
qual, sem a interferência de qualquer estação ferroviária, toda a cadeia
dos Alpes era visível, de um lado, e a bela cidade, dominada pelo Domo
com sua silhueta de cristal congelado, do outro. Depois a volta para
casa na carruagem aberta através do tranquilo entardecer, pelas longas
ruas, e em torno da base do Domo, o deslizar das rodas sobre o suave
pavimento contribuindo para a sensação de maravilhoso sonho que a tudo
permeava, --- o ar perfeito em absoluta calma, a majestade apenas
entrevista dos Alpes ao redor, a perfeição, --- assim me pereceu --- e a
pureza do suave e sólido mármore sem jacas contra o céu. O que mais se
poderia pedir deste bem aparentemente imutável neste mundo mutável?

%138. 
Em geral, pretendo evitar interferência com o julgamento do leitor
em questões as quais me esforço para narrar serenamente; mas aqui,
talvez, possa ser perdoado que eu observe a relativa vantagem da
abstração contemplativa do mundo, durante esta precoce viagem
continental, determinada parcialmente por nossa ignorância, e
parcialmente pelo nosso amor ao conforto. Existe algo peculiarmente
delicioso --- ainda mais deliciosamente inconcebível ao moderno turista
alemão fechado em si e ao turista francês refinado que passam pelas ruas
de uma cidade estrangeira sem entender uma palavra do que se diz! A
sensibilidade auditiva a todos os tipos de vozes torna"-se completamente
imparcial; não se é atraído pelos significados das sílabas, ao tentar
reconhecer suas qualidades guturais, líquidas ou melífluas: os gestos do
corpo e a expressão da face têm o mesmo valor que numa pantomima; cada
cena torna"-se uma melodiosa ópera ou um pitorescamente inarticulado
polichinelo. Considere, também, o ganho com tal consistente
tranquilidade. A maioria dos jovens, atualmente, ou mesmo os mais velhos
com espíritos joviais, viaja mais em busca de aventuras que de
informação. Uma das lembranças mais valiosas de minhas recentes
perambulações são os bosquejos feitos por uma amável e extremamente
habilidosa garota, de coisas que aconteceram a seu povo e a ela em cada
dia de sua viagem ao exterior. Aqui está o irmão Harry, lá está mamãe,
agora \textit{pater familias}, e então sua pequena e graciosa pessoa, logo
sua alegre e criticadora irmã, que enfrentam dificuldades arrebatadoras
e desventuras invejáveis; prendem"-se a amizades e insinuam entusiasmados
namoricos com todo tipo de gente usando chapéus cônicos e capas com
franjas: e isso tudo é muito deleitável e condescendente; e,
naturalmente, aprendem coisas sobre o país que não poderiam ser
aprendidas de outro modo, mas apenas sobre aquela parte que lhes
interessa conhecer, ou com a qual se tem interesse em se relacionar.
Virtualmente, está"-se pensando sobre si mesmo todo o tempo;
necessariamente fala"-se com as pessoas alegres, não com as tristes; e a
mente está, na maioria do tempo, vividamente ocupada com coisas muito
pequenas. Não digo que nosso isolamento tenha sido meritório, ou que as
pessoas, de modo geral, devam conhecer apenas a sua própria língua.
Entretanto, nossa humilde ignorância tem essas vantagens. Não viajamos
por aventuras, nem por companhias, mas para ver com nossos próprios
olhos, e para julgar com nossos corações. Se se tem simpatia, o aspecto
da humanidade é mais verdadeiro e profundo que as palavras; e mesmo em
minha própria terra, as coisas que pior decepcionaram foram aquelas que
conheci como Espectador.

\chapter{Papai e mamãe} %Capítulo \textsc{vii}. 

%139. 
O trabalho ao qual, como parcialmente descrito acima, dediquei"-me \label{139}
durante o ano de 1834, sob a excitação proveniente de minhas viagens ao
exterior, desenvolveu"-se em quatro direções diferentes, em cada uma das
quais minha energia poderia ter se concentrado se fosse encorajado
decisivamente. Houve, em primeiro lugar, o esforço para expressar
sentimentos em versos; o sentimento era realmente genuíno, sob todas as
vãs superficialidades em que eram exibidos; e os versos, ritmados,
embora sem conter ideias. Era impossível explicar, tanto a mim como aos
outros, por que eu gostava de contemplar o mar e vadiar pela charneca;
apenas meu prazer em produzir algum tipo de barulho melodioso sobre
aquilo, como as próprias ondas, ou os abibes. Então, em segundo lugar,
houve o verdadeiro amor pela gravura e pelos diferentes tipos de
superfícies e sombras que poderiam representar. Nunca vi desenhos,
feitos por um jovem, compostos com linhas tão delicadas; e havia,
realmente, em mim, um gravador destinado a esboçar paisagens ou
silhuetas. Mas tendo o destino querido diferente, lamento menos a perda
de minha capacidade como gravador que daquela anteriormente ponderada,
ou melhor, imponderável, como geólogo. Então houve, em terceiro lugar, o
poderoso impulso para a arquitetura; mas nunca poderia construir ou
esculpir algo, porque não tinha capacidade de desenhar; e fiz mais nesta
direção do que valeria a pena fazer tendo uma habilidade tão limitada. E
então, em quarto lugar, o inabalável, e jamais abalado, interesse
geológico, agora intensificado pelos Alpes. Solicitado a escolher meu
presente pelo décimo quinto aniversário, pedi \textit{Voyages dans les
Alpes}, de Saussure,\footnote{\textit{Viagens através dos Alpes}, de Horace"-Bénédict de Saussure. 
  Conferir nota na p. \pageref{saussure}.} e a
partir daí encetei um trabalho progressivo, lendo meu dicionário
mineralógico com a ajuda da Mineralogia em três volumes de
Jameson\footnote{Provavelmente Robert Jameson (1774--1854), geólogo
  escocês.} (um livro muito claro e útil), comparando
suas descrições de minerais com aquelas do Museu Britânico, e escrevendo
as minhas próprias, mais eloquentes e exaustivas, numa estenografia com
muitos símbolos engenhosamente criados, que me fizeram demorar mais em
escrevê"-los que se o fizesse com um texto comum, e dos quais, depois,
nem eu nem ninguém conseguia ler uma palavra.

%140. 
Sendo tal o plano quadrilateral de minhas disposições capazes de
serem desenvolvidas, é o momento de explicar, com a mesma diligência a
elas dedicada, o caráter algo peculiar de meus pais; cuja influência
sobre mim foi mais importante, então, e durante toda minha vida, que
qualquer condição externa, seja de amizade ou de tutoria, tanto na
universidade quanto no mundo.

Em primeiro lugar, foi uma questão de peso essencial na determinação das
linhas que seguiria, não apenas em termos de trabalho, mas também de
pensamento, o fato de meu pai, como dito anteriormente, ter me dado o
melhor exemplo de leitura emocional, --- \textit{leitura} propriamente,
observe"-se; não recitação, que ele desdenhava, e eu não apreciava, ---
minha mãe era capaz de me ensinar, e decidiu que eu deveria aprendê"-lo,
tanto a absoluta acurácia da dicção como a precisão do acento tônico na
prosa; e me ensinou, logo que pude falar corretamente, aquilo que
durante os anos posteriores tenho tentado inculcar em meus leitores, que
acurácia de dicção significa acurácia de sensação, e precisão de
acentuação, precisão de sentimento. Educada, na infância, apenas na
escola rural de Mrs.\,Rice, minha mãe ali aprendera severos e corretos
princípios de verdade, caridade e virtudes domésticas, além de
escrupuloso respeito pela pureza daquele inglês que em seu ambiente
familiar percebera não ser exatamente imaculado como a água das fontes
de Wandel. Ela era filha, como já mencionado, da proprietária,
precocemente viúva, da pousada e taverna King's Head, que ainda existe,
ou existia há um ou dois anos, cuja lateral dava para a praça do mercado
de Croydon, a frente e a porta de entrada para a estreita aleia que
descia, íngreme para pedestres, intransitável para carruagens, da High
Street à cidade baixa.

%141. 
Familiarizada, portanto, com os costumes e o dialeto da ágora de
Croydon, minha mãe, como percebo agora, deve ter sido uma moça
extremamente inteligente, de admirável praticidade e ambição ingênua;
mantendo, sem contenda, a liderança de sua classe, e aproveitando, com
serena discrição, cada vantagem que a escola rural e sua modesta mestra
podiam lhe oferecer. Nunca a ouvi, em sua vida adulta, lamentar a
disciplina imposta por Mrs.\,Rice, a qual raramente referia"-se sem
respeitoso louvor.

Não sei por que razão, ou sob quais condições, minha mãe foi viver com
meus avós escoceses, primeiro em Edimburgo, e depois na casa de Bower's
Well, na vertente da colina de Kinnoul, acima de Perth. Estive, de um
modo estúpido e insensível, desinteressado do passado de minha família
enquanto podia conhecê"-lo; apenas após a morte de minha mãe comecei
a desejar saber aquilo que ninguém mais poderia me contar.

Mas, certamente, para ela, a mudança significou a introdução a um
círculo social mais elevado, --- com cavalheiros e damas reais, embora
algumas vezes excêntricos, e frequentemente pobres. Ela deve, então, ter
rapidamente se transformado numa moça muito bem feita, alta e bonita,
com uma bela expressão de branda firmeza; uma dona de casa
irrepreensível e eficiente, e com natural, essencial, inatacável, embora
inofensiva, circunspeção. Nunca ouvi uma palavra sobre qualquer
sentimento, acidente, admiração ou afeição que perturbasse o curso de
sua administração doméstica na Escócia; notei, entretanto, que ela nunca
falava do Dr.\,Thomas Brown, sem uma ligeira timidez, diante de meu pai,
nem sem algum prazer, diante de outras pessoas.

%142. 
O fato do professor de filosofia moral ser um convidado frequente à
mesa de chá de minha avó, e aficionado da discussão amena com Miss
Margaret, é evidência suficiente da posição que ela ocupava nos círculos
sociais de Edimburgo; suas habilidades e deveres domésticos nunca eram
negligenciados --- pelo contrário, se algo pudesse ser objetado, era o
excessivo escrúpulo com que os praticava. Certa vez, quando tinha posto
seu vestido branco para o jantar, e correra à cozinha para uma última
olhada nas coisas por ali, a velha Mause, tendo"-se chocado contra o
vestido branco com uma caçarola de molho preto, e, sendo repreendida,
por sua jovem patroa com tão pouca resignação à vontade da Providência
nesta questão, apertou sua mão com pesar, dizendo: ``Ah, Miss Margaret,
`ocê é como Martha, cuidadosa e atrapalhada com tanta coisa''.

%143. 
Quando minha mãe tinha vinte anos, Desdêmona na flor de sua
feminilidade, atenta aos grandes problemas da filosofia moral --- ``até
que os deveres domésticos a reclamassem'', --- meu pai era um
brilhantemente ativo e sensível jovem de 16 anos, de olhos escuros.
Margaret tornou"-se para ele uma governanta e confidente absolutamente
respeitada e admirada, além de terna. Sua simpatia lhe era necessária,
com tantos amores rápidos e fugazes; seu conselho, nos negócios
domésticos ou na tristeza, e seu encorajamento, em todos os projetos de
vida. Os projetos já estavam dirigidos para o comércio; --- embora sem o
abandono das artes liberais. Ele aprendera perfeitamente latim, ainda
que sem grande variedade de leitura, na nobre tradição de Adam na Escola
Secundária de Edimburgo: quando, graças à então forte e universal
influência de Sir Walter, cada aspecto da cidade natal do escritor era
enaltecida em sua imaginação pela mais pura poesia e a mais gloriosa
história que já frequentou ou consagrou as ruas e rochas de uma capital
habitada por uma brilhante sociedade.

%144. 
Não tenho espaço nem desejo de estender meu relato das coisas
passadas, com base em correspondências; --- é comum entre os biógrafos
modernos confundir conversas epistolares com fatos reais. Mas a carta
apresentada a seguir, do Dr.\,Thomas Brown\footnote{Thomas Brown
  (1778--1820): filósofo e poeta escocês.} ao meu pai, nesse
momento crítico de sua vida, deve ser lida, em parte, como testemunho da
posição que ele já ocupava entre a juventude de Edimburgo, e mais ainda
como explicação de certos aspectos de seu caráter complexo, algo que,
posteriormente, foi de grande importância tanto para ele quanto para
mim: %---

\begin{quote}
\begin{flushright}
8, \textsc{n.\,st.\,david's street}\\
\textsc{edimburg}, \textit{February} 18th, 1807
\end{flushright}

Meu querido senhor,

Quando vejo a data da carta com a qual me
honrou ao pedir que o aconselhasse em questões literárias --- algo que
alguém com sua \textit{proficiência} pouco necessita, --- sinto"-me
envergonhado do tempo que levei para respondê"-la''. Posso assegurar,
contudo, que a demora foi inevitável, e que não se deveu a nenhum
desinteresse por seu progresso intelectual. Ainda quando você era um
simples garoto, eu me deleitava com seu entusiasmo e êxitos precoces;
por você mesmo e por sua admirável mãe, sempre o tive em grande estima,
e acreditei que você se distinguiria na profissão que escolhesse.

Você parece, creio, arrepender"-se do tempo excessivo que dedicou às
\textit{Belles Lettres}. Confesso não o lamentar por você. Estou certo
que deve ter percebido o efeito que esses estudos têm sobre o
refinamento das maneiras e dos sentimentos, algo que, para quem não está
destinado a ser estritamente um \textit{homem de ciências}, constitui o
mais benéfico efeito da literatura. Você deve ter em mente que existe
uma grande diferença entre o estudo \textit{profissional}, e o estudo por
prazer e para refinamento do espírito. Na sociedade à qual você
pertencerá, os escritores das \textit{Belles Lettres} serão mencionados
cinquenta vezes, enquanto os que se dedicam à ciência mais abstrata não
serão mencionados nenhuma; e a grande vantagem desse tipo de
conhecimento é que sua demonstração, a menos que se faça de maneira
muito imoderada, não é considerada pedantismo, enquanto que a
demonstração de outros conhecimentos intelectuais corre algum risco
daquela imputação. Existe, certamente, um mal na leitura de poesia e de
outros trabalhos intelectuais mais leves, que é o risco de se entregar a
uma franca \textit{glutonaria}, e com ela ocupar o tempo que deveria ser
dedicado ao trabalho; mas estou certo que posso confiar que \textit{você}
não desperdiçará seu tempo. Há, entretanto, \textit{uma ciência}, a
primeira e a maior das ciências, para todos os homens, e para os
negociantes em particular --- a ciência da economia política. Penso que
deva dirigir sua atenção a ela. Trata"-se, na verdade, da ciência de sua
profissão, que contrabalança o --- (palavra perdida com o lacre) --- e
estreitos hábitos que essa profissão algumas vezes inclina"-se a
produzir; e que é sempre de interesse nas discussões sobre negócios
mercantis e financeiros. Um comerciante bem instruído em economia
política estará sempre apto a ocupar papel preponderante nos debates
entre seus colegas comerciantes --- sem o que, será um simples mercador.
Não perca tempo, portanto, em obter um exemplar da \textit{Wealth of
Nations}, de Adam Smith, e em lê"-lo e relê"-lo com atenção --- como estou
certo que o fará com prazer. Ao lhe dar este conselho, considero"-o como
um \textit{comerciante}, pois esta será sua profissão; o teste para
avaliar a importância de qualquer conhecimento deve ser o quanto contribui
para torná"-lo um \textit{honrado e destacado comerciante}; personagem de
grande estima neste país comercial. Considero, portanto, as ciências
físicas como perfeitamente secundárias em relação aos seus objetivos na
vida, e à sociedade à qual será chamado a pertencer. Todas as ciências,
exceto a química, requerem um conhecimento da matemática maior do que
aquele que você provavelmente possui, e a química é quase impossível de
ser compreendida sem a oportunidade de observação sistemática de
experiências. Se, entretanto, você tiver a oportunidade de assistir a
alguma conferência sobre esta ciência, em Londres, valerá a pena, e,
neste caso, penso que você deverá adquirir o novo sistema químico do Dr.\,Thompson ou de Mr.\,Murray, o que lhe permitirá acompanhar a conferência.
Mesmo da física, é interessante se ter uma visão geral, ainda que
superficial, e embora você não possa esperar, sem o conhecimento da
matemática, nada além desta visão superficial, deve tentar obtê"-la. Este
conhecimento o permitirá ler \textit{Economy of Nature}, de Gregory, que,
embora não seja um bom livro, e nem sempre acurado, é, creio, a obra
mais popular que temos, e suficientemente exata para seus objetivos.
Lembre"-se, contudo, que embora lhe seja permitido ser um filósofo
natural superficial, esta indulgência não lhe será concedida em economia
política.

A única outra recomendação que me resta a fazer é que você não se
permita esquecer nenhuma das línguas que aprendeu. Em relação às
línguas modernas, o risco é menor, já que suas comunicações mercantis as
manterão, em certa medida, vivas; mas comerciantes não se correspondem
em latim, e talvez você perca esta língua inconscientemente.
Independentemente, contudo, dos admiráveis escritores dos quais você se
privará, e de se considerar esta língua meramente um dever de
cavalheiro, é demasiadamente valiosa para ser negligentemente
abandonada.

Adeus, meu querido senhor. Aceite a estima de todos os membros de minha
família, e, creia"-me, todo meu o desejo de lhe ser útil,

\bigskip

\hfill{}Seu amigo sincero,

\hfill{}\textsc{t. brown}
\end{quote}

%145. 
É fácil compreender que um jovem a quem uma carta como essa foi
dirigida por um dos principais intelectuais do mais elevado círculo
intelectual de Edimburgo, passasse a ser visto pela sua prima de Croydon
com um respeito maior do que aquele que jovens mulheres costumam devotar
a seus colegas de escola.

A relação entre eles continuou a ser a de primos, franca, sem que nenhum
pensasse em estreitar os laços, até que meu pai, com a idade de vinte e
dois ou vinte e três anos, após vários períodos de aprendizado em
Londres, decidiu estabelecer"-se nesta cidade com um negócio próprio.
Nessa época, ele havia se convencido de que Margaret, embora não fosse
exatamente seu ideal de heroína, era o tipo certo de pessoa para ter
como esposa, sobretudo por já estarem bem acostumados um com o outro; e,
de maneira tranquila mas resoluta, perguntou"-lhe se compartilhava sua
opinião, e se poderia esperá"-lo até que pudesse oferecer"-lhe uma vida
independente. A confidente de sua juventude consentiu com uma alegria
franca, não à maneira de Agnes Wickfield,\footnote{Personagem do romance
  \textit{David Copperfield}, de Charles Dickens.} ``Eu o
amei durante toda minha vida'', mas compreendendo o grande regozijo de
poder amá"-lo agora. As relações entre Grace Nugent e Lord Colambre no
\textit{Absentee}\footnote{\textit{O Ausente}, romance da escritora
  anglo"-irlandesa Maria Edgeworth (1767--1849).} de Miss
Edgeworth eram muito semelhantes àquelas entre meu pai e minha mãe,
exceto pelo fato de Lord Colambre ser um amante mais impetuoso. Meu pai
escolheu sua esposa com a mesma serenidade e firmeza com as quais
posteriormente escolheria seus empregados.

%146. 
Seguiu"-se uma época de contentamento ativo e esperançoso para os
dois jovens, minha mãe provavelmente a mais intensamente apaixonada,
enquanto John dependia mais estreitamente de sua compreensiva e sábia
amizade do que atualmente ocorre com os rapazes em suas relações com as
jovens que admiram. Mas nenhum deles jamais permitiu que seus
sentimentos se degenerassem numa paixão irritadiça e impaciente. Minha
mãe mostrava seu afeto sobretudo através do constante empenho em
desenvolver suas habilidades intelectuais e em aperfeiçoar as boas
maneiras, de modo a tornar"-se companhia digna de um homem a quem
considerava superior a ela: meu pai, na ininterrupta dedicação ao
negócio de cujo sucesso o casamento dependia, e através de uma metódica
e regular correspondência que nunca deixava a sua amada um momento de
ansiedade, ou dava"-lhe motivo para qualquer descontentamento.

Nessas condições, o noivado durou nove anos; ao fim deste tempo, tendo
os débitos de meu avô sido todos pagos, e o negócio que meu pai
estabelecera progredido gradualmente, não estando sujeito a qualquer
contingência grave, o agora não mais tão jovem par casou"-se numa noite
em Perth, após o jantar, os empregados da casa sem suspeitarem de nada
até que John e Margaret partiram juntos de carruagem na manhã seguinte
para Edimburgo.

%147. 
Ao olhar retrospectivamente para meus pensamentos e comportamentos,
o que mais me surpreende é a falta de curiosidade sobre todas essas
questões; e que, nas muitas vezes em que minha mãe me contava com
condescendência a história de seu casamento cuidadosamente mantido em
segredo, nunca tenha lhe perguntado: ``Mas, mãe, por que tão secreto,
quando era exatamente o que todos seus amigos esperavam há tanto tempo,
e todos seus melhores amigos tão sinceramente desejavam''?

Mas, até recentemente, nunca pensei em escrever sobre mim além daquilo
que registrei em diários, nem sobre minha família; e então,
descuidadamente, e, como agora penso, profanamente, negligenciei as
tradições de minha família. ``O que tudo isso importa, agora''? Eu
disse; ``nós somos o que somos, e seremos aquilo que fizermos de nós
mesmos''.

Também, até muito recentemente, acostumara"-me a considerar tudo o que
meus pais fizeram, em se tratando de sua felicidade pessoal, como
totalmente correto e exemplar. O leitor, entretanto, não deve supor que
o que disse em meu deliberado relato sobre a propriedade de longos
noivados tenha algo a haver com o caso singular que ocorreu em nossa
família. Do heroísmo e paciência que, dos dois lados, constituíram o
sacrifício, nada posso afirmar: --- mas que tenha sido maior do que
aquele que eu próprio teria sido capaz, posso asseverar, e creio que
tenha sido insensato. Pois durante os anos de espera, a saúde de meu pai
gradualmente declinou, a um ponto do qual nunca se recuperou
completamente; e, ao fim de suas vidas, meus pais tiveram que deixar seu
filho, exatamente quando estava começando a corresponder às expectativas
que eles haviam desenvolvido em relação a ele.

%148. 
Permiti"-me a narração do pouco que sabia a respeito das provações e
virtudes da juventude de meus pais porque creio que minha história será,
ao fim, mais completa, caso a escreva ao ritmo em que os temas
interligados se apresentem, do que se seguir um plano cronológico
formal. O motivo para incluí"-la neste espaço foi, sobretudo, explicar
como minha mãe obteve seu perfeito domínio da leitura em inglês, através
do árduo esforço que, durante os anos de espera, fez para eliminar seus
erros e suprir as deficiências de sua educação; esforço que foi
facilitado e infalivelmente guiado pela sua natural --- pela sua
intensidade, poderia chamá"-la de sobrenatural --- pureza de coração e de
conduta, que sempre a fazia ter o maior prazer com a linguagem correta e
clara que apenas pode narrar coisas agradáveis. Sua inquestionável fé
evangélica na verdade literal da Bíblia colocou"-me, logo que pude
compreender ou pensar, diante da presença de um mundo invisível; e
despertou, precocemente, minha capacidade de análise, obrigando"-me a
trabalhar sobre questões como consciência, livre arbítrio e
responsabilidade, facilmente determináveis nesses dias de inocência; mas
que são enfocadas muito frequentemente de modo preconceituoso, e sempre
desvantajosamente, depois que os homens tornam"-se entorpecidos pelas
opiniões, ou corrompidos pelos pecados do mundo externo: enquanto a
melancolia, e mesmo o terror, com os quais as restrições do domingo, e
as doutrinas do \textit{Pilgrim's Progress}, da \textit{Holy War},\footnote{\textit{The
  Holy War --- The Losing and Taking Again of the Town of Man"-Soul (The
  Holy War Made by Shaddai upon Diabolus, for the Regaining of the
  World)}: livro de John Bunyan.} e dos
\textit{Emblems}\footnote{\textit{Emblemas}, livro do poeta inglês Francis
  Quarles, muito popular em sua época (1592--1644).} de
Quarles oprimiam a sétima parte de meu tempo, eram"-me úteis como a única
forma de vexação que eu tinha que suportar; redimido, por outro lado,
pela ininterrupta alegria e tranquilidade de um lar cuja atmosfera era
sempre agradável, e conduzia, por um único e estreito caminho, à
perfeita paz.

%149. 
Os problemas da saúde de meu pai, consequência inevitável de longos
anos de responsabilidades e esforços, necessitavam apenas de repouso
para sua cura. Tímido ao extremo em sociedade, sobretudo pela sua
incapacidade de expressar, de maneira satisfatória, suas habilidades
naturais, --- seu talento para os negócios era soberbo e natural:
dedicava sua melhor energia à contabilidade da casa comercial, pela
manhã, e, à tarde, ao repouso doméstico. Dotado de percepção e decisão
instantâneas em todas as questões comerciais, lidando com elas com
princípios que não admitiam desonestidade, e não envolviam ansiedade ou
dissimulação, o trabalho era mais objeto de interesse, ou mesmo de
diversão, que de preocupação. Seu capital estava no banco, ou nas Docas
de St.\,Catherine, sob a forma de pipas seguradas contendo o mais fino
\textit{sherry} do mundo; seu sócio, Mr.\,Domecq, um espanhol tão altivo e
honrado quanto ele, e tendo completa confiança em meu pai, --- não apenas
em sua probidade, mas também em seu julgamento, --- seguia
escrupulosamente todas as indicações de meu pai para a preparação do
vinho destinado ao mercado inglês, não menos empenhado que ele em fazer
de cada tipo, em todas suas variedades, uma mercadoria de qualidade
incomparável. As cartas para a Espanha, portanto, requeriam apenas breve
informação sobre se o público daquele ano queria seu vinho jovem ou
envelhecido, claro ou escuro, e coisas semelhantes; e as cartas
destinadas aos clientes informavam, também com brevidade, que caso
encontrassem falhas nos vinhos, era porque não o souberam apreciar, e
caso desejassem uma extensão do crédito, esta não lhes seria concedida.
Essas breves disposições espartanas nas cartas, contudo, estavam sempre
amparadas no mais meticuloso cuidado na execução das ordens dos
clientes; e a rara atenção demonstrada por meu pai, ao visitar
pessoalmente seus clientes, para recolher as ordens, em vez de enviar um
agente comercial ou um empregado. Estas visitas domiciliares eram sempre
conduzidas por ele com grande \textit{savoir"-faire} e agradável cortesia,
além da mais atenta paciência: e resultavam na mais alta confiança entre
ele e o comerciante da província, sendo justo e honesto na apreciação
dos vinhos das casas rivais, enquanto seu fino paladar permitia"-lhe
sempre triunfar em qualquer prova às cegas à qual um cliente suspeitoso
o quisesse submeter. Além disso, quando clientes importantes vinham à
cidade, meu pai desviava"-se de seus afazeres e os convidava a jantar em
Herne Hill e a experimentar os vinhos de sua própria adega. Essas
visitas a Londres eram organizadas em grupos, em ocasiões em que havia
na metrópole algo de interesse especial para o espírito provinciano.
Nossos jantares de negócios eram organizados de modo a reunir dois ou
três visitantes da província, e a mesa simetricamente arranjada com
clientes londrinos selecionados, cuja conversa seria a mais instrutiva
para nossos amigos rurais.

Ainda muito jovem, comecei a detestar essas comemorações comerciais, e a
formar, através da observação atenta dos diálogos, quando tinham por
tema algo diferente que o vinho, uma opinião extremamente desfavorável
da mente dos comerciantes; --- opinião a qual nunca tive o menor motivo
para modificar.

Nossos vizinhos de Herne Hill, nós nunca os víamos, com apenas uma
exceção, que será posteriormente mencionada. Eles eram, em sua maioria,
ricos negociantes de Londres, pertencentes a uma classe social superior,
com pouca simpatia pelo estilo antiquado de minha mãe, e nenhuma pelos
sentimentos românticos de meu pai.

%150. 
Havia certamente outro motivo para o declínio de nossa intimidade
com nossos vizinhos imediatos, o fato da maioria deles serem mais ricos
que nós, e inclinados a demonstrar sua riqueza através da suntuosidade
de suas habitações. Meus pais viviam com rigorosa economia, mantinham
apenas criados femininos,\footnote{Thomas nos deixou, creio que, em
  parte, pelo meu lábio permanentemente lesado; e nunca tivemos, dentro
  da casa, outro criado masculino. {[}\textsc{n.\,a.}{]}} usavam apenas velas
de sebo em castiçais folheados,\footnote{Provavelmente significa
  ``folheados de prata'', para diferenciar dos castiçais de ``prata
  maciça''.} estavam satisfeitos com o arrendamento dos
jardins que ficavam à frente e atrás da casa, --- que juntos mal chegavam
a um acre, --- não tinham cavalo nem carruagem. Nossos vizinhos
proprietários de lojas, ao contrário, exibiam normalmente grande cortejo
de lacaios e baixela reluzente, extensos parques para lazer, caras
estufas, e carruagens conduzidas por cocheiros com perucas. Alguns
leitores talvez possam pensar que a frieza das relações com nossos
vizinhos originasse"-se de nós; mas, indubitavelmente, meu pai era
demasiadamente altivo para participar de divertimentos os quais ele não
poderia retribuir, e minha mãe não aceitava ir à pé entregar seu cartão
à porta de senhoras que passavam diante dela em suas caleches.

%151. 
Protegido por estas monásticas severidades e aristocráticas
dignidades das ciladas e perturbações da vida mundana, a rotina de minha
infância tornou"-se imutável, como o nascer e o pôr do sol para um
filhote no ninho. Pode parecer surpreendente para muitos de meus
leitores que eu lembre com o maior prazer a época em que minha vida foi
mais rotineira e solitária. A chegada de minha prima Mary a nosso lar
coincidiu com minha introdução aos mestres anteriormente mencionados, e
com outras mudanças na organização e nos objetivos de meu dia, os quais,
à medida que o tornavam mais interessante, também perturbavam sua
tranquilidade. A ideia do sucesso na escola ou na universidade, a mim
apresentada por meus mestres, era ignóbil e desconfortável, em
comparação com a culpa pesarosa de minha mãe, ou seu simples elogio: e
Mary, embora de disposição moderadamente alegre e inteiramente amigável,
inevitavelmente tocava o coração do lar com a tristeza de sua orfandade,
cuja harmonia também era perturbada pela diferença de sentimentos, a
qual minha mãe não podia evitar demonstrar, entre os dirigidos a sua
sobrinha e a seu filho.

%152. 
E embora eu tenha me detido, com gratidão, nas muitas alegrias e
vantagens daqueles anos de isolamento, o leitor atento não terá
interpretado, espero, este relato como elogio da educação doméstica como
a que tive nos arredores de Londres. Mas havia nela um benefício maior,
até agora não mencionado; o fato de que grande parte de minha aguda
percepção e profundo sentimento da beleza da arquitetura e da paisagem
exteriores, deveu"-se ao desenvolvimento do hábito de restringir minha
felicidade às quatro paredes de tijolos de nosso jardim de cinquenta por
cem jardas; e aceitar, com resignação, a estética de um subúrbio de
Londres, além daquela de uma capela de Londres. Pois a do Dr.\,Andrews
era o tipo perfeito da capela londrina, definível, com precisão, como
uma basílica romana, --- uma nave oblonga, de teto plano, iluminada por
janelas com extremidades superiores semicirculares sobre arcos de
tijolos, preenchidas com pequenas vidraças mantidas por barras de ferro,
como finas tessituras de aranhas; galerias sustentadas por pilares de
ferro dos dois lados; bancos, bem isolados uns dos outros por tabiques
de abeto com trancas de bronze bem polidas, ocupavam todo o interior da
nave, com exceção de duas passagens laterais em trançado de palha; o
púlpito, sublimemente isolado, central e afastado das grades do altar;
um sólido compartimento de lambris bem aplainados apoiado sobre quatro
pés, tão alto quanto as galerias frontais, decorado com uma almofada de
veludo carmesim de seis polegadas de espessura, com borlas douradas nos
cantos; que era de grande valia para mim quando estava cansado do
sermão, porque gostava de observar a rica coloração das dobras e pregas
quando o clérigo a golpeava.

%153. 
Imagine as mudanças entre um domingo e outro, --- do serviço
religioso matinal nesse edifício, assistido por famílias de pequenos
comerciantes de Walworth Road, com seus adornos dominicais;(a mulher de
nosso encanador, a gorda e boa Mrs.\,Goad, sentada no banco em frente ao
nosso, severamente susceptível à interrupção de sua devoção pelos nossos
atrasos); agora, desta situação à grande missa na catedral de Rouen, com
sua nave repleta de camponeses com toucas brancas de metade da
Normandia!

O contraste entre a arquitetura urbana que me era familiar e aquela de
Flandres e da Itália, como manifestação do gosto e do poder dos
comerciantes, não era menos maravilhoso ou encantador. A casa comercial
de meu pai ficava no centro da Billiter Street, há alguns anos
desaparecida da visão e da memória dos homens, mas que, então,
representava com perfeição a cidade inglesa. Agora construímos as
fachadas das casas para que sirvam como propaganda, gastando centenas de
milhares de libras numa máscara dissimuladora de nossa bancarrota. Mas,
no tempo de meu pai, tanto o comércio quanto as construções ainda eram
honestas. Sua casa de comércio era uma peça de cerca de quinze por vinte
pés, com escrivaninhas para dois empregados, e um pequeno armário para
amostras de \textit{sherry}, tendo em frente uma peça maior para refinadas
recepções privadas a visitantes elegantes, ou para se servir o jantar de
meu pai se ele tivesse que permanecer até tarde na cidade. O andar
térreo era ocupado pela Messrs.\,Wardell and Co., retalhistas de vinho em
garrafa, creio, com a qual meu pai mantinha relações amigáveis. A única
propaganda do negócio era uma placa de bronze sob a campainha, com a
inscrição ``Ruskin, Telford and Domecq'', polida até o brilho pela
única empregada do estabelecimento, a velha Maisie, --- uma abreviação ou
apelido carinhoso terminado em ``sie'', de um nome de batismo que não
sei precisar --- Marion, creio, como Mary deriva de Mause. Toda a casa,
com três andares e sótãos, estava sob sua autoridade, com a ajuda, sem
dúvida, de uma assistente matinal que arrumava a casa, --- cozinhar,
atender às pessoas, abrir a porta para visitantes ilustres, tudo era \label{153}
feito por Maisie; esperava"-se, naturalmente, que os visitantes se
anunciassem com batidas na aldrava proporcionais à posição que ocupavam
na sociedade. Os homens de negócio tocavam a campainha já mencionada,
(redonda, as muitas camadas anuais de pintura transformadas pelo
polimento numa bela gradação, como as de uma ágata;) e eram admitidos
através da abertura do trinco pela mão do principal empregado da casa,
sem que ele se movesse de seu assento.

%154. 
Este estabelecimento despretensioso, como disse, fazia parte do
lado ocidental da Billiter Street, uma via estreita --- deveria ter
trinta pés de largura --- pela qual dois carretões de cervejeiro, apenas
com muito cuidado, conseguiam passar um pelo outro. Não estou seguro de
que isso fosse possível nas extremidades da rua; provavelmente apenas em sua metade, 
no ligeiro alargamento diante da cervejaria.
Efetivamente, uma mera passagem estreita entre casas de três pavimentos
construídas com maravilhosos trabalhos em tijolos, muito bem assentados,
e não apresentando outro divertimento para o observador esteta que a
alternância de ângulos e lados das fileiras de tijolos belamente
alinhadas, e a habilidosa irradiação daqueles que formavam as vergas das
janelas.

Tipicamente, repito, uma construção do grupo de edifícios londrinos, a
leste de Mansion House, que se estendia até a Torre; a graciosidade
destas construções abaixo da colina estava"-me, entretanto, naqueles dias
da infância, inteiramente interdita, por temor que eu pudesse cair nas
docas; mas as ruas Fenchurch e Leadenhall eram"-me familiares como
representando à perfeição a grandeza do estado mercantil britânico, --- o
leitor deverá fazer um esforço para, ainda que de modo impreciso,
conceber o efeito sobre minha imaginação dos fantásticos espigões de
Ghent e dos \textit{cortiles}\footnote{Em italiano no original.
  \textit{Cortile}: pátio interno circundado por uma arcada,
  característico dos palácios do Renascimento italiano (século \textsc{xv}).} recendendo a laranja de Gênova.

%155. 
Apenas com muita dificuldade, recorrendo ao princípio da
resignação, posso explicar a imaculada tranquilidade do prazer com o
qual, após as infinitas excitações das terras estrangeiras, meu pai
retornava à sua escrivaninha em frente à parede de tijolos da cervejaria
e eu ao meu nicho atrás da chaminé da sala de visitas. Mas, para nós
dois, as ocupações regulares, a amada rotina, e os sagrados hábitos do
lar eram mais preciosos que toda a fervorosa admiração pelas coisas
novas, ou o deleite com as cenas de incomparável beleza. Muito cedo, de
fato, considerei que a novidade se exauria rapidamente, e que a beleza,
embora inexaurível, após certo ponto ou período de entusiasmo, não podia
mais ser desfrutada; mas não é tão frequentemente observado pelos
filósofos que um lar, saudavelmente organizado, seja algo deleitável?;
além disso, o intenso frêmito de prazer que perpassava minha mente e
coração, quando, mesmo após um mês ou dois de ausência, tinha a primeira
visão do cume de Herne Hill e buscava cada curva da bem conhecida
estrada e cada ramo das árvores familiares era --- embora não tão
profundo ou arrebatador --- mais intimamente e vitalmente poderoso que as
mais brilhantes paixões de alegria em terras estrangeiras, ou mesmo em
novas paisagens em meu próprio país. Para minha mãe, as obrigações
domésticas habituais, a leitura comigo e Mary, a oportunidade de uma
conversa com Mrs.\,Gray, e a imperturbável preparação do retorno de meu
pai, e de uma noite tranquila, significavam mais que todo o esplendor ou
as maravilhas do globo entre os polos e o equador.

%156. 
E assim retornamos --- repletos de novos pensamentos, e fiéis aos
antigos, à gloriosa tranquilidade do lar, no fim do ano de 1833. Uma
sombra imprevista pairava no céu de seu horizonte encantador.

A cada dia em Cornhill, Charles revelava"-se mais agradável e correto a
todos que o conheciam. Como um rapaz que passava todo o dia em Londres
podia exibir uma compleição tão luminosa, e cachos de cabelo anelados
como os de Aquiles --- e todo o alegre espírito da mãe de Croydon --- não
era algo facilmente concebível; mas ele se tornou uma perfeita
combinação da vivacidade de Jin Vin\footnote{Personagem do livro
  \textit{The Fortune of Nigel}, de Sir Walter Scott.} com
a constância de Tunstall,\footnote{Personagem do livro \textit{The Fortune
  of Nigel}, de Sir Walter Scott.} e imune ao charme de
qualquer inatingível Margaret, pois seu empregador não tinha filha; mas,
o que era pior, um filho: então as possibilidades de ascensão, que todo
aprendiz busca, pareceram a Charles nulas naquela casa, além da posição
de caixeiro, ou talvez a de empregado"-chefe. Seu irmão mais velho, que o
ensinara a nadar jogando"-o no canal do Croydon, estava progredindo
rapidamente como comerciante de diversos produtos na Austrália, e
naturalmente ansiava por ter seu amado irmão como sócio. Logo foi
decidido que Charles deveria ir para a Austrália. O Natal de 1833 foi
pesaroso, pois eu estava muito triste; e Mary ainda mais que eu: e meus
pais, embora em seus corações não houvesse lugar para ninguém além de
mim, preocupavam"-se com a ida de Charles para um lugar tão distante;
mas, pensavam, honesta e justificadamente, que era para o bem do jovem.
Creio que tudo relativo à viagem, os apetrechos de Charles, a cabine no
navio, a recomendação especial ao capitão da embarcação, foi decidido em
menos de duas semanas, e então ele foi a Portsmouth para embarcar
alegremente, tendo o mundo a conquistar. Pelos correios chegou a notícia
de que estava ancorado ao largo de Cowes, e que o navio não podia zarpar
devido ao vento do oeste. Cartas sucederam"-se a cartas, e o vento do
oeste ainda soprava. Gostávamos da existência do vento do oeste, mas
representava um prolongamento da despedida que nos era dolorosa, embora
Charles escrevesse que se divertia muito, que o capitão era agradável, e
que fizera amizade com cada marinheiro a bordo, além dos passageiros.

%157. 
E o vento do oeste ainda soprava. Não lembro por quanto tempo --- \label{157}
talvez dez dias ou duas semanas, creio. Finalmente, um dia minha mãe e
Mary foram com meu pai à cidade para compras ou alguma visita agradável;
e fiquei sozinho em casa, ocupado com algo que muito me comprazia, não
sei o quê; mas quando ouvi que a família retornava, subia as escadas até
a sala de visitas, desci precipitadamente para a sala, e comecei a
contar"-lhes sobre esta grande felicidade que me sucedera, seja lá qual
fosse. Eles ficaram paralisados como estátuas, meu pai e minha mãe muito
graves. Mary olhava pela janela --- a mais afastada da porta, das três
frontais. À medida que continuei, jactando"-me, ela se virou subitamente,
a face coberta de lágrimas, olhou"-me nos olhos, aproximou o rosto do meu
para que pudesse ouvir seu sussurro entre soluços: ``Charles se foi.''

%158. 
O vento do oeste ainda soprava, franco e forte, e no dia anterior
bafejara uma brisa fresca em torno da ilha, em Spithead, exatamente o
tipo de brisa que carrega as nuvens, e encrespa as ondas, no Gosport de
Turner.\footnote{Quadro do pintor inglês William Turner (1775--1851).}

O navio enviava o bote à praia para obter água, ou algo do gênero --- um
pequeno cúter, ou algum tipo de veleiro. O mar estava agitado e os
marinheiros, e, creio, também um ou dois passageiros, tiveram certa
dificuldade em embarcar. ``Posso ir também?'', disse Charles ao capitão,
em pé, enquanto os via descer pelo costado. ``Você não tem medo?'',
perguntou o capitão. ``Nunca tive medo de nada em minha vida'',
respondeu Charles, aproximou"-se do bordo do navio e saltou para dentro
do barco.

O barco não tinha se afastado cinquenta jardas do navio quando virou,
porém havia outros barcos navegando em torno dele, como um enxame de
moscas no alto verão. Dois ou três deles se precipitaram para o local, e
todos foram salvos, exceto Charles, que afundou como uma pedra.

\begin{center}
22 de janeiro de 1834
\end{center}

Apreendemos o fato pouco a pouco. Por um ou dois dias não pudemos
acreditar no que acontecera; pensamos que Charles pudesse ter sido
recolhido por algum outro barco e levado para o mar. Afinal, fomos
avisados que seu corpo dera à praia em Cowes, e que seu pai se dirigira
ao local para enterrá"-lo. Após a cerimônia, e tendo tido conhecimento de
todos os detalhes, pois o navio ainda se encontrava ancorado, ele veio a
Herne Hill para contar à ``tiazinha'' de Charles toda a história. (O
velho sempre chamava minha mãe de tiazinha.) Isso aconteceu pela manhã,
no salão da frente --- minha mãe tricotava no lugar habitual próximo à
lareira, eu desenhava, ou fazia algo do gênero, no lugar de sempre. Meu
tio contou toda a história da maneira calma e tranquila como comumente
fazem os ingleses, até que no fim começou a soluçar, dizendo (ainda
posso ouvir as palavras), ``Eles o agarraram pelo boné, mas não foi
possível salvá"-lo.''


\chapter{\textit{Vester, camenae}} %Capítulo \textsc{viii}. 

%159. 
A morte de Charles fechou novamente as portas de meu coração; e a
quietude autoabsorvente da vida em Herne Hill continuou por mais um ano,
com pouco a ser lembrado, e menos a ser contado. Meus pais, contudo,
fizeram um esforço para obter companhia saudável para mim, ao que
provavelmente devo mais do que naquele momento percebi.

Seis ou sete portões abaixo, descendo a colina em direção ao campo (devo
apresentar os mais sinceros agradecimentos ao seu atual proprietário, Mr.\,Sopper, por tê"-lo reaberto à visão do público; a grande perda ocasionada
pelo fechamento, tanto para os vizinhos como para o público, foi por mim
descrita num trecho acima),\footnote{Conferir capítulo \textsc{ii}, p.\,\pageref{herne}.} descendo seis ou sete portões, um belo gramado, sob a sombra
de um cedro com galhos baixos, abria"-se em frente a uma casa encantadora
e extremamente bem cuidada, onde viviam duas pessoas, com um estilo de
vida modesto como o de meus pais, --- Mr. e Mrs.\,Fall; muito felizes por
terem um filho e uma filha em vez de um único filho. O filho, Richard,
era um ano mais jovem que eu, mas já frequentava a escola em Shrewsbury,
e estava, portanto, um pouco à minha frente nas disciplinas
curriculares; extremamente afável e de boa natureza --- sua irmã, ainda
mais jovem, uma criança inteligente, era companhia constante da mãe:
ambos de maneiras despretensiosas, mas de conduta impecável, eram
exemplos das virtudes de Herne Hill, verdadeira religiosidade e
conhecimentos úteis. Ainda tremo ao lembrar o dia em que Mrs.\,Fall
franziu os sobrolhos ao ouvir"-me pronunciar ``\textit{naïveté}'' como
``\textit{navette}''.\footnote{\textit{Naïveté}: ingenuidade.}

%160. 
Creio que foi em 1832 que meu pai, ao perceber, com grande
admiração, a conduta irrepreensível dessa família, escreveu a Mr.\,Fall,
solicitando cortesmente que fosse permitido aos ``dois garotos'', quando
Richard estivesse em casa, fazer suas tarefas de férias ou se divertirem
juntos, caso isso lhes comprouvesse. A proposta foi gentilmente aceita:
os dois garotos travaram conhecimento, --- concordaram com o proposto, ---
e, como nesta época eu tinha obtido a posse de um estúdio, todo para
mim, enquanto Richard dispunha apenas de seu próprio quarto (sujeito aos
conselhos e à intrusão fraterna), o curso que as coisas seguiram foi
que, geralmente, quando Richard estava em casa, ele subia a colina,
passando pelos setes portões, por volta das dez horas da manhã; fazia as
lições que tinha que fazer na mesma mesa que eu, ocasionalmente
ajudando"-me com as minhas; e então saíamos para um passeio vespertino
com Dash, Gipsy, ou qualquer um dos cães que estivesse disponível.

%161. 
Não ouso afirmar que a neve daqueles Natais fosse mais branca do
que a de agora, embora tenha boas razões para acreditar que permanecesse
branca por mais tempo. Mas afirmo categoricamente que a neve que
costumava cair nos arredores de Londres era mais espessa do que a que tem
caído nos últimos vinte ou vinte e cinco anos. Era comum se encontrar,
nas depressões das Norwood Hills, as cercas dos campos soterradas sob
ondas de neve cristadas, quando então, dos pontos mais altos, metade dos
condados de Kent e de Surrey brilhava no horizonte como um mar ártico
sem nuvens e aprazível.

Richard Fall era muito espirituoso, de caráter sensato e prático; mas
não tinha gostos definidos; e certa aversão aos \textit{meus} estilos,
tanto em arte quanto em poesia. Declinava secamente do julgamento dos
méritos de minhas composições; e embora fosse agradavelmente cordial no
dia a dia, adotava a posição de quem tolera minha companhia e não a de
quem tem orgulho do privilégio de conhecer um escritor em seus
primórdios. Nunca era indelicado ou sarcástico; mas ria inexoravelmente
de mim se escrevesse em mal inglês para obter rimas, ou demonstrasse
falta de sentido em prosa ou em verso. Pouco a pouco, acostumamo"-nos a
estarmos juntos, e mais tarde, ao longo da vida, ficávamos felizes se o
acaso nos reunisse novamente.

%162. 
O ano de 1834 passou sem grandes acontecimentos mas com pouco
avanço nas quatro direções anteriormente descritas,\footnote{Conferir
  Capítulo \textsc{vi}, p.\,\pageref{139}.} perseguidas para meu próprio
prazer; --- com esporádicos esforços estéreis nos clássicos, cujo estudo
não sentia que me trouxesse nem que pudesse me trazer qualquer
benefício.

Sem \textit{grandes} acontecimentos, disse --- menosprezando um pouco as
travessuras que um garoto bem"-intencionado, virtualmente sem mestre,
poderia fazer agindo por conta própria, e diariamente persuadindo"-se
seriamente de que sua maneira de agir fosse sempre a melhor.

Não posso analisar, pelo menos sem fazer um esforço maior do que,
suponho, o leitor estaria disposto a fazer comigo, a mescla de coisas
boas e ruins proveniente da literatura de terceira categoria que eu
preferia aos clássicos em latim. Meu volume de
\textit{Forget"-me"-not},\footnote{Conferir capítulo \textsc{v}, p.\,\pageref{102}.}
que trazia aquela preciosa gravura de Verona (curiosamente também outra
de Prout, de São Marcos em Veneza), estava algo acima do nível das
publicações anuais em qualidade de impressão; e continha três histórias,
``The Red"-nosed Lieutenant'', pelo reverendo George Croly; ``Hans in
Kelder'', pelo autor de ``Chronicles of London Bridge''; e ``The
Comet'', por Henry Neele, Esq.,\footnote{Conferir capítulo \textsc{iii}, nota 4.} %Não encontrei a nota
 as quais, de maneiras diferentes, causaram"-me grande
impressão. O hábito, parcialmente infantil, parcialmente estúpido, ou
mesmo idiota, como já foi dito, que eu tinha de contemplar longamente as
mesmas coisas durante todo o dia, manifestava"-se também na leitura, de
modo que eu podia ler as mesmas coisas o ano inteiro. Como não era
vantajoso nem meritório lembrar circunstâncias ficcionais, eu tinha mais
orgulho de minhas habilidades de esquecimento, o que me permitia logo
redescobrir o prazer daquelas histórias; e creio que estas, favoritas, e
algumas tantas outras semelhantes, menos importantes, fossem lidas umas
vinte vezes por ano, no período inicial de minha adolescência.

%163. 
Surpreendo"-me um pouco por me ter sido permitido sentar tanto tempo
naquele canto da sala de visitas, tendo apenas meu \textit{Italy}, de
Rogers, meu \textit{Forget"-me"-not}, o \textit{Continental Annual},\footnote{\textit{Anuário
  Continental.}} e \textit{Friendship's
Offering}\footnote{\textit{Oferta de Amizade.}} como
biblioteca de trabalho; e surpreendo"-me ainda mais que meu pai, em sua
apaixonada esperança de que eu pudesse um dia escrever como Byron, nunca
tivesse percebido que o precoce talento deste autor devia"-se a extensas
leituras dos mestres de cada gênero literário, algo, creio, que não
encontra paralelo em nenhuma outra juventude, seja de estudante ou de
escritor. Mas eu era completamente incapaz de tal trabalho intelectual,
e o verdadeiro talento que eu tinha para desenhar consumia em sua
prática a melhor energia do dia. ``Hans in Kelder'' e ``The Comet'' eram
meu descanso.

Não sei quando meu pai começou a me ler Byron, com a expectativa de que
eu gostasse dele --- toda minha instrução, após a \textit{Ilíada}, havia
sido através de Scott; mas deve ter sido no início da adolescência,
se não me lembraria de seus primeiros efeitos. De
\textit{Manfred}\footnote{Poema dramático de Lord Byron, contendo
  elementos sobrenaturais, escrito entre 1816 e 1817.}
evidentemente gostei, como de \textit{Macbeth}, por causa das bruxas.
Várias mudanças questionáveis foram introduzidas, contudo, naquele 1831,
na passagem para os meus doze anos, na monástica disciplina de Herne
Hill. Foi"-me permitido provar vinho; fui levado ao teatro; e, nos dias
festivos, até mesmo jantar com meus pais, às quatro horas: quando
geralmente, à sobremesa, meu pai lia algo que em outras circunstâncias
seria suspeito: as \textit{Noctes} \textit{Ambrosiane},\footnote{Obra em
  quatro volumes do escritor escocês John Wilson (``Christopher North'')
  (1785--1854).} à medida que foram sendo publicadas ---
sem omitir as palavras de baixo calão; e, por fim, o naufrágio em
\textit{Don Juan},\footnote{Longo poema satírico de Lord Byron.} --- a partir do qual, percebendo meu interesse,
meu pai continuou até quase chegar ao fim da obra. Lembro que ele e
minha mãe olharam"-se através da mesa com algum alarme, quando, ao me
perguntar, numa refeição festiva posterior, qual deveria ser a leitura
após o jantar, eu respondesse imediatamente ``Juan and Haidée''.\footnote{\textit{The
  Poets's Song to Don Juan and Haydée}: título de um poema de cunho
  erótico da autoria de Lord Byron.} Minha sugestão
não foi aceita e, percebendo que havia alguma coisa errada com ela, não
insisti, tentando mesmo balbuciar alguma desculpa, o que piorou a
situação. Leram, então, alguns versos do \textit{Childe Harold},\footnote{\textit{Childe Harold' Pilgrimage}, 
título de um famoso poema narrativo de Lord Byron.} dos quais gostei muito àquela época; e, \label{childe}
sinceramente, a história de Haidée logo se tornou muito triste para mim. 
Mas, seguramente, no fim daquele ano de 1834, conhecia bem a obra de
Byron, com a exceção de \textit{Cain}, \textit{Werner}, \textit{Deformed}
\textit{Transformed} e \textit{Vision of Judgment},\footnote{Obras de Lord Byron.}
nenhuma das quais eu podia entender, nem papai e mamãe pensavam que eu
deveria tentá"-lo.

%164. 
O leitor singelo talvez possa ficar surpreso com o fato de mamãe
ter concordado com tudo isso, de modo que se torna necessário revelar
aqui algumas peculiaridades de seu recato, as quais não poderia
descobrir por si mesmo, a partir do que até agora foi dito sobre ela. O
leitor poderia inclusive supor que, após me ter feito ler toda a Bíblia
pelo menos seis vezes, ela não temia minha exposição à linguagem franca;
mas poderia não perceber que na energia e na afetividade de seu caráter
ela era tão simpática a tudo que fosse nobre e belo em Byron como meu
pai; e que seu puritanismo não excluía o bom senso de perceber que,
enquanto Shakespeare e Burns\footnote{Robert Burns (``Rabbie Burns''):
  poeta escocês (1759--1796).} permaneciam abertos sobre a
mesa durante todo o dia, não havia motivo para muito mistério acerca de
Byron (embora até mais tarde não me fosse permitido lê"-lo sozinho). Ela
confiava em meu caráter e em minha educação, e não tinha mais receio que
eu me tornasse um Corsair\footnote{Personagem de um poema homônimo de
  Lord Byron.} ou um Giaour\footnote{Personagem de um \label{giaour}
  poema homônimo de Lord Byron.} do que um Ricardo
\textsc{iii}\footnote{Personagem do drama histórico homônimo de William
  Shakespeare.} ou um --- Solomon.\footnote{Provável
  referência ao personagem bíblico Salomão.} E ela
estava perfeitamente certa. Nunca recebi a menor influência maléfica de
Byron: o mal que chegou a mim veio dos fatos da vida e de livros de
nível mais baixo, incluindo"-se uma grande variedade de obras de autores
popularmente considerados como altamente instrutivos --- de Victor Hugo
ao Doutor Watts.

%165. 
Posteriormente, deter"-me"-ei a explicar o que quis dizer ao
mencionar que minha mãe tinha um recato ``inofensivo''. Ela era tão
austera quanto Alice Bridgenorth;\footnote{Personagem da novela
  \textit{Peveril of the Peak} {[}\textit{Peveril do Cume}{]}, de Sir Walter
  Scott.} mas entendia a doutrina da religião que lhe
fora ensinada e, sem ostensivamente chamar a si mesma de miserável
pecadora, sabia que, de acordo com aquela doutrina, e provavelmente
também na realidade, Madge Wildfire\footnote{Personagem da novela
  \textit{The Heart of Midlothian} {[}\textit{O Coração de Midlothian}{]},
  de Sir Walter Scott.} não era mais pecadora que ela.
Era como a irmã em sua caridade universal --- tinha simpatia por todo
sofrimento, assim como por toda virtude, inerentes ao mundo feminino; e,
no fundo de seu coração, talvez gostasse tanto da verdadeira Margherita
Cogni\footnote{Amante veneziana de Lord Byron, iletrada e de
  temperamento violento, que foi abandonada pelo poeta.}
quanto da esposa ideal de Faliero.\footnote{Marino Faliero (1285--1355):
  quinquagésimo --- quinto Doge de Veneza; título de um drama de Lord
  Byron e de uma opera de Gaetano Donizetti.}

%166. 
E há mais um aspecto do caráter de minha mãe que deve ser
imediatamente asseverado para acabar com a ideia, da qual percebo ecos
em comentários de jornais sobre minhas descrições anteriores dela, de
que era em tudo semelhante à religiosa tia Esther de \textit{Black House}. Muito
pelo contrário, seu riso era vigoroso, franco, e algumas vezes até
irreprimível! Nunca sardônico, embora às vezes com um toque
\textit{smollettesco}!\footnote{De Tobias George Smollett (1721--1771):
  escritor escocês, conhecido por suas novelas picarescas.} De modo que, juntos, ela e meu pai liam com grande prazer a
Humphry Clinker,\footnote{\textit{The Expedition of Humphry Clinker}:
  título de um romance picaresco de Tobias Smollett.}
muito antes que \textit{eu} pudesse entender sua graça e seu conteúdo.
Mais ainda, ela se regozijava inocuamente com a realidade de Smollet.
Muitos anos depois desta época, numa de nossas passagens pelo
Simplon,\footnote{\textit{Simplon Pass}: passo alpino entre a Suíça e a
  Itália.} exatamente no topo, quando paramos para
admirar a paisagem a nossa volta, minha ama"-seca Anne sentou"-se para
descansar num parapeito ao lado da estrada, exatamente em frente do
monastério; --- do lado da estrada no qual o terreno descia abruptamente
após a cerca. Virando"-se para ver o panorama pitoresco, Anne perdeu o
equilíbrio e rolou de costas pelo declive. Meu pai não conseguiu se
conter e sugeriu que ela tinha feito aquilo de propósito, para o
divertimento dos Santos Pais; e nem ele nem minha mãe puderam depois
falar da ``performance'' (como a chamaram) sem rir durante quinze
minutos.

%167. 
Se, entretanto, houvesse qualquer traço de amargura ou ironia num
gracejo, minha mãe o rejeitava; mas meu pai e eu o apreciávamos ainda
mais, caso refletisse a realidade; e, na medida em que pude
compreendê"-lo, regozijei"-me com o sarcasmo de \textit{Don Juan}. Mas,
minha firme decisão, desde que me senti à vontade com seus outros
cantos, de que Byron seria meu mestre em verso, como Turner fora na cor,
foi tomada, claro, naquele período balbuciante da existência, sem
consciência dos profundos instintos que a motivaram: apenas duas coisas
reconheço, conscientemente, que a verdade advinda das observações de
Byron era a mais exata, e que o modo escolhido para exprimi"-las era o
mais pleno de significados que jamais encontrei na literatura. Por
aquela época, meu pai tinha me apresentado os dois primeiros livros de
Livy,\footnote{Titus Livius (provavelmente 59 a.C.--17 d.C):
  historiador romano.} e eu conhecia, consequentemente,
o que era uma linguagem precisa; mas percebi que o estilo de Livy, assim
como posteriormente o de Horácio\footnote{Quinto Horácio Flaco (65--8 a.C.): poeta (lírico e satírico) e filósofo romano.} e Tácito, era cuidadoso, frequentemente trabalhado, e algumas
vezes obscuro, intenso: enquanto Byron escrevia tão facilmente como um
falcão voa e tão claramente como um lago reflete a luz, a exata verdade
apresentada de modo preciso, em termos minuciosamente escolhidos; não
apenas a verdade exata, mas aquela mais essencial e útil.

%168. 
Naturalmente, naquela época eu era tão incapaz de avaliar a
grandeza de Byron quanto a de Turner; mas percebia que ambos concordavam
com todas as coisas que \textit{eu} considerava boas em oposição às más;
e, portanto, eles deveriam ser meus mestres, cada um em seu próprio
domínio. O leitor moderno, para não dizer também o acadêmico moderno, é
geralmente tão ignorante das qualidades essenciais de Byron, que não
posso prosseguir na história de meu próprio noviciado sob sua tutela sem
ilustrar, com um exemplo rápido, os aspectos que acredito serem
inigualáveis em sua obra.

Com este propósito, remeto"-me à sua prosa comum, ao invés de a seus
versos, porque a sequencia de seus ritmos envolve questões que não me
interessam. Leia"-se, por acaso, esta passagem sobre Sheridan,\footnote{Certamente
  Richard Brinsley Sheridan (1751--1816): dramaturgo e político
  \textit{whig} irlandês.} em sua carta a Thomas
Moore,\footnote{Thomas Moore (1779--1852): poeta, compositor e cantor
  irlandês.} escrita em Veneza em 1º de junho (ou da
madrugada de 2 de junho!) de 1818: %---

\begin{quote}
Os Whigs o insultaram; contudo, ele nunca os deixou, e tais
desastrados não merecem crédito ou compaixão. Quanto a seus credores ---
lembre"-se que Sheridan nunca teve um xelim, e fora arrebatado, por
grandes forças e paixões, para o turbilhão do mundo, e colocado sobre o
pináculo do sucesso sem meios externos que o pudessem sustentar nesta
ascensão. Fox pagou \textit{seus} débitos? Ou Sheridan fez uma subscrição?
A bebedeira de\ldots{} foi mais desculpável que a dele? Suas intrigas eram
mais notórias que a de seus contemporâneos? Sua memória deve ser
amaldiçoada e a deles respeitada? Não se deixe levar pelo clamor mas o
compare com Fox,\footnote{Provavelmente William Henry Fox Talbot
  (1800--1877): escritor e cientista inglês, pioneiro da fotografia.} ``o coligador'', e com o pensionista Burke,\footnote{Edmund
  Burke (1729--1797): filósofo e político anglo"-irlandês.}
como homem de princípio; e com milhões de outros desígnios pessoais; e
com alguém sem talento, pois Sheridan supera a todos estes. Sem
recursos, sem conexões, sem caráter (o que pode parecer falso, à
primeira vista, e levá"-lo à loucura pelo desespero), ele ganhava de
todos, em tudo em que se empenhava. Meu Deus! Pobre natureza humana!
Boa noite, ou melhor, bom dia. São quatro horas, e a madrugada surge
sobre o Grande Canal e desensombra o Rialto.
\end{quote}

%169. 
Agora, observe que esse trecho é nobre, principalmente porque
contém o máximo de pensamentos, absolutamente justos, sábios e
generosos, que poderiam ser apresentados juntos naquele espaço. Contudo,
é mais que nobre, é \textit{perfeito}, porque a quantidade de pensamentos
que contém não é artificial ou concentrada de maneira embaraçada, mas
são expostos com a serena celeridade com a qual o martelo do ferreiro
golpeia o ferro quente; e com uma escolha de palavras, na qual cada uma
ocupa o lugar correto, o que faz com que transmitam muito mais do que
significam no dicionário. Desse modo, ''\textit{however}''\footnote{Como
  conjunção, significa ``porém'', ``contudo'', ``todavia''.} é usado em lugar de ``\textit{yet}''\footnote{Como conjunção,
  também significa ``porém'', ``contudo'', ``todavia''.}
porque significa ``\textit{howsoever}'',\footnote{Advérbio, que significa
  ``de qualquer forma'', ``por mais que'', ``com tudo isso'', e do qual
  teria derivado \textit{however}.} ou, integralmente,
``\textit{yet whatever they did}''.\footnote{A despeito do que eles tenham feito.} ``\textit{Thick}'' \textit{of society},\footnote{O turbilhão do
  mundo, da sociedade.} porque significa não apenas a
multidão, mas a \textit{confusão} provocada por ela; ``\textit{ten hundred
thousand}''\footnote{Dezenas, centenas, milhares. Ruskin refere"-se ao
  uso que faz Byron das unidades do sistema decimal de numeração,
  apresentadas sem vírgulas entre elas, como recurso estilístico para
  que o leitor perceba que se trata de uma multidão de anônimos.} ao invés de ``um milhão'' ou ``mil milhares'', para
expressar a magnitude do número, e fazer"-nos sentir que se trata de
muitos anônimos. Então vem a frase entre parênteses, ``\textit{which might
be false}'',\footnote{Que poderia ser falso.} etc.,
que é realmente obscura, porque é impossível entendê"-la sem uma pausa
artificial e perda de tempo; e o leitor deve recorrer à própria
inteligência para desenvolvê"-la por si mesmo em ``\textit{it was,
perhaps, falsely said of him at first, that he had no
character}'',\footnote{Foi, talvez, falsamente dito, inicialmente, que
  ele não tinha caráter.} etc. Por fim, a madrugada
\textit{``unshadows''}\footnote{Desensombra.} --- diminui a
sombra sobre o Rialto, --- mas não o ilumina como às águas da laguna.

%170. 
A seguir, considere as duas frases sobre poesia, contidas nas
cartas a Murray, de 15 de setembro de 1817 e 12 de abril de 1818 (para
avaliar a força dessas afirmações, compare"-as com as declarações
publicadas, em resposta a Blackwood, em 1820).

\begin{quote}
\begin{enumerate}[label=(\scshape\alph*)]
%[label=(\arabic*)]
% \setcounter{enumi}{1817}
\item\,\textbf{15 de setembro de 1817}\quad Com respeito à poesia em geral, estou convencido, quanto mais penso no assunto, que ele (Moore), e todos nós --- Scott, Southey,\footnote{Robert Southey (1774--1843), poeta, prosador e historiador britânico; autor de \textit{História do Brasil} (1810--1819): primeiro trabalho panorâmico que abrange o período da colonização à chegada de Dom João \textsc{vi} ao Brasil.} Wordsworth,\footnote{William Wordsworth (1770--1850), um dos mais importantes poetas do romantismo inglês.} Moore, Campbell,\footnote{Thomas Campbell (1777--1844), poeta escocês, cuja poesia sentimental abordava com habilidade as questões humanas.} eu, --- estamos todos equivocados, um não menos que outro; que nos atemos a um sistema (ou sistemas) poético revolucionário equivocado, sem qualquer valor, do qual apenas Rogers e Crabbe\footnote{George Crabbe (1754--1832), poeta inglês.} escapam: e que a geração atual e as futuras acabarão tendo esta opinião. Obtive a confirmação disso após a leitura recente de alguns de nossos clássicos, particularmente Pope, cuja obra li da seguinte maneira: tomei os poemas de Moore, os meus, e os de alguns outros poetas, li"-os lado a lado com os de Pope, e fiquei realmente perplexo (não deveria ter ficado) e mortificado pela inefável distância em conteúdo, aprendizado, efeitos e mesmo imaginação, paixão, e invenção, entre o poeta da pequena Queen Anne e nós do Baixo Império. Dependente desta situação, esteve então Horace,\footnote{Quintus Horatius Flaccus (65--8 d.C.), poeta lírico romano durante o tempo do imperador Otávio (Augusto).} e agora Claudian,\footnote{Claudius Claudianus (fins do século 4 d.C. --- 404 d.C.), poeta latino de origem grega.} entre nós; e se eu tivesse que recomeçar, adaptar"-me"-ia ao modelo. Crabbe é o homem; mas ele escolheu um tema vulgar e impraticável, e\ldots{} aposentou"-se com metade do estipêndio, e fez o suficiente, a menos que devesse fazer como fizera no passado.

\item\, \textbf{12 de abril de 1818}\quad Pensei num prefácio defendendo Lord Hervey\footnote{Provavelmente John Hervey, Segundo Barão de Hervey (1696--1743), escritor, político e memorialista inglês.} contra o ataque de Pope, mas Pope --- seja Pope, o poeta --- contra todo o mundo, nas injustificáveis tentativas iniciadas por Warton\footnote{Provavelmente Thomas Warton (1728--1790), historiador literário, crítico e poeta inglês.} e mantidas até hoje pela nova escola de críticos e escrevinhadores que se consideram poetas porque \textit{não} escrevem como Pope. Não tenho paciência com tais abomináveis impostores e seu mau"-gosto; toda esta geração não vale um canto do \textit{Rape of the Lock},\footnote{\textit{O rapto da madeixa}, título de poema narrativo e satírico de Alexander Pope.} ou do \textit{Essay on Man},\footnote{\textit{Ensaio sobre o Homem}, título de um poema de caráter filosófico de Alexander Pope.} ou do \textit{Dunciad},\footnote{Título de um poema satírico de Alexander Pope.} ou ``de qualquer coisa desse gênero''. 
>>>>>>> 20a194fac21c961c04e6cddb981d3352e74fc726
\end{enumerate}
\end{quote}

%171. 
Não há nada que necessite explicação nesses dois fragmentos breves
e amenos, exceto, no primeiro deles, a singular e exaustiva enumeração
das qualidades da grande poesia, --- particularmente na ordem apresentada
por Byron.

\begin{quote}
\begin{enumerate}[label=(\scshape\alph*)]
\item\, \textbf{Conteúdo}\quad Significa que a primeira coisa que se deve pensar sobre
isso é se o poeta seria um homem sábio --- foi assim também na resposta a
\textit{Blackwood}, ``Eles o chamam (a Pope) o poeta da razão! --- existe
alguma razão pela qual ele não deveria ser um poeta?''

\item\,  \textbf{Aprendizado}\quad O lavrador de Ayrshire pode ter bons dotes, mas ele
não pode ser considerado como próximo a Homero, Dante ou Milton.

\item\,  \textbf{Efeito}\quad Existe \textit{eficiência} em seus versos? Eles falam ao
ouvido e ao espírito, ao mesmo tempo? Observe"-se o ``efeito'' sobre seu
público das ``oitavas'' da voz de Beatrice, na história relatava na
página 286 do livro \textit{Canções Rurais da Toscana}, de Miss
Alexander.\footnote{Francesca Alexander (1837--1917): ilustradora,
  escritora e tradutora (do italiano) norte"-americana, amiga de Ruskin,
  cujo livro, \textit{Roadside Songs of Tuscany}, este editou e ilustrou.}

\item\,  \textbf{Imaginação}\quad Colocada num dos últimos lugares porque muitos
romancistas e artistas possuem esta habilidade, embora não sejam poetas,
ou mesmo bons romancistas ou pintores; porque não sabem como utilizá"-la,
nem transformá"-la num efeito convincente.

\item\,  \textbf{Paixão}\quad Situada mais ainda para o fim da lista, porque todos os
bons homens e mulheres a possuem tanto quanto o poeta deveria ter.

\item\,  \textbf{Invenção}\quad Colocada em último lugar, porque se pode ser um bom poeta
sem a possuir. O próprio Byron tinha escassa imaginação, enquanto Scott,
embora a tivesse em grande quantidade --- nunca poderia ter escrito uma
peça de teatro.
\end{enumerate}
\end{quote}

%172. 
Mas nem a força e a precisão, nem o ritmo da linguagem de Byron,
constituíram os motivos decisivos pelos quais o elegi meu mestre.
Sabendo de memória o Cântico de Moisés e o Sermão da Montanha, além de
metade do Apocalipse, eu não tinha necessidade de mestres em relação à
majestade ou a simplicidade da língua inglesa; e para o encadeamento
lógico, eu tivera o próprio mestre de Byron, Pope, desde que pude
balbuciar. Mas o que havia de completamente novo e precioso para mim em
Byron, era sua \textit{verdade} viva e comedida --- comedida, em comparação
com Homero; viva, em comparação com todos os outros. Minha própria
batuta, inexoravelmente comedida --- não a do encantador, mas a do
artesão, a do construtor, --- reduzia à mera incredulidade todas as
declarações dos poetas considerados sublimes. Era inútil que Homero me
dissesse que Pelion fora colocado no topo do Ossa.\footnote{Ruskin cita
  trecho da \textit{Odisseia}, de Homero, na qual este autor escreve
  que dois personagens, Oto e Efialto, supostamente filhos de Efimédia com
  Posêidon, ameaçaram fazer guerra aos deuses no Olimpo, pretendendo
  ``colocar o Monte Ossa no topo do Monte Olimpo e o Monte Pelion no
  topo do Ossa'' (canto \textsc{xi}).} Eu sabia perfeitamente bem
que não havia ninguém no topo do Ossa. Era inútil que Pope me dissesse
que as árvores sob as quais sua amante o esperava coalesciam numa sombra
única, porque estava seguro de que elas não fariam nada semelhante. Não
apenas isso, mas o mundo inteiro, como me fora descrito tanto pela
poesia como pela teologia, estava se tornando, a cada momento, mais
quimérico e impossível. Regozijava"-me com todas as histórias sobre
Pallas\footnote{Personagem da mitologia grega, Pallas ou Palante, um dos
  titãs, era filho de Crios e Euríbia.} e Vênus, sobre
Aquiles\footnote{Também personagem da mitologia grega, Aquiles é um dos mais
  famosos heróis da Grécia Antiga; filho de Tétis e Peleu.} e Enéas,\footnote{Ainda outro personagem da mitologia grega, filho de
  Anquises e Vênus, e herói da \textit{Eneida}, de Virgílio.} sobre Elias e São João: mas sem que meu coração duvidasse de
serem eles espíritos reais de sabedoria e beleza, ou que tivessem sido
heróis invencíveis e profetas inspirados, já percebia, com crescente e
fatal tristeza, que não havia prova da existência de nenhum deles --- que
para mim eles não eram nem divindades"-guias nem mestres
proféticos; e que as histórias poéticas, sejam deste mundo ou do
próximo, eram para mim como as palavras de Pedro a seus discípulos mudos
--- ``são apenas fábulas; não acreditem nelas''.

%173. 
Mas, então, encontrei um homem que falava apenas do que tinha visto
e realmente sabia; e que falava sem exageros, sem mistérios, sem
hostilidade e sem piedade. ``As coisas são assim --- faça delas o
que quiser!'' Shakespeare disse que os Alpes expeliam sua reuma sobre os
vales, o que é verdadeiramente exato, tendo a verdade definitiva sobre a
questão sido apresentada por James Forbes,\footnote{James David Forbes
  (1809--1868), físico e geólogo escocês que viajou pelos Alpes e os
  estudou geologicamente, tendo publicado livros sobre suas observações.} mas de uma maneira mítica e com a desagradável
tendência dos britânicos para o vexatório. Mas Byron, ao dizer ``que a
massa glacialmente fria e inquieta avançava dia após dia'', descreveu
exatamente o que ele viu e sabia, --- e nada mais. Da mesma maneira,
\textit{Noites da Arábia}\footnote{Clássico da literatura oriental, sem
  autor definido, mais conhecido em português como \textit{As mil e uma
  noites.}} contou"-me sobre ladrões que viviam em
cavernas encantadas e de beldades que lutavam no ar com gênios; mas
Byron falou"-me de ladrões com os quais ele se escondeu nas montanhas em
que eles viviam, e de belos persas ou gregos que viveram e morreram sob
o mesmo sol que se levanta sobre as colinas de Norwood, visíveis aos
meus olhos.

E nessa verdade restrita mas exata, para Byron, como também já para mim,
o Amor parecia ter um caráter efêmero, e a Morte, ser terrível. Ele não
tentou me consolar pela morte de Jessie,\footnote{Conferir episódio da
morte de Jessie, Capítulo \textsc{iv}, p.\,\pageref{jessie}.} dizendo que ela
estava mais feliz no Paraíso; ou pela de Charles,\footnote{Conferir
episódio da morte de Charles, capítulo \textsc{vi}, p.\,\pageref{157}.}
dizendo que se tratava de uma dispensação à qual a Providência me
submetia sobre a Terra. Ele não me disse que a guerra era um preço justo
a pagar pela glória dos capitães, ou que o fato das mortes terem sido
ordenadas pelo governo da nação atenuava suas culpas. Ao analisar aquilo
que estivesse dentro do espectro do entendimento humano, ele se atinha
aos fatos, e discernia suas naturezas com justa precisão.

Mas mesmo com tudo que ele possa ter feito, não teria se tornado um
mestre para mim se não tivesse compartilhado meu amor reverencial à
beleza e minha indignada rejeição à fealdade. A bruxa do
Staubbach\footnote{Referência à Bruxa dos Alpes, personagem do poema
  gótico de Lord Byron, \textit{Manfred}. Ruskin visitou a catarata de
  Staubbach, localizada na cidade suíça de Lauterbrunnen, nos Alpes, e
  dela pintou uma aquarela, provavelmente em 1866.} em
seu arco"-íris era uma visão muito mais agradável que aquela de
Shakespeare, semelhante a um rato sem cauda, ou a de Burns,\footnote{Provavelmente
  John Elliot Burns (1858--1943), político inglês, antirracista e
  socialista, que apoiou o movimento sindicalista da Inglaterra.} em sua camisa curta. Conrad,\footnote{Provavelmente Joseph
  Conrad, pseudônimo de Jozéf Teodor Natecz Korzeniowski (1857--1924), escritor inglês
  de origem polonesa, que escreveu sobre marinheiros e o mar.} o rei do mar, apresentava, aos meus olhos, uma vantagem
imediata sobre os distantes, lânguidos, trigueiros e antigos marinheiros
de Coleridge;\footnote{Samuel Taylor Coleridge (1772--1834), poeta,
  crítico e ensaísta inglês; um dos expoentes do romantismo na
  Inglaterra.} e não importa o que Pope tenha
graciosamente dito, ou honestamente sentido dos bosques e das águas de
Windsor, ele estava apenas tilintando címbalos para mim, em comparação
ao amor de Byron por Lachin"-y"-Gair.

%174. 
Devo interromper, agora, a investigação das origens da influência
de Byron sobre mim, para que o leitor não confunda a análise que
apresentarei com uma descrição de meus sentimentos aos quinze anos. A
maioria deles, contudo, encontrava"-se, seguramente, em estado latente em
minha mente em desenvolvimento --- como o açafrão no croco ainda sob a
terra; e Byron --- embora não pudesse me ensinar a amar montanhas ou o
mar mais do que eu amara na infância, foi o primeiro a animá"-los com o
sentido da verdadeira nobreza e melancolia humanas. Ele me ensinou o \label{chillon}
significado de Chillon\footnote{Refere"-se ao poema de Lord Byron, 
  \textit{The Prisoner of Chillon} ou \textit{O Prisioneiro de
  Chillon}; sendo Chillon um castelo à margem do Lago Léman, de
  Genebra, visitado pelo autor.} e de
Meillerie,\footnote{Pequena cidade francesa em Haute"-Savoie, localizada às
  margens do Lago Léman, visitada por Lord Byron e
  mencionada por ele em seus diários.} e sugeriu buscar
em Veneza --- as ruínas das casas de Foscari\footnote{Refere"-se a Villa
  Foscari, construída para os irmãos Foscari por Andrea Palladio, entre
  1550 e 1560, nas proximidades de Veneza, classificada pela \textsc{unesco} como
  Patrimônio da Humanidade.} e Falier.\footnote{Refere"-se
  ao Palazzo Falier Canossa, construído à margem do Grande Canal de
  Veneza no século \textsc{xv}, por autor desconhecido.}

Observe que a força com a qual Byron se impôs dependia, novamente, da
existência, em seus personagens, de uma realidade inquestionável,
correspondente aos princípios de seus pensamentos. Romances de aventura,
para dar e vender, tive"-os com Scott --- mas sua \textit{Lady of the
Lake}\footnote{\textit{Senhora do lago} é o título de um poema narrativo de
  Sir Walter Scott.} era tão claramente fictícia como sua
\textit{White Maid of Avenel}:\footnote{\textit{A donzela branca de Avenel}:
  personagem do romance \textit{The Monastery} ou \textit{O monastério}, de
  Sir Walter Scott.} enquanto Rogers era um mero
diletante, que não distinguia entre desembarcar ``onde Tell saltou para
a praia''.\footnote{No original, ``where Tell leaped ashore''.} e
permanecer onde ``São Preux tinha ficado''\footnote{No original, ``St.\,Preux has
  stood''.} Mesmo a Veneza de Shakespeare era visionária;
e Portia tão impossível quanto Miranda.\footnote{Ambas personagens de William Shakespeare.} Mas Byron falou"-me, e reanimou"-as para mim, de
pessoas reais cujos pés marcaram o mármore sobre o qual ando.

%175. 
Uma única palavra, embora se refira a um tema futuro, devo"-me
permitir sobre o ritmo de Byron. Seu fluir natural, numa simplicidade e
tranquilidade quase prosaicas, interessou"-me com a mesma intensidade,
com a qual, embora por razões opostas, admirei as construções
logicamente simétricas de Pope e as estrofes equilibradas do verso
clássico e do verso hebreu. Mas, embora eu seguisse naturalmente o
estilo byroniano nos versos que escrevia para meu próprio deleite, meu
respeito pelo aspecto estrutural, em oposição à fluência, por força da
métrica clássica, fundamentado, parcialmente no desdém de Byron pelo seu
próprio trabalho, e parcialmente em meu próprio instinto arquitetônico
para ``o princípio da pirâmide'', fez com que me empenhasse, durante a
formação de meu estilo em prosa, em manter as cadências de Pope e
Johnson em qualquer trabalho sério que escrevesse. Sobre a influência de
Johnson, falarei no último capítulo deste volume; neste ínterim, devo
voltar àqueles dias em que o cantar era apenas o do arroio, em minha
pobre vida de plantinha à beira da água.

%176. 
Eu tive um ataque agudo de pleurisia na primavera de 1835, que me
fez ofegar dolorosamente, e me colocou em algum perigo durante três ou
quatro dias, durante os quais o velho médico da família, Dr.\,Walshman, e
minha mãe, protegeram"-me contra o desejo de outros homens de ciência de
me fazer uma sangria. ``Ele precisa de todo seu sangue para combater a
doença'', disse o velho médico, e me manteve sem sangria durante a
doença, embora estivesse suficientemente fraco para necessitar de uma
enfermeira durante duas semanas, seguidas de mimos, durante as quais li
\textit{Fair Maid of Perth},\footnote{\textit{Formosa donzela de
  Perth}: título de um romance de Sir Walter Scott.}
aprendi a canção \textit{Poor Louise} e regalei"-me com os desenhos que
Stanfield\footnote{Certamente William Clarkson Stanfield (1793--1867),
  pintor inglês de marinhas.} fez do Monte Saint Michel,
impressos em \textit{Coast Scenery}, e com a Santa Saba de Turner, Lago de
Betesda e Coríntios, que figuravam nas séries sobre a Bíblia, que me
>>>>>>> 20a194fac21c961c04e6cddb981d3352e74fc726
foram emprestadas pela irmã pequena de Richard Fall. Obtive uma imensa
quantidade de conhecimento útil dessas quatro gravuras, e estou muito
feliz por agora possuir os originais de Betesda e Coríntios.

Além disso, planejei todos os detalhes da viagem à Suíça, que começaria
quando estivesse plenamente recuperado. Envolvi em azul"-cobalto um \label{176}
\textit{cianômetro}, destinado a medir a intensidade do azul do céu; comprei
um caderno pautado para observações geológicas, e um grande \textit{in
quarto} para desenhos arquitetônicos, com páginas quadriculadas e
craveiras engenhosamente fixadas externamente. E decidi que os eventos e
sentimentos desta viagem deveriam ser descritos num diário poético ao
estilo do \textit{Childe Harold}.\footnote{Conferir nota na página \pageref{childe}.} Dois cantos deste
trabalho foram efetivamente finalizados --- no trecho através da França
até Chamouni --- onde os interrompi achando que, no Jura, exaurira todos
os termos descritivos à minha disposição, e que não sobrara nenhum para
os Alpes. Devo tentar fornecer, no próximo capítulo, em linguagem menos
exaltada, um relato válido desta mesma parte da viagem.

\chapter{O \textit{col de la faucille}} %Capítulo \textsc{ix}. 

%177. 
Naquele momento do fim da manhã em que o elegante viajante moderno,
destinado a Paris, Nice, e Mônaco, tendo iniciado a viagem em Charing
Cross pelo trem matinal, recupera"-se um pouco das náuseas provocadas
pela travessia e da irritação pela disputa por assentos em Boulogne, e
começa a olhar seu relógio para saber quão perto está do
\textit{buffet}\footnote{\textit{Buffet}, aqui, no sentido de compartimento
  em estações de viagem onde se servem comidas e bebidas.}
de Amiens, ele se encontra particularmente suscetível a sentir"-se
frustrado e molestado pelas paradas inúteis do trem numa estação
insignificante, como \textsc{Abbeville}. À medida que o trem começa a
movimentar"-se novamente, ele pode ver, se tiver o cuidado de erguer os
olhos por um instante do jornal, duas torres quadradas, com um arco
esculpido curiosamente acoplado, dominando sobre os álamos e os
salgueiros da área pantanosa que atravessa. Esta visão é provavelmente a
única que o viajante desejará guardar delas; quanto a mim, não sei como
fazer para que, mesmo o mais simpático dos leitores, entenda sua
importância em minha vida.

%178. 
Dessa cidade da província da qual são o centro, --- que fora, como
Croyland, mero refúgio de monges e camponeses (por isso, durante algum
tempo chamada \textit{refúgio}), --- situada entre os pântanos de Somme, e que
recebeu, por volta do ano 650, o nome de \textit{Abbatis Villa}, ---
\textit{Abbot's"-ford},\footnote{Em tradução, \textit{Vau do Abade}.} gostaria de
ter escrito: castelo e burgo, suponho que corretamente, --- visto a
dependência essencial que mantinha do grande monastério fundado por São
Riquier no lugar de seu nascimento, na encosta da colina cinco milhas a
leste da atual cidade. Em relação a este santo, traduzi do
\textit{Dictionnaire des sciences ecclésiastiques},\footnote{Em francês no
  original, \textit{Dicionário das Ciências Eclesiásticas}.} o
que talvez, diante das atuais conjunturas políticas, possa interessar
mais ao leitor do que as questões advindas de minha pobre e pequena
personalidade nascente: 

\begin{quote}
São Riquier, em latim, \textit{Sanctus Richarius}, nascido na vila
de Centula, a duas léguas de Abbeville, foi tão comovido pela piedade de
dois padres santos da Irlanda, os quais tinha hospitaleiramente
recebido, que também abraçou ``\textit{la pénitence}''. Tendo sido ordenado
padre, devotou"-se à pregação, e foi para a Inglaterra. Então, retornando
a Ponthieu, Deus lhe deu o dom de, através do trabalho e da palavra,
conduzir as pessoas ao arrependimento. Pregou na corte de
Dagoberto\footnote{Dagoberto \textsc{i} (602--639), rei merovíngio dos Francos.} e, pouco antes da morte do príncipe, fundou o
monastério que recebeu seu nome, e outro, chamado Forest"-Montier, na
floresta de Crécy, onde terminou seus dias em penitência.
\end{quote}

Encontrei, posteriormente, na \textit{História Eclesiástica de}
\textit{Abbeville}, publicada em Paris, em 1646, por François Pelican,
``Rue St.\,Jacques, \textit{à l'enseigne do Pelican}'',\footnote{Em francês
  no original: ``Rua Saint Jacques, na insígnia do Pelicano.''} que o próprio São Riquier era de sangue real, que Santo
Angilbert, o sétimo abade, casara"-se com Bertha, a segunda filha da
Carlos Magno --- ``\textit{qui se rendit aussi} \textit{Religieuse de l'ordre
de Saint Benoist}''.\footnote{Em francês no original: ``Que também se
  tornou Religiosa da ordem de São Benoist''.} Louis, o
décimo primeiro abade, era o primo alemão de Carlos o Calvo; o décimo
segundo, era filho de Santo Angilbert, neto de Carlos Magno. Rauol, o
décimo terceiro, era o irmão da Imperatriz Judith; e Carloman, o décimo
sexto, era o filho de Carlos, o Calvo.

%179. 
Erguendo novamente seus olhos, bom leitor, enquanto o trem ganha
velocidade, você poderá vislumbrar, no lado oposto, na encosta da
colina, a vila branca e sua abadia, --- não se trata de muralhas das
moradas desses príncipes e princesas (posteriormente, repetidamente
arruinadas), mas ainda da bela abadia construída sobre suas fundações
pelos monges de Saint"-Maur.

No ano em que a história acima mencionada de Abbeville foi escrita
(digamos, para ter segurança, 1600), a cidade, então familiarmente
chamada de ``Faithful Abbeville'',\footnote{Em inglês, ``Abbeville, a Fiel''.} contava com 40.000 \textit{almas}, 


\begin{quote}
vivendo em grande harmonia
entre si, numa sinceridade maravilhosa, temendo fazer algo que pudesse
ferir o vizinho, as mulheres modestas, honestas, tendo muita fé e
caridade, dotadas de uma bondade e uma beleza \textit{toute
innocente}:\footnote{Em francês no original, \textit{toda inocente}.} a nobreza numerosa, robusta, destra em armas, os mestres em artes e comércio com excelentes trabalhadores em cada profissão, sob
sessenta e quatro Mayor"-Bannerets, que chefiam as corporações de ofício,
e elegem o prefeito da cidade, que é um administrador independente,
\textit{de grande probité, d'authorité, et sans reproche},\footnote{Em
  francês no original, \textit{de grande probidade, de autoridade, e sem
  repreensão}.} auxiliado por quatro almotacés do ano
atual, e quatro do ano passado; tendo autoridade em matéria de justiça,
polícia, e guerra, e o dever de zelar pela veracidade e imutabilidade
dos pesos e das medidas, e de punir aqueles que os adulterem, ou vendam
com falso peso ou falsa medida, ou vendam algo que não contenha a marca
da cidade.
\end{quote}

Além disso, a cidade contava com a grande igreja de São Vulfran, treze
igrejas paroquiais, seis monastérios, oito conventos, e cinco hospitais;
entre as igrejas, inclino"-me particularmente em mencionar aquela de São
Jorge, iniciada pelo nosso Eduardo em 1368, em 10 de janeiro;
transferida ao Bispo de Bethlehem em 1469, que a reconsagrou neste mesmo
ano, e ampliada pelos \textit{marguilliers}\footnote{Em francês no
  original: membro do conselho de fábrica de uma igreja paroquial,
  particularmente durante o regime do Concordat. Período entre 1801 e
  1905 em que o catolicismo foi reconhecido como a religião da maioria
  dos franceses e Estado Francês e a Igreja Católica mantiveram
  estreitos vínculos.} em 1536, ``porque a congregação
aumentou tanto que muitos fiéis tinham que ficar do lado de fora durante
as solenidades''.

Essas reconstruções ocorreram com muita facilidade e rapidez em
Abbeville; em parte devido ao número de homens trabalhando em uníssono;
em parte devido à qualidade da pedra utilizada, que permitia ser
facilmente trabalhada; em parte devido à insegurança das fundações
assentadas sobre estacas; fato responsável pela existência de escassos
vestígios dos edifícios construídos antes do século quinze. A própria
igreja de São Vulfran, assim como a de São Riquier, e tudo que resta das
igrejas paroquiais (apenas quatro, agora, creio, além da de São
Vulfran), são construídas no mesmo Gótico extravagante, --- paredes e
torres contemporâneas das casas com espigão de madeira que formavam as
ruas comerciais quando as vi pela primeira vez.

%180. 
Devo aqui adiantar ao leitor que houve, em suma, três centros que
influenciaram meu pensamento: Rouen, Genebra e Pisa. Tudo que fiz em
Veneza foi de importância secundária, porque sua história já havia sido
erroneamente escrita, sem que nem seu próprio povo a pudesse entender; e
porque, no mundo da pintura, Tintoretto era praticamente invisível,
Veronese não era apreciado, Carpaccio era no máximo citado, quando
comecei a estudá"-los; algo de meu interesse pela cidade também se deve
ao imenso prazer em deslizar em gôndolas. Mas Rouen, Genebra e Pisa são a
origem de tudo que sei, e me guiaram em tudo que fiz, desde o primeiro
momento em que passei por seus portões.

Nessa viagem de 1835, conheci primeiramente Rouen e Veneza --- Pisa, não a
conheci até 1840; mas não fui capaz de compreender plenamente o poder de
nenhuma dessas grandes cidades até muito mais tarde. Mas em relação a
Abbeville, que é o prefácio e a interpretação de Rouen, eu já estava
pronto naquele 5 de junho, e percebi imediatamente que representava para
mim o início de um trabalho saudável e fonte de alegria.

%181. 
Pois nesta cidade vi que a arte (em sua variedade local), a
religião e a vida humana ainda estavam em perfeita harmonia. Não havia
seis dias mortos seguidos por um sétimo lúgubre naquelas igrejas
esculpidas; não havia sacristão para me impedir a entrada nem guardião
para me trancar dentro das tribunas particulares. Eu poderia assustá"-las
como se fosse um fantasma; espreitar por trás de seus pilares, como Rob
Roy;\footnote{Robert Roy Macgregor (1671--17340), herói folclórico da
  Escócia.} ajoelhar"-me sem a ninguém escandalizar;
desenhar sem a ninguém perturbar. Fora das igrejas, a velha e fiel
cidade congregava"-se, aninhando"-se sob seus arcobotantes como um filhote
sob as asas da mãe; a tranquila e inofensiva nova cidade ramificava"-se
aristocraticamente por ruas silenciosas, entre habitações fleumáticas e
de dignidades contidas, cada uma com um quintal e um jardim com rica
latada. A praça comercial, cortada pela rua principal, compunha"-se de
lojas sem rivais, necessárias à venda dos produtos regionais; roupas e
malharia, tecidas e tricotadas, pelas paredes; queijo da vizinha
Neufchâtel; frutos dos jardins da cidade; pão proveniente dos campos
sobre as colinas verdes; carne de seus rebanhos, não contaminada pelo
estanho das conservas americanas; boas foices e relhas malhadas na
bigorna pelo ferreiro, à vista de todos; delicadas iguarias, o café
olorosamente torrado nas ruas, diante das portas; quanto às modistas, ---
bem, talvez um ou outro chapéu elegante fosse trazido de Paris, o
restante apenas artigos para os camponeses e as damas de Ponthieu. Acima
das lojas prósperas, com movimento sereno e reconfortante, ficavam as
residências dos mestres que exerciam suas profissões há gerações;
graciosamente esculpidas, com telhados altivos, mantendo seu lugar, a
ordem estabelecida, com sua função reconhecida, infalíveis, não
ampliadas, durante séculos. Em torno da cidade, fortificações batidas
pelo vento, com suas longas avenidas sinuosas; fluindo limpa e tranquila
através dos contornos variados do rio navegável e com moinhos ativos, a
água verde"-giz do Somme.

Minha mais intensa felicidade aconteceu, naturalmente, entre as
montanhas. Mas para um prazer jubiloso, puro, do qual não se cansa, a
vista de Abbeville numa bela tarde de verão, ao descer da carruagem no
jardim do Hôtel de l'Europe, e correr para a rua para ver as torres da
igreja de São Wulfran ainda iluminadas pelo sol, é algo para se lembrar
com carinho, --- para sempre.

%182. 
De Rouen e sua catedral, o que ainda tenho a dizer, se me foram
concedidos os dias, figurará em \textit{Our Fathers have Told
Us}.\footnote{\textit{Nossos pais nos contaram}, título de uma obra de
  Ruskin.} A visão da cidade e da igreja, e jornada
seguinte, subindo o Sena em direção a Paris, depois a Soissons e Reims,
determinaram, como já foi dito, o primeiro centro e círculo do futuro
trabalho de toda minha vida. Além de Reims, em Bar"-le"-Duc, entrei
novamente num raio que permitia visão dos Alpes, e meu pai teve a
bondade de descer através de Plombières até Dijon, de modo que eu
pudesse abordá"-los através do passo mais estreito do Jura.

O leitor deve perdoar a extensão de meu relato, pois penso que se
interessaria em saber sobre esta viagem de 1835, mais do que usualmente
acontece, sem se limitar àquele momento específico; é extremamente
difícil para mim, agora, separar as circunstâncias de qualquer uma
dessas viagens dos dias que as sucederam, nos quais nos hospedamos nas
mesmas pousadas, mudando apenas do quarto azul para o verde, vimos as
mesmas paisagens e nos regozijamos ainda mais com cada prazer --- que não
era novo.

E esta última parte da estrada de Paris a Genebra, bela sem nada de
impressionante ou arrebatador, mas admirável e agradável, tornou"-se
posteriormente tão querida e familiar, que não tentarei aqui reprimir
minha bisbilhotice.

%183. 
Tínhamos o hábito de deixar o pátio do La Cloche, em Dijon, bem
cedo pela manhã --- sete horas, após um alegre desjejum às seis e meia.
Havia um quarto na extremidade oeste do pequeno salão no primeiro andar,
situado na parte da frente do hotel, cujas janelas permitiam visão das
torres da catedral, acima de um teto baixo do outro lado da rua. Eu
ocupava sempre este quarto, cuja cama ficava numa alcova, ao fundo,
separada apenas por uma divisão de ripas da estreitíssima passagem que
levava da galeria exterior ao quarto de Anne. Era um prazer para Anne,
pelo qual creio que ela ansiasse durante toda a travessia da França,
abrir a pequena porta escondida pela passagem, no fundo da alcova,
exatamente sobre meu travesseiro, e me surpreender, ou acordar, pela
manhã.

Creio que me lembro apenas de um dia que começou com chuva.
Geralmente o sol da manhã brilhava através da nuvem úmida de diamantes
lançados bem alto pela fonte do subúrbio sudeste, e seus raios
projetavam longas sombras dos choupos sobre a estrada para Genlis.

Genlis, Auxonne, Dôle, Mont"-sous"-Vaudrey --- três etapas de 12 ou 14
quilômetros cada, duas de 18; no total, cerca de 70 quilômetros ou 42
milhas, das portas de Dijon aos pés dos Jura --- íamos sempre diretamente
para as montanhas, almoçando ameixas francesas e pão.

Até Auxonne, a planície era de pouco interesse. Perguntava"-me como uma
criatura mortal poderia se contentar em viver dentro de uma distância
que permitia visão dos Jura e nunca ir visitá"-los, durante toda sua
vida! Em Auxonne, travessia do Saone, largo e belo com claros baixios de
correntes esverdeadas --- pouco mais que uma nobre torrente de montanha;
percebe"-se imediatamente que procede dos Jura. Outra hora de paciência,
e das escarpas friáveis de pedra calcária amarelada de Dole ---
finalmente lá estavam elas --- a longa ondulação azul dos Jura
desvanecendo"-se, até onde a vista alcançava, na direção sul, mais
abruptamente perto do nordeste, onde um grupo separado e escarpado de
montanhas, quase uma ilha, elevava"-se como um íngreme Wrekin, acima de
Salins. Além do Dôle, outra extensão desértica surgia numa região mais
plana, que se distinguia principalmente por seus chalés de argila com
enormes telhados de colmo pontiagudos. Estranho que eu nunca tenha
procurado saber o motivo particular daquela forma, nem olhado o interior
de um único chalé para conhecer como viviam seus habitantes!

%184. 
A cidade, ou a vila rural de Poligny, era formada pela aglomeração
de velhas casas de pedra bem"-construídas, com jardins e pomares, que se
dispondo à maneira de uma pretensa rua, ao centro, dispersavam"-se ao
longo do sopé dos Jura na abertura de um pequeno vale, o qual, em pedra
calcária de Yorkshire ou Derbyshire, seria uma garganta entre falésias
pendentes, com uma corrente sonora ao fundo, mas, nos Jura, é um
anfiteatro muito retirado com terraços elevados, nos quais pequenas
extensões de campo e jardim se assentam em diferentes alturas; um
convento com torres espiraladas na depressão, e pequenos ninhos bem
construídos de edifícios agrícolas nos cantos do prado e nas saliências
das rochas; --- nenhuma corrente, da qual se possa falar, nem fontes de
água, nem a mínima razão concebível para que estivesse ali; apenas
porque Deus a fez.

Anfiteatro \textit{muito} retirado, eu disse, --- talvez uma milha para dentro
das montanhas, a partir da planície, por meia milha de largura,
permitindo que a estrada principal, de Paris a Genebra, com uma
construção simples, serpenteasse e ziguezagueasse caprichosamente até os
terraços elevados, fazendo com que o viajante se encontrasse,
esporadicamente, onde não imaginava que pudesse estar, e ao olhar
circunflexamente deste nível que o surpreendia, não soubesse aonde iria
a seguir; --- recordava a planície de Burgundy, retrocedendo num
movimento de varredura, até que finalmente, sob a base de um último
penhasco escarpado, fletia"-se para o lado e para fora, sobre a borda da
ravina, onde esta se fechava de um modo tão surpreendente quanto se
abria, e o maravilhado viajante encontrava"-se, magicamente, como se
fosse \textit{Jack of the Beanstalk},\footnote{Personagem de uma fábula dos
  Irmãos Grimm.} na nova planície de um mundo mais
elevado. Um mundo ao nível das rochas, em cuja superfície surgiam
trechos de solo amarelo, onde nascia uma grama rarefeita mas robusta, e
um matagal disperso e espesso; e, ao longe, aqui e ali, a ondulação azul
dos pinheiros, e, para além desses, se a tarde ou a manhã estivesse
clara, sempre um pequeno brilho prateado semelhante a uma nuvem.

%185. 
As primeiras terras dos Jura diferem de muitas agradáveis maneiras
das regiões de pedra calcária em torno de Ingleborough, que lhes
equivalem na paisagem inglesa. Os pântanos de Yorkshire têm
provavelmente cem ou duzentos pés de altura e estão expostos a bátegas
de chuvas pelos violentos e quase constantes ventos. Eles são
interrompidos por vastos campos de rochas soltas, e declives xistosos
acidentados; e misturam"-se com areia e argila provenientes da poeira do
moinho, que nutrem a grama viçosa e se acumulam em depressões
alagadiças: os ventos selvagens também impedem qualquer vestígio ou
crescimento de árvores, exceto aqui e ali, aquelas recém"-plantadas em
recessos abrigados. Mas o céu dos Jura é tranquilo e claro como o do
resto da França; se o dia é belo na planície, o é também nas montanhas
circundantes; a rocha dos Jura, harmoniosamente constituída de calcário
e mármore, desintegra"-se em curiosas fissuras e estrias, mas raramente
se desprende, e há muito se cobriu com florações de florestas, ou com
delicada grama curta, e com todas as flores que amam o sol. O ar puro,
mesmo nesta borda mais baixa a mil pés acima do mar, enche"-se com os
perfumes mais doces e as cores mais vivas das flores, e o inverno lhes
dá repouso sob a serenidade da neve eterna.

%186. 
Uma diferença ainda maior e mais surpreendente existe entre os
sistemas de correntes de água. A despeito de se perderem, se esconderem,
se intermitirem, a presença das correntes de água é claramente percebida
nos pântanos de Yorkshire; veem"-se os lugares onde estiveram antes, os
poços para onde fluirão após a próxima chuva; um gotejar na base de um
penhasco, um tilintar a partir do topo, fazem pensar se se trata de uma
das fontes do Aire, das origens do Ribble, ou do início do Bolton Strid,
ou filetes de prata a serem tecidos para formar o Tees.

Mas nenhum sussurro, nenhum murmúrio, nenhum tamborilar, nenhuma canção
de regatos perturba o encantado silêncio dos Jura. As nuvens de chuva
envolvem seus penhascos e flutuam através de seus campos; passam, e em
uma hora as rochas estão secas, e apenas gotas de orvalho restam nas
folhas de alchemilla,\footnote{Herbácea perene do gênero das \textit{Rosaceae},
  muito comum em jardins europeus.} --- mas nada de
riachos, ou arroios, --- nenhum vestígio deles ontem, hoje, ou amanhã.
Através de fissuras imperceptíveis e fendas estreitas as águas dos
penhascos e dos planaltos também desaparecem; mas apenas lá longe, nas
profundezas do vale principal por onde corre o rio vigoroso, indiferente
às mudanças.

%187. 
Aprende"-se muito, portanto, para uma primeira lição, nas duas
etapas entre Poligny e Champagnole, que atravessam a grama absolutamente
viçosa e as rochas brilhantes ao sol, com água tão escassa que ali o
agrião não poderia crescer, nem o girino agitar sua cauda --- e então,
através do zigue"-zague da estrada ensombrada, contornando o Parque e o
Boulevard da desejada pequena cidade, desce"-se em direção à ponte de um
único arco que transpõe o Ain, que, abaixo, forma magníficos remansos de
um verde glauco e luminoso: o verde das folhas primaveris; para então
estrepitar em espuma, em parte estagnada, em parte cascada natural, numa
confusa competição de correntes sob as guirlandas de folhagens que
pendem dos penhascos.

%188. 
A única surpresa, para os que conhecem a estrutura dos Jura, é que
os rios deveriam ser completamente visíveis em qualquer lugar e as
rochas deveriam ser suficientemente sólidas para mantê"-los ao ar livre
através dos grandes vales, sem constantes \textit{pertes}\footnote{Em
  francês, no original, \textit{perdas}.} como as do Ródano. Sob o
Lac de Joux o próprio Orbe desaparece, ressurgindo setecentos
pés\footnote{Seiscentos e oitenta pés franceses. Saussure, § 385. {[}\textsc{n.\,a.}{]}} abaixo num cenário, cuja descrição, existente no Saussure
de papai, permito"-me citar:

\begin{quote}
Um rochedo semicircular, com pelo menos duzentos pés de altura,
composto por grandes rochas horizontais talhadas verticalmente, e
dividido\footnote{Em francês no original: ``Talhadas a pique, e
  entrecortadas''. {[}\textsc{n.\,a.}{]}} por fileiras de pinheiros que crescem
em suas bordas salientes, fecha a oeste o vale de Valorbe. Montanhas
ainda mais elevadas e cobertas por florestas formam um círculo em torno
do rochedo, que se abre apenas para dar passagem ao Orbe, cuja nascente
encontra"-se a seus pés. Suas águas, de perfeita limpidez, fluem a
princípio com uma majestosa tranquilidade sobre um leito atapetado com
belo musgo verde, \textit{Fontinalis antipyretica}; mas logo, lançado
sobre uma escarpa abrupta, o fio da corrente desfaz"-se em espuma contra
as rochas que ocupam o centro do leito, enquanto nas margens, menos
agitadas, segue fluindo sobre o fundo verde, em contraste com a brancura
do meio do rio; e então desaparece da vista, ao seguir o curso de vale
profundo coberto com pinheiros, cujos tons escuros tornam"-se mais
intensos pelo verde vívido das faias dispersas entre eles\ldots{}

Ah, se Petrarca tivesse visto esta fonte e nela tivesse encontrado sua
Laura, ele certamente a teria preferido àquela do Vaucluse, mais
abundante, talvez, e mais rápida, mas cujas rochas estéreis não têm nem
a grandeza das nossas, nem os ricos ornamentos que as embelezam.
\end{quote}

Nunca vi a nascente do Orbe, mas relataria, para conhecimento do leitor,
a beleza constante desses grandes cursos d'água que literalmente
\textit{sobem} pela base dos penhascos, ao invés de cair deles, partir das
fendas existentes em sua superfície, como se imaginaria provável. Em
nossa nascente inglesa antípoda à do Orbe, Malham Cove, o fluxo d'água,
de maneira semelhante, reúne totalmente na base do rochedo, e parece se
erguer até a saliência da qual brota a partir de um profundo lago
interior.

%189. 
O velho Hôtel de la Poste, em Champagnole, encontra"-se exatamente
acima da ponte de Ain, em frente à cidade, onde a estrada torna"-se
novamente plana e arremessa"-se firmemente em direção a Genebra. Creio
que 1842 foi o primeiro ano em que alongamos o percurso a partir de
Dijon, com duas paradas depois de Poligny; mas posteriormente, o Hôtel
da la Poste, em Champagnole, tornou"-se uma espécie de casa para nós: na
ida, ficávamos muito felizes com a hospedagem; na volta, havia muitas
recordações do belo período que passamos na tranquilidade do hotel.
Nunca havia ninguém na casa, apenas nós; se uma família se hospedasse a
cada três dias, em média, seria o suficiente para manter a pousada, a
qual tinha também sua própria fazenda; e aqueles que paravam, partiam
cedo pela manhã rapidamente para Genebra. Nós, que dormiríamos novamente
em Morez, não tínhamos pressa; e, ao retornarmos, deixávamos Genebra
sempre numa sexta"-feira, para passa o domingo em Champagnole.

%190. 
Mas minha grande alegria acontecia no fim da tarde de junho,
quando, tendo chegado de Dijon, após a truta e a costeleta rapidamente
preparadas, saía para o primeiro passeio pelas rochas e sob os
pinheiros.

Com todos meus preconceitos Tory (quero dizer, princípios), tenho que
confessar que uma grande alegria que a Suíça me trouxe --- sobretudo a
Suíça dos Jura --- provém de sua efetiva, e não meramente teórica,
liberdade. Em meio as grandes montanhas, não é sempre possível se
ir exatamente onde se deseja, --- tudo em volta é longe demais, ou
íngreme demais, --- quer"-se ir a um lugar, ou escalar outro, e não se
pode fazer nem uma coisa nem outra; --- mas nos Jura pode"-se ser feliz em
qualquer lugar a que se vá. Em geral, quando tinha tempo, costumava
escalar o penhasco que formava uma ilhota ao norte da cidade, onde
restam algumas muralhas cinzas de um castelo em ruínas, e as ainda
identificáveis aleias de seu \textit{deleite},\footnote{\textit{Pleasance}, em
  inglês: \textit{prazer}, \textit{deleite}, mas também \textit{jardim de recreio}. Ruskin
  emprega a palavra em seu duplo sentido.} de onde podia
ver se a imagem da nuvem branca ainda estava no horizonte. Estando lá,
na claridade do entardecer, sempre e sempre, a cada ano maravilhava"-me
ainda mais; eram as últimas rochas e a \textit{calotte}\footnote{Em
  francês, no original, \textit{calota}.} do Mont Blanc. Apenas
isso; quer dizer, o que pode ser visto sobre o Dôme du Goûter, se
olharmos do monte Saint Martin. Mas o cume do Mont Blanc parece tão
grande visto de Champagnole como daqui --- brilhando na última luz como
seara sob a lua.

Se não houvesse tempo para ir ao penhasco com o castelo, poderia, ao
menos, visitar os bosques sobre o Ain, e colher minhas primeiras flores
alpinas. Mais uma vez, sentia"-se grato às formalidades, e mesmo às
vulgaridades de Herne Hill, por fazer"-me perceber, por contraste, o
esplendor sáfaro das florestas dos Jura.

Então vinha a excursão matinal de carruagem até o vale estreito e
elevado do Ain, ponto no qual a estrada começava a serpentear o rio em
queda. Ninguém consegue entender como essas estradas sinuosas
escamoteiam"-se tranquilamente pelas encostas, de um pico a outro; era
apenas uma hora de caminhada ao lado da carruagem, --- uma hora que
parecia um minuto; e emergia"-se na alta planície de Saint Laurent,
quando as gencianas começavam a surgir entre a grama ao lado da estrada
e os pinheiros ondulavam"-se no horizonte como a escura infinitude do
oceano.

%191. 
Toda a Suíça estava ali em esperança e sensação, e o que era menos
do que a Suíça era, de algum modo, ainda melhor, em sua meiga
simplicidade e saudável pureza. As casas dos Jura não são esculpidas com
a mesma rica imponência das de Berna, nem agrupadas com a força
tradicional das de Uri. São cobertas com finas telhas de madeira
inclinadas que descem pelos lados até o solo para manter a casa seca; as
treliças entrecruzadas, aqui e ali, sob as janelas, mero ornamento. Elas
não têm o requinte de um jardim nem a prosperidade de uma fazenda em
torno delas, --- são, na verdade, pouco mais que um chalé delicadamente
construído, embora bem cuidado e doméstico, moderadamente aberto ao não
campestre, parecendo montar guarda no meio do prado, as gencianas na
porta, os lírios do vale selvagem pelo chão duro do matagal.

Meu encanto com essas casas, e com a engenhosidade humana e sua
capacidade de desfrutá"-las, em todo o cenário, estava na origem de todo
o prazer com sua beleza; veja a passagem posteriormente escrita em
\textit{Seven Lamps},\footnote{\textit{As sete lâmpadas da arquitetura},
  título de uma obra de Ruskin.} na qual insisto neste
ponto, como se fosse comum à natureza humana admirar através da
simpatia. Percebi, desde então, com uma precisão que me entristeceu,
quantas pessoas existem que, não importa onde estejam, pensam apenas \textit{a
partir de suas posições}. Mas o sentimento que me trouxe maior
felicidade, tanto naquele momento quanto em toda a vida, diferia de modo
curioso, pelo seu caráter impessoal, dos sentimentos de muitos, até
mesmo daqueles das melhores e mais generosas pessoas.

%192. 
No início da correspondência entre Carlyle e Emerson,\footnote{Ralph
  Waldo Emerson (1803--1882), escritor e filósofo norte"-americano,
  criador do Transcendentalismo.} editadas com
comentários demasiadamente escassos pelo meu caro amigo Charles
Norton,\footnote{Charles Eliot Norton (1827--1908), escritor, historiador
  e crítico de arte norte"-americano; professor em Harvard;
  correspondente epistolar de Ruskin.} encontro, na
página 18 esta observação --- inteiramente contestável, e em minha
opinião, na medida em que incontestável, muito condenável e deplorável,
de meu mestre: ``Só quando pudermos considerar que em toda parte alguém
está pensando em nós, alguém está nos amando, esta terra inculta
tornar"-se"-á um jardim habitado.'' Minha experiência, como o leitor
talvez tenha percebido, produziu em mim exatamente o sentimento oposto.
Meu tempo de felicidade sempre foi quando \textit{ninguém} estava
pensando em mim; e o maior constrangimento e estorvo a todas as ações e
projetos, a atenção e a interferência do público --- representadas por
minha mãe e pelo jardineiro. O jardim não era um lugar deserto para mim,
entretanto, não me imaginava objeto de interesse nem das formigas nem
das borboletas; e o único óbice ao completo deleite de minha caminhada
vespertina por Champagnole ou Saint Laurent era o sentimento de que meu
pai e minha mãe estivessem pensando em mim, e que ficariam
preocupados se eu me atrasasse cinco minutos para o chá.

Não quero absolutamente dizer que poderia passar sem meus pais. Eles
eram, para mim, muito mais do que a esposa de Carlyle para ele; e se
Carlyle tivesse escrito que, ao invés de querer que Emerson pensasse
nele na América, quisesse que seu pai e sua mãe pensassem nele em
Ecclefechan, teria sido perfeito. Mas que o resto do mundo fosse um
deserto para ele a menos que tivesse admiradores em toda parte,
correspondia a um estado de espírito lamentável; e estou de algum modo
tentado, pelo menos uma vez, a admirar o sentimento exatamente oposto
àquele de minha própria solidão. Todo meu deleite estava em observar sem
ser notado, --- e se eu pudesse ser invisível, melhor ainda. Estava
absolutamente interessado nos homens e em seus comportamentos, como
estava interessado em marmotas e camurças, em canários"-da"-terra e em
trutas. Se eles pudessem permanecer imóveis e deixar que os observasse,
e não se meter em seus buracos e subir às alturas! A viva habitação
deste mundo --- pastar e aninhar"-se nele, --- o poder espiritual do ar, as
rochas, as águas, o estar no meio deles, e regozijar"-se e maravilhar"-se
com eles, e ajudá"-lo se pudesse, --- ainda mais feliz se não precisasse
de mim, --- este era essencialmente meu amor pela \textit{Natureza}, a raiz
de tudo de útil que fiz, e a luz de tudo de valioso que aprendi.

%193. 
Quando dormíamos em Saint Laurent ou Morez, a manhã seguinte era
muito excitante. Com bom tempo, a subida de Morez a Les Rousses, quase
toda feita a pé, era puro encantamento; assim como o desjejum e a
colheita de gencianas, em Les Rousses. Então, geralmente, seguia"-se uma
hora de inquietação observando"-se as nuvens que aumentavam ao meio"-dia;
pois, mesmo que nos levantássemos muito cedo, era impossível atingir o
Col de la Faucille antes das duas horas, ou mais tarde ainda, se
tivéssemos cavalos ruins, e, às duas horas, se houvesse nuvens sobre os
Jura, certamente haveria nuvens sobre os Alpes.

É digno de nota, o próprio Saussure não o tendo mencionado, que este
passo principal dos Jura, diferentemente dos grandes passos dos Alpes,
atinge seu ponto de travessia quase sob o pico mais alto daquela parte
da cadeia montanhosa. O passo, separando a nascente do Bienne, que desce
em direção a Morez e a Saint Claude, da nascente do Valserine, que corre
por entre os Jura até o Ródano, em Bellegarde, é um contraforte do
próprio Dôle, sob cujas prolongadas massas a estrada avança por seis
milhas, subindo muito ligeiramente até o Col de la Faucille, onde a
cadeia se abre repentinamente, e surge uma curva, percorrida em cinco
minutos a trote, que leva à visão de todo o lago de Genebra e da cadeia
dos Alpes estendendo"-se por centenas de milhas pelo horizonte.

%194. 
Apenas uma vez pude ver a vista com perfeição --- neste ano de 1835;
quando a desenhei cuidadosamente segundo meu estilo à época, e tenho
estado feliz em olhá"-la retrospectivamente, como confirmação do
resultado de minha primeira visão dos Alpes, a partir de Schaffhausen.
Muito poucos viajantes, mesmo em tempos antigos, viram"-na inteiramente;
cansados das longas etapas da viagem desde Paris, quando chegavam ao Col
pensavam apenas em seus jantares e no descanso em Genebra; os guias de
viagem nada diziam sobre o passo; e embora, para todos, fosse uma tarefa
inevitável a ascensão do Righi, a ninguém ocorreu que nada houvesse para
ser visto do Dôle.

Ambas as montanhas tiveram enorme influência em toda minha vida; --- o
Dôle, de modo contínuo e tranquilo; o Righi, intermitente e
dolorosamente, como será visto. Mas o Col de la Faucille, naquele dia de
1835, abriu"-se para mim como uma clara visão da Terra Prometida de meu
futuro trabalho e de uma verdadeira casa neste mundo. Meus olhos foram
abertos, e meu coração abriu"-se junto com eles, para ver e possuir como
um rei tal reino! Até onde os olhos podiam ver --- aquela terra e suas
águas móveis ou imóveis; o Arve, que atravessa Cluse, com suas águas
provenientes de geleiras liquefeitas; o Ródano, e a infinitude de seu
lago cor de safira, --- a paz que transmite correndo por entre os prados
de narcisos de Vevay --- a força cruel com a qual se move sob os
promontórios de Sierre. E tudo que se ergue contra e se funde com o céu,
montanhas e montanhas nevadas; e toda aquela planície viva, fervilhando
de alegria humana --- repleta de casas brancas, --- uma Via Láctea de
moradas"-estrelas salpicadas através da imensidão azul da luz solar.

\chapter{Quem tu, Melpomene} 
%Capítulo \textsc{x}. 
%\footnote[*]{Início da Ode \textsc{iv} do poeta romano Quinto Horácio Flaco (65--8 a.C.), em que ele agradece a Melpomene, musa da tragédia na mitologia grega, o dom de ser um grande poeta lírico.}}

%195. 
Seja a biografia de uma nação, seja a de uma simples pessoa, é
impossível escrevê"-la de modo constante ao longo dos anos. Algumas
forças desvanecem enquanto outras ganham revelo, e a maioria atua
irregularmente, por períodos de renovado entusiasmo após intervalos de
lassidão, sem correspondência na realidade. Para a clareza da exposição,
é necessário primeiro seguir uma força, depois a outra, sem que nos
confundamos com o que ocorre em outras situações.

Devo, portanto, neste ponto do livro, parar de falar sobre meus esforços
para dominar a pintura e o ritmo poético no ano de 1835; e voltar às
atenções a outro aspecto de meu aprendizado, que poderia ter tido
consequência melhor do que aquela que dele proveio, quisessem os astros.

%196. 
Não sou capaz, e talvez o leitor me agradeça por isso, de recordar
nada daquela inspiração apolínea sob a qual assegurei a meus incrédulos
pais que, ``embora não pudesse falar bem, poderia tocar violino''. Mas,
mesmo hoje, recordo com pesar a oportunidade perdida para mostrar o que
havia de genialidade em mim, por ocasião do grande jantar oferecido
pelos militares no salão oficial de Sussex, em Tunbridge Wells; onde,
quando tinha oito ou nove anos, encontrávamo"-nos de modo completamente
fortuito, admirando as telhas curvas, o prado comum, a vista, o gosto da
bela fonte, e fazendo passeios de carruagem a High Rocks. Após o jantar
dos militares, houve música militar, e, com a conivência dos garçons,
Anne e eu conseguimos adentrar o salão, durante o serviço da sobremesa.
Creio que fosse então um belo rapaz, mesmo não estando vestido de
maneira perfeitamente civilizada, com uma espécie de sobretudo
guarnecido com passamanaria e botões. Minha mente estava completamente
absorta na contemplação dos instrumentistas da banda, --- admirava a
todos, mas ardia de inveja pelo tamborileiro.

O coronel percebeu minha enlevada atenção e enviou um alferes para que
me conduzisse até ele; sem que soubesse como, descobriu meus sentimentos
mais íntimos, e disse que se eu pedisse ao tamborileiro as lindas
baquetas com extremidades arredondadas, ele mas emprestaria. Eu estava
dividido quanto a isso, tendo confiança em minha capacidade de manter o
ritmo. Mas a funesta timidez venceu: --- balancei negativamente minha
cabeça, infeliz, e minha carreira musical estava arruinada. Ninguém
nunca saberá o que eu poderia ter tirado daquele tambor, ou (se meu pai
tivesse me levado, por acaso, à Espanha) de um tamborim.

%197. 
Minha mãe, ocupada com questões mais sérias, nunca cultivara o
pouco que aprendera de música, embora tivesse grande sensibilidade
musical. Mrs.\,Richard Gray algumas vezes tocava para mim, graciosamente,
mas se emitisse uma nota falsa, seu marido colocava os dedos nos ouvidos
e, dançando pela sala, exclamava: ``Oh Mary, Mary querida!'' Ela parava
de tocar. Nossa Mary de Perth executava aplicadamente suas escalas e ia
um pouco além; obtive importante ajuda, entretanto, quase
inconscientemente, de uma família de jovens, a qual, se minha cronologia
fosse sistemática, deveria a muito ter me referido afetuosamente.

Quando, anteriormente, descrevi a casa comercial de meu pai, disse que a
porta era aberta através de um trinco manipulado pelo empregado
principal.\footnote{Conferir capítulo \textsc{vii}, p.\,\pageref{153}.} Este
empregado principal, ou, colocando"-o de um modo mais modesto, o chefe
dos dois outros empregados, Henry Watson, foi uma pessoa de muita
importância na vida de meu pai e na minha; importância que foi,
percebo"-o olhando retrospectivamente, em muitos aspectos, terna e
afortunada, embora para outros tenha sido extremamente lastimosa, tanto
para nós quanto para ele.

A principal falha do caráter de meu pai (digo"-o reverentemente, pois
eram muito poucas, embora seja necessário relatá"-las, por serem
decisivas), era não gostar de ser superado. Ele conhecia suas próprias
capacidades --- sentia que não tinha a coragem para utilizá"-las ou
demonstrá"-las, em sua plenitude; e não podia suportar, em seu próprio
âmbito, qualquer tentativa de se equiparar a ele. Escolhia seus
empregados, primeiro pela confiabilidade, segundo pela ---
\textit{in}capacidade. Não creio que despedisse um empregado inteligente,
se o tivesse admitido acidentalmente; mas certamente não buscava gênios
mercantis, mas simplesmente subordinados que permanecessem sempre
subordinados. Frederico O Grande escolhia seus subalternos da mesma
maneira; embora, supostamente, eles nunca pensassem em se tornar reis,
enquanto o empregado de um comerciante está apto a esperar, ao menos,
tornar"-se sócio, senão seu sucessor. Além disso, os subalternos de
Frederico eram absolutamente aptos a suas funções; enquanto os
empregados de meu pai eram, em muitos aspectos, completamente inaptos às
suas. Desta inaptidão meu pai lamentava"-se muito, entretanto, nada fazia
para encontrar empregados mais capazes. Costumava enviar Henry Watson em
viagens de negócio e, em seguida, dizer que isso era mais maléfico que
benéfico: esporadicamente deixava que Henry Ritchie escrevesse uma carta
comercial; e, creio, depois ficava satisfeito por ter que escrever duas
cartas, para corrigir aquela. Raramente passava"-se um dia em que não
chegasse a casa irritado com algo que um ou outro empregado houvesse
feito ou deixado de fazer. Mas ficava com ele até sua morte.

%198. 
Do segundo em comando, Mr.\,Ritchie, direi o que é necessário em
outra oportunidade; pois o empregado de confiança, Henry Watson, já foi
negligenciado por demasiado tempo. Ele era, creio, o principal arrimo de
uma mãe viúva e de três irmãs adultas, todas amáveis, bem educadas e
sensatas; com um refinamento acima da média daqueles de sua posição
social, --- e desejosas, de um modo não vulgar, de manterem"-se no círculo
superior da classe média. De um modo não vulgar, eu disse; gostavam de
ter carruagens de entrega parando em suas portas, mas tinham
discernimento do que realmente importava na boa sociedade londrina, como
os melhores desta sociedade. Como se expressavam com elegância,
estimavam a conversa com pessoas que também assim o fizessem; de saber o
que estava acontecendo nos círculos políticos; de obter
\textit{entrée}\footnote{Em francês, no original.} em bailes
elegantes ou em concertos oferecidos pelos motivos apropriados. Sendo
ambas boas musicistas e agradáveis executantes (as duas virtudes não
estão presentes em todos os músicos), o acesso não era algo difícil para
elas; --- para isso, contudo, era necessário manter uma casa numa rua de
bom"-tom perto do Hyde Park, estar elegantemente vestidas e eventualmente
oferecer, elas próprias, uma pequena recepção. Juntos, esses gastos
absorviam a totalidade do salário de Henry, e dos ganhos, em ocupações
oficiais ou informais, dos dois outros irmãos, David e William. O
último, lembro"-me agora, era um comerciante de vinhos de West"-End que
supria a nobreza com Clos"-Vougeot, Hochheimer, Champagne especial e
outras bebidas néctareas, nas quais o fundo atinge a metade da garrafa,
e que só são encontradas nas adegas de grandes duques e condes do
Império. A família vivia no limite de seus meios, --- mas não havia
penúria: as jovens se divertiam muito, estudavam alemão --- naquela época
era considerado muito distinto e poético estudar esta língua; ---
cantavam muito bem, com graça e desembaraço; tinham bom gosto ao se
vestir, com ligeira tendência ao matronal e ao antiquado: e toda a
família considerava a si mesmo extremamente \textit{élite}, de uma maneira autêntica e
virtuosa.

%199. 
Quando Henry Watson foi contratado (então, creio, um rapaz de 16
anos), não sei em que situação, ou sob qual recomendação, para trabalhar
na casa comercial de meu pai, sua família considerou que a oportunidade
era magnífica; ficaram muito agradecidos e felizes e, claro, ansiosos
para fazer tudo que pudessem para agradar meus pais, em nome de seu
irmão. Descobriram, entretanto, que minha mãe não era fácil de ser
agradada; e logo passaram a não se surpreender e a não achar
\textit{des}agradável o modo como as coisas aconteciam, tanto na casa
comercial quanto em Herne Hill. Na primeira, trabalho assíduo; na
segunda, discrição: os empregados eram terminantemente proibidos de
deixar suas escrivaninhas para participar de uma \textit{garden"-party} e,
quando escurecia, contavam apenas com velas de sebo. O fato do dono da
firma viver na metade de uma casa dividida por um muro, para além do
subúrbio de Camberwell, significava degradação e desgraça para todos
aqueles relacionados ao negócio! E de Henry ser obrigado, toda manhã, a
tomar um ônibus para o leste da City, e trabalhar sentido os odores de
Billingsgate,\footnote{Billingsgate Market é o mercado de peixes, em Londres.} ao invés de caminhar elegantemente através de
Piccadilly até um escritório na rua Saint James, era também ofensivo
para ele e depreciativo do bom gosto de meu pai e de seu conhecimento
mundano. Além disso, para o círculo feminino, minha mãe era um fenômeno
singular, lamentavelmente intratável. Como não tinha interesse pelo
estudo do alemão, fosse pouco curiosa em relação aos eventos de Mayfair,
cujas opiniões lhe eram irrelevantes, ela estava apta a olhar com alguma
severidade, talvez matizada de ciúme, para aquilo que considerava
atividades pretensiosas ou afetações de maneiras das jovens: enquanto
elas, de sua parte, embora sensíveis ao mundo de minha mãe, gratas por
sua boa vontade e, com o tempo, realmente ligadas a ela, não estavam
dispostas a prestar muita atenção às opiniões de uma mulher que conhecia
apenas seu próprio mundo; e eram mais refratárias que responsivas a uma
amabilidade que frequentemente adquiria o aspecto de advertência.

%200. 
Essas diferenças de pensamento, embora irreconciliáveis, não
impediram o desenvolvimento de relações consistentes, agradáveis e
sinceras entre minha mãe e as jovens. Com o melhor de sua naturezas de
moças, Fanny, Helen, e a pequena June, a mais alegre, a mais
inteligente, gostavam de, durante a primavera, trocar a empoeirada
dignidade da rua de bom tom em Mayfair pelos lilases e laburnos de Herne
Hill: e estavam sempre prontas, juntas com o irmão Henry, a nos visitar
em qualquer ocasião em que a colina hospedasse algum correspondente da
casa comercial, e a cantar para nós as mais belas árias da nova ópera,
nas quais se percebia a tônica do alemão clássico.

Henry tinha uma voz de tenor singularmente bela; e as três irmãs, embora
nenhuma tivesse voz particularmente poderosa, cantavam seus trechos com
precisão, bom gosto, e com a agradável união de vozes fraternas. Assim,
desde a tenra infância, estive acostumado a ouvir grande variedade de
boa música, executada apropriadamente, sem interrupções, afetações de
maneiras, ou vulgar proeminência da execução. Se o quarteto tivesse
cantado alegres canções inglesas, ou baladas escocesas, ou canções
marítimas britânicas, ou se qualquer uma das jovens tivesse talento
suficiente para entoar música mais elevada em todo seu esplendor,
provavelmente teria deixado de lado por alguns momentos meus mapas e a
mineralogia para audições mais atentas. Na verdade, as composições
científicas alemãs me eram enfadonhas, e as belas modulações do
italiano, do qual não entendia qualquer sílaba, eram agradáveis ao
ouvido apenas como o trinado dos melros, que ouvia frequentemente
através da janela para o jardim dos fundos, expressando juntos sua
satisfação, em festas sonoras nas tardes de primavera. A educação de meu
ouvido e de meu gosto continuaram, entretanto, sem sobressaltos. Não
creio que tenha jamais ouvido música executada por um profissional, até
que, por um feliz acaso, ouvi o melhor possível, durante um curto espaço
de tempo naqueles dias da juventude.

%201. 
Também deixei, negligentemente, sem explicação, a sentença fortuita
sobre o ``funesto jantar em Mr.\,Domecq'' quando tinha 14 anos, capítulo
\textsc{iv}, parágrafo 94. O sócio espanhol de meu pai morava, naquela época, nos Champs \label{201}
Élysées, com sua mulher inglesa e suas cinco filhas; a mais velha,
Diana, prestes a se casar com um oficial de Napoleão, o conde Maison; as
outras quatro, muito mais jovens, estavam eventualmente em casa passando
férias da escola religiosa: e tivemos alegres jantares familiares com os
noivos, mamãe, as garotas e um encantador cavaleiro francês, já idoso,
Mr.\,Badell, o qual, após as refeições, interpretava comigo ``la toilette
de Madame''; não consigo lembrar se eu era o colar ou a
jarreteira;\footnote{Relativo à Ordem da Jarreteira: ordem de cavalaria
  instituída na Inglaterra por Eduardo \textsc{iii}, no século \textsc{xiv}.}
e então Clotilde e Cécile tocavam ``les Echos'' e outras melodias
dançantes fascinantes, --- mas eu não sabia dançar; e, afinal, Elise teve
pena de mim, como anteriormente descrevi. Mas a melhor, se não a maior
parte da conversa entre os adultos, versava sobre a recente morte de
Bellini,\footnote{Vincenzo Salvatore Carmelo Francesco Bellini
  (1801--1835), compositor italiano de óperas, nascido na Sicília.} a consternação de toda a Paris, e o vigor com o qual sua
\textit{I Puritani}\footnote{Ópera de Vincenzo Bellini.}
estava sendo interpretada pelos quatro grandes cantores para os quais
fora escrita.

%202. 
Surpreende"-me que não tenha qualquer lembrança da primeira vez que
assisti a uma ópera. Nem mesmo, a propósito do tema, da primeira vez que
fui ao teatro, embora já tivesse 12 anos quando me levaram; a seguir, o
intenso arrebatamento ao assistir a uma pantomima, foi algo banal. E amo
o teatro até hoje --- é um dos prazeres que menos se desgastaram;
entretanto, embora me lembre de Friar's Craig em Derwentwater quando
tinha quatro anos, e do pátio interno de nossa hospedaria em Paris, aos
cinco anos, não tenho qualquer lembrança, o que me dá certo orgulho, de
minha primeira peça de teatro. Ter sido levado a Paris para assistir à
ópera \textit{Puritani}, fraca do ponto de vista dramático, não me deu
grande alegria; mas ouvi, algo que será sempre raro, apenas possível
duas ou três vezes num século, quatro grandes cantores, todos podendo
ser chamados de gênios, cantando juntos, com o desejo sincero de se
harmonizarem, e não de se ofuscarem; e de valorizar, não apenas a
potência de seu canto, mas a beleza da música que interpretavam.

%203. 
Ainda mais afortunadamente, aconteceu de uma mulher de
\textit{indiscutível} gênio conduzir as danças que se seguiram, ---
Taglioni;\footnote{Marie Taglioni (1804--1884), célebre bailarina sueca
  que fascinou a Europa com suas performances nos ballets \textit{La
  sylphide} e \textit{La fille du Danube}.} pessoa dotada
das mais elevadas faculdades naturais, caráter simples e imaculado,
reunidos com ardor e reverência em sua arte. Minha mãe, embora
permitisse sem sérias objeções que meu pai me levasse ao teatro, nutria
o mais estrito preconceito puritano contra o palco; entretanto,
apreciava"-o tanto que creio que acreditasse que o sacrifício que fazia
ao não nos acompanhar fosse uma espécie de preço exigido pelas leis da
virtude, pelas quais era pecaminosa a concessão que fazia a meu pai e a
mim. Ela foi, contudo, assistir a esse grupo de renomados dançarinos,
que se apresentavam sem rivais por todas as cidades da Europa; --- e,
diga"-se, com grata surpresa --- seu instinto para perceber a inocência, a
beleza, e enlevar"-se com cada movimento da Graça de seu século foi tão
forte que, a partir de então, minha mãe sempre foi ver conosco Taglioni,
mediante um simples convite.

Depois disso, não se passava uma temporada sem que eu não ouvisse duas
ou três vezes, ao menos, aqueles quatro cantores; e apreendia da melhor
maneira possível, porque meus ouvidos nunca se cansavam, o propósito da
música que fora escrita para eles ou estudada por eles; e estou
extremamente feliz agora por ter ouvido suas interpretações de
Mozart e Rossini, dos quais não se pode dizer que ainda possam ser
entendidos, devido ao detestável hábito de acelerar o tempo musical.
Grisi\footnote{Giula Grisi (1811--1896), soprano italiana.}
e Malibran\footnote{Maria Malibran (1808--1836), cantora lírica francesa,
  célebre em sua época.} cantavam, ao menos, um terço
mais lento que qualquer \textit{cantatrice}\footnote{Em francês, no
  original, \textit{cantora famosa}.} moderna,\footnote{É um belo
  recurso os músicos chamarem a si mesmos de científicos, quando ainda
  nem definiram suas unidades de tempo! {[}\textsc{n.\,a.}{]}} e
Patti,\footnote{Adelina Patti (1843--1919), cantora lírica italiana,
  célebre em sua época.} na última vez que a ouvi,
massacrou o papel de Zerlina\footnote{Personagem da ópera \textit{Don
  Giovanni}, de Wolfgang Amadeus Mozart (1765--1791).} em
\textit{La Cidarem},\footnote{\textit{La ci darem la mano}, ária da ópera
  \textit{Don Giovanni}, de Mozart.} como se a plateia e
ela tivessem o único objetivo de acabar com a ária de Mozart o mais
rápido possível.

%204. 
Posteriormente, (a confissão poderia muito bem ter sido feita
imediatamente) quando estava acomodado em meu banco na Igreja Cristã,
descobri que os melhores homens desta comunidade haviam fundado uma
sociedade musical, sob instrução do organista da catedral, Mr.\,Marshall,
pessoa muito simples, de boa índole e bem"-humorada, que me encorajou a
tentar aprender a cantar \textit{Come mai posso vivere se Rosina
non m'ascolta},\footnote{Em italiano, no original, ``como posso continuar
  a viver se Rosina não me ouve''.} e a tocar as duas
linhas do prelúdio de \textit{A te o cara}\footnote{Ária da ópera
  \textit{I Puritani}, de Vincenzo Bellini.} e as
notas que eu conseguisse ler dos acompanhamentos de outras canções com
propósitos igualmente ternos. Nos quais, embora nunca tivesse avançado a
ponto de ler com facilidade, consegui, contudo, entre a sensibilidade de
meu ouvido para ritmos e meu amor verdadeiro pela música, entender
alguns princípios da arte musical, que eu talvez pudesse, ainda hoje,
inculcar beneficamente ao público com interesse pela música, desde que
acabe esta autobiografia.

O que aquele banco na Igreja Cristã, no qual me sentava, representaria,
ou aonde me levaria, ninguém de minha família parecia considerar naquela
época. Minha mãe, percebendo minha inclinação para a metodologia e as
ciências naturais, estava, creio, tranquila, acreditando que me tornaria
outro White of Selborne,\footnote{Gilbert White of Selbourne (1720--1793),
  reverendo e naturalista inglês, inspirador de Charles Darwin,
  considerado um dos primeiros ecologistas ingleses.} o
Vigário de Wakefield,\footnote{Refere"-se a \textit{The Vicar of Wakefield},
  romance do escritor e poeta anglo"-irlandês Oliver Goldsmith (1728 ou
  1730--1774).} ou vitorioso na controvérsia
whistoniana\footnote{Refere"-se ao teólogo, historiador e matemático
  inglês William Whiston (1667--1752), que se envolveu em várias
  controvérsias religiosas sobre a natureza da fé cristã e da religião
  católica; previu que o mundo se acabaria em outubro de 1736.} e em todas as outras. Meu pai talvez concebesse para mim uma
carreira mais ligada aos cometas e aos meteoros, mas nenhum deles
colocava explicitamente a questão, que calava fundo no coração: e a mim
era permitido, sem objeções, continuar medindo o azul do céu\footnote{Conferir
  capítulo \textsc{viii}, p.\,\pageref{176}.} e observando o movimento das
nuvens, até que tivesse esquecido todo o latim que sabia, além de todo o
grego, exceto a ode de Anacreonte sobre a rosa.

%205. 
Certo esforço foi feito, em 1836, no sentido de me reconduzir
àquela direção, ao me enviarem para assistir às conferências de Mr.\,Dale
no King's College, onde, ao encontrá"-lo um dia no vestíbulo,
expliquei"-lhe que pórticos não deveriam figurar sobre arcos; e
considerei"-me elevado socialmente por ter entrado pela mesma porta dos
estudantes que usavam barretes quadrados. As conferências versavam sobre
literatura inglesa antiga, sobre a qual, embora não tivesse lido uma
linha anterior a Pope, já pensava ter uma avaliação superior a de Mr.\,Dale. Sua citação de ``O rei Knut chegou num barco à vela''\footnote{``Knut
  the king came sailing by'', no original. Refere"-se, provavelmente, a
  um verso da obra do cronista inglês do século \textsc{xii}, Thomas de Ely,
  \textit{Historia Eliensis}.} permaneceu comigo; e creio
que foi tudo que aprendi durante o verão. Pois, como os astros me foram
adversos, naquele ano o sócio de meu pai, Mr.\,Domecq, considerou que
seria conveniente fazer uma rodada extra de visitas aos seus clientes
britânicos, e perguntou se poderia deixar suas filhas em Herne Hill para
ver os leões na Torre de Londres e coisas do gênero. Como conseguimos
acomodá"-las todas nos vãos e armários de Herne Hill seria inexplicável
sem uma planta dos três andares! A arrumação teve algo de Arca de Noé e
algo de casa de bonecas, mas conseguimos alojá"-las: Clotilde, graciosa
loura com rosto oval, de quinze anos; Cécile, bela morena com
sobrancelhas bem desenhadas, de treze anos; Elise, também loura, com o
rosto redondo de uma garota inglesa, preciosidade de boa índole e bom
senso; Caroline, um pequeno e delicado ser, quase irreal, de onze anos.
Todas elas nasceram no exterior; Clotilde em Cadiz; e, naturalmente, foi
educada num convento; mas posteriormente habituou"-se a frequência à
sociedade durante as férias em Paris. A impressão que a visão dessas
garotas nos Champs"-Élysées, as primeiras bem"-nascidas e bem"-vestidas que
eu via --- ou com as quais falava, deixou em mim, foi mais profunda do
que alguém poderia imaginar. Com bem"-vestidas, quero dizer,
naturalmente, vestidas de uma maneira perfeitamente simples, com cortes
e ajustes seguindo a moda parisiense. Elas eram \textit{beatas} --- como
diriam os protestantes; com uma firmeza suave, quando queriam dizer ---
católicas romanas; falavam espanhol e francês com perfeição e graça, e
inglês com uma correção artificial: todas muito ajuizadas; Clotilde,
austera e corretamente; Elise, alegre e afavelmente; Cécile,
serenamente; Caroline, incisivamente. Uma galáxia rara, ou um Cruzeiro
do Sul, de estrelas inconcebíveis, subitamente flutuando em meu obscuro
firmamento num subúrbio londrino.

%206. 
Como meus pais puderam permitir que seu jovem noviço fosse lançado
sem ajuda na ardente fornalha do mundo exterior, o leitor pode imaginar,
mas só o destino sabe. Havia a desculpa de que eles nunca tinham
percebido em mim o menor interesse ou ansiedade em relação a garotas ---
eu nunca quisera permanecer nas esplanadas de Cheltenham ou Bath, ou nas
paradas de Dover; ao contrário, resmungava e choramingava se me
mantivessem ali e corria para o mar ou para o campo assim que me via sem
vigilância. Eles me haviam educado nos extremamente ortodoxos torismo e
evangelismo ingleses e não podiam conceber que seu jovem filho
reverenciasse a ciência, George Terceiro e, em religião, se inclinasse
ao catolicismo francês. E eu nunca \textit{dissera} nada sobre os
Champs"-Elysées!\footnote{Conferir capítulo \textsc{x}, p.\,\pageref{201}.}
Virtualmente ``educado num convento'', e ainda mais severamente que as
garotas, sem a afeição de uma irmã ou de uma prima como refúgio ou
varinha de condão, sem ter habilidades atléticas ou prazeres que
pudessem se contrapor à minha tendência ao devaneio, fui atirado, com as
mãos e os pés atados, com minha candura imaculada, na fornalha ardente,
ou cruz ardente, dessas quatro garotas, --- que, naturalmente,
reduziram"-me a um mero monte de cinzas brancas em quatro dias. Quatro
dias, no máximo, levaram para me reduzir a cinzas, mas esta Quarta"-Feira
de Cinzas\footnote{Em francês, no original, \textit{Mercredi des cendres}.} durou quatro anos.

Nada mais cômico na aparência, nada mais trágico na essência, poderia
ter sido inventado pelo mais talentoso criador. Em comportamento social
e opinião eu era uma curiosa combinação de Mr.\,Traddles, Mr.\,Toots e Mr.\,Winkle. Tinha a constância e a concentração de Mr.\,Traddles, o gosto pela
conversação de Mr.\,Toots, e a heroica ambição de Mr.\,Winkle; --- todas as
três habilidades iluminadas pela imaginação, como Mr.\,Copperfield,\footnote{Referência ao personagem do romance \textit{David
  Copperfield}, de Charles Dickens.} no primeiro jantar
em Norwood.

%207. 
Clotilde (Adèle Clotilde era seu nome, mas suas irmãs a chamavam de \label{207}
Clotilde, em homenagem a rainha"-santa, e eu, de Adèle, porque rimava com
concha, encanto e dobrar de sinos\footnote{No original, ``Shell, spell
  and knell''.}) tornava"-se ainda mais resplandecente no
pequeno círculo da beleza de suas irmãs; enquanto minha própria timidez
e insegurança eram acentuadas, melhor, encorajadas, por uma concepção
patriótica e protestante, não suavizada nem pela polidez nem pela
simpatia; de modo que sentado, e ciumentamente miserável como um
bacalhau seco (na verdade, imagino, como uma arraia num aquário tentando
passar por cima do vidro), na abençoada ocasião \textit{tête"-à"-tête},
esforçava"-me para entreter minha querida, nascida na Espanha, criada em
Paris e católica de coração, com visões pessoais sobre a Armada
Espanhola, a batalha de Waterloo e a doutrina da transubstanciação.

A essas maneiras de me valorizar, não me abstive, contudo, de
acrescentar a demonstração dos talentos que julgava possuir. Escrevi,
com grande dificuldade, e considerável esforço de invenção, uma história
sobre Nápoles (que nunca visitara), e ``Bandit Leoni'',\footnote{Referência
  à novela de John Ruskin, \textit{Leoni: a Legend of Italy}, publicada em
  1868.} representado como um tipo que reunia minhas
próprias tendências ao crime e à aventura, caso tivesse sido criado para
me tornar um bandido; e ``Maiden Giuletta'',\footnote{Personagem de
  \textit{Leoni: a Legend of Italy}.} a quem atribuí todas
as perfeições de minha querida. Nossa conexão com Messrs.\,Smith \& Elder
permitiu que minha história fosse impressa no \textit{Friendship's
Offering}; e Adèle riu da história, convulsionada por êxtases de
derrisão; suportei a dor bravamente, pelo prazer de vê"-la se divertir
intensamente.

Não ousei lhe enviar diretamente meus sonetos; mas quando ela voltou a
Paris, escrevi"-lhe uma carta em francês, de sete páginas
\textit{in"-quarto}, descrevendo a desolação e a solidão de Herne Hill
desde que partira. Elise, ou Caroline, escreveu"-me para dizer que ela
realmente tinha lido a carta, e ``rido muito de meu francês''. Tanto
Caroline como Elise compadeceram"-se de mim, e apenas a contragosto
disseram que ela também rira do conteúdo da carta.

%208. 
Neste ínterim, os mais velhos não viam grande perigo em tudo
aquilo. Mr.\,Domecq, de muito boa índole e ótimo avaliador de caráteres,
gostava de mim, porque percebia que eu também tinha boa índole e algumas
sementes de habilidades intelectuais que desabrochariam com o tempo:
para o bem dos negócios, estaria disposto a me conceder a mão de
qualquer uma de suas filhas, que também viesse a gostar de mim, mas
considerava que ainda não havia chegado o momento para falar dessas
coisas. Meu pai era exatamente da mesma opinião, além de sentir"-se feliz
por eu ter publicado um relato no \textit{Frindship's Offering},
satisfeito por eu ter conhecido algumas garotas bem"-educadas e
esperançoso de que se eu escrevesse poesia inspirado por elas, fosse tão
boa quanto \textit{Hours of Idleness}.\footnote{Título de um livro de
  poesia de Lord Byron.} Minha mãe, para quem a
ideia de meu casamento com uma católica romana era monstruosa demais
para ser um desígnio dos Céus e prepóstera demais para merecer prevenção
aqui na Terra, estava agastada com a história como se uma de suas
chaminés tivesse começado a fumaçar, --- e ela não tivesse a menor ideia
de que a casa estava pegando fogo. Mais do que meu pai, ela, contudo,
compreendeu a profundidade de meu sentimento, mas não quis, com sua
ternura materna, constranger"-me, questionando"-me seriamente. Esperava
que, quando os Domecqs voltasse para Paris, não os veríamos mais, e que
a influência e a lembrança de Adèle desvaneceriam --- como a neve do
próximo inverno.

%209. 
Sob essas circunstâncias indulgentes, --- amargamente envergonhado
pelo papel que fizera, sem que a borbulhante espuma de minha furiosa
presunção cotidiana tivesse, contudo, sido diminuída, justificada pela
real profundidade do sentimento, e (note bem, caro leitor) por uma
verdadeira e gloriosa percepção do milagre recentemente revelado do amor
humano, em sua exaltação da beleza física do mundo que eu até então
buscara unicamente em sua luz, --- entrei no meu décimo --- sétimo ano, num
estado de majestosa imbecilidade, para escrever uma tragédia sobre um
tema veneziano, no qual as dores de minha alma deveriam ser
imortalizadas em versos --- a bela heroína, Bianca, deveria possuir as
perfeições de Desdêmona\footnote{Personagem da peça \textit{Otelo, O mouro
  de Veneza}, de William Shakespeare.} e o brilhantismo
de Juliet,\footnote{Julieta, personagem da tragédia \textit{Romeu e
  Julieta}, de William Shakespeare.} --- e Veneza e o
Amor seriam descritos como nunca anteriormente. Devo dizer, \textit{en
passant}, que ao ver pela primeira o Palácio Ducal, no ano anterior,
anunciei deliberadamente a meus pais e a Mary --- algo que me pareceu
ridículo e dificilmente crível --- que eu pretendia fazer um desenho do
Palácio Ducal como nunca antes fora feito. Com este objetivo, fiz
esboços apressados no local, e concluí o desenho, de memória, em
Treviso. O desenho ainda existe --- para minha surpresa, desconsidera a
perspectiva, porque então eu era demasiadamente pretensioso para seguir
suas regras, --- no padrão geométrico dos painéis em relevo destacado dos
mármores vermelho e branco. Nenhuma figura perturba a solene
tranquilidade da Riva,\footnote{Provavelmente trata"-se de Riva degli
  Schiavoni, famosa \textit{promenade} à beira da água, em Veneza.} e das gôndolas --- cada uma como uma crescente turca
estendendo"-se na água --- flutuando sem a ajuda dos gondoleiros.

Não me lembro de nada mais daquele ano, 1836, além de sentar sob a
amoreira no jardim dos fundos, e escrever minha tragédia. Não sei se
fizemos ou não uma viagem, ou o que fiz no restante dos dias. A memória
está vazia, exceto pela lembrança de Veneza, Bianca, e de olhar a partir
de Shooter's Hill, de onde podia ver a última curva da estrada que leva
a Paris.

Devo ter lido alguns gregos, embora não me lembre de quais, e alguma
matemática foi aprendida, porque certamente sabia a diferença entre a
raiz quadrada e a raiz cúbica quando fui para Oxford, onde meu tutor me
fez estudar Heródoto, de cuja obra imediatamente reuni material
suficiente para escrever uma canção que se canta ao se beber, à moda dos
citas, imitando o \textit{Giaour}.\footnote{Conferir capítulo \textsc{viii}, nota
à página \pageref{giaour}.}

%210. 
O leitor crítico deve ter começado a duvidar, a partir desse
momento, da veracidade de minha afirmação de não ter recebido qualquer
influência negativa de Byron. Mas não precisaria. A forma específica de
expressão que minha tolice adquiriu deve"-se realmente à influência dele;
mas esta forma foi a melhor possível. Desenvolvi mais meu inglês ao
imitar o \textit{Giaour} e a \textit{Bride of Abydos}\footnote{Título de um
  poema de Lord Byron.} do que teria desenvolvido se
imitasse qualquer outro mestre, (a tragédia foi, evidentemente, de
influência shakespeariana!) e meu estado de espírito não provinha de
Byron --- devia"-se a uma deficiência pessoal e a má utilização de minhas
circunstâncias. Nesse mesmo ano, 1836, também comecei a ler Shelley, e
perdi muito tempo com \textit{Sensitive Plant}\footnote{\textit{Planta
  sensível}, título de um poema de Percy Byssche Shelley (1792--1822).} e \textit{Epipsychidion};\footnote{Título de um poema, também de
  P.\,B.\,Shelley.} e recebi uma
razoável influência negativa \textit{dele}, ao tentar escrever linhas como
``prickly and pulpous and blistered and blue'';\footnote{``Espinhoso
  e polposo e bolhoso e azul''.} ou ``it was a
little lawny islet by anemone and vi'let, --- like mosaic
paven'',\footnote{``Era uma pequena ilhota gramada com anêmones e
  vi'oletas, --- como se pavimentada por um mosaico''.}
etc.; mas no estado de efervescência febril em que me encontrava,
resultava pouco benéfico, para mim, escapar às influências externas. A
perseverança com a qual tentei avançar através da \textit{Revolt of
Islam}\footnote{\textit{Revolta do Islã}: título de um poema de P.\,B.\,Shelley.} e tentar saber (até hoje
não o consegui) quem se revoltou contra quem, ou contra o que, me deve
ser creditada; e o \textit{Prometeu}\footnote{Referência à tragédia
  \textit{Prometeu Acorrentado}, de Ésquilo (c. 525--456 a.C).} me fez realmente entender algo de Ésquilo. Não estou
certo de que, para aquilo a que me dedicaria, meus dias de efervescência
poderiam ter transcorrido mais tranquilamente; de qualquer modo, foi
melhor do que se tivesse aprendido a atirar, ou a caçar, ou a fumar, ou
a jogar. O que me é totalmente incompreensível, olhando
retrospectivamente, é minha absoluta falta de motivação, de vontade ou
de objetivo naquela questão: eu não tinha nem a determinação de
conquistar Adèle nem a coragem para viver sem ela, nem a inteligência
para considerar o que finalmente resultaria de tudo aquilo, nem a
perspicácia para perceber quão desagradável estava me tornando para
todos aqueles à minha volta. Realmente não havia em mim mais capacidade
ou inteligência do que numa corujinha que acabou de ganhar penas ou num
cachorrinho que acabou de abrir os olhos e descobre, desconsolado, a
existência da lua.

%211. 
Enquanto estava envolvido em fracas e melodiosas lamúrias dirigidas
àquele astro luminoso, contudo, fui surpreendido por uma carta do Christ
Church a meu pai, avisando"-o de que não haveria alojamento para mim em
janeiro de 1837, e que eu deveria me matricular em outubro no ano em
curso, 1836.

Estranhamente, meu pai nunca procurou informações sobre a natureza e os
procedimentos da matrícula, até que me trouxe a Oxford; --- ele, e quase
tanto como eu; nós nada sabíamos do que se tratava. Ele nunca teve
qualquer dúvida sobre colocar"-me na universidade mais famosa e,
naturalmente, meu nome fora enviado ao Christ Church muitos anos antes
de eu ser convocado; mas nunca ocorrera a meu pai que existissem duas
ordens, ou castas, de estudantes em Christ Church, de diferentes níveis
sociais, uma chamada de estudantes"-\textit{gentlemen,} a outra de
estudantes"-comuns; esta última parecia ocupar um ponto intermediário
entre os estudantes"-\textit{gentlemen} e os servis. Todas essas
``odiosas'' distinções foram abolidas pela reforma de nossas
universidades. Ninguém é elevado à posição especial de \textit{gentleman},
nem ninguém é rebaixado à condição de estudante"-comum; e se, entre os
mais velhos, alguém solicitar um lugar gratuito para seus filhos,
ficaria furioso com a simples ideia de eles usarem, na universidade, a
beca dos servis.

%212. 
Até onde concordo com os modernos cidadãos britânicos em relação a
esses elevados sentimentos, meus escritos já o demonstraram
suficientemente; mas deixo que o leitor forme sua própria opinião, sem
qualquer comentário negativo de minha parte, acerca dos efeitos dessa
explosiva mudança do estado de coisas sobre minha vida universitária.

Meu pai não gostava da expressão ``homem do povo'',\footnote{\textit{Commoner},
  em inglês.} --- sobretudo, porque nossas relações, em
geral, eram com as pessoas chamadas comuns. Além disso, embora tivesse
seu orgulho satisfeito por estar à frente de um negócio de
\textit{sherry}, sentia e via em seu filho capacidades que ultrapassavam o
âmbito do comércio de \textit{sherry}. Seu ideal para meu futuro, --- agora
inteiramente baseado na convicção de meu gênio, --- era que, através da
universidade, eu deveria ter acesso a melhor sociedade, ganhar todos os
prêmios todos os anos, encerrá"-la com dois primeiros lugares; casar"-me
com Lady Clara Vere de Vere; escrever poesia tão bem quanto Byron, mas \label{212}
com um caráter religioso; fazer sermões tão bons quanto dos de Bossuet,
mas que fossem protestantes; ser consagrado, aos quarenta, Bispo de
Winchester, e, aos cinquenta, Primado da Inglaterra.

%213. 
Com todas essas esperanças, e diante de todas essas tentações, meu
pai sentia"-se, contudo, em grande medida, limitado e embaraçado pelo seu
velho e seguro senso sobre sua posição social: e ansiosamente, mas
honestamente, consultava o Deão do Christ Church (Gaisford), e aquele
que seria meu tutor na universidade, Mr.\,Walter Brown, sobre a
possibilidade de uma pessoa, em sua posição, matricular, sem cometer uma
impropriedade, seu filho como estudante"-\textit{gentleman}. Não ouvi os
diálogos, mas o velho Deão deve ter respondido com um grunhido, que meu
pai tinha todo o direito de fazer de mim um estudante"-\textit{gentleman},
se assim o quisesse, e pudesse pagar as taxas; o tutor, analisando mais
cuidadosamente as condições para responder à questão, talvez tenha dito,
cortesmente, que seria bom para a universidade ter alguém que gostava de
ler entre os estudantes"-\textit{gentlemen}, os quais, em geral, não eram
muito inclinados ao estudo; mas ele também foi compelido a advertir meu
pai de que, embora eu tivesse realizado consideráveis leituras, não era
certo que conseguisse passar no exame de admissão ao qual os estudantes
comuns deveriam ser submetidos. A última observação foi decisiva. Não
seria tolerável que um rapaz de quem se esperava que vencesse todos os
obstáculos diante dele, ficasse retido na primeira barreira. Fui
matriculado como estudante"-\textit{gentleman}, sem mais discussões, e
ainda me lembro, como se fosse ontem, do orgulho com o qual caminhei do
Angel Hotel e passei pela universidade, de braço dado com meu pai, com
meu gorro de veludo e minha beca de seda.

%214. 
Sim, caro leitor, o veludo e a seda fizeram a diferença, não apenas
para minha mãe, mas também para mim! Um dos pontos cruciais nas
discussões domésticas concernentes a esta escolha de Hércules, foi que a
beca dos estudantes"-comuns, além de ser de material ordinário, e de não
flutuar graciosamente, consistia virtualmente num trapo negro amarrado
sobre os ombros. Era"-se três vezes mais estudante numa beca flutuante.

Na verdade, agora, nos anos da maturidade, encontro"-me tão pouco
disposto a escarnecer desses sentimentos não filosóficos, que, em lugar
da abolição das distinções de traje na universidade (com a exceção dos
clubes náuticos), ficaria feliz se as visse estendidas por toda a ordem
social do país. Penso que apenas às duquesas deveria ser permitido o uso
de diamantes; que os lordes deveriam ser reconhecidos, pelo povo, pelas
suas estrelas, a um quarto de milha de distância; que cada jovem
camponesa deveria homenagear seu condado com um gracioso chapéu ou
justilho alusivo; e que nas cidades um negociante de vinhos deveria
distinguir"-se de um peixeiro pelo corte de sua jaqueta.

Aquela caminhada até a universidade, e a espera, do lado de fora da
Divinity School, admirando confortavelmente sua porta, pelo momento de
minha matrícula, continua sendo, ainda para mim, uma fonte de prazer.
Mas não me lembro de nada mais daquele ano; nem nada dos primeiros dias
do ano seguinte, até que no início de janeiro fomos para Oxford, apenas
minha mãe e eu, através da bela estrada Henley, um pouco fatigados
quando trocamos os cavalos na última posta, em Dorchester; solenes, a
despeito do veludo e da seda, adentramos por entre as torres no
crepúsculo; e após uma noite de repouso sob o teto familiar do
\textit{Angel}, encontrei"-me, na tarde do dia seguinte, sozinho, ao lado da
lareira, no comando de minha própria vida, em meu quarto na
universidade, em Peckwater.

\chapter{O coro do Christ Church} %Capítulo \textsc{xi}. 

%215. 
Sentado sozinho, ao lado da lareira do pequeno quarto dos fundos
que dava para a rua estreita, então repleta de estrebarias, passei em
revista minhas decisões a respeito da vida universitária.

Não havia muito que considerar; nem contra nem a favor, pelo que sabia.
Tinha, sobre o lugar em que me encontrava, ou sobre o destino que me
esperava, uma compreensão tão clara quanto aquela que Davie
Gellatley\footnote{Personagem do romance \textit{Waverley}, de Walter
  Scott. Um tolo que, ao invés de responder as perguntas que lhe são
  feitas, recita baladas.} teria tido em meu lugar; com
algumas desvantagens em relação a Davie; eu não sabia dançar, nem
cantar, nem fazer ovos fritos. Não havia o menor temor quanto a jogar,
pois nunca tocara numa carta, e via os dados como as pessoas agora veem
a dinamite. Nem o receio de que fosse seduzido por alguma mulher, pois
eu não estava apaixonado? E, além disso, não era permitido estar fora do
quarto após as nove e meia. Nenhum temor de que incorresse em débito,
pois não havia Turners a serem adquiridos em Oxford, e nada mais de
material me interessava no mundo. Nenhum receio de que quebrasse o
pescoço caçando, pois não conseguiria descer a High Street montado num
pangaré; nem receio de que me arruinasse nas corridas de cavalo, pois
jamais assistira a uma corrida, e não tinha o menor desejo de ganhar
dinheiro de outra pessoa.

Na verdade, conjeturava que minhas maneiras simples pudessem ser
ridicularizadas, mas estava protegido contra isso por minha autoestima:
a única coisa sobre a qual tinha dúvidas, e muito fortes, era sobre
minha capacidade de me dedicar durante três anos a uma atividade pela
qual não tinha o menor interesse. Decidi, contudo, dar a meus pais e a
mim mesmo quanto crédito fosse possível, rezei seriamente, e fui para
cama com muita esperança.

%216. 
E aqui devo me deter, por um ou dois minutos, para explicar meu
estado de espírito à medida que avançava o já mencionado processo de
minha educação, no que se refere às questões religiosas.

Até onde me lembro, a leitura regular da Bíblia com minha mãe
encerrou"-se com nossa primeira viagem continental, quando tinha quatorze
anos; não se podiam ler três capítulos após o desjejum enquanto os
cavalos esperavam na porta. Essa tarefa foi substituída pela leitura,
que devia fazer sozinho, de um capítulo, pela manhã e à noite, seguida,
naturalmente, pelo Pai Nosso e por preces para que tudo corresse bem
comigo e com minha família; após o que, eu me levantava ou ia dormir,
sem pensar em nada mais além de afazeres terrenos, fosse noite ou fosse
dia.

Nunca passou pela minha cabeça duvidar de uma palavra da Bíblia, embora
já tivesse percebido que suas palavras deveriam ser entendidas num
sentido diferente daquele que me havia sido ensinado; porém, quanto mais
nelas acreditava, menos elas me faziam bem. Considerava certo que Abraão
fizesse aquilo que os anjos lhe ordenassem, --- eu também o faria, se
algum anjo me ordenasse; mas nenhum tinha aparecido para mim, pelo que
sabia, nem mesmo Adèle, que não poderia ser um anjo porque era católica
romana.

%217. 
Além disso, se eu tivesse vivido na época de Cristo, naturalmente o
teria acompanhado à montanha, ou navegado com ele pelo lago da Galileia;
mas isso era muito diferente de ir à igreja de Beresford, em Walworth,
ou a de Saint Bride, em Fleet Street. E também, embora me sentisse, de
algum modo, inclinado a imitar Christian, em \textit{Pilgrim's
Progress},\footnote{Conferir capítulo \textsc{i}, nota à página \pageref{peregrino}.} não
conseguia perceber que a Billiter Street e a Tower Wharf, onde meu pai
tinha suas adegas, ou o jardim com cerejeiras em flor, onde minha mãe
plantava suas flores em vasos, pudessem ser lugares dos quais eu devesse
fugir como da Cidade da Destruição. Sem muito refletir sobre o tema,
havia virtualmente concluído, a partir da leitura da Bíblia que, nunca
tendo pretendido fazer ou feito qualquer mal, que eu soubesse, não
corria risco de ir para o inferno: ao mesmo tempo, observava que mesmo o
\textit{crème de la crème} dos religiosos não parecia ansioso para a
chegada ao Paraíso. Em suma: parece que tudo que me requeriam era rezar
minhas preces, ir à igreja, aprender minhas lições, obedecer meus pais,
e degustar meu jantar.

%218. 
Com esta disposição de espírito, na luz lentamente revelada da
manhã de inverno, olhei através da janela da universidade para ver a
biblioteca de Christ Church e a Praça de Peckwater, coberta por fino
cascalho, um pouco contrariado por não estar olhando através de uma
janela ogival para uma igreja gótica: despercebido da verdadeira
condenação da qual fora vítima, ou do quanto perderia por ter apenas a
biblioteca de Christ Church e uma praça coberta por cascalhos para ver
através da janela durante as primaveras de dois anos da juventude.

Naquele momento compreendi que, embora enfadonha, a vista era grandiosa
e que sua arquitetura, embora renascentista, era arrojada, erudita, bem
proporcionada, e didática em vários aspectos. Na verdade, eu poderia ter
sido enviado à masmorra de Chillon,\footnote{Conferir capítulo \textsc{viii},
  nota à página \pageref{chillon}.} exceto pela umidade; melhor seria se
pudesse ver as três pequenas árvores através da fenda da janela, e belas
nervuras de abóbadas e pavimentos, ao invés da moderna e vulgar
tapeçaria que cobria os móveis de meu quarto.

Após ter visto as grandes igrejas do exterior, decepcionei"-me com a
capela da universidade à primeira visão; mas sob suas abóbadas
desenrolavam"-se diferentes serviços.

No geral, dos lugares e dos serviços importantes para as almas cristãs
da Inglaterra, o coro da Christ Church representava, naquele período da
história inglesa, virtualmente o umbigo, a sede da vida. Ali remanesciam
as tradições religiosas dos saxões, dos normandos, dos elisabetanos,
intactas, --- a memória da lealdade, da erudição, e, ao menos em
obediência nominal, e no coração com verdadeira docilidade, comparecia,
a cada manhã, animada pelo mais alto sentimento de dever perante a
nação, a mais nobre juventude da Inglaterra. A maioria dos pares da
Inglaterra e, como regra, o melhor de sua nobreza rural, passavam
necessariamente por Christ Church.

A própria catedral era um epítome da história inglesa. Cada pedra, cada
vitral, cada painel de madeira esculpida, era autêntico e pertencia a
seu tempo, --- e não a detestável imitação feita por um arquiteto. O
primeiro santuário de Saint Frideswide\footnote{Lendária princesa e
  abadessa inglesa (c. 650--727), a quem se atribui a fundação da catedral
  Christ Church, em Oxford.} foi destruído, e seu corpo
despedaçado e transformado em pó pelos puritanos; mas o segundo
santuário era ainda mais belo, --- a mais admirável realização dos
ingleses, com a mão e com o coração. As abóbadas acima eram
verdadeiramente normandas; reles e grosseiras, mas o melhor que podíamos
fazer com nossas próprias habilidades, sem ajuda dos franceses. O teto
era verdadeiramente Tudor, --- grotesco, construído com inventividade,
delicadamente esculpido; e junto com o teto do átrio da escada resumia
as habilidades construtivas do século quinze. A janela do oeste, com seu
canhestro vitral da Adoração dos Pastores, monumento da transição entre
o vitral e a tela que conduziu a quadros holandeses com gado, mas sem
pastores ou Cristo, --- mas, ainda assim, o melhor que se podia fazer
naquela época; e o trabalho em madeira das cadeiras do coro bem
representava a arte de viver da Inglaterra na forma de uma honesta e
confortável carpintaria.

%219. 
Nesse coro, tão íntima e consecutivamente associado à autêntica
histórica britânica, reunia"-se, a cada manhã, uma congregação que
representava o melhor que a Grã"-Bretanha obtivera, --- disciplinadamente,
como uma tripulação em tempo de guerra, no magnífico navio que é seu
templo. Cada homem em seu lugar, de acordo com seu posto, idade, seus
conhecimentos; cada homem, em razão e coração, compreendia estar
preparado, ou estar sendo preparado, para desempenhar as árduas tarefas
exigidas dos ingleses. Um estrangeiro bem"-preparado, admitido àquele
serviço matinal, poderia compreender e avaliar mais rápida e
precisamente o passado do país, e seu potencial futuro, do que se
passasse meses na corte ou na cidade. Ali, nas cadeiras do coro,
sentavam"-se os maiores teólogos da Inglaterra, --- e abaixo de seu nicho
de comando, seus maiores acadêmicos, --- entre os diretores de estudo,
encontrava"-se o deão Liddell, e um homem de singular capacidade
intelectual e virtudes simples, Osborne Gordon.\footnote{Osborne Gordon
  (1813--1883), clérigo inglês; tutor acadêmico de Ruskin em Oxford.} O grupo de nobres forneceu, através de homens como o
marquês de Kildare,\footnote{Lord William FiztGerald (1749--1804),
  político liberal e proprietário de terras irlandês.} o
conde de Desart,\footnote{Provavelmente John Otway O'Connor Cuffe, terceiro
  Conde de Desart (1818--1865), político conservador irlandês.} o conde de Emlyn e Francis Charteris,\footnote{Francis Wemyss
  Charteris (1723--1808), aristocrata escocês, sétimo Conde de Wemyss; um dos
  amigos de Ruskin em Oxford.} atualmente Lord Wemyss,
--- os mais brilhantes exemplos de elevada estirpe e capacidade de ação.
Henry Acland\footnote{Sir Henry Wentworth Dyke Acland (1815--1900),
  médico e pesquisador inglês, professor em Oxford.} e
Charles Newton,\footnote{Sir Charles Thomas Newton (1816--1894),
  arqueólogo e diplomata inglês; membro do Museu Britânico.} entre os estudantes não graduados mais velhos, e eu, entre os
mais jovens, mostravam, para os que pudessem reconhecê"-lo, traços das
singulares possibilidades dos dias futuros. Nenhum de nós tinha, então,
consciência de qualquer necessidade ou oportunidade de mudança, e, menos
ainda, o severo capitão, o qual, com suas sobrancelhas arredondadas e os
olhos escuros e brilhantes, conduzia, com seu vetusto e trovejante
latim, as preces matinais.

Por tudo o que vi e por tudo que aquele coro de catedral me inspirou,
sou profundamente grato.

%220. 
A influência sobre mim da segunda mais bela construção entre as
edificações da universidade, --- o vestíbulo, --- era de caráter
diferente, curiosamente mesclado. Tivesse sido unicamente utilizado,
como deveria ter sido, para festividades e pompas, --- refeições diárias,
recepção de convidados, discursos em ocasiões oficiais e situações
semelhantes, --- o vestíbulo, assim como a catedral, teria tido sobre mim
um efeito inteiramente salutar e beneficamente solene, consagrando o pão
nosso de cada dia, ou, caso o Abade tivesse, algumas vezes,
condescendido a jantar conosco, nossa eventual carne de caça. Mas, com o
extremo mau gosto (que, para mim, é nosso moderno pecado capital,
pináculo de nossa atração pelo dinheiro, e desprezo pelo valor do
dinheiro e por qualquer outro valor) --- mau gosto que, afirmo, fez com
que o Abade permitisse que nosso vestíbulo fosse utilizado para
``coletas''.\footnote{Coletas de dinheiro na igreja.} A
palavra me é absolutamente abominável, seja expressando caridades
extorquidas na igreja, seja significando conhecimentos arrancados
durante os exames.\footnote{Refere"-se, aqui, também à confissão extraída
  sob tortura durante a Inquisição e considerada como verdadeira.} ``Coletas'', no sentido acadêmico, significa o exame
realizado ao fim de cada trimestre na universidade, no qual o Abade
demonstra seu péssimo gosto, ao desempenhar o papel de nosso inquisidor,
embora nunca tenha estado presente em nossa mesa como anfitrião.
Naturalmente, a soma dos conhecimentos de grego existentes nas mentes de
todos os estudantes da graduação presentes no vestíbulo era, para
ele, infinitesimal. A princípio desdenhoso, e vingativo, sempre
trovejante, mais rabugento e ameaçador à medida que o dia avançava,
caminhava com terrível glacialidade gorgônia do estrado à porta, da
porta ao estrado, pela majestosa câmara de tortura, --- vasta como o
grande vestíbulo do conselho de Veneza, mas agora degradada pelo terror
de lastimáveis criaturas sem direito à opinião, que, assustadas como
andorinhas sob beirais, nada podiam fazer, exceto esconder suas colas a
cada passagem fatal do Abade. Claro que \textit{eu} nunca colei, mas creio
que o Deão teria preferido que tivesse colado cinquenta vezes a
apresentar o aspecto desconcertado e desesperançado que sempre exibia
durante a tarde, diante do trabalho que tinha a fazer. E como meu
conhecimento de latim fosse, suponho, o pior da universidade, --- já que
nunca pude distinguir o futuro do presente do futuro do pretérito, ou
mesmo, ao fim de meus estudos em Oxford, saber onde os pelasgos viviam,
ou de onde Heráclito retornou, --- pode"-se imaginar com que expressão
fisionômica o Deão me estendeu dois dedos de sua mão quando parti, e com
que alívio recebi os questionamentos de meus pais sobre as expectativas
que deveria ter em relação ao futuro, baseados na opinião de meus
professores.

%221. 
Com o tempo, o vestíbulo da universidade passou a representar, para
mim, um pouco mais do que o medo e a vergonha daqueles dias de exames;
mas mesmo diante da surpresa e da sublimidade de estar jantando ali,
compreendi serem várias as razões de meu prazer. A mudança de nossa sala
em Herne Hill, com quinze por dezoito pés, e da carne e do pudim com
minha mãe e Mary, para um vestíbulo tão grande quanto a nave da catedral
de Canterbury, com a extremidade se perdendo na névoa e o teto na
escuridão, e as pessoas estendendo"-se ao infinito, numa visão
inumerável, imensurável, era, em si mesma, mais assustadora do que
apetecível; mas, também, do primeiro ao último dia, tive o burlesco
sentimento de nada ter a fazer ali.

Na catedral, contudo, fosse pelo meu nascimento, fosse pela minha
educação, sentia que minha presença era tão justificada quanto a do
bispo, --- e, mais ainda, que pelas lições que dele se depreendia e pelas
suas tradições, o lugar pertencesse mais a mim do que a ele. Mas, à
mesa, diante da perspectiva de hóspedes cultos e nobres, afeitos ao
mundanismo, não estava à vontade; meu próprio estilo de jantar, percebi,
dissentiu para sempre daquele --- impassivelmente. Com as batatas cozidas
em volta do carneiro, recém"-saídas do fogão para a pequena antessala à
loja em Market Street ou ao lado de uma chaleira de ciganos na Addington
Hill (não que já estivesse estado ao lado de uma chaleira de ciganos,
mas muitas vezes quis estar); ou com um bolo de aveia e manteiga --- pois
sempre fui guloso --- numa cabana de pastor escocês, que seria dividido
com seu \textit{collie},\footnote{Cão pastor escocês de pelo longo.} eu era eu mesmo, estava em meu lugar: mas, à mesa dos
estudantes"-comuns, na sala de jantar do cardeal Wolsey,\footnote{Thomas
  Wolsey (1471--1530) foi um religioso e estadista inglês que desempenhou papel
  político essencial durante o reino de Henrique \textsc{viii}, empenhando"-se em
  manter a Inglaterra fora das guerras continentais e em centralizar os
  poderes políticos religiosos nas mãos do rei.} era,
por diversos motivos, menos que eu mesmo, e em todos os tipos de
situações equivocadas, sentia"-me fora de lugar.

%222. 
Posso também recordar um incidente algo cômico, extremamente
trivial, que aconteceu a seguir; o qual, a despeito de sua trivialidade,
posteriormente contribuiu para diminuir, em meu espírito, o encanto de
Christ Church. Fui recebido como um bem"-humorado e inofensivo pequeno
vira"-lata, com uma gentileza desdenhosa, à mesa dos cães de raça dos
estudantes"-\textit{gentlemen}; mas, pouco a pouco, meu tutor acadêmico e
meus colegas de classe começaram a reconhecer que eu tinha certo talento
para ler com boa entoação, refletir sobre o que tinha lido, e mesmo
fazer algumas perguntas embaraçosas sobre o que lera; até, um dia, ser
entusiástica e vigorosamente congratulado por toda a classe, ao sairmos
para o pátio quadrangular, pela maneira pela qual confrontei nosso
tutor. Tive tanto intenção de confrontar o tutor, ou a consciência de
fazê"-lo, quanto um recém"-nascido! Para grande júbilo de meus colegas,
simplesmente acontecera de perguntar"-lhe algo que não sabia responder.
Porém, um pouco antes de obter aquela aprovação pública, fiz uma
tentativa direta para ser visto de um modo favorável, que foi muito
menos bem"-sucedida.

Era um instituto da universidade que, a cada semana, o estudante da
graduação deveria escrever sobre um tema filosófico, no qual explicasse
algum breve texto em latim de Horácio, Juvenal,\footnote{Juvenal (Decimus
  Iunius Iuvenalis) (provavelmente 60--140), foi um poeta satírico romano, autor
  de \textit{As sátiras}.} ou qualquer outro escritor
reconhecido e com um estilo vigoroso; e, suponho, como uma espécie de
garantia aos estudantes de que seus textos seriam realmente avaliados, o
ensaio considerado o melhor seria lido publicamente no vestíbulo, no
sábado à tarde, com a presença obrigatória de todos os estudantes da
graduação. Ali havia, ao menos, uma situação, na qual, minhas parcas
capacidades tinham uma oportunidade de se manifestar e, tanto
conscientemente, quanto com um verdadeiro interesse pela tarefa, escrevi
meu ensaio semanal com toda sagacidade e eloquência que possuía. E
assim, embora muito lisonjeado, não fiquei surpreso, quando, algumas
semanas depois, meu tutor acadêmico me informou, com um olhar de
aprovação, que deveria ler meu ensaio no vestíbulo no próximo sábado.

%223. 
Serenamente, seguro de mim, confiante em minha capacidade de ler
corretamente, e com a grave decência que acreditava ser conveniente
àquela minha primeira distinção pública, li meu ensaio, de uma maneira,
a qual, tenho razões para crer, não foi deselegante; e desci da tribuna
para receber --- como não duvidara --- os agradecimentos dos
estudantes"-\textit{gentlemen} pela respeitável apresentação do
conhecimento daquele grupo. Mas, para a pobre Clara,\footnote{Conferir
  capítulo \textsc{x}, p.\,\pageref{212}.} após seu primeiro baile, receber os
cumprimentos de seu primo no vestiário foi menos surpreendente do que as
boas"-vindas que recebi de meus primos da mesa comunal. Não com
verdadeira inveja, mas com veemente desdém, cuja expressão variava nas
formas e maneiras da língua inglesa, desde o sarcasmo olímpico de \label{tom}
Charteris à saraivada de injúrias de Grimston,\footnote{Robert Grimston
  (1816--1884), famoso jogador de cricket e boxeador inglês, primeiro Conde de
  Verulam; um dos amigos de Ruskin em Oxford.} eles me
explicaram que eu tinha cometido o mais grosseiro crime de
lesa"-majestade\footnote{Em francês no original: \textit{lèse"-majesté}.} contra a ordem dos estudantes"-\textit{gentlemen}:
nenhum ensaio de estudante"-\textit{gentleman} deveria conter mais do que
doze linhas, com quatro palavras em cada linha; e que mesmo perdoando
minha insensatez, minha vaidade e meu desejo de demonstrar \textit{savoir
faire}, a impropriedade de redigir um ensaio sem qualquer conteúdo, como
um estudante vulgar, --- a negligência e a audácia de escrever um ensaio
que levasse um quarto de hora para ser lido, e o ler todo, poderia, uma
vez, ser relevada para um neófito como eu, mas que seria, no mínimo,
enviado para o hospício de Coventry caso tal fato se repetisse. Fico
feliz em recordar que caí de muito alto sem muito me ferir, ou mesmo sem
um embaraço excessivamente ridículo. Imediatamente reconheci a justeza
daquelas observações, embora não me lembre de ter modificado meu estilo
em futuros ensaios, nem que linha de conduta previ para o caso de
novamente obter o privilégio de dirigir"-me de maneira edificante à
congregação reunida aos sábados. Talvez o nível de meus ensaios tenha
realmente caído, ou talvez meus tutores tenham se cansado deles. Tudo o
que sei, entretanto, é que nunca mais me pediram para os ler.

%224. 
Deveria ter observado que meu primeiro contato com os estudantes, à
mesa, foi facilitado pelo acaso: uma tempestade encerrou"-me durante dois
dias no albergue de Grimsel, em 1835, com cerca de trinta viajantes de
vários países, entre os quais um estudante"-\textit{gentleman} de Christ
Church, Mr.\,Strangways,\footnote{Provavelmente William Fox"-Strangways, quarto
  Conde de Ilchester (1795--1865), político e diplomata inglês.} que jogou xadrez comigo e mostrou algum interesse pela
maneira pela qual eu desenhava rochas de granito sobre a neve. No
vestíbulo, ele me reconheceu imediatamente como igual; e o resto de sua
turma, acreditando que poderia se divertir à minha custa sem que eu o
percebesse, e que eu não pretendia modificar sua maneira de viver, a
partir de um ponto de vista evangélico, ou de qualquer outro igualmente
impertinente, tratou"-me gentilmente; de modo que, em cerca de duas
semanas, pude escolher as companhias que me agradavam, em toda a
faculdade.

Felizmente para mim --- felicidade que não pode ser expressa em palavras
--- Henry Acland, aproximadamente um ano e meio mais velho do que eu,
\textit{me} escolheu; percebeu possibilidades em mim que necessitavam de
apoio, e afetuosamente me estendeu a mão. Seus aposentos, próximos ao
portão do lado norte de Canterbury, a cinquenta jardas dos meus,
tornaram"-se o único lugar onde eu era feliz. Tranquilamente, mostrou"-me
como vivia um jovem inglês sensato, de boa família, e ampla educação;
nós dois já vivíamos distantes dos muros quadrangulares da faculdade.
Ele me falou sobre a planta da cidade de Troia; um ou dois anos mais
tarde, quando estava em sua viagem de núpcias, indiquei"-lhe o caminho
até Montanvert; nossa amizade nunca se alterou, tendo se estreitado até
os dias de hoje.

%225. 
Os outros amigos, alguns eram sensíveis e muitos, generosos; um
excelente tutor na faculdade; e, mais tarde, como professor particular,
o já citado Osborne Gordon, muito erudito e correto de caráter. Numa
extremidade do grande quadrilátero vivia Dr.\,Buckland,\footnote{William
  Buckland (1784--1856), religioso, geólogo e paleontologista inglês; o
  primeiro a descrever de modo completo um fóssil de dinossauro.} sempre disposto a me ajudar, --- ou, o que seria um favor
ainda maior, a ser ajudado por mim, ao aceitar que meus desenhos
ilustrassem suas aulas. Minha pintura dos veios de granito em Trewavas
Head, com um cúter que dobrava o cabo em meio a rajadas de vento, ao
estilo de Copley Fielding,\footnote{Anthony Vandyke Copley Fielding
  (1787--1855), aquarelista inglês, cujas marinhas obtiveram grande
  sucesso em sua época.} creio que ainda conste dos
arquivos do departamento geológico. Mr.\,Parker, fundador da Sociedade
Arquitetônica, e Charles Newton, já notável pela sua intensa e curiosa
maneira de perscrutar as coisas, estavam presentes para simpatizar
comigo e para me incentivar a um estudo mais preciso da arquitetura. A
cerca de oito milhas se encontravam as pinturas de Blenheim.
Considerando todas as oportunidades e privilégios, não poderia conceber
que um jovem de minha idade pudesse ser colocado em posição mais
favorável --- se tivesse a perspicácia para reconhecê"-los e a disposição
para utilizá"-los. Infelizmente permaneci --- ou titubeei --- em parte
indeciso, em parte estúpido, em meio a essa situação favorável: não
consigo imaginar, entre homens, pássaros ou bestas, nenhuma outra imagem
capaz de me representar, exceto a da pobre e pequena pintura que a
pastora Agnes fez do ``patinho perdido''.

%226. 
Computo a meu favor o fato de não ficado envergonhado, mas
contente, com a vinda de minha mãe a Oxford, a fim de cuidar de mim,
logo que lhe foi possível. Ao longo dos três anos de residência, durante
o período letivo, ela se instalou em High Street (a princípio, na bela
casa de madeira trabalhada, do século dezesseis, de Mr.\,Adam), e meu pai
permaneceu sozinho durante a semana em Herne Hill, aceitando a separação
da mulher e do filho, em benefício do último. Aos sábados, ele vinha nos
visitar, e nós três, juntos, íamos, no velho estilo doméstico, assistir
ao serviço dominical na igreja de Saint Peter: em outras ocasiões, não
nos mostrávamos juntos, com receio de que meus colegas rissem de mim, ou
que alguém fizesse perguntas maliciosas a respeito do papai comerciante
de vinhos e de sua esposa antiquada.

Nenhum de meus colegas, durante todo o período da faculdade, jamais
disse qualquer palavra depreciativa em relação a meus pais, ou de
sarcasmo, por eu passar as \textit{soirées} com minha mãe. Porém, certa
vez, quando a irmã mais velha de Adèle veio com seu marido visitar
Oxford, e mencionei, desnecessariamente, no jantar, que ela era a
condessa Diane de Maison, eles foram impiedosos comigo durante mais de
um mês.

O leitor deverá compreender que minha mãe não veio a Oxford porque não
podia se separar de mim, --- menos ainda, porque não confiava em mim. Mas
simplesmente porque gostaria de estar por perto em caso de acidente ou
de doença repentina. Ela sempre fora minha médica e minha enfermeira; em
várias ocasiões, sua pronta vigilância salvou"-me de graves perigos; o
desenrolar dos acontecimentos mostrará que sua cautela não era
injustificada. Durante os dois primeiros anos de minha vida
universitária não lhe causei preocupações; e meu dia era sempre mais
feliz porque podia lhe contar, à hora do chá, como o tinha aproveitado,
ou o que nele mais me agradara.

%227. 
A rotina diária certamente merece ser mencionada. Nunca deixei de
ir à capela; no inverno, lia durante uma hora antes da missa. Desjejum
às nove horas, --- a meia"-hora que lhe era destinada, dedicava ao Capitão
Marryat\footnote{Capitão Frederick Marryat (1792--1848), escritor inglês,
  um dos pioneiros das histórias que se passam no mar.}
enquanto comia meu pão com manteiga. Aulas na faculdade até uma hora.
Almoço, conversa com quem se aproximasse, disposto a dividir suas
trivialidades comigo. Às duas, aula de Buckland ou de outro professor.
Caminhada até às cinco, jantar no vestíbulo, vinho em meu quarto ou no
de outro estudante, bate"-papo tranquilo durante a bebida, ou
brincadeiras com os colegas de minha mesa; mas sempre comparecia ao chá
com minha mãe, às sete horas, na High Street, e me divertia até que
Tom\footnote{Tento escrever sem notas, porém, em benefício dos leitores
  não ingleses, devo explicar que ``Tom'' é o nome do grande sino de
  Oxford, que fica na torre ocidental de Christ Church. {[}\textsc{n.\,a.}{]}}
soasse e eu corresse até o portão de Canterbury, e lesse ainda um pouco
até às dez horas. Não passavam de seis horas de atividade por dia, mas
era trabalho constante e persistente.

%228. 
Minha história herodotiana\footnote{Refere"-se ao livro
  \textit{Histórias}, de Heródoto (485?--420 a.C.), geógrafo e historiador
  grego, considerado o ``pai da história''.} foi, de
qualquer modo, plenamente assimilada, e permanece como propriedade
valiosa até os dias de hoje. Além disso, meu tutor acadêmico, Mr.\,Walter
Brown, tornou"-se alguém muito querido por mim, e gentilmente me
encorajou a adquirir algum conhecimento dos verbos gregos. Meu progresso
na matemática também se fez sob a orientação de outro tutor, de quem
gostava muito, Mr.\,Hill; minha inclinação natural para a geometria pura
era veemente e minha compreensão da matéria, até onde avancei, foi
completa. Em meu \textit{little go},\footnote{``\textit{Little go}'' ou
  ``\textit{small go}'' refere"-se ao primeiro exame de admissão em Oxford
  e Cambridge.} na primavera de 1838, tendo me sido
apresentado os diagramas de Euclides, como era habitual durante os
exames, devolvi os livros ao examinador, dizendo, com desdém: ``Não
preciso de qualquer desenho, \textit{Sir}''. ``É melhor aceitá"-los'', ele
retrucou, benevolente; o que fiz, atendendo a seu pedido; mas era capaz,
como ainda sou, de apresentar, com os olhos vendados, a demonstração de
qualquer problema, se indicarem as letras que correspondem aos diversos
pontos. Fiz o mínimo para passar, e não mais que isso, no exame de
latim; fui honrosamente aprovado nas disciplinas restantes, deixando meu
tutor acadêmico satisfeito comigo --- sem, entretanto, reconhecer que o
\textit{litlle go} tinha exigido, e obtido de mim, quase tudo que eu
sabia ou poderia saber naquele campo do conhecimento.

%229. 
Era um grande infortúnio para mim que dois dos maiores professores
da faculdade, Kynaston (posteriormente mestre do Colégio de Saint Paul),
em grego, e Hussey, o censor, embora não me recorde o que exatamente me
desagradava neles, fossem"-me ambos antipáticos. Para começar, eles me
desprezavam, por ter sido educado em casa; Kynaston, com razão, pois meu
grego não permitia entender nada do que dizia; e um dia, dotado de
boa"-vontade, disposto a permitir que meus supostos talentos e
conhecimentos aflorassem, ao me pedir para traduzir\ldots{} (em grego, no
original), da \textit{Ifigênia em Tauris},\footnote{Título de uma tragédia
  do dramaturgo grego Eurípides (c. 480--406 a.C.).} descobriu, para sua surpresa e desgosto e de toda a classe,
que eu não sabia o que era um tríglifo; --- nunca mais se dirigiu a mim
com a mesma paciência, até muito depois em Saint Paul, quando, por
ocasião de uma cerimônia universitária, recebeu"-me com afeição e
respeito.

Hussey era considerado, exceto pelos melhores estudantes da
universidade, um censor muito rigoroso; e o comportamento dos estudantes
da faculdade era tal, que, infelizmente, faria qualquer censor sensato
adotar o rigor. Ele tinha, como o quisera a Providência, um semblante
sinistro; e fora visto por mim, do primeiro ao último dia, como uma
Górgone\footnote{Cada uma das três personagens da mitologia grega:
  Esteno, Euríale, Medusa. Mulheres que tinham serpentes por cabelos e
  que transformariam em pedra quem as encarasse.} ou uma
Erínia\footnote{Uma das divindades infernais da mitologia --- Tisífone,
  Aleto e Megera, --- filhas da deusa Gaia, cuja missão era vingar os
  crimes dos homens.} de Christchurch, cuja passagem
lançasse uma sombra tanto no ar como sobre o pavimento de pedra.

Ao lançar um olhar retrospectivo, divirto"-me ao perceber que tinha de
meus tutores acadêmicos e colegas uma visão estética, --- quão
consistentemente eles adquiriram para mim o aspecto de quadros, e como,
desde o início, recusei"-me a prestar atenção àqueles que não tinham sido
bem pintados. Meu ideal de tutor era baseado na representação que
Holbein\footnote{Hans Holbein, O Jovem (1497/\,98--1543). Pintor e
  desenhista alemão, um dos mestres do retrato no Renascimento.} ou Dürer\footnote{Albrecht Dürer (1471--1528), gravador, pintor
  e ilustrador alemão.} fizera de Erasmo\footnote{Desiderius
  Erasmus Roterodamus, conhecido como Erasmo de Roterdã
  (1466--1536).
  Teólogo e humanista neerlandês, autor
  de \textit{Elogio da Loucura}, espiritualmente considerado como o
  ``primeiro europeu''.} ou na
\textit{Melancolia},\footnote{Obra de Albrecht Dürer.} ou, até mesmo de um modo mais solene, dos Magníficos
de Ticiano\footnote{Tiziano Vecellio ou Vecelli (1490--1576), pintor
  italiano, um dos principais representantes da escola veneziana durante
  o Renascimento.} ou dos Bispos de Bonifazio.\footnote{Bonifazio
  Veronese (1487--1553): pintor italiano, também conhecido como Bonifácio
  de Pitati.} Nenhum personagem desse tipo existia na
torre com o grande relógio de Oxford nem no pátio quadrangular de
Peckwater; e mesmo o doutor Pusey (que também nunca me dirigiu a
palavra) não era uma figura pitoresca ou portentosa, mas apenas um
clerical gentil"-homem inglês, doentio e um tanto incompetente, que nunca
olhava alguém nos olhos, e parecia nunca se dar conta das condições
meteorológicas.

%230. 
Meu tutor acadêmico tinha olhos escuros, era vivaz, agradável, mas
não exatamente uma pessoa que causasse forte impressão; movia"-se com um
ar de empedernida presunção, que o desfavorecia diante de nós, seus
alunos. Kynaston era jocoso como um escolar gordo. Hussey, sinistro e
moreno, como já disse, algo lânguido, incapaz de gracejar, igualmente
incapaz de entusiasmar"-se; de resto, realizava suas tarefas com
seriedade, e era um membro estimado da universidade, --- mas, para mim, a
maior calamidade residente que conheci, cuja maléfica influência apenas
retrospectivamente reconheci.

Finalmente, o próprio deão, que embora venerável para mim, desde o
início, por sua evidente honestidade e respeito próprio, e pelo seu real
poder, exercido de um modo rude, era, entretanto, em seu aspecto geral,
muito parecido com o pudim de passas na forma de Porco Vermelho que
posteriormente vi um criativo merceeiro fazer na feira de Chartres,
utilizando groselhas negras como olhos; e em toda sua presença física e
fantasmagórica, ele era para mim apenas um rotundo e progressivo terror,
ou o Anátema firmemente acomodado e entronizado.

Havia, entretanto, um tutor acadêmico, com o qual não mantinha relação
alguma, que personificava meu ideal, mas, para o desapontamento de minha
esperança, naquele momento, --- como talvez da sua própria, desde sempre;
um homem pesarosamente sob o domínio do conceito grego de
destino  --- o atual deão. Ele era, e é, um dos raros ingleses com um
porte naturalmente nobre, mas imagino que a maléfica influência dos
astros fora a responsável por fazê"-lo nascer inglês --- a praticidade e o
pragmatismo prevaleciam sobre a sensibilidade. Ele era o único professor
em Oxford, em minha época, que sabia alguma coisa sobre arte; e sua
perspicaz observação sobre Turner, ao afirmar que ele tinha ``se
prendido a um falso ideal'', me teria sido de imensa utilidade àquela
época, se o tivesse explicitado e o feito prevalecer. Mas suponho que
não me considerasse suficientemente interessante para gastar seu tempo
comigo, --- e, o que era mais grave, que não visse em si mesmo, nesse
campo, nada que justificasse levar"-se a sério.

%231. 
Existia um espírito mais humano e mais vivo, contudo, habitante da
esquina noroeste da Praça do Cardeal: e a maioria dos infortúnios que me
aconteceram, por minha própria leviandade, puderam ser contrabalançados
por um único golpe de sorte, o qual tive a sabedoria de reconhecer e de
aproveitar. Dr.\,Buckland era cônego da catedral, e ele, a mulher e a
família, eram todos sensatos e afáveis, com uma originalidade suficiente
para dar vida e colorido a toda a universidade.

A originalidade era tal que, ao tocar levemente o grotesco, diminuía seu
efetivo poder. O humor do doutor alcançava até mesmo os aspectos mais
fastidiosos das questões. Frank\footnote{Certamente Frank Buckland, apelido de Francis Trevelyan Buckland (1826--1880), foi cirurgião, zoólogo, escritor
  e naturalista inglês; filho de William Buckland.}
gostava muito do filhote de urso que existia dentro dele para se
preocupar em lhe ensinar boas maneiras; e raramente passava"-se um dia
sem que sua filha Mit não incorresse em algum comportamento desaprovado
pelas senhoras que dirigiam a faculdade. Mas todos eles eram francos,
amáveis, inteligentes e vitais no mais elevado grau; para mim,
medicinais e salvadores.

Dr. Buckland parecia"-se muito com Sydney Smith\footnote{Sidney Smith
  (1771--1845), escritor e clérigo inglês.} nos traços
gerais do caráter; não o rivalizava em perspicácia, mas o igualava em
humor, sensatez e no conhecimento alegre e benevolente da doutrina da
Divindade. Em sua mesa, durante o desjejum, eu encontrava os homens de
ciência mais notáveis da época, a começar com Herschel,\footnote{John
  Frederick William Herschel (1792--1871), matemático e astrônomo inglês,
  filho do famoso astrônomo William Herschel.} e,
frequentemente, estrangeiros inteligentes e corteses, --- com os quais
meu francês tartamudeante, cuja pronúncia fora refinada por
Adèle,\footnote{Conferir capítulo \textsc{x}, p.\,\pageref{207}.} algumas
vezes era útil. Todos estavam à vontade e se divertiam naquela mesa de
desjejum, --- o cardápio e o serviço também eram interessantes. Lamento
sempre o dia, em que, devido a um malfadado compromisso, perdi uma
delicada ``torrada de rato'';\footnote{Refere"-se, provavelmente, a uma
  refeição com o excêntrico William Buckland, que costumava servir
  estranhos animais --- ratos, crocodilos, panteras --- como alimentos a
  seus convidados.} e recordo"-me, deliciado, ser
esperado numa manhã de verão particularmente quente por dois pequenos e
graciosos lagartos da Carolina,\footnote{Outra provável referência aos
  animais que o excêntrico naturalista William Buckland mantinha em
  cativeiro.} que mantinham as moscas afastadas.

%232. 
Já mencionei a imensa e incalculável felicidade que foi para mim o
fato de Acland ter me acolhido nos meus primeiros e insensatos dias, e
que, com uma gentil ironia e uma amável perspicácia, --- ou, melhor, com
simpatia para com o que havia de melhor em mim, e olhos fechados para o
que havia de pior, --- tenha me dado a ventura de observar a vida de um
jovem nobre inglês em sua pureza, sagacidade, honra, despretensiosa
coragem, alegre piedade; seu orgulho inglês irradiava seu brilho através
de tudo, com uma garota em sua beleza. É extremamente interessante, para
mim, contrastar o brio silenciosamente consciente de um inglês, no qual
realmente ele \textit{é}, com a inquietação irritada e ignóbil de um
francês, em sua sede de \textit{gloire},\footnote{Glória, em francês, no
  original.} a ser obtida através de esforços
desesperados para tornar"-se algo que ele \textit{não é}.

Um dia, quando o Cherwell corria caudaloso por sobre um de seus diques
mais escorregadios, surgiu a questão, entre mim e Acland, se o rio
poderia ser vadeado, e tendo eu declarado peremptoriamente que seria
intransponível, Acland instantaneamente tirou as botas e as meias e
caminhou sobre o dique, indo e voltando. Ele não correu risco, exceto o
de um estrondoso banho, porque, naturalmente, era um grande nadador: e
suponho que ele fosse suficientemente sensato para não tê"-lo feito, caso
houvesse perigo real. Certamente a margem de segurança era estreita; mas
ele possuía, no grau mais elevado, e, em certo sentido, no grau mais
risível, a serenidade constitucional inglesa diante do perigo, a qual,
nos néscios, degenera no prazer do risco pelo risco, mas que nos sábios,
sejam eles soldados ou médicos, alicerça a mais afortunada e rápida
decisão graças à pronta reflexão. Quando, trinta anos mais tarde, Dr.\,Acland, naufragou a bordo do vapor \textit{Tyne}, ao largo da costa de
Dorset, tendo o barco permanecido toda a noite preso nas rochas, --- não
se sabia em quais rochas, --- e o amanhecer revelado a existência de meia
milha de perigosa ressaca entre o navio e a praia, --- os oficiais,
discutindo ansiosamente, a tripulação, confusa, os passageiros,
histéricos ou rezando; todos ficaram perplexos, muitos escandalizados,
quando o Dr.\,Acland surgiu do salão, num meticuloso traje matinal,
anunciando que ``o desjejum estava pronto''. Ao impaciente clamor de
indignação com o qual sua antipática conduta foi recebida, ele retrucou,
mostrando que nenhum barco poderia chegar até a praia, muito menos de
ela vir, devido às condições do mar, e que enquanto esperavam, estando a
maioria molhados, todos com frio, e que, no melhor dos casos, seriam
puxados com cordas através das ondas, ou talvez tivessem que nadar por
suas vidas, seria extremamente prudente começar o dia, como
habitualmente, com o desjejum. A histeria cessou, a confusão arrefeceu,
cada um retomou sua presença de espírito, e nenhuma vida se perdeu.

%233. 
Com todo esse orgulhoso e jovial heroísmo da juventude, Henry
Acland deleitava"-me como o leopardo ou o falcão o fariam, sem que seu
comportamento afetasse meu próprio caráter, pelo exemplo. Eu fora
demasiadamente vezes instado e comandado para me importar comigo, até
mesmo para segui"-lo através de diques escorregadios, ou para
acompanhá"-lo num escaler através de águas turbulentas e perigosas; mas,
tanto em arte como em ciência, ele fora capaz de me incentivar,
encontrando"-se anos a minha frente, e ficara grato pelo meu interesse,
pois, até minha vinda, era literalmente o único na universidade a se
interessar por esses temas. Para o Dr.\,Buckland, geologia era apenas a
agradável ocupação de sua feliz vida. Para Henry Acland, fisiologia era
um evangélico o qual ele fora solitariamente incumbido de pregar aos
pagãos; e ainda em seus aposentos de estudante de graduação, em
Canterbury, desenvolvera o projeto --- que poucos anos mais tarde, já em
seu quarto de professor, no quadrângulo do Tom,\footnote{Conferir
  Capítulo \textsc{xi}, nota à página \pageref{tom}.} começara a realizar, --- de
introduzir o estudo da fisiologia, o qual fez da universidade o que ela
é hoje.

Na verdade, o aspecto curioso do caráter de Acland era sua precoce
plenitude. Já nesses anos juvenis, seu julgamento era infalível, seus
objetivos definidos, suas capacidades desenvolvidas; e, não tivesse sido
ele, com o tempo, dirigido à rotina do trabalho profissional, e se não
tivesse ficado satisfeito pela serenidade, para não dizer, seduzido
pelos prazeres de uma bela vida doméstica --- inútil imaginar o que ele
poderia ser sido; aqueles que o conhecem melhor são os mais gratos por
ele ser o que é.

%234. 
Depois de Acland, mas dele separado por uma parede com muitos pés
de espessura, em minha estética eleição de ídolos, que requeria, de
homem ou mulher, sobretudo graça, antes de me interessar por suas outras
qualidades, vinha Francis Charteris. Sempre considerei Charteris o tipo
ideal de escocês, e, no geral, o mais perfeito exemplo de europeu
circassiano que já conhecera; e seu sutil, natural, inevitável, não
malicioso sarcasmo, juntamente com o firme e sempre presente bom senso,
conferia uma grandeza natural e inofensiva à sua delicada beleza. Ele
podia fazer o que quisesse com qualquer um, --- ao menos com alguém
bem"-humorado e que lhe fosse simpático; e um dia, quando o velho
vice"-deão saía do portão de Canterbury, no momento em que Charteris
desmontava do cavalo vestido de rosa, cor proibida, este se virou
calmamente para o vice"-deão e, enquanto tirava o pé do estribo,
informou"-lhe que ``estivera fora com os cães de caça do deão''; tanto o
velho professor quanto o jovem regojizaram"-se.

Charteris nunca negligenciou qualquer assunto, mas, também, nunca foi
perturbado por qualquer questão. Dotado de natural habilidade, expedito,
fazia todas as coisas com desembaraço, --- nunca caía nas caçadas, nunca
tropeçava nas leituras, nunca se mostrava presunçoso ou ansioso durante
os exames, --- nunca se desesperava diante dos infortúnios da vida. Ele
era parcialmente confrontado, o que em certa medida o enfraquecia, pela
natureza tísica de sua constituição, certamente o verdadeiro motivo pelo
qual não deixou uma marca mais profunda na vida.

%235. 
Após Charteris, vinha o conde de Desart, que, dentre aqueles com os
quais compartilhava as refeições, era o que mais me interessava. Um
jovem com o mesmo brilho promissor, o mesmo tipo de distinção, mas com
menos diligência natural e menos --- sendo irlandês --- bom senso, que o
escocês; e a universidade não fez qualquer tentativa de lhe tornar mais
sensato. O orgulho de nossa época tem sido igualar as posições sociais,
e atenuar as distinções entre nobres e serviçais. Talvez fosse mais
sábio, ao invés de se abolir as diferenças, se inverter os
comportamentos. Naqueles dias, o trabalho feliz de um serviçal num
quarto da universidade e sua renda dependiam de seu zelo, enquanto que
era privilégio dos nobres manter a universidade através de generosas
doações, sem esperar nada em troca, e comprar, com somas correspondentes
às suas dignidades, os privilégios de rejeitar tanto a instrução
oferecida como o controle exercido sobre eles. Parece"-me estranho, e
pouco favorável à sagacidade do caráter inglês, que nunca tenha ocorrido
a um velho deão, ou a um jovem duque, a possibilidade da igreja da
Inglaterra e da Câmara dos Lordes ocuparem uma posição diferente no
país, no futuro, caso o exame de admissão à universidade tivesse sido
mais rigoroso para os ricos do que para os pobres; e que a nobreza e a
boa educação de um estudante pudesse ser consistentemente demonstradas
tanto pelo brasão gravado em seu selo, pela borla de seu barrete, como
pela graça de sua conduta, pela exatidão de seus conhecimentos.

Com respeito ao último aspecto, na verdade, os rapazes de Eton e Harrow
serão sempre distintos, --- não importa se ociosos ou diligentes após a
escola, --- de toda a juventude da Inglaterra; porém, quanto da melhor
capacidade da nobreza é perdida pela negligência da educação
universitária, o país logo terá motivos mais sérios para se preocupar do
que minha disposição para o presságio.

Tenho pouco a recordar de meu admirado colega irlandês além de que ele
ofereceu o jantar durante o qual minha iniciação de calouro diante dos
estudantes"-\textit{gentlemen} foi apropriada e formalmente ratificada.
Olhares curiosos me eram dirigidos enquanto me provia das torradas
necessárias, --- mas não tinha ocorrido à hospitalidade de meus
anfitriões que provavelmente eu soubesse sobre vinhos tanto quanto eles.
Quando nos separávamos, em horas avançadas, eu ajudava a carregar,
escada abaixo, o filho do diretor de minha faculdade, e depois caminhava
através de Peckwater até meus aposentos, considerando, durante o
percurso, a possibilidade de um método trigonométrico empírico capaz de
determinar se andava em linha reta até a lâmpada acima da porta.

%236. 
A partir desse momento --- isso é, a partir da terceira semana em
que cheguei à residência --- começou"-se a reconhecer que, embora eu
pudesse ser desajeitado ou efeminado, podia me garantir, quando
necessário; o que foi confirmado, no semestre seguinte, quando tive que
retribuir as amabilidades, e servi bons vinhos, de cepas raras, sem
qualquer gosto de rolha; e encarei, com bom"-humor, as frutas que fizera
vir até Londres serem lançadas pela janela aos filhos do porteiro: mais
ainda, quando me mostrei capaz de aceitar qualquer tipo de brincadeiras,
embora fosse incapaz de fazê"-las, e quando estive extremamente
interessado em escutar conversas sobre tópicos dos quais nada sabia, ---
a um ponto que Bob Grimston concedeu em me levar com ele à taberna além
de Magdalen Bridge, para ouvi"-lo esclarecer, junto ao proprietário,
questões relacionadas aos cavalos que correriam o Derby,\footnote{Grande
  corrida anual de cavalos que acontece em Epsom, Inglaterra, criada por
  Lord Derby, em 1780.} algo que só se pode propriamente
fazer, indiferentemente sentado, no canto de uma mesa de cozinha,
acompanhando o diálogo repleto de pausas cuidadosas, e mais com
piscadelas de olhos do que com palavras.

Os estudantes mais tranquilos do grupo eram também os mais interessados
em meus desenhos; um ou dois --- Scott Murray,\footnote{Provavelmente
  Charles Robert Scott Murray (1818--1882), aristocrata e político inglês.} por exemplo, e Lord Kildare --- eram tão assíduos
quanto eu à capela, e tinham algumas ideias concernentes à vida
universitária e suas questões, as quais ficavam felizes em compartilhar
comigo. No segundo ano de residência, minha posição na faculdade
continuou igualmente confortável, e notavelmente satisfatória para meus
pais: e fui admitido sem objeções na Christ Church Society, que mantinha
um tranquilo clube na esquina de Oriel Lane, em frente ao ``belo
portão'' da faculdade de Saint Mary; em cujos livros eram inscritos os
nomes da maioria dos bons homens partícipes da mesa privilegiada, que
haviam passado por Christ Church nos últimos dez ou doze anos.

%237. 
Sob essas luxuosas, e --- aos olhos do mundo --- honrosas condições,
meu espírito recobrou, gradualmente, sua tranquilidade e vitalidade, e
no avanço diário, embora infinitesimal, em direção à sensatez, acredito
ter feito um trabalho melhor e mais árduo nas leituras da faculdade do
que consigo me lembrar. Parece"-me, agora, como se eu conhecesse
Tucídides\footnote{Aristocrata, historiador e militar grego. Nasceu
  provavelmente entre 460 e 455 e morreu por volta de 400 a.C. Autor
  do clássico \textit{História da Guerra do Peloponeso}, na qual
  inaugura uma concepção historiográfica realista e crítica,
  afastando"-se do gosto pelo fabuloso e pelo exótico dos historiadores
  de sua época.} como conhecia Homero (o de Pope!), desde
que fui capaz de soletrar; mas a verdade é que, para um jovem que sabia
tão pouco grego aos dezessete anos, conhecer cada sílaba de seu
Tucídides aos dezoito anos e meio, implicava estudo regular da obra. A
perfeita honestidade do soldado grego, sua nobre educação, sua
consciência política, e o desdém pela forma, o que lhe permitia urdir o
conteúdo num ritmo forte que se ondulava e se contorcia
intermitentemente, tudo isso o fazia extremamente interessante como
escritor; enquanto que seu tema, a tragédia central do mundo, o suicídio
da Grécia, percebi"-o com uma simpatia que conduziu, por muitos anos, as
melhores potencialidades de meu coração e de minha mente à excelsitude.

Abro, e deixo ao meu lado, enquanto escrevo, o terceiro volume de
Arnold,\footnote{Provavelmente Thomas Arnold (1795--1842), educador e
  historiador inglês.} perfeitamente limpo e bem
preservado, no qual muito trabalhei e que muito me consumiu; meus
resumos, cuidadosamente escritos, ainda se encontram encerrados entre
suas páginas; e leio, com grata surpresa, a frase final do editor no
prefácio datado ``Fox How, Ambleside, janeiro de 1835''.

\begin{quote}
Nem a mais selvagem extravagância do ateísmo perverso dos tempos
modernos pode ir além de onde foram os sofistas da Grécia. Tudo que a
audácia pode ousar e a sutileza conceber para fazer as palavras \textit{bom}
e \textit{mau} mudarem seus significados já foi tentado nos dias de Platão,
que pela sua eloquência, sua sabedoria, sua fé inabalável, as
derrotou.
\end{quote}

\chapter{A capela Roslyn} %Capítulo \textsc{xii}. 

%238. 
Devo ainda retornar, antes de encerrar o relato descontínuo desses
primeiros vinte anos, a alguns dias dispersos pelo ano de 1836, quando
se deram os acontecimentos que conduziram aos períodos de trabalho dos
quais falarei no próximo volume.

Não consigo lembrar a data na qual meu pai comprou seu primeiro Copley
Fielding, --- ``Between King's House and Inveroran,
Argyllshire''.\footnote{Em inglês, ``Entre a morada do rei e Inveroran, Argyllshire''.} Para nós, custou uma fortuna ---
quarenta e sete guinéus;\footnote{Moeda de ouro inglesa existente entre
  1663 e 1813, equivalente, a princípio a 20 xelins; posteriormente a 21
  xelins. Utilizada como moeda de cálculo para o pagamento de salários
  profissionais e transações de alto valor, como obras de arte, joias,
  tecidos finos, cavalos, terras.} e no dia em que o
quadro chegou a nossa casa houve festa; passamos os dias seguintes
admirando"-o, e imaginando que as colinas e a chuva fossem verdadeiras.

Meu pai e eu concordávamos plenamente sobre o valor de Copley Fielding,
e, no fundo de meu coração, agora desejava ter vivido em Land's End, e
nunca ter conhecido outras obras de arte além das de Prout e as de meu
pai. Ficamos muito emocionados em conhecer o artista, e muito felizes
com sua amizade: era o mais modesto dos presidentes, o mais simples dos
pintores, sem qualquer traço de romantismo, mas como o mais puro amor
pela luz do sol que brilha a cada dia e pelas colinas eternas. Enquanto
Stanfield e Harding e Roberts\footnote{David Roberts (1796--1864) foi um pintor
  escocês que viajou pela Europa, norte da África e a Terra Santa.
  Conhecido pelas representações de paisagens do Egito e da Palestina.} faziam o grande \textit{tour} pela Itália, Sicília,
Stíria, Boêmia e Ilíria, pelos Alpes e pelos Pirineus, e pela Serra
Morena, --- imagine que Fielding nunca fez a travessia até Calais, mas,
ano após ano, retornava a Saddleback e a Ben Venue, ou, menos
ambiciosamente ainda, a Sandgate e às dunas de Sussex.

%239. 
Os desenhos que fiz em 1835 eram realmente interessantes, mesmo
para artistas, e pareceram suficientemente promissores a meu pai para
justificar minha promoção da tutoria de Mr.\,Runciman\footnote{Conferir
  capítulo \textsc{iv}, p.\,\pageref{84}.} aos privilégios mais elevados da instrução
artística. Aulas ministradas pelos membros da Water"-Colour
Society\footnote{Sociedade da Aquarela.} custavam um
guinéu; consideravam"-se seis lições como suficientes para a produção de
um aquarelista amador de razoável habilidade. Naturalmente, o professor
escolhido só poderia ser um. E não sei quem mais se deleitou com as seis
horas passadas na Newman Street, se meu pai ou eu: o grande fascínio que
ele tinha pela obra de Fielding permitia ao pintor conversar
prazerosamente com meu pai enquanto ministrava minhas aulas. A conversa
de meu pai (quando se dispunha a falar) não era indigna da atenção do
pintor, embora todo o potencial de sua inteligência só se manifestasse
através da escrita.


%240. 
Tive a sorte de encontrar uma carta de Northcote,\footnote{Thomas
  James Northcote (1746--1831), pintor e escritor inglês.}
escrita em 1830, que revela o valor que o velho pintor atribuía à
opinião de meu pai sobre um trabalho literário que permanece clássico
até hoje e que, indubitavelmente, consiste no melhor exemplo de crítica
com base nos princípios da escola de Sir Joshua:\footnote{Sir Joshua
  Reynolds (1723--1792) foi influente pintor inglês, notável por seus
  retratos, um dos fundadores da Royal Academy of Arts e seu primeiro
  presidente.} %---

\begin{quote}
\begin{flushright}
\hfill\textsc{13 de outubro de 1830}\\
\textit{Argyll House}
\end{flushright}
\bigskip

Caro senhor,

Recebi sua amável e reconfortante carta; embora
lamente muito por sua grave doença; espero que já tenha recuperado a
saúde. Os elogios que o senhor teve a bondade de me fazer, assim como o
\textit{Volume das Conversações} enviado, deram"-me mais prazer que talvez
tenha imaginado, visto que o livro foi publicado sem a minha permissão,
e, sua primeira aparição nas revistas ocorreu com meu total
desconhecimento. Fiz tudo que era possível para evitar sua publicação,
porque contém várias opiniões de certas pessoas, duras e cruéis, as
quais não gostaria de ver impressas num livro; além disso, embora
Hazlitt\footnote{William Hazlitt (1778--1830), escritor inglês, conhecido
  por seus ensaios e aforismos.} seja um homem de grande
talento, possui um desejo de ferir os outros, e frequentemente apresenta
de modo exagerado aquilo que lhe disse confidencialmente. Contudo,
agradeço a Deus que este livro, cujo aparecimento diante do mundo me fez
tremer, tenha sido recebido com uma generosidade que eu não esperava; e
a aprovação de um espírito como o vosso oferece"-me (\textit{sic} ---
abreviação de: ``só pode oferecer"-me'') a maior consolação que poderia
receber, e deixa"-me mais tranquilo.

Por favor, apresente meus respeitosos cumprimentos à senhora Ruskin,
quem espero que se encontre em boa saúde, e minhas saudações a seu
filho. --- Sempre, caro senhor, seu mais grato amigo\footnote{Em memória
  desse tranquilo ancião que nos honrou com sua amizade, e pelo valor de
  sua obra, espero reimprimir os textos das \textit{Conversações} que,
  creio, ele gostaria que fossem preservados. {[}\textsc{n.\,a.}{]}} e mais
humilde servo.

\smallskip

\begin{flushright}
\hfill\textit{James Northcote}
\end{flushright}

\end{quote}

\medskip

%241. 
E, então, as seis lições propostas na Newman Street alongaram"-se
talvez em oito ou nove, durante as quais Copley Fielding me ensinou a
diluir suavemente as cores em sucessivos matizes, a escurecer o cobalto
com o rosa garanço para obter o amarelo ocre para os céus, a utilizar um
traço irregular e eriçado para o topo das montanhas, a representar lagos
tranquilos através de largas faixas de sombra com linhas luminosas entre
elas (geralmente na mesma distância que separa as linhas impressas neste
texto), a produzir nuvens escuras e a chuva com doze ou vinte pinceladas
sucessivas, e a fragmentar a terra de Siena com o pincel seco para fazer
a folhagem e o primeiro plano. Graças a essas instruções, consegui
copiar um desenho que Fielding fez à minha frente, com cerca de doze por
nove polegadas, de Ben Venue e das montanhas Trosachs, com vacas marrons
pastando em Loch Achray, do qual gostei tanto que o coloquei sobre a
cornija da lareira de meu quarto: era a última coisa que contemplava
antes de dormir e a primeira ao acordar, com arrebatamento e um misto de
autocomplacência e consciência da nova habilidade, nos quais flutuava
durante todo o dia, como uma recém"-descoberta espécie que boiasse no ar.

Pouco tempo depois, contudo, concluí que este primeiro grande passo não
significava progresso consistente e regular. Observei que minhas
pinceladas, embora numerosas e cuidadosamente executadas, não pareciam
tão homogêneas como as de Fielding, e que minha fragmentação da terra de
Siena se tornava desinteressante após algumas repetições.

Para minha grande decepção, percebi que os processos de Fielding não
podiam ser aplicados para desenhar os Alpes. Meu traço irregular e
eriçado não representava satisfatoriamente o pico das montanhas, nem
minhas linhas regulares de sombra, o lago de Genebra. As pinturas com
aquarela foram abandonadas com um obscuro e vago sentimento de que não
tinha talento para este tipo de trabalho, --- e, na verdade, não tinha
nenhum para combinação de cores, --- e retomei os esboços a lápis com
resoluta energia.

%242. 
Até aquele momento eu não tinha visto qualquer desenho de Turner, e
não sei dizer se se devia à estupidez ou à prudência a tranquilidade com
a qual copiava as reproduções das vinhetas de Rogers,\footnote{Conferir
  capítulo \textsc{iv}, p.\,\pageref{87}.} sem me preocupar em saber onde se
encontravam os originais. O que fato é que eles repousavam no fundo de
uma velha gaveta em Queen Anne Street, tão inacessível a mim quanto o
fundo do mar, --- e que, se as tivesse visto, elas certamente teriam
destruído o prazer que sentia com as reproduções, --- minha
despreocupação quanto aos originais era, sob certo aspecto, sinal de boa
sorte: e quanto mais considerava essa e outras faltas daquilo que a
maioria das pessoas chamariam de louvável curiosidade, mais estava
disposto a ver com gratidão, e mesmo com respeito, o hábito que
permaneceu comigo durante a vida, de sempre trabalhar resignadamente
naquilo que estivesse em minhas mãos até concluir a tarefa, e de olhar
exclusivamente para o objeto diante de meus olhos até que
verdadeiramente o visse.

Por outro lado, os Turners acadêmicos estavam muito além de qualquer
possibilidade de imitação para que pudessem me perturbar, e as
impressões que produziram antes de 1836 eram confusas; muitos deles,
como o Quilleboeuf, ou os ``Keelmen heaving in coals'',\footnote{Refere"-se
  ao quadro de Turner: \textit{Keelmen heaving in coals by Moonlight}.
  \textit{Estivadores carregados de carvão ao luar}.}
tinhas cores não muito interessantes; e a \textit{Fonte da Indolência}, ou
o \textit{Ramo Dourado}, talvez já me parecessem fantásticos, comparados
ao naturalismo de Landseer\footnote{Certamente Charles Landseer
  (1799--1879): pintor, desenhista e gravador inglês; veio ao Brasil em
  1825, acompanhando a missão diplomática que negociou os termos para o
  reconhecimento da independência do país pela Grã"-Bretanha. Durante o
  pouco mais de um ano que permaneceu no Brasil, realizou cerca de 300
  desenhos e aquarelas de cenas do Rio de Janeiro, São Paulo, Bahia e
  Pernambuco.} e ao interesse humano e ao acabamento
inteligível de Wilkie.\footnote{Certamente Sir David Wilkie (1785--1841):
  pintor escocês.}

%243. 
Porém, em 1836, Turner exibiu três pinturas, nas quais as
características que posteriormente marcariam sua obra se encontravam
desenvolvidas no mais alto grau: \textit{Juliet e sua ama}, \textit{Roma
vista do Monte Aventino} e \textit{Mercúrio e Argus}. A excentricidade de
colocar Juliet em Veneza em vez de em Verona, e os misteriosos lampiões
e rojões através dos quais mal se distinguia Veneza, provocaram um
artigo na \textit{Blackwood's Magazine} repleto de ribaldaria, que
expressava com certa intensidade e extrema descortesia os sentimentos
dos discípulos de Sir George Beaumont\footnote{Sir George Beaumont
  (1753--1827) foi um nobre inglês, patrono das artes e pintor amador. Realizou
  a primeira doação de pinturas recebida pela London National Gallery.} diante dessas estranhas visões da natureza.

O artigo conduziu"-me ao píncaro da ``ira negra'', onde permaneci
praticamente até hoje; e, tendo àquela época certa confiança em meu
poder com as palavras, e --- não apenas capacidade de julgamento, mas
experiência \textit{sincera} --- do encanto que emanava do trabalho de
Turner, escrevi uma resposta para a \textit{Blackwood}, da qual gostaria
agora de encontrar algum fragmento. Mas meu pai considerou que seria
necessário solicitar a permissão de Turner para sua publicação; copiei"-a
com minha mais bela caligrafia e enviei"-a a Queen Anne Street; o velho
homem respondeu"-me gentilmente, como se segue: %---

\medskip

\begin{quote}
\begin{flushright}
\smallskip\hfill\textsc{6 de outubro de 1836}\\
\textit{Queen Ann (\textit{sic}) Street West, 47}
\end{flushright}
\bigskip
Meu caro senhor,

Rogo que aceite meu agradecimento pelo seu zelo,
bondade, e pelo trabalho que teve em escrever em minha defesa, a
respeito da crítica da \textit{Blackwood's Magazine} de outubro em relação
a meus trabalhos; porém, essas questões jamais me tocam; elas são de
menor importância, brincadeira de mau"-gosto, como a praga que minha
governanta receia que invada a barrica de cereais.

\textsc{p.\,s.} Se o senhor deseja que seu manuscrito seja devolvido, tenha a
bondade de me informar. Caso contrário, com sua permissão, o enviarei ao
proprietário do quadro sobre Juliet.
\end{quote}

\medskip

Não posso reproduzir a assinatura da carta, pois foi cortada por algum
amigo! Anos mais tarde, costumava ser, para meu pai, ``Yours most
truly'',\footnote{Aproximadamente ``muito verdadeiramente seu amigo''.} e para mim, ``Yours truly''.\footnote{Aproximadamente
  ``verdadeiramente seu amigo''. Ambas as saudações refletem,
  sutilmente, o grau de apreço que Turner tinha por Ruskin e por seu pai.}

O proprietário do quadro era Mr.\,Munro of Novar.\footnote{Hugh Andrew
  Johnstone Munro of Novar (?--1865) foi um proprietário rural, colecionador de
  arte e pintor amador inglês.} que nunca me dirigiu
qualquer comentário sobre o primeiro capítulo de \textit{Pintores Modernos},
que acabou lhe caindo nas mãos. Nem eu tive a preocupação de saber sua
opinião; e, ainda durante um ou dois anos, perseverei no estudo das
gravuras de Turner, e utilizei o método de Copley Fielding para criar os
efeitos de cor que produzi nas viagens de férias durante os dias de
Oxford.

%244. 
Fizemos três excursões naqueles verões, sem cruzar o Canal da
Mancha. Em 1837, fomos a Yorkshire e aos Lagos; em 1838, à Escócia; em
1839, a Cornwall.

Na viagem de 1837, quando eu tinha dezoito anos, senti, pela última vez,
o puro amor infantil pela natureza, que Wordsworth tão idilicamente
considera um vislumbre da imortalidade. Seguimos pela North Road, como
habitualmente; e, no quarto dia, chegamos a Catterick Bridge, onde corre
um riacho de águas claras sobre um leito de seixos e, tanto a oeste como
a leste, erguem"-se colinas que anunciam as charnecas e os vales
estreitos das terras altas de Yorkshire; e ali voltei a ter aquele
sentimento --- que nunca mais recuperaria.

É um sentimento apenas possível na juventude, pois qualquer preocupação,
pesar ou conhecimento do mal, o destrói; e requer também a completa
sensibilidade dos nervos e do sangue, a força consciente do coração, e
esperança; não que eu creia que a pureza da juventude não possa atingir
seu apogeu diante da doença e da morte, mas que isso acontece apenas ao
se considerar a própria morte como enviada por Deus.

%245. 
Em mim, este sentimento esteve sempre exclusivamente confinado ao
\textit{selvagem}, quer dizer, a lugares completamente naturais, e
especialmente a cenários animados por cursos d'água, ou pelo mar. A
percepção do livre, espontâneo e impoluto poder da natureza, era
essencial nele. Deleitava"-me com um gramado, um jardim, um campo de
margaridas, como qualquer criança; mas, diante das fontes de Wandel, ou
das dunas de Sandgate, ou dos rios que correm por ravinas em Yorkshire,
eu era diferente das outras crianças: mas este sentimento não pode ser
descrito por quem o tenha experimentado. As palavras de Wordsworth,
``obsedou"-me como uma paixão'',\footnote{Em inglês, \textit{haunted me like
  a passion}.} não o descreve, pois o sentimento não
é \textit{como} alguma coisa, mas simplesmente \textit{é}, uma paixão; a
questão é esclarecer como ele \textit{difere} de outras paixões, --- que
tipo de sentimento humano, eminentemente humano, é este que permite se
amar uma pedra pelo amor à pedra, e uma nuvem pelo amor à nuvem. Um
macaco ama um macaco pelo amor dos macacos, e uma noz pela amêndoa que
contém, mas não uma pedra pela pedra. Posso considerar a pedra como pão,
mas certamente não sob incitação do demônio.

Deve ser mais uma vez dito que eu era diferente das outras crianças,
mesmo daquelas semelhantes a mim, não tanto na verdadeira natureza do
sentimento, mas em sua composição. Meu pequeno vaso de argila, minha
redoma, estava repleto da reverência religiosa de Wordsworth, da
sensibilidade de Shelley e da precisão de Turner, fundidas num todo. Uma
gota de neve era para mim, como para Wordsworth, parte do Sermão da
Montanha; mas jamais poderia escrever sonetos a quelidônia, devido a seu
amarelo grosseiro e a sua forma imperfeita. Como Shelley, eu amava os
céus azuis e os olhos azuis, mas nem por um instante os confundia com
minha pobre e pequena Psychidion.\footnote{Em grego, \textit{Pequena alma}.} E a reverência pela natureza e a paixão eram
igualmente mantidas em seus lugares pelo elemento construtivo
turneriano. Não pretendia que uma margarida pudesse ver a beleza de sua
própria sombra, mas me empenhava em tentar desenhá"-la corretamente, eu
mesmo.

%246. 
Mas as leis prescritas eram tão inflexíveis e quimicamente
inalteráveis, que agora, olhando retrospectivamente de 1886 à margem
daquele riacho, em 1886, sou capaz de ver toda minha juventude, e
percebo que nada em mim mudou. Uma parte de mim morreu, mas
outras se tornaram mais fortes. Aprendi umas poucas coisas, esqueci
muitas outras; no cômputo geral, sou o mesmo jovem, decepcionado e
reumático.

E como ilustração dessa inflexibilidade, que não ocorre pelo
enrijecimento da madeira com a idade, mas à custa da própria estrutura
do cerne, permitam"-me deter"-me mais um ou dois minutos na singular
alegria que senti em 1837, ao retornar dos lugares que frequentava na
infância. Nenhuma criança poderia ter ficado mais comovida do que eu ao
ver a Itália e os Alpes; nenhuma criança ou adulto soube jamais mais bem
distinguir a diferença entre uma cabana de Cumberland e um palácio
veneziano, ou entre um arroio de Cumberland e o Ródano: --- minha
verdadeira percepção dessa diferença seria demonstrada no ano seguinte,
quando produzi a primeira peça literária promissora; mas, depois de
todas as intensas emoções e selvagens alegrias sentidas no continente,
retornar às margens de um riacho em Yorkshire foi como retornar ao
Paraíso. Continuamos a percorrer o bem conhecido condado de Cumberland;
meu pai levou"-me até Scawfell e Helvellyn, em companhia de um hábil guia
de Keswick, que conhecia mineralogia, Mr.\,Wright; e o verão transcorreu
benéfico e pacífico.

%247. 
Um pequeno incidente que aconteceu, suponho, no começo de 1838,
mostra que eu tinha, então, recuperado alguma tranquilidade e bom senso,
e que poderia, àquela época, ter facilmente adotado uma vida simples e
saudável, se meus pais tivessem percebido a oportunidade.

Esqueci"-me de dizer, ao falar de Mr. e Mrs.\,Richard Gray, que, quando eu
era criança, minha mãe tinha outros amigos religiosos, que viviam no
alto de Camberwell Grove, ou entre o topo da colina e White Gate, --- Mrs.\,Withers; uma pessoa extremamente amável e caridosa, que, creio a
ajudasse na distribuição de donativos, quando suas tarefas domésticas a
impediam de fazê"-lo. Mr.\,Withers era um comerciante de carvão, não
exatamente bem"-sucedido. Dele, recordo"-me apenas de um rosto avermelhado
e geralmente vazio de expressão; de Mrs.\,Withers, nenhum aspecto físico,
apenas vagamente os fatos acima narrados; e que era uma presença
familiar no universo de minha mãe, dele se afastando, contudo, sem que
percebêssemos ou sentíssemos sua falta, antes que eu fosse
suficientemente adulto para formar uma ideia precisa dela.

Nesta primavera de 1838, porém, Mr.\,Withers, já viúvo, tendo se retirado
para os distritos rurais em condições materiais modestas, veio à cidade
para tratar de questões que remanesciam de seu negócio carvoeiro, e
trouxe com ele sua única filha para mostrá"-la à minha mãe; --- que,
surpreendentemente, convidou"-a para ficar conosco enquanto seu pai
visitava sua antiga clientela nas docas. Charlotte Withers era uma
frágil, formosa, sardenta e sensível garota de uns dezesseis anos; uma
espécie de flor selvagem pequena e inacabada, extremamente inteligente,
afetuosa, muito sensata e moderadamente devota. Uma criatura comum,
delicada e terna, não particularmente bonita, mas muito agradável de
ver, especialmente se seus olhos estivessem olhando para você e sua
mente os acompanhasse.

%248. 
No curso de uma semana, passamos a gostar um do outro, tendo
nascido entre nós uma confiança moderada. Discutíamos sobre a primazia
da música ou da pintura; e escrevi um ensaio com nove páginas de papel
almaço, no qual afirmava solidamente minhas opiniões e arrasava
totalmente as dela, segundo minha maneira habitual de fazer a corte a
minha amada. Charlotte Withers, contudo, considerou o ensaio uma grande
honra e o guardou como se se tratasse de um prêmio escolar.

E, como disse, se meus pais tivessem decidido ficar com ela por mais um
mês, nós teríamos, suave e melodiosamente, nos apaixonado; e eles teriam
me proporcionado uma excelente e agradável pequena esposa, e teria me
estabelecido, dado meu interesse pela geologia, no negócio de carvão,
sem qualquer resistência ou futuros problemas de minha parte. Mas não
creio que a ideia tenha sequer lhes ocorrido; Charlotte não era o tipo
de pessoa que eles imaginavam para mim. Assim Charlotte foi embora ao
fim de uma semana, quando seu pai terminou seus negócios. Caminhei com
ela até Camberwell Green e lhe disse adeus, um tanto pesarosamente, na
esquina de New Road; e a possibilidade daquela felicidade singela
desapareceu para sempre. Um pouco depois, seu pai \textit{negociou} um
casamento para ela com um rico comerciante de Newcastle, que foi aceito
por obrigação. Ela a tratava como um de seus sacos de carvão e, em um ou
dois anos, ela morreu.

%249. 
De um modo muito indireto, e a contragosto, o episódio mostrou"-me
aquilo que minha mãe havia uma ou duas vezes me dito, para minha grande
indignação, que Adèle não era a única garota no mundo; e minha alegria
durante a excursão pelas Trossachs não foi mais descrita em heroicos
versos byronianos; a tragédia também foi abandonada, porque, ao
descrever uma gôndola, um bandido, a heroína Bianca, e o luar sobre o
\textit{Grand Canal}, percebi que não tinha muito mais a dizer.

O país de Walter Scott afastou"-me definitivamente de tudo isso. Não há
muito interesse para o leitor dizer agora que, àquela época, a praia do
Loch Katrine, na extremidade leste do lago, permanecia exatamente como
Scott a tinha visto, e descrito,

\begin{verse}
Adiante, entre o matagal surge,\\
Um fio d'água, tranquilo e profundo.\footnotemark
\end{verse}  
\footnotetext{No original: \textit{Onward, amid the copse gang peep,\,/\,A narrow inlet, still and deep}.}

\noindent{}A literal e adorável verdade era essa: --- ao lado da vereda (não passava
disso) que cortava as Trossachs, tranquilo e profundo sob os bosques de
mirtilo, um sinuoso curso d'água marrom"-claro, à primeira vista com não
mais que cinco pés de largura, refletia o emaranhado de musgos em suas
margens, e o arco de galhos que o sobrejazia, deixando vislumbrar
pedaçinhos do céu.

O pequeno meandro do Loch Katrine era algo extremamente raro; nunca
tinha visto margens de lagos como essas. Um sinuoso recanto de águas
profundas, sem qualquer fonte que o alimentasse --- fenômeno só possível,
creio, na desordem incomum das rochas das Trossachs; além disso, a
natural suavidade e beleza do lago, tornada sagrada pelo mais belo poema
que a Escócia jamais cantou para suas margens de lagos. E tudo que o
século dezenove soube fazer para preservar esse canto de montanha foi
meter"-lhe o nariz de um barco a vapor, cobrir seus mirtilos com uma
plataforma de pranchas de madeiras e conduzir o populacho através dela
da maneira mais apressada e tumultuada possível.

Teria sido perfeito escalar o Ben Venue e o Ben Ledi, martelo na mão,
como fiz em Scawfell e Helvellyn. Mas, ao invés disso, realizei alguns
trabalhos literários, cuja inspiração proveio, sobretudo, da visão da
capela de Roslyn e da abadia de Melrose.

%250. 
A ideia me ocorreu no verão de 1837, e, creio, tenha surgido
imediatamente após minha percepção do contraste entre as cabanas de
Westmoreland e aquelas da Itália. De qualquer modo, o número de novembro
da \textit{Architectural Magazine} de Londres, de 1837, iniciava"-se com
uma ``Introdução à Poesia da Arquitetura das Nações da Europa,
considerada em sua Associação com o Cenário Natural e o Caráter
Nacional'', por Kataphusin.\footnote{Pseudônimo de Ruskin.}
Eu seria incapaz de expressar em menos e mais significativas palavras a
definição daquilo que viria a passar metade de minha vida discutindo; o
pseudônimo que escolhi, ``Segundo a Natureza'', era igualmente
expressivo da disposição de espírito com a qual discutiria este e
qualquer outro tema. A própria adoção de um pseudônimo implicava (assim
como o fato de não colocar meu nome na primeira publicação de
\textit{Pintores Modernos}) a percepção de um poder de julgamento,
existente em mim mesmo, que não se poderia esperar de um jovem de
dezoito anos. Se meu pai ou meu tutor tivesse me dito, ``escreva como um
jovem escreve, --- deixe o leitor descobri o que você sabe e o convença
pela argumentação'', talvez eu não tivesse ficado envergonhado de meus
ensaios juvenis. Se eles tivessem me dito, de um modo mais severo,
``segure sua língua até não precisar que o leitor condescenda em
ouvi"-lo'', talvez eu tivesse ficado satisfeito com meu trabalho quando
ele estivesse maduro.

Assim como são, esses ensaios juvenis, embora deformados pela pretensão,
e superficiais em conteúdo, vão direto ao ponto; e distinguem"-se da
maioria da literatura de sua época pela habilidade no uso da linguagem,
percebida imediatamente pelo público como um dom que me foi concedido.

%251. 
Já disse que se não fosse pela leitura constante da Bíblia, teria
tomado Johnson como meu único modelo em inglês. Até certo ponto ele me
serviu de modelo, e isso me foi útil nesses primeiros ensaios; em parte
porque não o pude evitar; em parte porque fora esta minha intenção.

Em nossas viagens ao exterior, sendo naturalmente desejável levar a
menor quantidade possível de bagagem, meu pai considerou que quatro
pequenos volumes de Johnson --- o \textit{Idler} e o \textit{Rambler} ---
deveriam conter, com nomes apropriados às circunstâncias,\footnote{\textit{Idler}
  pode ser entendido como ocioso, preguiçoso; \textit{Rambler} como vagabundo, vadio.} mais alimento intelectual concentrados num reduzido volume do
que os de qualquer outro autor. E assim, nas horas vagas, e nos dias
chuvosos, os repetidos torneios de frases de \textit{Rambler} e
\textit{Idler} fixaram"-se em meus ouvidos e em minha mente; não foi
possível para mim, até muito tempo depois, abandonar a simetria e o
equilíbrio das frases de Johnson, que se destinavam, como um golpe do
esgrimista ou do calceteiro, a perfurar o elmo do inimigo, ou a demolir
o pilar de carvalho de um princípio. Nunca, nem por um instante,
comparei Johnson a Scott, a Pope, a Byron, ou a qualquer um daqueles
realmente grandes escritores os quais amava. Embora imediatamente, e
para sempre, tenha reconhecido nele um homem completamente sincero,
infalivelmente sábio na opinião e no julgamento das questões correntes,
dos negócios e das coisas mundanas. Apreciei suas frases não porque
fossem simétricas, mas porque eram precisas, claras; esse é um método de
julgamento raramente utilizado pelo leitor médio, que sempre esperam de
um autor, em primeiro lugar, argumentos em favor de suas próprias
opiniões, em termos elegantes; e estão sempre dispostos a aplaudir uma
frase de Macaulay,\footnote{Thomas Babington Macaulay (180--1859) foi poeta,
  historiador e político britânico.} que tem menos
sentido do que um borrão comprimido num papel dobrado, e a rejeitar uma
frase de Johnson, o que reforça seus próprios preconceitos, --- embora
sua simetria seja como um trovão que ecoasse em dois horizontes.

%252. 
Credito como felicidade o fato de que, durante nossas viagens
continentais, a vívida excitação que prevalecia durante a maior parte do
dia me deixasse algumas horas livres para estudar um livro que levava a
reflexão, e que Johnson fosse o único autor disponível. Nenhum outro
autor teria me protegido, como ele fez, das possibilidades de me deixar
levar pelo meu temperamento, ao mesmo tempo sanguíneo e metafísico. Ele
me ensinou a mensurar cuidadosamente a vida, e a desacreditar do acaso;
e me protegeu para sempre, com seu pétreo bom"-senso, de ser apanhado nas
teias da metafísica alemã, ou de me atolar no pântano da versão inglesa
dessa filosofia.

Abro, neste momento, o maior dos volumes do \textit{Idler}, ao qual muito
devo. Após folhear algumas páginas, deparo"-me com o parágrafo final do
número 65, o qual aqui transcrevo para mostrar ao leitor, em suma, o que ele
me ensinou, --- em palavras tais que, ao escrever esta autobiografia,
terminantemente as obedeço: %---

\begin{quote}
Desses sábios, que aqueles que lhes aspirem a mesma glória,
imitem"-lhes a diligência e evitem"-lhes a escrupulosidade. Que sempre se
lembre de que a vida é curta, que o conhecimento é limitado, que muitas
dúvidas não merecem ser esclarecidas. Que aqueles cuja natureza e estudo
os qualificaram para ensinar à humanidade, diga"-nos o que aprenderam
enquanto estão aptos a fazê"-los, e creditem sua reputação apenas a si
mesmos.
\end{quote}

É"-me impossível agora saber quanto do meu honesto desejo de verdade e da
minha compaixão, que me coloca imediatamente à disposição das criaturas
que a cada instante correm perigo, fizeram com que, no momento adequado,
eu pensasse e julgasse como Johnson pensava e julgava, --- mesmo se nada
tivesse aprendido com ele. Ele, pelo menos, me colocou no caminho certo
desde o início e, não importa quanto tempo eu tenha gasto com prazeres
vãos, ou débeis esforços, ele me salvou, para sempre, de pensamentos
falsos e especulações fúteis.

%253. 
Não sei por que, --- pois certamente Mr.\,Loudon não estava cansado de
mim --- os artigos de Kataphusin cessaram abruptamente de aparecer, como
se nada mais tivessem a dizer sobre as mais elevadas formas da
arquitetura civil e religiosa, sem qualquer alusão ao seu fim, ou
justificativa pela falta de alusão. Na verdade, encontrei, casualmente,
uma indicação de algum propósito ulterior numa pomposa frase de um
artigo sobre as cabanas de Westmoreland, a qual anuncia que ``será visto
posteriormente, quando deixarmos o modesto vale para entrarmos na
abrupta ravina, e a colina relvada para o precipício estriado de rochas;
se os arquitetos continentais não conseguem adornar as pastagens com um
humilde teto, eles podem coroar as cristas das montanhas com ameias
inexpugnáveis.'' Mas essa magnífica promessa não leva a nada de mais
extraordinário do que a um ``capítulo sobre chaminés'', ilustrado, como,
para minha grande surpresa esta manhã percebi, por um desenho muito bom
do edifício que agora é o aspecto dominante da vista que tenho da janela
de meu estúdio, --- Coniston Hall.

Em seu conjunto, entretanto, esses papéis, escritos intermitentemente
durante o ano de 1838, mostram um progresso constante e uma gama de
pensamentos corretamente consolidados sobre esses temas, na crisálida em
torpor dentro de mim.

%254. 
Das Trossachs nos dirigimos a Edimburgo: e, em algum lugar da
estrada, perto de Linlithgow, meu pai, ao ler algumas cartas que
chegaram pelo correio daquele dia, informou"-nos calmamente que Mr.\,Domecq
iria novamente trazer suas quatro filhas para a Inglaterra, a fim de que
concluíssem seus estudos em New Hall, perto de Chelmsford.

E não me lembro de nada mais daquela viagem, ou do período que a seguiu,
até que me encontrasse dentro de um carro em direção a Chelmsford.
Naturalmente, não havia qualquer motivo para que minha mãe me
levasse com ela numa visita a um convento; mas suponho que tenha
considerado demasiadamente cruel me deixar para trás. As jovens foram
autorizadas a conversar conosco no parlatório e convidadas (com
aceitação) a passar suas férias sempre em Herne Hill. E assim começou
uma segunda época daquele período de minha vida, que \textit{não} é
``digno de memória'', mas apenas de \textit{guarda e passa}.\footnote{Célebre
  verso da \textit{Divina comédia}, de Dante Alighieri.}

Houve algum consolo durante meus estudos, no outono, ao pensar que ela
se encontrava na Inglaterra, na verdade bem \textit{ali}, --- eu podia ver
o céu sobre Chelmsford da janela de meu estúdio, --- e ao saber que ela
estava enclausurada num convento e não poderia ser vista por ninguém, e
que visitaria Herne Hill outra vez, e teria que conviver um pouco
comigo.

%255. 
Pergunto"-me insistentemente em que tipo de pessoa eu teria me
transformado, caso àquela época o Amor estivesse comigo ao invés de
contra mim; e, se ao invés da dor inútil e perturbadora, eu tivesse tido
a alegria do amor aceito, e a inefável, incalculável força que provém de
sua harmonia e de seu louvor.

Parece"-me que tais coisas não são permitidas neste mundo. Os homens
capazes das mais elevadas paixões imaginativas são sempre sacudidos por
ondas ígneas: os homens que nas paixões encontram águas tranquilas, e
nelas não se escaldam, são de outro tipo. O segundo empregado de meu
pai, Mr.\,Ritchie, escreveu insensivelmente a seu colega, o solteiro
Henry, que decidira não se casar para cuidar de sua mãe e de suas irmãs,
''se você quer saber o que é a felicidade, tenha uma esposa e meia dúzia
de filhos, e venha para Margate''. Mas Mr.\,Ritchie permaneceu toda sua
vida um imponente cavalheiro com olhos como groselhas, de convicção
Irvingite.\footnote{Refere"-se à seita presbiteriana cujo nome deriva do
  pastor e teólogo escocês Edward Irving (1792--1834).}

Deve haver grande felicidade nos casamentos do típico nobre rural
inglês. Embora os nobres rurais ingleses façam de suas vidas felizes
apenas pastos para raposas.\footnote{Provável citação bíblica: Salmo 63,
  versículo 10.}

%256. 
Naturalmente, quando Adèle e suas irmãs voltaram no Natal, e
ficaram conosco durante quatro ou cinco semanas, cada sentimento e cada
sandice, os quais eu havia subjugado ou esquecido, retornou com força
redobrada. Não sei o que poderia ter acontecido se Adèle tivesse sido
uma garota perfeitamente bela e amigável, e demonstrado um mínimo de
afeto por mim. Suponho que minha mãe teria capitulado. Entretanto,
embora extremamente adorável aos quinze anos, Adèle não era mais bonita
do que geralmente o são as moças francesas aos dezoito anos; ela era
firme, ardente, e de sólidos princípios; mas, como os traços de seu
caráter já esboçados demonstram, não exatamente uma pessoa simpática; e
embora pudesse ter se casado comigo, caso seu pai assim o tivesse
desejado, mostrou"-se sempre feliz em me ver fora de seu caminho. Meu
amor era demasiadamente elevado e fantástico para ser diminuído por sua
menor beleza; mas eu estava perfeitamente ciente de seus defeitos e os
admitia, e, em nenhum momento apresentei o menor grau de cegueira por
amor, como percebo que acontece com outros homens, nem tive meu senso
crítico diminuído. E os dias se seguiram aos dias, e os meses se
seguiram aos meses, de complexo absurdo, dor, erro, afeto desperdiçado,
a despeito de minha semi"-virtude, os quais me agradam varrer do caminho
para que possa recordar as melhores coisas daquela época, e colocá"-las
no menor montículo possível de poeira, desejando que o Senhor Lixeiro do
Oblívio faça deles boa limpeza.

Com essa observação genérica, quanto à conduta dos filhos em relação a
seus pais, constato que a grande quantidade de obediência externa e
contrafeita demonstrada por eles, não é virtualmente obediência, por não
ser alegre nem total. O desejo de obedecer já é desobediência; e,
embora nessa época eu realmente fizesse muitas coisas das quais não
gostava, para agradar meus pais, não tenho agora um sentimento de
autoaprovação ou consolação por tê"-las feito, tanto que a rabugice e a
mutilação poluem esse magro sacrifício.

%257. 
Mas, antes que abandone, por esta vez, o tema do romance,
permitam"-me escrever o epitáfio de uma de suas doces sombras, o que fará
com que aqueles que a tenham conhecido fiquem felizes por eu a ter
escrito. O térreo, sob o escritório de meu pai na Billiter Street, como
já mencionei, era ocupado pela Messrs.\,Wardell \& Co. O gerente da
empresa era um cavalheiro idoso, extremamente inteligente e refinado,
moreno, cabelos escuros, salientes, encaracolados e vivazes, e olhos
brilhantes; de boa convivência e amável, no mais alto grau, bem educado,
não demasiadamente erudito, sempre satisfeito consigo mesmo, feliz com
sua sensata esposa, e sua única filha, muito bonita, absolutamente
adorável e generosa. Não demasiado erudito, repito, mas um excelente
homem de negócios; mais velho e, suponho, já consideravelmente mais rico
do que meu pai. Ele tinha uma bela casa em Hampstead, e não poupava
esforços para dar uma boa educação a sua filha.

Deve ter sido em algum momento do ano de 1839, ou do ano anterior, que,
tendo meu pai confiado a Mr.\,Wardell sua preocupação com o lamentável
estado de ânimo em mim provocado por Adèle, Mr.\,Wardell sugeriu"-lhe uma
visita a Hampstead, como tentativa de distração de meus pensamentos
obsessivos. A preferência de meu pai continuava sendo Lady Clara Vere de
Vere; mas Miss Wardell tinha todas as qualidades que uma garota deveria
ter, e era uma herdeira, --- de uma fortuna provavelmente maior do que
aquela que seria minha. E os dois pais concordaram que nada poderia ser
mais conveniente, racional e desejável do que um acordo nesse sentido.
Assim, fui enviado para passar uma tarde de verão, e para jantar em
Hampstead.

%258. 
Aquela seria uma tarde extremamente agradável para qualquer jovem
que não fosse um simplório. Miss Wardell ouvira muitas vezes meu pai me
descrever como um jovem bem comportado, já dotado de certa reputação
literária --- autor de \textit{A poesia da arquitetura} --- ganhador do
Newdigate,\footnote{Refere"-se ao Newdigate Prize, o Prêmio Newdigate,
  conferido anualmente ao estudante de graduação na Universidade de
  Oxford, admitido nos últimos quatro anos, que escrevesse a melhor
  composição em versos. Instituído por Sir Roger Newdigate (1719--1806).} --- expectativa de um homem de primeira classe. Ela
própria fora educada de uma maneira muito semelhante à minha, mantida em
severo isolamento pelos seus devotados pais, numa casa nos subúrbios de
Londres cercada de jardins, nos quais podia brincar, pular, e colher
flores. A principal diferença entre nós era que, desde o início, Miss
Wardell tivera excelentes mestres, e, aos dezessete anos, tornara"-se uma
garota muito bem"-sucedida, inteligente e irrepreensível; frágil e
delicada a um grau que acrescentava à sua beleza a gravidade que provém
do temor, embora em perfeita saúde, na medida em que uma garota em
rápido crescimento o poderia ser; uma morena esbelta e de formas suaves,
na qual os cabelos encaracolados do pai haviam se transformados em
graciosos cachos que adornavam uma bela, modesta e pensativa face de
olhos acinzentados. Daquela tarde em Hampstead, lembro"-me apenas que o
tempo estava bom, e que caminhamos pelo jardim; mamãe, por mero dever de
cortesia numa primeira visita, nos supervisionava, --- teria sido mais
sensato deixar que nos entendêssemos como pudéssemos. Eu admirava
sincera e reverentemente a bela criatura e teria feito, ou dito, de bom
grado, tudo que pudesse lhe dar prazer. Literalmente para lhe dar
\textit{prazer}, pois, essa era, na verdade, minha disposição em relação a
todas as garotas, a despeito dos mencionados equívocos cometidos em
minha apresentação. Minha primeira preocupação era como servi"-las
e fazê"-las feliz, e se elas quisessem me usar como prancha para cruzar
um rio, ou me fazer de poste para amarrar um balanço, ou qualquer coisa
semelhante na qual eu não precisasse falar, ficaria sempre feliz com tal
promoção. Essa sincera devoção às garotas, acompanhada pelo intenso
deleite pela graça ou beleza que pudessem apresentar, e, na maioria dos
casos, pela simpatia percebida, acentuada pela crença na correção de
seus sentimentos, geralmente concedia"-me considerável prestígio com as
moças: mas tudo isso fazia com que apenas raramente me sentisse à
vontade com elas, --- e, não duvido que durante toda àquela tarde em
Hampstead tenha dado pouco prazer à minha companhia. De resto, embora
admirasse muito a Miss Wardell, ela não correspondia ao meu tipo de
beleza. Gosto de rostos ovais, cabelos cristalinamente louros, retos, no
máximo, ondulados (ou presos numa trança longa), e um andar elástico e
um passo firme. A graça delicada e de morena de Miss Wardell não
produzia efeitos sobre mim, exceto o temor de me cansar terrivelmente
dela. No cômputo geral, suponho que tenha me saído muito bem, pois,
posteriormente, ela aceitou passar uma temporada em Herne Hill para ver
nossos quadros e outras coisas; e lembro"-me de sua expressão ao mesmo
tempo assustada e prazerosa, quando me ajoelhei ao seu lado para lhe
mostrar um livro, ou algo do gênero.

%259. 
Após esse segundo encontro, contudo, meu pai e minha mãe
perguntaram"-me seriamente o que pensava dela, e expliquei"-lhes que
embora reconhecesse sua beleza, seus méritos e sua amabilidade, ela não
era meu tipo de garota, --- as negociações foram interrompidas, naquele
ponto, e, pouco tempo depois, abandonadas para sempre; enquanto em
Hampstead eles continuaram a ensinar, à delicada criatura, alto alemão,
o francês de Paris, \textit{A metafísica} de Kant e \textit{Os princípios} de Newton; e então a levaram a Paris, e a esgotaram
fazendo com que visitasse, durante muitos dias, tudo que merecessem ser
visto; isso, além da ofuscação e da excitação da primeira temporada fora
de Hamspstead, tornaram"-na muito pálida e débil, e eles a trouxeram de
volta para algum lugar à beira"-mar na Inglaterra, cujo nome esqueci:
onde ela foi acometida de uma febre nervosa e faleceu, a luz da morte
bruxuleando cada vez mais fraca em seus olhos ternos, e ela nunca mais
brincará nos jardins de Hampstead.

Como seus pais, sobretudo o pai, continuaram a viver, nunca pude
entender; mas creio que fossem honestamente religiosos sem que o
declarassem, e não havia nada de que pudessem culpar a si mesmos, exceto
não conhecer melhor sua filha. O pai, embora com profundas rugas que
alteraram para sempre seu rosto, continuou a conduzir seus negócios e
viveu até a velhice.

%260. 
Não tenho certeza das datas das mortes de Miss Withers e de Miss
Wardell; a morte de Sybilla Dowie (relatada em \textit{Fors}), mais
tristes que essas duas, foi muito posterior; mas a perda deste espírito
gracioso, que se seguiu a de seu amado, foi por nós sentida muito tempo
antes do momento em que escrevo. Pessoalmente, nunca vi a Morte, nem
participei da dor e da aflição de um quarto de moribundo; nem vi, algo
ainda muito menos concebível, a miséria da pobreza desassistida. Mas fui
levado a pensar sobre ela; e nos falecimentos de criaturas que tinha
visto felizes, o sentimento de profundo pesar, não por mim, mas por
elas, começou a se fundir com todos aqueles pensamentos que se
originaram das tragédias de Homero, Ésquilo e Shakespeare, e a modificar
a fé infantil ainda não posta à prova. O azul das montanhas tornou"-se
mais profundo com a púrpura do luto, --- as nuvens que se reuniam em
torno do sol poente, não esmaecidas, mas reunidas em reverência como as
harmonias de uma Miserere,\footnote{Refere"-se, provavelmente, à obra do
  compositor e religioso italiano, Gregorio Allegri (1582--1652),
  \textit{Miserere, mei, Deus} --- em português,\textit{Tende misericódia de mim, Deus}), baseada nos Salmos 50 e 51.} --- e toda a força e
estrutura de minha mente, lúrica, como as abóbadas de Roslyn, enquanto
um fogo misterioso cintilava em seus pilares, cobertos pela vegetação, e
longe, nas profundezas do crepúsculo, ``resplandecia a beleza de cada
contraforte cinzelado com rosas''.
