\chapter{Introdução}

\begin{flushright}
\textsc{marcos a.\,p. ribeiro}
\end{flushright}\medskip

\noindent{}John Ruskin (1819--1900), cuja vida praticamente coincide com a da Rainha
Vitória (1819--1901), foi, provavelmente, o mais importante intelectual da
era vitoriana. Estudou em Oxford, onde posteriormente foi professor.
Escritor, crítico de arte e de arquitetura, crítico social, desenhista e
pintor. Defensor e patrono da Irmandade Pré"-Rafaelita, suas ideias
inspiraram o movimento \emph{Arts \& Crafts} e os movimentos em defesa
dos direitos dos trabalhadores.

Nascido numa família escocesa, seu pai era comerciante de Sherry
(Jerez), atividade que, embora eminentemente comercial, conferia"-lhe
certo prestígio social na Inglaterra oitocentista. A profissão paterna
obrigava"-o a extensas viagens pela Europa, nas quais o jovem John era
levado. Sua educação foi particularmente severa, rigorosa. Não
frequentou regularmente escolas até os quinze, mas, desde cedo, lia os
clássicos ingleses (Shakespeare, Alexander Pope, Byron, Walter Scott) e
a Bíblia. Não lhe era permitido possuir brinquedos sofisticados,
industrializados, mas apenas objetos simples e artesanais, como
carrinhos de madeira e blocos de montar. A família considerava que essa
era uma forma de desenvolver sua percepção do mundo e sensibilidade,
obrigando"-o a observar detalhadamente o que existia à sua volta: os
materiais com os quais as casas eram construídas, as características
físicas, pictóricas e tácteis dos móveis e da decoração, como papéis de
parede, tapetes etc.

O espectro de seus interesses era amplo: literatura, a Bíblia, artes
plásticas, arquitetura, história, economia, geologia. Seus escritos
influenciaram Gandhi, Tolstói e Proust. Foi o principal teórico da
preservação arquitetônica e ambiental da Inglaterra do século \textsc{xix} e
crítico perspicaz das transformações sociais trazidas ao país pela
industrialização, a qual veementemente combateu. Excêntrico, vinculado
ao romantismo, grande esteta, valorizava a sensibilidade subjetiva em
contraponto à razão; contraditório --- ao mesmo tempo aristocrático,
reacionário e simpático ao socialismo.

Em sua vasta e multifacetada obra, encontram"-se tomos sobre história,
geologia, artes plásticas, literatura, ornitologia, botânica, economia,
mitologia, poluição do meio"-ambiente. Nos últimos anos da vida, com o
agravamento dos episódios de loucura, recolheu"-se a sua casa em
Brantwood (noroeste da Inglaterra), onde faleceu em 1900.

A autobiografia --- \emph{Praeterita} --- foi sua última obra e,
certamente, seu testamento literário. Escrita ao longo de 27 anos, a
desigualdade de suas partes reflete o estado mental do autor durante o
período de sua elaboração.

O livro, heterodoxo como gênero e de caráter ``experimental'', não segue
os modelos de autobiografias então vigentes --- geralmente apresentados
em termos de confissão religiosa ---, e poderia também ser considerado
como narrativa de viagem, elegia, memória filial ou coleção de excertos
de diários.

Esta tradução contempla apenas o primeiro dos três volumes de
\emph{Praeterita}. Na França, o primeiro volume foi publicado
isoladamente, sob o título de \emph{Les Sources de Wandel} (Éditions Le
Temps Singulier, Nantes, 1980). O livro é comercializado envolto numa
tarja vermelha com a frase: ``\emph{Le Maître de Proust}''.

Ruskin foi uma das grandes influências literárias de Proust, que
traduziu \emph{The Bible of Amiens} para o francês, embora seu inglês
fosse precário --- contou com a ajuda da mãe e da amiga inglesa Marie
Nordlinger. A tradução de \emph{Praeterita} foi encetada e abandonada.

A influência de Ruskin sobre Proust se manifesta essencialmente em dois
aspectos: a narrativa não linear, mas espiralada, cíclica, na qual os
temas são retomados e retrabalhados ao longo do texto; e uma espécie de
``solipsismo literário'', no qual cada ente (animado ou inanimado) que
entra no campo existencial do escritor ganha importância e
transcendência. Seja um empregado da casa comercial de seu pai, seja um
pôr do sol sobre os Alpes, sejam cabanas entrevistas numa viagem de
carruagem, tudo que toca os sentidos e o intelecto de Ruskin ganha uma
importância quase mística e merece maravilhosas páginas de descrições,
análises e reflexões. Tais exacerbações da subjetividade são
características tanto do estilo de Ruskin quanto de Proust.

Além disso, as digressões encadeadas e razoavelmente longas de Ruskin ---
uma característica da literatura britânica, muito bem exemplificada em
Ford Madox Ford (\emph{The Good Soldier}) e conduzida ao paroxismo em
Laurence Sterne (\emph{The Life and Opinions of Tristam Shandy,
Gentleman}) --- também estão presentes em \emph{A la Recherche du Temps
Perdu}.


\begin{bibliohedra}
\tit{Ford}, Madox Ford. \emph{O bom soldado}. São Paulo: Editora 34, 1997.

\tit{Proust}, Marcel. \emph{A la recherce du temps perdu}. Paris: Éditions
Robert Laffont, 1987.

\tit{Ruskin}, John. \emph{A lâmpada da memória.} São Paulo: Ateliê Editorial, 2008.

\titidem. \emph{Praeterita and Dialecta}. New York: Alfred A. Knopf, 2005.

\titidem. \emph{Praeterita -- Souvenirs de Jeunesse}. Paris: Librairie Hachette et Cie, 1911.

\titidem. \emph{Les Sources de Wandel} Nantes: Éditions Le Temps Singulier, 1980

\tit{Sterne}, Laurence. \emph{A vida e as opiniões do cavaleiro Tristam
Shandy}. São Paulo: Companhia das Letras, 1998.
\end{bibliohedra}
